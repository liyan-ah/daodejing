\chapter{陳情表}

臣密言:“臣以險釁,夙遭閔凶。生孩六月,慈父見背;行年四歲,舅奪母志。祖母劉,湣臣孤弱,躬親撫養。臣少多疾病。九歲不行。零丁孤苦,至於成立。既無叔伯,終鮮兄弟。門衰祚薄,晚有兒息。外無期功強近之親,內無應門五尺之童。煢煢孑立,形影相吊。而劉夙嬰疾病,常在床蓐;臣待湯藥,未嘗廢離。

逮奉聖朝,沐浴清化。前太守臣逵,察臣孝廉;後刺史臣榮,舉臣秀才。

臣以供養無主,辭不赴命。詔書特下,拜臣郎中。尋蒙國恩,除臣洗馬。猥以微賤,當待東宮,非臣隕首所能上報。臣具以表聞,辭不就職。詔書切峻,責臣逋慢。郡縣逼迫,催臣上道。州司臨門,急於星火。臣欲奉詔賓士,則以劉病日篤;欲苟順私情,則告訴不許。臣之進退,實為狼狽。

伏惟聖朝,以孝治天下。凡在故老,猶蒙矜育;況臣孤苦,特為尤甚。且臣少事為朝,曆職郎署,本圖宦達,不矜名節。今臣亡國賤俘,至微至陋。

過蒙拔擢,寵命優渥,豈敢盤桓,有所希冀?但以劉日薄西山,氣息奄奄,人命危淺,朝不慮夕。臣無祖母,無以至今日?祖母無臣,無以終餘年。母孫二人,更相為命。是以區區不能廢遠。

臣密今年四十有四,祖母劉今年九十有六;是以臣盡節于陛下之日長,報劉之日短也。烏鳥私情,願乞終養!臣之辛苦,非獨蜀之人士,及二州牧伯,所見明知;皇天後土,實所共鑒。願陛下矜湣愚誠,聽臣微志。庶劉僥幸,卒保餘年。臣生當隕首,死當結草。臣不勝犬馬怖懼之情,謹拜表以聞!”

\section{詩詞賞析}
怒髮衝冠,憑闌處、瀟瀟雨歇。抬望眼、仰天長嘯,壯懷激烈。三十功名塵與土,八千里路雲和月。莫等閒、白了少年頭,空悲切! 靖康恥,猶未雪。臣子恨,何時滅?駕長車踏破,賀蘭山缺。壯志飢餐胡虜肉,笑談渴飲匈奴血。待從頭、收拾舊山河,朝天闕!
\index{怒髮}