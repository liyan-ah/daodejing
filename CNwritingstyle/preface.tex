\begin{preface}
\index{序} \index{cnwritingCJK 環境!preface}


本範例僅假設使用者有基礎的\LaTeX 知識。知道如何安裝{\color {red}免費軟體\LaTeX},不然需要找朋友幫你安裝。或到中央大學教務處網址下載中央大學碩博士論文\LaTeX 範例。

中文直書一直未突破,主要是東西方人士書寫方式的不同,西方的書寫方式多是橫向書寫,由左至右。所以由\LaTeX 排版中文直書,由上至下,由右至左有些困難。首先,將紙張橫向放置(選用{\tt landscape}),然後輸入資料,因其輸出是橫向,故必須旋轉90度(這可由{\tt PDF READER}改變)。另外\LaTeX{}內部設定長寬的定義不變,故技巧上須讓\LaTeX{}內部之長寬互換,以便橫向輸入不受內定的寬度限制。最重要的是頁碼安排,這是藉由tikz巨集的絕對座標來達成。這些設定都已在{\tt cnwritingCJK.cls}內安排完成,使用者({\tt user})不須學習或擔心。目前設定只允許更改標楷體字型大小及間距(\verb|fontsize{大小}{間距}\selectfont|)。
此體裁檔({\tt cnwritingCJK})的目的在提供中文直書的排版。若想改變排版外觀,則需有相當知識更改此體裁檔,故本體裁檔可再增修,複製,直接採用做個人用途,或供單位使用,唯不可做商業用途。此套件係自助編寫屬非賣品,但可自由使用,期望提供學生/作者便利性,以\LaTeX 做出符合中文直書格式的文稿,但不隱含任何商業價值。

\clearpage

{\color{red}已知問題}

\begin{itemize}
\item 主目錄{\tt (titletoc)}和圖表目錄{\tt (LoF,LoT)}之頁碼未成一直線。
\end{itemize}


歷史更新
\begin{itemize}
\item 2013/01/05:索引頁碼符合中文閱讀。
\item 2013/01/01:本檔案適用於 {\tt Windows},{\tt MiKTeX}引擎。
\begin{itemize}
\item 引述文獻格式符合中文閱讀。但需一篇一篇用\verb|\cite{tag}|引述。
\item 封面有頁碼,雖曾設\verb|\thispagestyle{empty}|,不成功。但用\verb|\pagenumbering{gobble}|解決了。
\item 頁面下方之頁碼,可由\verb|\pagestyle{empty}|除去。
\item 再加選項{\tt draft},會於書本左側,印出日期,未完稿,作者等資訊。圖檔則以方框表示。
\end{itemize}
\item 2012/06/30:{\tt cnwritingCJK.cls} 完成。
\begin{itemize}
\item 額外選項可有{\tt frame},{\tt noframe},標準選項同\LaTeX。
\item 加入檔案結構之解釋。
\item 中文化章節,目錄,表目錄,圖目錄,皆已完成。
\item 中文書封面製作,變化很多,建議用其他方法製作\cite{interm},轉成{\tt PDF}檔後,再用插頁方式加入。
\item 作者製作的中央大學碩博士論文體裁檔(中文橫書),有\LaTeX{}初步介紹,可參考。
\item 完稿最後要以{\tt PDF Reader}旋轉90度。
\end{itemize}
\end{itemize}
\end{preface}