%!TEX program = uplatex + dvipdfmx
%!TEX encoding = UTF-8

\documentclass{utbook}

%% \usepackage{wallpaper}
%% \ThisULCornerWallPaper{1}{竹.jpg}

%% \usepackage{geometry}
%% \newgeometry{
%%   top=50pt, bottom=50pt, left=46pt, right=46pt,
%%   headsep=25pt,
%% }
%% \savegeometry{mdGeo}
%% \loadgeometry{mdGeo}

%% fontset=windows
%% 正文  =songti=宋体
%% textit=kaishu=楷体
%% textbf=heiti=黑体
%% textsf=yahei=微软雅黑
%% texttt=仿宋
%% textbf+textsf=微软雅黑粗体
%% lishu=隶书
%% youyuan=幼圆
\usepackage[fontset=windows]{ctex}
\ctexset{
  today = big,
  %% punct = quanjiao, %% banjiao,
  autoindent = 0pt,
  %% section = {
  %%   numbering = false, % 标题不显示编号
  %%   titleformat = \Huge,
  %%   format = \texttt
  %% },
  %% chapter = {
  %%   numbering = true, % 标题不显示编号
  %%   titleformat = \centering\Huge, % 小四号字
  %%   format = \textbf % 楷体
  %% }
}

\newcommand{\yt}{\yahei}
\newcommand{\ytz}{\yahei}
\newcommand{\ytza}{\yahei}
\newcommand{\ytzn}{\yahei}

\usepackage{titlesec,lipsum}

\titleformat{\chapter}[display]{\normalfont\huge\bfseries}{}{0pt}{\Huge\yahei}
\titleformat{\section}[display]{\normalfont\huge\bfseries}{}{0pt}{\huge\lishu}


\title{\zihao{0}\textbf{黃帝內經}}

\author{\normalsize 由 \textit{喝普洱茶的麥兜} 进行編輯整理及异体字注音}
\date{\normalsize\today 版}

\begin{document}
\maketitle

\LARGE
\tableofcontents

\part{黃帝內經·素問}
\LARGE
\section{上古天真論篇第一}
%%  全角空格
昔在黃帝生而神靈弱而能言幼而徇齊長而敦敏成而登天乃問於天師曰余聞上古之人春秋皆度百歲而動作不衰今時之人年半百而動作皆衰者時世異耶人將失之耶岐伯對曰上古之人其知道者法於陰陽和於術數食飲有節起居有常不妄作勞故能形與神俱而盡終其天年度百歲乃去今時之人不然也以酒為漿以妄為常醉以入房以欲竭其精以耗散其真不知持滿不時御神務快其心逆於生樂起居無節故半百而衰也

夫上古聖人之教下也皆謂之虛邪賊風避之有時恬惔虛無真氣從之精神內守病安從來是以志閒而少欲心安而不懼形勞而不倦氣從以順各從其欲皆得所願故美其食任其服樂其俗高下不相慕其民故曰朴是以嗜欲不能勞其目淫邪不能惑其心愚智賢不肖不懼於物故合於道所以能年皆度百歲而動作不衰者以其德全不危也

帝曰人年老而無子者材力盡耶將天數然也岐伯曰女子七歲腎氣盛齒更髮長二七而天癸至任脈通太衝脈盛月事以時下故有子三七腎氣平均故真牙生而長極四七筋骨堅髮長極身體盛壯五七陽明脈衰面始焦發始墮六七三陽脈衰於上面皆焦發始白七七任脈虛太衝脈衰少天癸竭地道不通故形壞而無子也丈夫八歲腎氣實髮長齒更二八腎氣盛天癸至精氣溢寫陰陽和故能有子三八腎氣平均筋骨勁強故真牙生而長極四八筋骨隆盛肌肉滿壯五八腎氣衰發墮齒槁六八陽氣衰竭於上面焦髮鬢頒白七八肝氣衰筋不能動天癸竭精少腎藏衰形體皆極八八則齒發去腎者主水受五藏六府之精而藏之故五藏盛乃能寫今五藏皆衰筋骨解墮天癸盡矣故髮鬢白身體重行步不正而無子耳帝曰有其年已老而有子者何也岐伯曰此其天壽過度氣脈常通而腎氣有餘也此雖有子男不過盡八八女不過盡七七而天地之精氣皆竭矣帝曰夫道者年皆百數能有子乎岐伯曰夫道者能卻老而全形身年雖壽能生子也

黃帝曰余聞上古有真人者提挈天地把握陰陽呼吸精氣獨立守神肌肉若一故能壽敝天地無有終時此其道生中古之時有至人者淳德全道和於陰陽調於四時去世離俗積精全神遊行天地之間視聽八達之外此蓋益其壽命而強者也亦歸於真人其次有聖人者處天地之和從八風之理適嗜欲於世俗之間無恚嗔之心行不欲離於世被服章舉不欲觀於俗外不勞形於事內無思想之患以恬愉為務以自得為功形體不敝精神不散亦可以百數其次有賢人者法則天地象似日月辨列星辰逆從陰陽分別四時將從上古合同於道亦可使益壽而有極時


\section{四氣調神大論篇第二}

春三月此謂發陳天地俱生萬物以榮夜臥早起廣步於庭被發緩形以使志生生而勿殺予而勿奪賞而勿罰此春氣之應養生之道也逆之則傷肝夏為寒變奉長者少夏三月此謂蕃秀天地氣交萬物華實夜臥早起無厭於日使志無怒使華英成秀使氣得洩若所愛在外此夏氣之應養長之道也逆之則傷心秋為痎瘧奉收者少冬至重病秋三月此謂容平天氣以急地氣以明早臥早起與雞俱興使志安寧以緩秋刑收斂神氣使秋氣平無外其志使肺氣清此秋氣之應養收之道也逆之則傷肺冬為飧洩奉藏者少冬三月此謂閉藏水冰地坼無擾乎陽早臥晚起必待日光使志若伏若匿若有私意若已有得去寒就溫無洩皮膚使氣亟奪此冬氣之應養藏之道也逆之則傷腎春為痿厥奉生者少

天氣清淨光明者也藏德不止故不下也天明則日月不明邪害空竅陽氣者閉塞地氣者冒明雲霧不精則上應白露不下交通不表萬物命故不施不施則名木多死惡氣不發風雨不節白露不下則菀槁不榮賊風數至暴雨數起天地四時不相保與道相失則未央絕滅唯聖人從之故身無奇病萬物不失生氣不竭逆春氣則少陽不生肝氣內變逆夏氣則太陽不長心氣內洞逆秋氣則太陰不收肺氣焦滿逆冬氣則少陰不藏腎氣獨沉夫四時陰陽者萬物之根本也所以聖人春夏養陽秋冬養陰以從其根故與萬物沉浮於生長之門逆其根則伐其本壞其真矣故陰陽四時者萬物之終始也死生之本也逆之則災害生從之則苛疾不起是謂得道道者聖人行之愚者佩之從陰陽則生逆之則死從之則治逆之則亂反順為逆是謂內格

是故聖人不治已病治未病不治已亂治未亂此之謂也夫病已成而後藥之亂已成而後治之譬猶渴而穿井而鑄錐不亦晚乎


\section{生氣通天論篇第三}

黃帝曰夫自古通天者生之本本於陰陽天地之間六合之內其氣九州九竅五藏十二節皆通乎天氣其生五其氣三數犯此者則邪氣傷人此壽命之本也蒼天之氣清淨則志意治順之則陽氣固雖有賊邪弗能害也此因時之序故聖人傳精神服天氣而通神明失之則內閉九竅外壅肌肉衛氣散解此謂自傷氣之削也陽氣者若天與日失其所則折壽而不彰故天運當以日光明是故陽因而上衛外者也因於寒欲如運樞起居如驚神氣乃浮因於暑汗煩則喘喝靜則多言體若燔炭汗出而散因於濕首如裹濕熱不攘大筋緛短小筋弛長緛短為拘弛長為痿因於氣為腫四維相代陽氣乃竭陽氣者煩勞則張精絕辟積於夏使人煎厥目盲不可以視耳閉不可以聽潰潰乎若壞都汨汨乎不可止陽氣者大怒則形氣絕而血菀於上使人薄厥有傷於筋縱其若不容汗出偏沮使人偏枯汗出見濕乃生痤疿高粱之變足生大丁受如持虛勞汗當風寒薄為皶郁乃痤
陽氣者精則養神柔則養筋開闔不得寒氣從之乃生大僂陷脈為瘻留連肉腠俞氣化薄傳為善畏及為驚駭營氣不從逆於肉理乃生癰腫魄汗未盡形弱而氣爍穴俞以閉發為風瘧
故風者百病之始也清靜則肉腠閉拒雖有大風苛毒弗之能害此因時之序也

故病久則傳化上下不併良醫弗為故陽畜積病死而陽氣當隔隔者當寫不亟正治粗乃敗之
故陽氣者一日而主外平旦人氣生日中而陽氣隆日西而陽氣已虛氣門乃閉是故暮而收拒無擾筋骨無見霧露反此三時形乃困薄

岐伯曰陰者藏精而起亟也陽者衛外而為固也陰不勝其陽則脈流薄疾並乃狂陽不勝其陰則五藏氣爭九竅不通是以聖人陳陰陽筋脈和同骨髓堅固氣血皆從如是則內外調和邪不能害耳目聰明氣立如故
風客淫氣精乃亡邪傷肝也因而飽食筋脈橫解腸澼為痔因而大飲則氣逆因而強力腎氣乃傷高骨乃壞
凡陰陽之要陽密乃固兩者不和若春無秋若冬無夏因而和之是謂聖度故陽強不能密陰氣乃絕陰平陽秘精神乃治陰陽離決精氣乃絕
因於露風乃生寒熱是以春傷於風邪氣留連乃為洞洩夏傷於暑秋為痎瘧秋傷於濕上逆而咳發為痿厥冬傷於寒春必溫病四時之氣更傷五藏
陰之所生本在五味陰之五宮傷在五味是故味過於酸肝氣以津脾氣乃絕味過於咸大骨氣勞短肌心氣抑味過於甘心氣喘滿色黑腎氣不衡味過於苦脾氣不濡胃氣乃厚味過於辛筋脈沮弛精神乃央是故謹和五味骨正筋柔氣血以流腠理以密如是則骨氣以精謹道如法長有天命


\section{金匱真言論篇第四}

黃帝問曰天有八風經有五風何謂岐伯對曰八風發邪以為經風觸五藏邪氣發病所謂得四時之勝者春勝長夏長夏勝冬冬勝夏夏勝秋秋勝春所謂四時之勝也
東風生於春病在肝俞在頸項南風生於夏病在心俞在胸脅西風生於秋病在肺俞在肩背北風生於冬病在腎俞在腰股中央為土病在脾俞在脊故春氣者病在頭夏氣者病在藏秋氣者病在肩背冬氣者病在四支
故春善病鼽衄仲夏善病胸脅長夏善病洞洩寒中秋善病風瘧冬善病痹厥故冬不按蹻春不鼽衄春不病頸項仲夏不病胸脅長夏不病洞洩寒中秋不病風瘧冬不病痹厥飧洩而汗出也
夫精者身之本也故藏於精者春不病溫夏暑汗不出者秋成風瘧此平人脈法也

故曰陰中有陰陽中有陽平旦至日中天之陽陽中之陽也日中至黃昏天之陽陽中之陰也合夜至雞鳴天之陰陰中之陰也雞鳴至平旦天之陰陰中之陽也

故人亦應之夫言人之陰陽則外為陽內為陰言人身之陰陽則背為陽腹為陰言人身之藏府中陰陽則藏者為陰府者為陽肝心脾肺腎五藏皆為陰膽胃大腸小腸膀胱三焦六府皆為陽所以欲知陰中之陰陽中之陽者何也為冬病在陰夏病在陽春病在陰秋病在陽皆視其所在為施針石也故背為陽陽中之陽心也背為陽陽中之陰肺也腹為陰陰中之陰腎也腹為陰陰中之陽肝也腹為陰陰中之至陰脾也此皆陰陽表裡內外雌雄相俞應也故以應天之陰陽也

帝曰五藏應四時各有收受乎岐伯曰有東方青色入通於肝開竅於目藏精於肝其病發驚駭其味酸其類草木其畜雞其穀麥其應四時上為歲星是以春氣在頭也其音角其數八是以知病之在筋也其臭臊
南方赤色入通於心開竅於耳藏精於心故病在五藏其味苦其類火其畜羊其谷黍其應四時上為熒惑星是以知病之在脈也其音徵其數七其臭焦
中央黃色入通於脾開竅於口藏精於脾故病在舌本其味甘其類土其畜牛其谷稷其應四時上為鎮星是以知病之在肉也其音宮其數五其臭香
西方白色入通於肺開竅於鼻藏精於肺故病在背其味辛其類金其畜馬其穀稻其應四時上為太白星是以知病之在皮毛也其音商其數九其臭腥
北方黑色入通於腎開竅於二陰藏精於腎故病在谿其味咸其類水其畜彘其谷豆其應四時上為辰星是以知病之在骨也其音羽其數六其臭腐故善為脈者謹察五藏六府一逆一從陰陽表裡雌雄之紀藏之心意合心於精非其人勿教非其真勿授是謂得道


\section{陰陽應像大論篇第五}

黃帝曰陰陽者天地之道也萬物之綱紀變化之父母生殺之本始神明之府也治病必求於本故積陽為天積陰為地陰靜陽躁陽生陰長陽殺陰藏陽化氣陰成形寒極生熱熱極生寒寒氣生濁熱氣生清清氣在下則生飧洩濁氣在上則生䐜脹此陰陽反作病之逆從也
故清陽為天濁陰為地地氣上為雲天氣下為雨雨出地氣雲出天氣故清陽出上竅濁陰出下竅清陽發腠理濁陰走五藏清陽實四支濁陰歸六府
水為陰火為陽陽為氣陰為味味歸形形歸氣氣歸精精歸化精食氣形食味化生精氣生形味傷形氣傷精精化為氣氣傷於味
陰味出下竅陽氣出上竅味厚者為陰薄為陰之陽氣厚者為陽薄為陽之陰味厚則洩薄則通氣薄則發洩厚則發熱壯火之氣衰少火之氣壯壯火食氣氣食少火壯火散氣少火生氣
氣味辛甘發散為陽酸苦湧洩為陰陰勝則陽病陽勝則陰病陽勝則熱陰勝則寒重寒則熱重熱則寒寒傷形熱傷氣氣傷痛形傷腫故先痛而後腫者氣傷形也先腫而後痛者形傷氣也
風勝則動熱勝則腫燥勝則干寒勝則浮濕勝則濡寫
天有四時五行以生長收藏以生寒暑燥濕風人有五藏化五氣以生喜怒悲憂恐故喜怒傷氣寒暑傷形暴怒傷陰暴喜傷陽厥氣上行滿脈去形喜怒不節寒暑過度生乃不固故重陰必陽重陽必陰
故曰冬傷於寒春必溫病春傷於風夏生飧洩夏傷於暑秋必痎瘧秋傷於濕冬生咳嗽

帝曰余聞上古聖人論理人形列別藏府端絡經脈會通六合各從其經氣穴所發各有處名谿谷屬骨皆有所起分部逆從各有條理四時陰陽盡有經紀外內之應皆有表裡其信然乎
岐伯對曰東方生風風生木木生酸酸生肝肝生筋筋生心肝主目其在天為玄在人為道在地為化化生五味道生智玄生神神在天為風在地為木在體為筋在藏為肝在色為蒼在音為角在聲為呼在變動為握在竅為目在味為酸在志為怒怒傷肝悲勝怒風傷筋燥勝風酸傷筋辛勝酸

南方生熱熱生火火生苦苦生心心生血血生脾心主舌其在天為熱在地為火在體為脈在藏為心在色為赤在音為徵在聲為笑在變動為憂在竅為舌在味為苦在志為喜喜傷心恐勝喜熱傷氣寒勝熱苦傷氣咸勝苦

中央生濕濕生土土生甘甘生脾脾生肉肉生肺脾主口其在天為濕在地為土在體為肉在藏為脾在色為黃在音為宮在聲為歌在變動為噦在竅為口在味為甘在志為思思傷脾怒勝思濕傷肉風勝濕甘傷肉酸勝甘

西方生燥燥生金金生辛辛生肺肺生皮毛皮毛生腎肺主鼻其在天為燥在地為金在體為皮毛在藏為肺在色為白在音為商在聲為哭在變動為咳在竅為鼻在味為辛在志為憂憂傷肺喜勝憂熱傷皮毛寒勝熱辛傷皮毛苦勝辛

北方生寒寒生水水生咸咸生腎腎生骨髓髓生肝腎主耳其在天為寒在地為水在體為骨在藏為腎在色為黑在音為羽在聲為呻在變動為栗在竅為耳在味為咸在志為恐恐傷腎思勝恐寒傷血燥勝寒咸傷血甘勝咸

故曰天地者萬物之上下也陰陽者血氣之男女也左右者陰陽之道路也水火者陰陽之徵兆也陰陽者萬物之能始也故曰陰在內陽之守也陽在外陰之使也

帝曰法陰陽奈何岐伯曰陽勝則身熱腠理閉喘粗為之仰汗不出而熱齒干以煩冤腹滿死能冬不能夏陰勝則身寒汗出身常清數栗而寒寒則厥厥則腹滿死能夏不能冬此陰陽更勝之變病之形能也
帝曰調此二者奈何岐伯曰能知七損八益則二者可調不知用此則早衰之節也年四十而陰氣自半也起居衰矣年五十體重耳目不聰明矣年六十陰痿氣大衰九竅不利下虛上實涕泣俱出矣故曰知之則強不知則老故同出而名異耳智者察同愚者察異愚者不足智者有餘有餘則耳目聰明身體輕強老者復壯壯者益治是以聖人為無為之事樂恬憺之能從欲快志於虛無之守故壽命無窮與天地終此聖人之治身也

天不足西北故西北方陰也而人右耳目不如左明也地不滿東南故東南方陽也而人左手足不如右強也帝曰何以然岐伯曰東方陽也陽者其精並於上並於上則上明而下虛故使耳目聰明而手足不便也西方陰也陰者其精並於下並於下則下盛而上虛故其耳目不聰明而手足便也故俱感於邪其在上則右甚在下則左甚此天地陰陽所不能全也故邪居之
故天有精地有形天有八紀地有五里故能為萬物之父母清陽上天濁陰歸地是故天地之動靜神明為之綱紀故能以生長收藏終而復始惟賢人上配天以養頭下象地以養足中傍人事以養五藏天氣通於肺地氣通於嗌風氣通於肝雷氣通於心谷氣通於脾雨氣通於腎六經為川腸胃為海九竅為水注之氣以天地為之陰陽陽之汗以天地之雨名之陽之氣以天地之疾風名之暴氣象雷逆氣象陽故治不法天之紀不用地之理則災害至矣

故邪風之至疾如風雨故善治者治皮毛其次治肌膚其次治筋脈其次治六府其次治五藏治五藏者半死半生也故天之邪氣感則害人五藏水谷之寒熱感則害於六府地之濕氣感則害皮肉筋脈
故善用針者從陰引陽從陽引陰以右治左以左治右以我知彼以表知裡以觀過與不及之理見微得過用之不殆善診者察色按脈先別陰陽審清濁而知部分視喘息聽音聲而知所苦觀權衡規矩而知病所主按尺寸觀浮沉滑澀而知病所生以治無過以診則不失矣
故曰病之始起也可刺而已其盛可待衰而已故因其輕而揚之因其重而減之因其衰而彰之形不足者溫之以氣精不足者補之以味其高者因而越之其下者引而竭之中滿者寫之於內其有邪者漬形以為汗其在皮者汗而發之其慓悍者按而收之其實者散而寫之審其陰陽以別柔剛陽病治陰陰病治陽定其血氣各守其鄉血實宜決之氣虛宜掣引之

\section{陰陽離合論篇第六}

黃帝問曰余聞天為陽地為陰日為陽月為陰大小月三百六十日成一歲人亦應之今三陰三陽不應陰陽其故何也岐伯對曰陰陽者數之可十推之可百數之可千推之可萬萬之大不可勝數然其要一也
天覆地載萬物方生未出地者命曰陰處名曰陰中之陰則出地者命曰陰中之陽陽予之正陰為之主故生因春長因夏收因秋藏因冬失常則天地四塞陰陽之變其在人者亦數之可數
帝曰願聞三陰三陽之離合也岐伯曰聖人南面而立前曰廣明後曰太沖太沖之地名曰少陰少陰之上名曰太陽太陽根起於至陰結於命門名曰陰中之陽中身而上名曰廣明廣明之下名曰太陰太陰之前名曰陽明陽明根起於厲兌名曰陰中之陽厥陰之表名曰少陽少陽根起於竅陰名曰陰中之少陽是故三陽之離合也太陽為開陽明為闔少陽為樞三經者不得相失也搏而勿浮命曰一陽

帝曰願聞三陰岐伯曰外者為陽內者為陰然則中為陰其衝在下名曰太陰太陰根起於隱白名曰陰中之陰太陰之後名曰少陰少陰根起於湧泉名曰陰中之少陰少陰之前名曰厥陰厥陰根起於大敦陰之絕陽名曰陰之絕陰是故三陰之離合也太陰為開厥陰為闔少陰為樞

三經者不得相失也搏而勿沉名曰一陰陰陽\yt{𩅞}\yt{𩅞}積傳為一週氣裡形表而為相成也


\section{陰陽別論篇第七}

黃帝問曰人有四經十二從何謂岐伯對曰四經應四時十二從應十二月十二月應十二脈脈有陰陽知陽者知陰知陰者知陽凡陽有五五五二十五陽所謂陰者真藏也見則為敗敗必死也所謂陽者胃脘之陽也別於陽者知病處也別於陰者知死生之期
三陽在頭三陰在手所謂一也別於陽者知病忌時別於陰者知死生之期謹熟陰陽無與眾謀
所謂陰陽者去者為陰至者為陽靜者為陰動者為陽遲者為陰數者為陽凡持真脈之藏脈者肝至懸絕急十八日死心至懸絕九日死肺至懸絕十二日死腎至懸絕七日死脾至懸絕四日死
曰二陽之病發心脾有不得隱曲女子不月其傳為風消其傳為息賁者死不治
曰三陽為病發寒熱下為癰腫及為痿厥腨㾓其傳為索澤其傳為頹疝
曰一陽發病少氣善咳善洩其傳為心掣其傳為隔
二陽一陰發病主驚駭背痛善噫善欠名曰風厥
二陰一陽發病善脹心滿善氣
三陽三陰發病為偏枯痿易四支不舉
一陽曰鉤鼓一陰曰毛鼓陽勝急曰弦鼓陽至而絕曰石陰陽相過曰溜
陰爭於內陽擾於外魄汗未藏四逆而起起則熏肺使人喘鳴陰之所生和本曰和是故剛與剛陽氣破散陰氣乃消亡淖則剛柔不和經氣乃絕
死陰之屬不過三日而死生陽之屬不過四日而死所謂生陽死陰者肝之心謂之生陽心之肺謂之死陰肺之腎謂之重陰腎之脾謂之辟陰死不治
結陽者腫四支結陰者便血一升再結二升三結三升陰陽結斜多陰少陽曰石水少腹腫二陽結謂之消三陽結謂之隔三陰結謂之水一陰一陽結謂之喉痹陰搏陽別謂之有子陰陽虛腸辟死陽加於陰謂之汗陰虛陽搏謂之崩
三陰俱搏二十日夜半死二陰俱搏十三日夕時死一陰俱搏十日死三陽俱搏且鼓三日死三陰三陽俱搏心腹滿發盡不得隱曲五日死二陽俱搏其病溫死不治不過十日死


\section{靈蘭秘典論篇第八}
%% 羶
黃帝問曰願聞十二藏之相使貴賤何如岐伯對曰悉乎哉問也請遂言之心者君主之官也神明出焉肺者相傅之官治節出焉肝者將軍之官謀慮出焉膽者中正之官決斷出焉膻中者臣使之官喜樂出焉脾胃者倉廩之官五味出焉大腸者傳道之官變化出焉小腸者受盛之官化物出焉腎者作強之官伎巧出焉三焦者決瀆之官水道出焉膀胱者州都之官津液藏焉氣化則能出矣
凡此十二官者不得相失也故主明則下安以此養生則壽歿世不殆以為天下則大昌主不明則十二官危使道閉塞而不通形乃大傷以此養生則殃以為天下者其宗大危戒之戒之
至道在微變化無窮孰知其原窘乎哉消者瞿瞿孰知其要閔閔之當孰者為良恍惚之數生於毫氂毫氂之數起於度量千之萬之可以益大推之大之其形乃制
黃帝曰善哉余聞精光之道大聖之業而宣明大道非齋戒擇吉日不敢受也黃帝乃澤吉日良兆而藏靈蘭之室以傳保焉

\section{六節藏象論篇第九}

黃帝問曰余聞天以六六之節以成一歲人以九九制會計人亦有三百六十五節以為天地久矣不知其所謂也岐伯對曰昭乎哉問也請遂言之夫六六之節九九制會者所以正天之度氣之數也天度者所以制日月之行也氣數者所以紀化生之用也
天為陽地為陰日為陽月為陰行有分紀周有道理日行一度月行十三度而有奇焉故大小月三百六十五日而成歲積氣余而盈閏矣立端於始表正於中推余於終而天度畢矣
帝曰余已聞天度矣願聞氣數何以合之岐伯曰天以六六為節地以九九制會天有十日日六竟而周甲甲六復而終歲三百六十日法也夫自古通天者生之本本於陰陽其氣九州九竅皆通乎天氣故其生五其氣三三而成天三而成地三而成人三而三之合則為九九分為九野九野為九藏故形藏四神藏五合為九藏以應之也
帝曰余已聞六六九九之會也夫子言積氣盈閏願聞何謂氣請夫子發蒙解惑焉岐伯曰此上帝所秘先師傳之也帝曰請遂聞之岐伯曰五日謂之候三候謂之氣六氣謂之時四時謂之歲而各從其主治焉五運相襲而皆治之終期之日週而復始時立氣布如環無端候亦同法故曰不知年之所加氣之盛衰虛實之所起不可以為工矣
帝曰五運之始如環無端其太過不及何如岐伯曰五氣更立各有所勝盛虛之變此其常也帝曰平氣何如岐伯曰無過者也帝曰太過不及奈何岐伯曰在經有也帝曰何謂所勝岐伯曰春勝長夏長夏勝冬冬勝夏夏勝秋秋勝春所謂得五行時之勝各以氣命其藏帝曰何以知其勝岐伯曰求其至也皆歸始春未至而至此謂太過則薄所不勝而乘所勝也命曰氣淫不分邪僻內生工不能禁至而不至此謂不及則所勝妄行而所生受病所不勝薄之也命曰氣迫所謂求其至者氣至之時也謹候其時氣可與期失時反候五治不分邪僻內生工不能禁也
帝曰有不襲乎岐伯曰蒼天之氣不得無常也氣之不襲是謂非常非常則變矣帝曰非常而變奈何岐伯曰變至則病所勝則微所不勝則甚因而重感於邪則死矣故非其時則微當其時則甚也
帝曰善余聞氣合而有形因變以正名天地之運陰陽之化其於萬物孰少孰多可得聞乎岐伯曰悉哉問也天至廣不可度地至大不可量大神靈問請陳其方草生五色五色之變不可勝視草生五味五味之美不可勝極嗜欲不同各有所通天食人以五氣地食人以五味五氣入鼻藏於心肺上使五色修明音聲能彰五味入口藏於腸胃味有所藏以養五氣氣和而生津液相成神乃自生
帝曰藏象何如岐伯曰心者生之本神之變也其華在面其充在血脈為陽中之太陽通於夏氣肺者氣之本魄之處也其華在毛其充在皮為陽中之太陰通於秋氣腎者主蟄封藏之本精之處也其華在發其充在骨為陰中之少陰通於冬氣肝者罷極之本魂之居也其華在爪其充在筋以生血氣其味酸其色蒼此為陽中之少陽通於春氣脾胃大腸小腸三焦膀胱者倉廩之本營之居也名曰器能化糟粕轉味而入出者也其華在唇四白其充在肌其味甘其色黃此至陰之類通於土氣凡十一藏取決於膽也
故人迎一盛病在少陽二盛病在太陽三盛病在陽明四盛已上為格陽寸口一盛病在厥陰二盛病在少陰三盛病在太陰四盛已上為關陰人迎與寸口俱盛四倍已上為關格關格之脈羸不能極於天地之精氣則死矣


\section{五藏生成篇第十}

心之合脈也其榮色也其主腎也肺之合皮也其榮毛也其主心也肝之合筋也其榮爪也其主肺也脾之合肉也其榮唇也其主肝也腎之合骨也其榮發也其主脾也
是故多食咸則脈凝泣而變色多食苦則皮槁而毛拔多食辛則筋急而爪枯多食酸則肉胝而唇揭多食甘則骨痛而發落此五味之所傷也故心欲苦肺欲辛肝欲酸脾欲甘腎欲咸此五味之所合也
五藏之氣故色見青如草茲者死黃如枳實者死黑如炲者死赤如衃血者死白如枯骨者死此五色之見死也青如翠羽者生赤如雞冠者生黃如蟹腹者生白如豕膏者生黑如烏羽者生此五色之見生也生於心如以縞裹朱生於肺如以縞裹紅生於肝如以縞裹紺生於脾如以縞裹栝樓實生於腎如以縞裹紫此五藏所生之外榮也
色味當五藏白當肺辛赤當心苦青當肝酸黃當脾甘黑當腎咸故白當皮赤當脈青當筋黃當肉黑當骨
諸脈者皆屬於目諸髓者皆屬於腦諸筋者皆屬於節諸血者皆屬於心諸氣者皆屬於肺此四支八谿之朝夕也
故人臥血歸於肝肝受血而能視足受血而能步掌受血而能握指受血而能攝臥出而風吹之血凝於膚者為痹凝於脈者為泣凝於足者為厥此三者血行而不得反其空故為痹厥也人有大谷十二分小谿三百五十四名少十二俞此皆衛氣之所留止邪氣之所客也針石緣而去之
診病之始五決為紀欲知其始先建其母所謂五決者五脈也
是以頭痛巔疾下虛上實過在足少陰巨陽甚則入腎徇蒙招尤目冥耳聾下實上虛過在足少陽厥陰甚則入肝腹滿䐜脹支鬲胠脅下厥上冒過在足太陰陽明咳嗽上氣厥在胸中過在手陽明太陰心煩頭痛病在鬲中過在手巨陽少陰
夫脈之小大滑澀浮沉可以指別五藏之象可以類推五藏相音可以意識五色微診可以目察能合脈色可以萬全赤脈之至也喘而堅診曰有積氣在中時害於食名曰心痹得之外疾思慮而心虛故邪從之白脈之至也喘而浮上虛下實驚有積氣在胸中喘而虛名曰肺痹寒熱得之醉而使內也青脈之至也長而左右彈有積氣在心下支胠名曰肝痹得之寒濕與疝同法腰痛足清頭痛黃脈之至也大而虛有積氣在腹中有厥氣名曰厥疝女子同法得之疾使四支汗出當風黑脈之至也上堅而大有積氣在小腹與陰名曰腎痹得之沐浴清水而臥
凡相五色之奇脈面黃目青面黃目赤面黃目白面黃目黑者皆不死也面青目赤面赤目白面青目黑面黑目白面赤目青皆死也


%% \section{五藏別論篇第十一}

%%   黃帝問曰:余聞方士,或以腦髓為藏,或以腸胃為藏,或以為府,敢問更相反,皆自謂是,不知其道,願聞其說。
%%   岐伯對曰:腦髓骨脈膽女子胞,此六者地氣之所生也,皆藏於陰而像於地,故藏而不寫,名曰奇恆之府。夫胃大腸小腸三焦膀胱,此五者,天氣之所生也,其氣象天,故寫而不藏,此受五藏濁氣,名曰傳化之府,此不能久留,輸瀉者也。魄門亦為五藏使,水谷不得久藏。所謂五藏者,藏精氣而不寫也,故滿而不能實。六府者,傳化物而不藏,故實而不能滿也。所以然者,水谷入口,則胃實而腸虛;食下,則腸實而胃虛。故曰:實而不滿,滿而不實也。
%%   帝曰:氣口何以獨為五藏主?岐伯曰:胃者,水穀之海,六府之大源也。五味入口,藏於胃,以養五藏氣,氣口亦太陰也。是以五藏六府之氣味,皆出於胃,變見於氣口。故五氣入鼻,藏於心肺,心肺有病,而鼻為之不利也。
%%   凡治病必察其下,適其脈,觀其志意與其病也。拘於鬼神者,不可與言至德。惡於針石者,不可與言至巧。病不許治者,病必不治,治之無功矣。


%% \section{異法方宜論篇第十二}

%%   黃帝問曰:醫之治病也,一病而治各不同,皆愈何也?岐伯對曰:地勢使然也。故東方之域,天地之所始生也,魚鹽之地,海濱傍水,其民食魚而嗜咸,皆安其處,美其食,魚者使人熱中,鹽者勝血,故其民皆黑色疏理,其病皆為癰瘍,其治宜砭石,故砭石者,亦從東方來。
%%   西方者,金玉之域,沙石之處,天地之所收引也,其民陵居而多風,水土剛強,其民不衣而褐薦,其民華食而脂肥,故邪不能傷其形體,其病生於內,其治宜毒藥,故毒藥者,亦從西方來。
%%   北方者,天地所閉藏之域也,其地高陵居,風寒冰冽,其民樂野處而乳食,藏寒生滿病,其治宜灸焫,故灸焫者,亦從北方來。
%%   南方者,天地所長養,陽之所盛處也,其地下,水土弱,霧露之所聚也,其民嗜酸而食胕,故其民皆致理而赤色,其病攣痹,其治宜微針,故九針者,亦從南方來。
%%   中央者,其地平以濕,天地所以生萬物也眾,其民食雜而不勞,故其病多痿厥寒熱,其治宜導引按蹻,故導引按蹻者,亦從中央出也。
%%   故聖人雜合以治,各得其所宜,故治所以異而病皆愈者,得病之情,知治之大體也。
%% \section{移精變氣論篇第十三}

%%   黃帝問曰:余聞古之治病,惟其移精變氣,可祝由而已。今世治病,毒藥治其內,針石治其外,或愈或不愈,何也?
%%   岐伯對曰:往古人居禽獸之間,動作以避寒,陰居以避暑,內無眷慕之累,外無伸宦之形,此恬憺之世,邪不能深入也。故毒藥不能治其內,針石不能治其外,故可移精祝由而已。當今之世不然,憂患緣其內,苦形傷其外,又失四時之從,逆寒暑之宜,賊風數至,虛邪朝夕,內至五藏骨髓,外傷空竅肌膚,所以小病必甚,大病必死,故祝由不能已也。
%%   帝曰:善。余欲臨病人,觀死生,決嫌疑,欲知其要,如日月光,可得聞乎?岐伯曰:色脈者,上帝之所貴也,先師之所傳也。上古使僦貸季,理色脈而通神明,合之金木水火土四時八風六合,不離其常,變化相移,以觀其妙,以知其要,欲知其要,則色脈是矣。色以應日,脈以應月,常求其要,則其要也。夫色之變化,以應四時之脈,此上帝之所貴,以合於神明也,所以遠死而近生。生道以長,命曰聖王。中古之治病,至而治之,湯液十日,以去八風五痹之病,十日不巳,治以草蘇草荄之枝,本末為助,標本已得,邪氣乃服。暮世之治病也則不然,治不本四時,不知日月,不審逆從,病形已成,乃欲微針治其外,湯液治其內,粗工凶凶,以為可攻,故病未已,新病復起。
%%   帝曰:願聞要道。岐伯曰:治之要極,無失色脈,用之不惑,治之大則。逆從到行,標本不得,亡神失國。去故就新,乃得真人。帝曰:余聞其要於夫子矣,夫子言不離色脈,此余之所知也。岐伯曰:治之極於一。帝曰:何謂一?岐伯曰:一者,因得之。帝曰:奈何?岐伯曰:閉戶塞牖,系之病者,數問其情,以從其意,得神者昌,失神者亡。帝曰:善。


%% \section{湯液醪醴論篇第十四}

%%   黃帝問曰:為五穀湯液及醪醴,奈何?岐伯對曰:必以稻米,炊之稻薪,稻米者完,稻薪者堅。帝曰:何以然?岐伯曰:此得天地之和,高下之宜,故能至完,伐取得時,故能至堅也。
%%   帝曰:上古聖人作湯液醪醴,為而不用,何也?岐伯曰:自古聖人之作湯液醪醴者,以為備耳,夫上古作湯液,故為而弗服也。中古之世,道德稍衰,邪氣時至,服之萬全。帝曰:今之世不必已何也。岐伯曰:當今之世,必齊毒藥攻其中,鑱石針艾治其外也。
%%   帝曰:形弊血盡而功不立者何?岐伯曰:神不使也。帝曰:何謂神不使?岐伯曰:針石道也。精神不進,志意不治,故病不可愈。今精壞神去,榮衛不可復收。何者,嗜欲無窮,而憂患不止,精氣弛壞,營泣衛除,故神去之而病不愈也。
%%   帝曰:夫病之始生也,極微極精,必先入結於皮膚。今良工皆稱曰:病成名曰逆,則針石不能治,良藥不能及也。今良工皆得其法,守其數,親戚兄弟遠近音聲日聞於耳,五色日見於目,而病不愈者,亦何暇不早乎。岐伯曰:病為本,工為標,標本不得,邪氣不服,此之謂也。
%%   帝曰:其有不從毫毛而生,五藏陽以竭也,津液充郭,其魄獨居,孤精於內,氣耗於外,形不可與衣相保,此四極急而動中,是氣拒於內,而形施於外,治之奈何?岐伯曰:平治於權衡,去宛陳莝,微動四極,溫衣,繆刺其處,以復其形。開鬼門,潔淨府,精以時服,五陽已布,疏滌五藏,故精自生,形自盛,骨肉相保,巨氣乃平。帝曰:善。


%% \section{玉版論要篇第十五}

%%   黃帝問曰:余聞揆度奇恆,所指不同,用之奈何?岐伯對曰:揆度者,度病之淺深也。奇恆者,言奇病也。請言道之至數,五色脈變,揆度奇恆,道在於一。神轉不回,回則不轉,乃失其機,至數之要,迫近於微,著之玉版,命曰合玉機。
%%   容色見上下左右,各在其要。其色見淺者,湯液主治,十日已。其見深者,必齊主治,二十一日已。其見大深者,醪酒主治,百日已。色夭面脫,不治,百日盡已。脈短氣絕死,病溫虛甚死。色見上下左右,各在其要。上為逆,下為從。女子右為逆,左為從;男子左為逆,右為從。易,重陽死,重陰死。陰陽反他,治在權衡相奪,奇恆事也,揆度事也。
%%   搏脈痹躄,寒熱之交。脈孤為消氣,虛洩為奪血。孤為逆,虛為從。行奇恆之法,以太陰始。行所不勝曰逆,逆則死;行所勝曰從,從則活。八風四時之勝,終而復始,逆行一過,不復可數,論要畢矣。


%% \section{診要經終論篇第十六}

%%   黃帝問曰:診要何如?岐伯對曰:正月二月,天氣始方,地氣始發,人氣在肝。三月四月,天氣正方,地氣定發,人氣在脾。五月六月,天氣盛,地氣高,人氣在頭。七月八月,陰氣始殺,人氣在肺。九月十月,陰氣始冰,地氣始閉,人氣在心。十一月十二月,冰復,地氣合,人氣在腎。
%%   故春刺散俞,及與分理,血出而止,甚者傳氣,間者環也。夏刺絡俞,見血而止,盡氣閉環,痛病必下。秋刺皮膚,循理,上下同法,神變而止。冬刺俞竅於分理,甚者直下,間者散下。春夏秋冬,各有所刺,法其所在。
%%   春刺夏分,脈亂氣微,入淫骨髓,病不能愈,令人不嗜食,又且少氣。春刺秋分,筋攣逆氣,環為嗽,病不愈,令人時驚,又且哭。春刺冬分,邪氣著藏,令人脹,病不愈,又且欲言語。
%%   夏刺春分,病不愈,令人解墮,夏刺秋分,病不愈,令人心中欲無言,惕惕如人將捕之。夏刺冬分,病不愈,令人少氣,時欲怒。
%%   秋刺春分,病不已,令人惕然,欲有所為,起而忘之。秋刺夏分,病不已,令人益嗜臥,又且善夢。秋刺冬分,病不已,令人灑灑時寒。
%%   冬刺春分,病不已,令人欲臥不能眠,眠而有見。冬刺夏分,病不愈,氣上,發為諸痹。冬刺秋分,病不已,令人善渴。
%%   凡刺胸腹者,必避五藏。中心者,環死;中脾者,五日死;中腎者,七日死;中肺者,五日死;中鬲者,皆為傷中,其病雖愈,不過一歲必死。刺避五藏者,知逆從也。所謂從者,鬲與脾腎之處,不知者反之。刺胸腹者,必以布憿著之,乃從單布上刺,刺之不愈,復刺。刺針必肅,刺腫搖針,經刺勿搖,此刺之道也。
%%   帝曰:願聞十二經脈之終,奈何?岐伯曰:太陽之脈,其終也,戴眼反折,瘛瘲,其色白,絕汗乃出,出則死矣。少陽終者,耳聾,百節皆縱,目寰絕系,絕系一日半死,其死也,色先青白,乃死矣。陽明終者,口目動作,善驚忘言,色黃,其上下經盛,不仁,則終矣。少陰終者,面黑齒長而垢,腹脹閉,上下不通而終矣。太陰終者,腹脹閉不得息,善噫善嘔,嘔則逆,逆則面赤,不逆則上下不通,不通則面黑皮,毛焦而終矣。厥陰終者,中熱嗌干,善溺心煩,甚則舌卷卵上縮而終矣。此十二經之所敗也。


%% \section{脈要精微論篇第十七}

%%   黃帝問曰:診法何如?岐伯對曰:診法常以平旦,陰氣未動,陽氣未散,飲食未進,經脈未盛,絡脈調勻,氣血未亂,故乃可診有過之脈。
%%   切脈動靜而視精明,察五色,觀五藏有餘不足,六府強弱,形之盛衰,以此參伍,決死生之分。
%%   夫脈者,血之府也,長則氣治,短則氣病,數則煩心,大則病進,上盛則氣高,下盛則氣脹,代則氣衰,細則氣少,澀則心痛,渾渾革至如湧泉,病進而色弊,綿綿其去如弦絕,死。
%%   夫精明五色者,氣之華也。赤欲如白裹朱,不欲如赭;白欲如鵝羽,不欲如鹽;青欲如蒼璧之澤,不欲如藍;黃欲如羅裹雄黃,不欲如黃土;黑欲如重漆色,不欲如地蒼。五色精微象見矣,其壽不久也。夫精明者,所以視萬物,別白黑,審短長。以長為短,以白為黑,如是則精衰矣。
%%   五藏者,中之守也,中盛藏滿,氣勝傷恐者,聲如從室中言,是中氣之濕也。言而微,終日乃復言者,此奪氣也。衣被不斂,言語善惡,不避親疏者,此神明之亂也。倉廩不藏者,是門戶不要也。水泉不止者,是膀胱不藏也。得守者生,失守者死。
%%   夫五藏者,身之強也,頭者精明之府,頭頃視深,精神將奪矣。背者胸中之府,背曲肩隨,府將壞矣。腰者腎之府,轉搖不能,腎將憊矣。膝者筋之府,屈伸不能,行則僂附,筋將憊矣。骨者髓之府,不能久立,行則振掉,骨將憊矣。得強則生,失強則死。
%%   岐伯曰:反四時者,有餘為精,不足為消。應太過,不足為精;應不足,有餘為消。陰陽不相應,病名曰關格。
%%   帝曰:脈其四時動奈何,知病之所在奈何,知病之所變奈何,知病乍在內奈何,知病乍在外奈何,請問此五者,可得聞乎。岐伯曰:請言其與天運轉大也。萬物之外,六合之內,天地之變,陰陽之應,彼春之暖,為夏之暑,彼秋之忿,為冬之怒,四變之動,脈與之上下,以春應中規,夏應中矩,秋應中衡,冬應中權。是故冬至四十五日,陽氣微上,陰氣微下;夏至四十五日,陰氣微上,陽氣微下。陰陽有時,與脈為期,期而相失,知脈所分,分之有期,故知死時。微妙在脈,不可不察,察之有紀,從陰陽始,始之有經,從五行生,生之有度,四時為宜,補寫勿失,與天地如一,得一之情,以知死生。是故聲合五音,色合五行,脈合陰陽。
%%   是知陰盛則夢涉大水恐懼,陽盛則夢大火燔灼,陰陽俱盛則夢相殺毀傷;上盛則夢飛,下盛則夢墮;甚飽則夢予,甚飢則夢取;肝氣盛則夢怒,肺氣盛則夢哭;短蟲多則夢聚眾,長蟲多則夢相擊毀傷。
%%   是故持脈有道,虛靜為保。春日浮,如魚之遊在波;夏日在膚,泛泛乎萬物有餘;秋日下膚,蟄蟲將去;冬日在骨,蟄蟲周密,君子居室。故曰:知內者按而紀之,知外者終而始之。此六者,持脈之大法。
%%   心脈搏堅而長,當病舌卷不能言;其耎而散者,當消環自已。肺脈搏堅而長,當病唾血;其耎而散者,當病灌汗,至今不復散發也。肝脈搏堅而長,色不青,當病墜若搏,因血在脅下,令人喘逆;其耎而散色澤者,當病溢飲,溢飲者喝暴多飲,而易入肌皮腸胃之外也。胃脈搏堅而長,其色赤,當病折髀;其耎而散者,當病食痹。脾脈搏堅而長,其色黃,當病少氣;其耎而散色不澤者,當病足胻腫,若水狀也。腎脈搏堅而長,其色黃而赤者,當病折腰;其而散者,當病少血,至今不復也。
%%   帝曰:診得心脈而急,此為何病,病形何如?岐伯曰:病名心疝,少腹當有形也。帝曰:何以言之。岐伯曰:心為牡藏,小腸為之使,故曰少腹當有形也。帝曰:診得胃脈,病形何如?岐伯曰:胃脈實則脹,虛則洩。
%%   帝:病成而變何謂?岐伯曰:風成為寒熱,癉成為消中,厥成為巔疾,久風為飧洩,脈風成為癘,病之變化,不可勝數。
%%   帝曰:諸癰腫筋攣骨痛,此皆安生。岐伯曰:此寒氣之腫,八風之變也。帝曰:治之奈何?岐伯曰:此四時之病,以其勝治之,愈也。
%%   帝曰:有故病五藏發動,因傷脈色,各何以知其久暴至之病乎。岐伯曰:悉乎哉問也。徵其脈小色不奪者,新病也;徵其脈不奪其色奪者,此久病也;徵其脈與五色俱奪者,此久病也;徵其脈與五色俱不奪者,新病也。肝與腎脈並至,其色蒼赤,當病毀傷,不見血,已見血,濕若中水也。
%%   尺內兩傍,則季脅也,尺外以候腎,尺裡以候腹。中附上,左外以候肝,內以候鬲;右,外以候胃,內以候脾。上附上,右外以候肺,內以候胸中;左,外以候心,內以候羶中。前以候前,後以候後。上竟上者,胸喉中事也;下竟下者,少腹腰股膝脛足中事也。
%%   粗大者,陰不足陽有餘,為熱中也。來疾去徐,上實下虛,為厥巔疾;來徐去疾,上虛下實,為惡風也。故中惡風者,陽氣受也。有脈俱沉細數者,少陰厥也;沉細數散者,寒熱也;浮而散者為眴僕。諸浮不躁者皆在陽,則為熱,其有躁者在手。諸細而沉者皆在陰,則為骨痛;其有靜者在足。數動一代者,病在陽之脈也,洩及便膿血。諸過者,切之,澀者陽氣有餘也,滑者陰氣有餘也。陽氣有餘,為身熱無汗,陰氣有餘,為多汗身寒,陰陽有餘,則無汗而寒。推而外之,內而不外,有心腹積也。推而內之,外而不內,身有熱也。推而上之,上而不下,腰足清也。推而下之,下而不上,頭項痛也。按之至骨,脈氣少者,腰脊痛而身有痹也。
%% \section{平人氣象論篇第十八}

%%   黃帝問曰:平人何如?岐伯對曰:人一呼脈再動,一吸脈亦再動,呼吸定息脈五動,閏以太息,命曰平人。平人者,不病也。常以不病調病人,醫不病,故為病人平息以調之為法。人一呼脈一動,一吸脈一動,曰少氣。人一呼脈三動,一吸脈三動而躁,尺熱曰病溫,尺不熱脈滑曰病風,脈澀曰痹。人一呼脈四動以上曰死,脈絕不至曰死,乍疏乍數曰死。
%%   平人之常氣稟於胃,胃者,平人之常氣也,人無胃氣曰逆,逆者死。
%%   春胃微弦曰平,弦多胃少曰肝病,但弦無胃曰死,胃而有毛曰秋病,毛甚曰今病。藏真散於肝,肝藏筋膜之氣也,夏胃微鉤曰平,鉤多胃少曰心病,但鉤無胃曰死,胃而有石曰冬病,石甚曰今病。藏真通於心,心藏血脈之氣也。長夏胃微軟弱曰平,弱多胃少曰脾病,但代無胃曰死,軟弱有石曰冬病,弱甚曰今病。藏真濡於脾,脾藏肌肉之氣也。秋胃微毛曰平,毛多胃少曰肺病,但毛無胃曰死,毛而有弦曰春病,弦甚曰今病。藏真高於肺,以行榮衛陰陽也。冬胃微石曰平,石多胃少曰腎病,但石無胃曰死,石而有鉤曰夏病,鉤甚曰今病。藏真下於腎,腎藏骨髓之氣也。
%%   胃之大絡,名曰虛裡,貫鬲絡肺,出於左乳下,其動應衣,脈宗氣也。盛喘數絕者,則病在中;結而橫,有積矣;絕不至曰死。乳之下其動應衣,宗氣洩也。
%%   欲知寸口太過與不及,寸口之脈中手短者,曰頭痛。寸口脈中手長者,曰足脛痛。寸口脈中手促上擊者,曰肩背痛。寸口脈沉而堅者,曰病在中。寸口脈浮而盛者,曰病在外。寸口脈沉而弱,曰寒熱及疝瘕少腹痛。寸口脈沉而橫,曰脅下有積,腹中有橫積痛。寸口脈沉而喘,曰寒熱。脈盛滑堅者,曰病在外。脈小實而堅者,病在內。脈小弱以澀,謂之久病。脈滑浮而疾者,謂之新病。脈急者,曰疝瘕少腹痛。脈滑曰風。脈澀曰痹。緩而滑曰熱中。盛而緊曰脹。
%%   脈從陰陽,病易已;脈逆陰陽,病難已。脈得四時之順,曰病無他;脈反四時及不間藏,曰難已。
%%   臂多青脈,曰脫血。尺脈緩澀,謂之解(亻亦)。安臥脈盛,謂之脫血。尺澀脈滑,謂之多汗。尺寒脈細,謂之後洩。脈尺常熱者,謂之熱中。
%%   肝見庚辛死,心見壬癸死,脾見甲乙死,肺見丙丁死,腎見戊己死,是謂真藏見,皆死。
%%   頸脈動喘疾欬,曰水。目裹微腫如臥蠶起之狀,曰水。溺黃赤安臥者,黃疸。已食如飢者,胃疸。面腫曰風。足脛腫曰水。目黃者曰黃疸。婦人手少陰脈動甚者,妊子也。
%%   脈有逆從,四時未有藏形,春夏而脈瘦,秋冬而脈浮大,命曰逆四時也。風熱而脈靜,洩而脫血脈實,病在中,脈虛,病在外,脈澀堅者,皆難治,命曰反四時也。
%%   人以水谷為本,故人絕水谷則死,脈無胃氣亦死,所謂無胃氣者,但得真藏脈不得胃氣也。所謂脈不得胃氣者,肝不弦腎不石也。
%%   太陽脈至,洪大以長;少陽脈至,乍數乍疏,乍短乍長;陽明脈至,浮大而短。
%%   夫平心脈來,纍纍如連珠,如循琅玕,曰心平,夏以胃氣為本,病心脈來,喘喘連屬,其中微曲,曰心病,死心脈來,前曲後居,如操帶鉤,曰心死。
%%   平肺脈來,厭厭聶聶,如落榆莢,曰肺平,秋以胃氣為本。病肺脈來,不上不下,如循雞羽,曰肺病。死肺脈來,如物之浮,如風吹毛,曰肺死。
%%   平肝脈來,軟弱招招,如揭長竿末梢,曰肝平,春以胃氣為本。病肝脈來,盈實而滑,如循長竿,曰肝病。死肝脈來,急益勁,如新張弓弦,曰肝死。
%%   平脾脈來,和柔相離,如雞踐地,曰脾平,長夏以胃氣為本。病脾脈來,實而盈數,如雞舉足,曰脾病。死脾脈來,銳堅如烏之喙,如鳥之距,如屋之漏,如水之流,曰脾死。
%%   平腎脈來,喘喘纍纍如鉤,按之而堅,曰腎平,冬以胃氣為本。病腎脈來,如引葛,按之益堅,曰腎病。死腎脈來,發如奪索,辟辟如彈石,曰腎死。

%% \section{玉機真藏論篇第十九}

%%   黃帝問曰:春脈如弦,何如而弦?岐伯對曰:春脈者肝也,東方木也,萬物之所以始生也,故其氣來,軟弱輕虛而滑,端直以長,故曰弦,反此者病。帝曰:何如而反。岐伯曰:其氣來實而強,此謂太過,病在外;其氣來不實而微,此謂不及,病在中。帝曰:春脈太過與不及,其病皆何如?岐伯曰:太過則令人善忘,忽忽眩冒而巔疾;其不及,則令人胸痛引背,下則兩胠脅滿。帝曰:善。
%%   夏脈如鉤,何如而鉤?岐伯曰:夏脈者心也,南方火也,萬物之所以盛長也,故其氣來盛去衰,故曰鉤,反此者病。帝曰:何如而反。岐伯曰:其氣來盛去亦盛,此謂太過,病在外;其氣來不盛去反盛,此謂不及,病在中。帝曰:夏脈太過與不及,其病皆何如?岐伯曰:太過則令人身熱而膚痛,為浸淫;其不及,則令人煩心,上見欬唾,下為氣洩。帝曰:善。
%%   秋脈如浮,何如而浮?岐伯曰:秋脈者肺也,西方金也,萬物之所以收成也,故其氣來,輕虛以浮,來急去散,故曰浮,反此者病。帝曰:何如而反。岐伯曰:其氣來,毛而中央堅,兩傍虛,此謂太過,病在外;其氣來,毛而微,此謂不及,病在中。帝曰:秋脈太過與不及,其病皆何如?岐伯曰:太過則令人逆氣而背痛,慍慍然;其不及,則令人喘,呼吸少氣而欬,上氣見血,下聞病音。帝曰:善。
%%   冬脈如營,何如而營?岐伯曰:冬脈者腎也,北方水也,萬物之所以合藏也,故其氣來,沉以搏,故曰營,反此者病。帝曰:何如而反。岐伯曰:其氣來如彈石者,此謂太過,病在外;其去如數者,此謂不及,病在中。帝曰:冬脈太過與不及,其病皆何如?岐伯曰:太過,則令人解(亻亦),脊脈痛而少氣不欲言;其不及,則令人心懸如病飢,眇中清,脊中痛,少腹滿,小便變。帝曰:善。
%%   帝曰:四時之序,逆從之變異也,然脾脈獨何主。岐伯曰:脾脈者土也,孤藏以灌四傍者也。帝曰:然則脾善惡,可得見之乎。岐伯曰:善者不可得見,惡者可見。帝曰:惡者何如可見。岐伯曰:其來如水之流者,此謂太過,病在外;如鳥之喙者,此謂不及,病在中。帝曰:夫子言脾為孤藏,中央土以灌四傍,其太過與不及,其病皆何如?岐伯曰:太過,則令人四支不舉;其不及,則令人九竅不通,名曰重強。
%%   帝瞿然而起,再拜而稽首曰:善。吾得脈之大要,天下至數,五色脈變,揆度奇恆,道在於一,神轉不回,回則不轉,乃失其機,至數之要,迫近以微,著之玉版,藏之藏府,每旦讀之,名曰玉機。
%%   五藏受氣於其所生,傳之於其所勝,氣舍於其所生,死於其所不勝。病之且死,必先傳行至其所不勝,病乃死。此言氣之逆行也,故死。肝受氣於心,傳之於脾,氣舍於腎,至肺而死。心受氣於脾,傳之於肺,氣舍於肝,至腎而死。脾受氣於肺,傳之於腎,氣舍於心,至肝而死。肺受氣於腎,傳之於肝,氣舍於脾,至心而死。腎受氣於肝,傳之於心,氣舍於肺,至脾而死。此皆逆死也。一日一夜五分之,此所以佔死生之早暮也。
%%   黃帝曰:五藏相通,移皆有次,五藏有病,則各傳其所勝。不治,法三月若六月,若三日若六日,傳五藏而當死,是順傳所勝之次。故曰:別於陽者,知病從來;別於陰者,知死生之期。言知至其所困而死。
%%   是故風者百病之長也,今風寒客於人,使人毫毛畢直,皮膚閉而為熱,當是之時,可汗而發也;或痹不仁腫痛,當是之時,可湯熨及火灸刺而去之。弗治,病入舍於肺,名曰肺痹,發欬上氣。弗治,肺即傳而行之肝,病名曰肝痹,一名曰厥,脅痛出食,當是之時,可按若刺耳。弗治,肝傳之脾,病名曰脾風,發癉,腹中熱,煩心出黃,當此之時,可按可藥可浴。弗治,脾傳之腎,病名曰疝瘕,少腹冤熱而痛,出白,一名曰蠱,當此之時,可按可藥。弗治,腎傳之心,病筋脈相引而急,病名曰瘛,當此之時,可灸可藥。弗治,滿十日,法當死。腎因傳之心,心即復反傳而行之肺,發寒熱,法當三歲死,此病之次也。
%%   然其捽髮者,不必治於傳,或其傳化有不以次,不以次入者,憂恐悲喜怒,令不得以其次,故令人有大病矣。因而喜大虛則腎氣乘矣,怒則肝氣乘矣,悲則肺氣乘矣,恐則脾氣乘矣,憂則心氣乘矣,此其道也。故病有五,五五二十五變,及其傳化。傳,乘之名也。
%%   大骨枯槁,大肉陷下,胸中氣滿,喘息不便,其氣動形,期六月死,真藏脈見,乃予之期日。大骨枯槁,大肉陷下,胸中氣滿,喘息不便,內痛引肩項,期一月死,真藏見,乃予之期日。大骨枯槁,大肉陷下,胸中氣滿,喘息不便,內痛引肩項,身熱脫肉破(月囷),真藏見,十月之內死。大骨枯槁,大肉陷下,肩髓內消,動作益衰,真藏來見,期一歲死,見其真藏,乃予之期日。大骨枯槁,大肉陷下,胸中氣滿,腹內痛,心中不便,肩項身熱,破(月囷)脫肉,目匡陷,真藏見,目不見人,立死,其見人者,至其所不勝之時則死。
%%   急虛身中卒至,五藏絕閉,脈道不通,氣不往來,譬如墮溺,不可為期。其脈絕不來,若人一息五六至,其形肉不脫,真藏雖不見,猶死也。
%%   真肝脈至,中外急,如循刀刃責責然,如按琴瑟弦,色青白不澤,毛折,乃死。真心脈至,堅而搏,如循薏苡子纍纍然,色赤黑不澤,毛折,乃死。真肺脈至,大而虛,如以毛羽中人膚,色白赤不澤,毛折,乃死。真腎脈至,搏而絕,如指彈石辟辟然,色黑黃不澤,毛折,乃死。真脾脈至,弱而乍數乍疏,色黃青不澤,毛折,乃死。諸真藏脈見者,皆死,不治也。
%%   黃帝曰:見真藏曰死,何也。岐伯曰:五藏者,皆稟氣於胃,胃者,五藏之本也,藏氣者,不能自致於手太陰,必因於胃氣,乃至於手太陰也,故五藏各以其時,自為而至於手太陰也。故邪氣勝者,精氣衰也,故病甚者,胃氣不能與之俱至於手太陰,故真藏之氣獨見,獨見者病勝藏也,故曰死。帝曰:善。
%%   黃帝曰:凡治病,察其形氣色澤,脈之盛衰,病之新故,乃治之無後其時。形氣相得,謂之可治;色澤以浮,謂之易己;脈從四時,謂之可治;脈弱以滑,是有胃氣,命曰易治,取之以時。形氣相失,謂之難治;色夭不澤,謂之難已;脈實以堅,謂之益甚;脈逆四時,為不可治。必察四難,而明告之。
%%   所謂逆四時者,春得肺脈,夏得腎脈,秋得心脈,冬得脾脈,其至皆懸絕沉澀者,命曰逆。四時未有藏形,於春夏而脈沉澀,秋冬而脈浮大,名曰逆四時也。
%%   病熱脈靜,洩而脈大,脫血而脈實,病在中脈實堅,病在外,脈不實堅者,皆難治。
%%   黃帝曰:余聞虛實以決死生,願聞其情。岐伯曰:五實死,五虛死。帝曰:願聞五實五虛。岐伯曰:脈盛,皮熱,腹脹,前後不通,悶瞀,此謂五實。脈細,皮寒,氣少,洩利前後,飲食不入,此謂五虛。帝曰:其時有生者,何也。岐伯曰:漿粥入胃,洩注止,則虛者活;身汗得後利,則實者活。此其候也。


%% \section{三部九候論篇第二十}

%%   黃帝問曰:余聞九針於夫子,眾多博大,不可勝數。余願聞要道,以屬子孫,傳之後世,著之骨髓,藏之肝肺,歃血而受,不敢妄洩,令合天道,必有終始,上應天光星辰歷紀,下副四時五行,貴賤更互,冬陰夏陽,以人應之奈何,願聞其方。
%%   岐伯對曰:妙乎哉問也!此天地之至數。帝曰:願聞天地之至數,合於人形,血氣通,決死生,為之奈何?岐伯曰:天地之至數,始於一,終於九焉。一者天,二者地,三者人,因而三之,三三者九,以應九野。故人有三部,部有三候,以決死生,以處百病,以調虛實,而除邪疾。
%%   帝曰:何謂三部。岐伯曰:有下部,有中部,有上部,部各有三候,三候者,有天有地有人也,必指而導之,乃以為真。上部天,兩額之動脈;上部地,兩頰之動脈;上部人,耳前之動脈。中部天,手太陰也;中部地,手陽明也;中部人,手少陰也。下部天,足厥陰也;下部地,足少陰也;下部人,足太陰也。故下部之天以侯肝,地以候腎,人以候脾胃之氣。
%%   帝曰:中部之候奈何?岐伯曰:亦有天,亦有地,亦有人。天以候肺,地以候胸中之氣,人以候心。帝曰:上部以何候之。岐伯曰:亦有天,亦有地,亦有人,天以候頭角之氣,地以候口齒之氣,人以候耳目之氣。三部者,各有天,各有地,各有人。三而成天,三而成地,三而成人,三而三之,合則為九,九分為九野,九野為九藏。故神藏五,形藏四,合為九藏。五藏已敗,其色必夭,夭必死矣。
%%   帝曰:以候奈何?岐伯曰:必先度其形之肥瘦,以調其氣之虛實,實則寫之,虛則補之。必先去其血脈而後調之,無問其病,以平為期。
%%   帝曰:決死生奈何?岐伯曰:形盛脈細,少氣不足以息者,危。形瘦脈大,胸中多氣者,死。形氣相得者,生。參伍不調者,病。三部九候皆相失者,死。上下左右之脈相應如參舂者,病甚。上下左右相失不可數者,死。中部之候雖獨調,與眾藏相失者,死。中部之候相減者,死。目內陷者死。
%%   帝曰:何以知病之所在。岐伯曰:察九候,獨小者病,獨大者病,獨疾者病,獨遲者病,獨熱者病,獨寒者病,獨陷下者病。以左手足上,去踝五寸按之,庶右手足當踝而彈之,其應過五寸以上,蠕蠕然者,不病;其應疾,中手渾渾然者,病;中手徐徐然者,病;其應上不能至五寸,彈之不應者,死。是以脫肉身不去者,死。中部乍疏乍數者,死。其脈代而鉤者,病在絡脈。九候之相應也,上下若一,不得相失。一候後則病,二候後則病甚,三候後則病危。所謂後者,應不俱也。察其府藏,以知死生之期。必先知經脈,然後知病脈,真藏脈見者勝死。足太陽氣絕者,其足不可屈伸,死必戴眼。
%%   帝曰:冬陰夏陽奈何?岐伯曰:九候之脈,皆沉細懸絕者為陰,主冬,故以夜半死。盛躁喘數者為陽,主夏,故以日中死。是故寒熱病者,以平旦死。熱中及熱病者,以日中死。病風者,以日夕死。病水者,以夜半死。其脈乍疏乍數乍遲乍疾者,日乘四季死。形肉已脫,九候雖調,猶死。七診雖見,九候皆從者不死。所言不死者,風氣之病,及經月之病,似七診之病而非也,故言不死。若有七診之病,其脈候亦敗者死矣,必發噦噫。必審問其所始病,與今之所方病,而後各切循其脈,視其經絡浮沉,以上下逆從循之,其脈疾者不病,其脈遲者病,脈不往來者死,皮膚著者死。
%%   帝曰:其可治者奈何?岐伯曰:經病者治其經,孫絡病者治其孫絡血,血病身有痛者,治其經絡。其病者在奇邪,奇邪之脈則繆刺之。留瘦不移,節而刺之。上實下虛,切而從之,索其結絡脈,刺出其血,以見通之。瞳子高者,太陽不足,戴眼者,太陽已絕,此決死生之要,不可不察也。手指及手外踝上五指,留針。


%% \section{經脈別論篇第二十一}

%%   黃帝問曰:人之居處動靜勇怯,脈亦為之變乎。岐伯對曰:凡人之驚恐恚勞動靜,皆為變也。是以夜行則喘出於腎,淫氣病肺。有所墮恐,喘出於肝,淫氣害脾。有所驚恐,喘出於肺,淫氣傷心。度水跌僕,喘出於腎與骨,當是之時,勇者氣行則已,怯者則著而為病也。故曰:診病之道,觀人勇怯,骨肉皮膚,能知其情,以為診法也。
%%   故飲食飽甚,汗出於胃。驚而奪精,汗出於心。持重遠行,汗出於腎。疾走恐懼,汗出於肝。搖體勞苦,汗出於脾。故春秋冬夏,四時陰陽,生病起於過用,此為常也。
%%   食氣入胃,散精於肝,淫氣於筋。食氣入胃,濁氣歸心,淫精於脈。脈氣流經,經氣歸於肺,肺朝百脈,輸精於皮毛。毛脈合精,行氣於府。府精神明,留於四藏,氣歸於權衡。權衡以平,氣口成寸,以決死生。
%%   飲入於胃,游溢精氣,上輸於脾。脾氣散精,上歸於肺,通調水道,下輸膀胱。水精四布,五經並行,合於四時五藏陰陽,揆度以為常也。
%%   太陽藏獨至,厥喘虛氣逆,是陰不足陽有餘也,表裡當俱寫,取之下俞,陽明藏獨至,是陽氣重並也,當寫陽補陰,取之下俞。少陽藏獨至,是厥氣也,蹻前卒大,取之下俞,少陽獨至者,一陽之過也。太陰藏搏者,用心省真,五脈氣少,胃氣不平,三陰也,宜治其下俞,補陽寫陰。一陽獨嘯,少陽厥也,陽並於上,四脈爭張,氣歸於腎,宜治其經絡,寫陽補陰。一陰至,厥陰之治也,真陰(疒肙)心,厥氣留薄,發為白汗,調食和藥,治在下俞。
%%   帝曰:太陽藏何象。岐伯曰:像三陽而浮也。帝曰:少陽藏何象。岐伯曰:像一陽也,一陽藏者,滑而不實也。帝曰:陽明藏何象。岐伯曰:象大浮也,太陰藏搏,言伏鼓也。二陰搏至,腎沉不浮也。


%% \section{藏氣法時論篇第二十二}

%%   黃帝問曰:合人形以法四時五行而治,何如而從,何如而逆,得失之意,願聞其事。岐伯對曰:五行者,金木水火土也,更貴更賤,以知死生,以決成敗,而定五藏之氣,間甚之時,死生之期也。
%%   帝曰:願卒聞之。岐伯曰:肝主春,足厥陰少陽主治,其日甲乙,肝苦急,急食甘以緩之。心主夏,手少陰太陽主治,其日丙丁,心苦緩,急食酸以收之。脾主長夏,足太陰陽明主治,其日戊己,脾苦濕,急食苦以燥之。肺主秋,手太陰陽明主治,其日庚辛,肺苦氣上逆,急食苦以洩之。腎主冬,足少陰太陽主治,其日壬癸,腎苦燥,急食辛以潤之,開腠理,致津液,通氣也。
%%   病在肝,愈於夏,夏不愈,甚於秋,秋不死,持於冬,起於春,禁當風。肝病者,愈在丙丁,丙丁不愈,加於庚辛,庚辛不死,持於壬癸,起於甲乙。肝病者,平旦慧,下晡甚,夜半靜。肝欲散,急食辛以散之,用辛補之,酸寫之。
%%   病在心,愈在長夏,長夏不愈,甚於冬,冬不死,持於春,起於夏,禁溫食熱衣。心病者,愈在戊己,戊己不愈,加於壬癸,壬癸不死,持於甲乙,起於丙丁。心病者,日中慧,夜半甚,平旦靜。心欲軟,急食咸以軟之,用咸補之,甘寫之。
%%   病在脾,愈在秋,秋不愈,甚於春,春不死,持於夏,起於長夏,禁溫食飽食濕地濡衣。脾病者,愈在庚辛,庚辛不愈,加於甲乙,甲乙不死,持於丙丁,起於戊己。脾病者,日昳慧,日出甚,下晡靜。脾欲緩,急食甘以緩之,用苦寫之,甘補之。
%%   病在肺,愈在冬,冬不愈,甚於夏,夏不死,持於長夏,起於秋,禁寒飲食寒衣。肺病者,愈在壬癸,壬癸不愈,加於丙丁,丙丁不死,持於戊己,起於庚辛。肺病者,下晡慧,日中甚,夜半靜。肺欲收,急食酸以收之,用酸補之,辛寫之。
%%   病在腎,愈在春,春不愈,甚於長夏,長夏不死,持於秋,起於冬,禁犯焠(火矣) 熱食溫灸衣。腎病者,愈在甲乙,甲乙不愈,甚於戊己,戊己不死,持於庚辛,起於壬癸。腎病者,夜半慧,四季甚,下晡靜。腎欲堅,急食苦以堅之,用苦補之,咸寫之。
%%   夫邪氣之客於身也,以勝相加,至其所生而愈,至其所不勝而甚,至於所生而持,自得其位而起。必先定五藏之脈,乃可言間甚之時,死生之期也。
%%   肝病者,兩脅下痛引少腹,令人善怒,虛則目(目巟)(目巟)無所見,耳無所聞,善恐,如人將捕之,取其經,厥陰與少陽,氣逆,則頭痛耳聾不聰頰腫。取血者。
%%   心病者,胸中痛,脅支滿,脅下痛,膺背肩甲間痛,兩臂內痛;虛則胸腹大,脅下與腰相引而痛,取其經,少陰太陽,舌下血者。其變病,刺隙中血者。
%%   脾病者,身重善肌肉痿,足不收行,善瘈,腳下痛;虛則腹滿腸鳴,飧洩食不化,取其經,太陰陽明少陰血者。
%%   肺病者,喘咳逆氣,肩背痛,汗出,尻陰股膝髀腨(骨行)足皆痛;虛則少氣不能報息,耳聾嗌干,取其經,太陰足太陽之外厥陰內血者。
%%   腎病者,腹大脛腫,喘咳身重,寢汗出,憎風;虛則胸中痛,大腹小腹痛,清厥意不樂,取其經,少陰太陽血者。
%%   肝色青,宜食甘,粳米牛肉棗葵皆甘。心色赤,宜食酸,小豆犬肉李韭皆酸。肺色白,宜食苦,麥羊肉杏薤皆苦。脾色黃,宜食咸,大豆豕肉栗藿皆咸。腎色黑,宜食辛,黃黍雞肉桃蔥皆辛。辛散,酸收,甘緩,苦堅,咸軟。
%%   毒藥攻邪,五穀為養,五果為助,五畜為益,五菜為充,氣味合而服之,以補精益氣。此五者,有辛酸甘苦咸,各有所利,或散,或收,或緩,或急,或堅,或軟,四時五藏,病隨五味所宜也。


%% \section{宣明五氣篇第二十三}

%%   五味所入:酸入肝,辛入肺,苦入心,咸入腎,甘入脾,是謂五入。
%%   五氣所病:心為噫,肺為咳,肝為語,脾為吞,腎為欠為嚏,胃為氣逆,為噦為恐,大腸小腸為洩,下焦溢為水,膀胱不利為癃,不約為遺溺,膽為怒,是謂五病。
%%   五精所並:精氣並於心則喜,並於肺則悲,並於肝則憂,並於脾則畏,並於腎則恐,是謂五並,虛而相併者也。
%%   五藏所惡:心惡熱,肺惡寒,肝惡風,脾惡濕,腎惡燥,是謂五惡。
%%   五藏化液:心為汗,肺為涕,肝為淚,脾為涎,腎為唾,是謂五液。
%%   五味所禁:辛走氣,氣病無多食辛;咸走血,血病無多食咸;苦走骨,骨病無多食苦;甘走肉,肉病無多食甘;酸走筋,筋病無多食酸;是謂五禁,無令多食。
%%   五病所發:陰病發於骨,陽病發於血,陰病發於肉,陽病發於冬,陰病發於夏,是謂五發。
%%   五邪所亂:邪入於陽則狂,邪入於陰則痹,搏陽則為巔疾,搏陰則為瘖,陽入之陰則靜,陰出之陽則怒,是謂五亂。
%%   五邪所見:春得秋脈,夏得冬脈,長夏得春脈,秋得夏脈,冬得長夏脈,名曰陰出之陽,病善怒不治,是謂五邪。皆同命,死不治。
%%   五藏所藏:心藏神,肺藏魄,肝藏魂,脾藏意,腎藏志,是謂五藏所藏。
%%   五藏所主:心主脈,肺主皮,肝主筋,脾主肉,腎主骨,是謂五主。
%%   五勞所傷:久視傷血,久臥傷氣,久坐傷肉,久立傷骨,久行傷筋,是謂五勞所傷。
%%   五脈應像:肝脈弦,心脈鉤,脾脈代,肺脈毛,腎脈石,是謂五藏之脈。


%% \section{血氣形志篇第二十四}

%%   夫人之常數,太陽常多血少氣,少陽常少血多氣,陽明常多氣多血,少陰常少血多氣,厥陰常多血少氣,太陰常多氣少血,此天之常數。足太陽與少陰為表裡,少陽與厥陰為表裡,陽明與太陰為表裡,是為足陰陽也。手太陽與少陰為表裡,少陽與心主為表裡,陽明與太陰為表裡,是為手之陰陽也。今知手足陰陽所苦,凡治病必先去其血,乃去其所苦,伺之所欲,然後寫有餘,補不足。
%%   欲知背俞,先度其兩乳間,中折之,更以他草度去半已,即以兩隅相拄也,乃舉以度其背,令其一隅居上,齊脊大柱,兩隅在下,當其下隅者,肺之俞也。復下一度,心之俞也。復下一度,左角肝之俞也,右角脾之俞也。復下一度,腎之俞也。是謂五藏之俞,灸刺之度也。
%%   形樂志苦,病生於脈,治之以灸刺。形樂志樂,病生於肉,治之以針石。形苦志樂,病生於筋,治之以熨引。形苦志苦,病生於咽嗌,治之以百藥。形數驚恐,經絡不通,病生於不仁,治之以按摩醪藥。是謂五形志也。
%%   刺陽明出血氣,刺太陽,出血惡氣,刺少陽,出氣惡血,刺太陰,出氣惡血,刺少陰,出氣惡血,刺厥陰,出血惡氣也。

%% \section{寶命全形論篇第二十五}

%%   黃帝問曰:天覆地載,萬物悉備,莫貴於人,人以天地之氣生,四時之法成,君王眾庶,盡欲全形,形之疾病,莫知其情,留淫日深,著於骨髓,心私慮之,余欲針除其疾病,為之奈何?
%%   岐伯對曰:夫鹽之味咸者,其氣令器津洩;弦絕者,其音嘶敗;木敷者,其葉發;病深者,其聲噦。人有此三者,是謂壞府,毒藥無治,短針無取,此皆絕皮傷肉,血氣爭黑。
%%   帝曰:余念其痛,心為之亂惑,反甚其病,不可更代,百姓聞之,以為殘賊,為之奈何?岐伯曰:夫人生於地,懸命於天,天地合氣,命之曰人。人能應四時者,天地為之父母;知萬物者,謂之天子。天有陰陽,人有十二節;天有寒暑,人有虛實。能經天地陰陽之化者,不失四時;知十二節之理者,聖智不能欺也;能存八動之變,五勝更立;能達虛實之數者,獨出獨入,呿吟至微,秋毫在目。
%%   帝曰:人生有形,不離陰陽,天地合氣,別為九野,分為四時,月有小大,日有短長,萬物並至,不可勝量,虛實呿吟,敢問其方。
%%   岐伯曰:木得金而伐,火得水而滅,土得木而達,金得火而缺,水得土而絕,萬物盡然,不可勝竭。故針有懸布天下者五,黔首共余食,莫知之也。一曰治神,二曰知養身,三曰知毒藥為真,四曰制砭石小大,五曰知府藏血氣之診。五法俱立,各有所先。今末世之刺也,虛者實之,滿者洩之,此皆眾工所共知也。若夫法天則地,隨應而動,和之者若響,隨之者若影,道無鬼神,獨來獨往。
%%   帝曰:願聞其道。
%%   岐伯曰:凡刺之真,必先治神,五藏已定,九候已備,後乃存針,眾脈不見,眾凶弗聞,外內相得,無以形先,可玩往來,乃施於人。人有虛實,五虛勿近,五實勿遠,至其當發,間不容瞚。手動若務,針耀而勻,靜意視義,觀適之變,是謂冥冥,莫知其形,見其烏烏,見其稷稷,從見其飛,不知其誰,伏如橫弩,起如發機。
%%   帝曰:何如而虛?何如而實?岐伯曰:刺虛者須其實,刺實者須其虛,經氣已至,慎守勿失,深淺在志,遠近若一,如臨深淵,手如握虎,神無營於眾物。


%% \section{八正神明論篇第二十六}

%%   黃帝問曰:用針之服,必有法則焉,今何法何則?岐伯對曰:法天則地,合以天光。
%%   帝曰:願卒聞之。岐伯曰:凡刺之法,必候日月星辰四時八正之氣,氣定乃刺之。是故天溫日明,則人血淖液而衛氣浮,故血易寫,氣易行;天寒日陰,則人血凝泣,而衛氣沉。月始生,則血氣始精,衛氣始行;月郭滿,則血氣實,肌肉堅;月郭空,則肌肉減,經絡虛,衛氣去,形獨居。是以因天時而調血氣也。是以天寒無刺,天溫無疑。月生無寫,月滿無補,月郭空無治,是謂得時而調之。因天之序,盛虛之時,移光定位,正立而待之。故日月生而寫,是謂藏虛;月滿而補,血氣揚溢,絡有留血,命曰重實;月郭空而治,是謂亂經。陰陽相錯,真邪不別,沉以留止,外虛內亂,淫邪乃起。
%%   帝曰:星辰八正何候?
%%   岐伯曰:星辰者,所以制日月之行也。八正者,所以候八風之虛邪以時至者也。四時者,所以分春秋冬夏之氣所在,以時調之也,八正之虛邪,而避之勿犯也。以身之虛,而逢天之虛,兩虛相感,其氣至骨,入則傷五藏,工候救之,弗能傷也,故曰天忌不可不知也。
%%   帝曰:善。其法星辰者,余聞之矣,願聞法往古者。
%%   岐伯曰:法往古者,先知針經也。驗於來今者,先知日之寒溫、月之虛盛,以候氣之浮沉,而調之於身,觀其立有驗也。觀其冥冥者,言形氣榮衛之不形於外,而工獨知之,以日之寒溫,月之虛盛,四時氣之浮沉,參伍相合而調之,工常先見之,然而不形於外,故曰觀於冥冥焉。通於無窮者,可以傳於後世也,是故工之所以異也,然而不形見於外,故俱不能見也。視之無形,嘗之無味,故謂冥冥,若神彷彿。虛邪者,八正之虛邪氣也。正邪者,身形若用力,汗出,腠理開,逢虛風,其中人也微,故莫知其情,莫見其形。上工救其萌牙,必先見三部九候之氣,盡調不敗而救之,故曰上工。下工救其已成,救其已敗。救其已成者,言不知三部九候之相失,因病而敗之也,知其所在者,知診三部九候之病脈處而治之,故曰守其門戶焉,莫知其情而見邪形也。
%%   帝曰:余聞補寫,未得其意。
%%   岐伯曰:寫必用方,方者,以氣方盛也,以月方滿也,以日方溫也,以身方定也,以息方吸而內針,乃復候其方吸而轉針,乃復候其方呼而徐引針,故曰寫必用方,其氣而行焉。補必用員,員者行也,行者移也,刺必中其,復以吸排針也。故員與方,非針也。故養神者,必知形之肥瘦,榮衛血氣之盛衰。血氣者,人之神,不可不謹養。
%%   帝曰:妙乎哉論也。合人形於陰陽四時,虛實之應,冥冥之期,其非夫子孰能通之。然夫子數言形與神,何謂形,何謂神,願卒聞之。
%%   岐伯曰:請言形、形乎形、目冥冥,問其所病,索之於經,慧然在前,按之不得,不知其情,故曰形。
%%   帝曰:何謂神?
%%   岐伯曰:,神乎神,耳不聞,目明,心開而志先,慧然獨悟,口弗能言,俱視獨見,適若昏,昭然獨明*請言神*,若風吹雲,故曰神。三部九候為之原,九針之論,不必存也。


%% \section{離合真邪論篇第二十七}

%%   黃帝問曰:余聞九針九篇,夫子乃因而九之,九九八十一篇,余盡通其意矣。經言氣之盛衰,左右頃移,以上調下,以左調右,有餘不足,補瀉於滎輸,余知之矣。此皆榮衛之頃移,虛實之所生,非邪氣從外入於經也。余願聞邪氣之在經也,其病人何如?取之奈何?
%%   岐伯對曰:夫聖人之起度數,必應於天地,故天有宿度,地有經水,人有經脈。天地溫和,則經水安靜;天寒地凍,則經水凝泣;天暑地熱,則經水沸溢;卒風暴起,則經水波湧而隴起。夫邪之入於脈也,寒則血凝泣,暑則氣淖澤,虛邪因而入客,亦如經水之得風也,經之動脈,其至也亦時隴起,其行於脈中循循然,其至寸口中手也,時大時小,大則邪至,小則平,其行無常處,在陰與陽,不可為度,從而察之,三部九候,卒然逢之,早遏其路,吸則內針,無令氣忤;靜以久留,無令邪布;吸則轉針,以得氣為故;候呼引針,呼盡乃去;大氣皆出,故命曰寫。
%%   帝曰:不足者補之,奈何?
%%   岐伯曰:必先捫而循之,切而散之,推而按之,彈而怒之,抓而下之,通而取之,外引其門,以閉其神。呼盡內針,靜以久留,以氣至為故,如待所貴,不知日暮,其氣以至,適而自護,候吸引針,氣不得出,各在其處,推闔其門,令神氣存,大氣留止,故命曰補。
%%   帝曰:候氣奈何?
%%   岐伯曰:夫邪去絡入於經也,舍於血脈之中,其寒溫未相得,如湧波之起也,時來時去,故不常在。故曰方其來也,必按而止之,止而取之,無逢其沖而寫之。真氣者,經氣也,經氣太虛,故曰其來不可逢,此之謂也。故曰候邪不審,大氣已過,寫之則真氣脫,脫則不復,邪氣復至,而病益蓄,故曰其往不可追,此之謂也。不可掛以發者,待邪之至時而髮針寫矣,若先若後者,血氣已盡,其病不可下,故曰知其可取如發機,不知其取如扣椎,故曰知機道者不可掛以發,不知機者扣之不發,此之謂也。
%%   帝曰:補寫奈何?
%%   岐伯曰:此攻邪也,疾出以去盛血,而復其真氣,此邪新客,溶溶未有定處也,推之則前,引之則止,逆而刺之,溫血也。刺出其血,其病立已。
%%   帝曰:善。然真邪以合,波隴不起,候之奈何?
%%   岐伯曰:審捫循三部九候之盛虛而調之,察其左右上下相失及相減者,審其病藏以期之。不知三部者,陰陽不別,天地不分,地以候地,天以候天,人以候人,調之中府,以定三部,故曰刺不知三部九候病脈之處,雖有大過且至,工不能禁也。誅罰無過,命曰大惑,反亂大經,真不可復,用實為虛,以邪為真,用針無義,反為氣賊,奪人正氣,以從為逆,榮衛散亂,真氣已失,邪獨內著,絕人長命,予人夭殃,不知三部九候,故不能久長。因不知合之四時五行,因加相勝,釋邪攻正,絕人長命。邪之新客來也,未有定處,推之則前,引之則止,逢而寫之,其病立已。


%% \section{通評虛實論篇第二十八}

%%   黃帝問曰:何謂虛實?岐伯對曰:邪氣盛則實,精氣奪則虛。
%%   帝曰:虛實何如?岐伯曰:氣虛者肺虛也,氣逆者足寒也,非其時則生,當其時則死。余藏皆如此。
%%   帝曰:何謂重實?岐伯曰:所謂重實者,言大熱病,氣熱脈滿,是謂重實。
%%   帝曰:經絡俱實何如?何以治之?岐伯曰:經絡皆實,是寸脈急而尺緩也,皆當治之,故曰滑則從,澀則逆也。夫虛實者,皆從其物類始,故五藏骨肉滑利,可以長久也。
%%   帝曰:絡氣不足,經氣有餘,何如?岐伯曰:絡氣不足,經氣有餘者,脈口熱而尺寒也,秋冬為逆,春夏為從,治主病者。
%%   帝曰:經虛絡滿,何如?岐伯曰:經虛絡滿者,尺熱滿,脈口寒澀也,此春夏死秋冬生也。
%%   帝曰:治此者奈何?岐伯曰:絡滿經虛,灸陰刺陽;經滿絡虛,刺陰灸陽。 帝曰:何謂重虛?岐伯曰:脈氣上虛尺虛,是謂重虛。帝曰:何以治之?岐伯曰:所謂氣虛者,言無常也。尺虛者,行步恇然。脈虛者,不像陰也。如此者,滑則生,澀則死也。
%%   帝曰:寒氣暴上,脈滿而實何如?岐伯曰:實而滑則生,實而逆則死。
%%   帝曰:脈實滿,手足寒,頭熱,何如?岐伯曰:春秋則生,冬夏則死。脈浮而澀,澀而身有熱者死。
%%   帝曰:其形盡滿何如?岐伯曰:其形盡滿者,脈急大堅,尺澀而不應也,如是者,故從則生,逆則死。帝曰:何謂從則生,逆則死?岐伯曰:所謂從者,手足溫也;所謂逆者,手足寒也。
%%   帝曰:乳子而病熱,脈懸小者何如?岐伯曰:手足溫則生,寒則死。
%%   帝曰:乳子中風熱,喘鳴肩息者,脈何如?岐伯曰:喘鳴肩息者,脈實大也,緩則生,急則死。
%%   帝曰:腸澼便血何如?岐伯曰:身熱則死,寒則生。帝曰:腸澼下白沫何如?岐伯曰:脈沉則生,脈浮則死。帝曰:腸下膿血何如?岐伯曰:脈懸絕則死,滑大則生。帝曰:腸澼之屬,身不熱,脈不懸絕何如?岐伯曰:滑大者曰生,懸澀者曰死,以藏期之。
%%   帝曰:癲疾何如?岐伯曰:脈搏大滑,久自已;脈小堅急,死不治。帝曰:癲疾之脈,虛實何如?岐伯曰:虛則可治,實則死。
%%   帝曰:消癉虛實何如?岐伯曰:脈實大,病久可治;脈懸小堅,病久不可治。
%%   帝曰:形度骨度脈度筋度,何以知其度也?
%%   帝曰:春亟治經絡;夏亟治經輸;秋亟治六府;冬則閉塞,閉塞者,用藥而少針石也。所謂少針石者,非癰疽之謂也,癰疽不得頃時回。癰不知所,按之不應手,乍來乍已,刺手太陰傍三痏與纓脈各二,掖癰大熱,刺足少陽五;刺而熱不止,刺手心主三,刺手太陰經絡者大骨之會各三。暴癰筋軟,隨分而痛,魄汗不盡,胞氣不足,治在經俞。
%%   腹暴滿,按之不下,取手太陽經絡者,胃之募也,少陰俞去脊椎三寸傍五,用員利針。霍亂,刺俞傍五,足陽明及上傍三。刺癇驚脈五,針手太陰各五,刺經太陽五,刺手少陰經絡傍者一,足陽明一,上踝五寸刺三針。
%%   凡治消癉、僕擊、偏枯、痿厥、氣滿發逆,肥貴人,則高梁之疾也。隔塞閉絕,上下不通,則暴憂之疾也。暴厥而聾,偏塞閉不通,內氣暴薄也。不從內,外中風之病,故瘦留著也。蹠跛,寒風濕之病也。
%%   黃帝曰:黃疸暴痛,癲疾厥狂,久逆之所生也。五藏不平,六府閉塞之所生也。頭痛耳鳴,九竅不利,腸胃之所生也。


%% \section{太陰陽明論篇第二十九}

%%   黃帝問曰:太陰陽明為表裡,脾胃脈也,生病而異者何也?岐伯對曰:陰陽異位,更虛更實,更逆更從,或從內,或從外,所從不同,故病異名也。
%%   帝曰:願聞其異狀也。岐伯曰:陽者,天氣也,主外;陰者,地氣也,主內。故陽道實,陰道虛。故犯賊風虛邪者,陽受之;食飲不節,起居不時者,陰受之。陽受之,則入六府,陰受之,則入五藏。入六府,則身熱不時臥,上為喘呼;入五藏,則(月真)滿閉塞,下為飧洩,久為腸澼。故喉主天氣,咽主地氣。故陽受風氣,陰受濕氣。故陰氣從足上行至頭,而下行循臂至指端;陽氣從手上行至頭,而下行至足。故曰陽病者上行極而下,陰病者下行極而上。故傷於風者,上先受之;傷於濕者,下先受之。
%%   帝曰:脾病而四支不用何也?岐伯曰:四支皆稟氣於胃,而不得至經,必因於脾,乃得稟也。今脾病不能為胃行其津液,四支不得稟水谷氣,氣日以衰,脈道不利,筋骨肌肉,皆無氣以生,故不用焉。
%%   帝曰:脾不主時何也?岐伯曰:脾者土也,治中央,常以四時長四藏,各十八日寄治,不得獨主於時也。脾藏者常著胃土之精也,土者生萬物而法天地,故上下至頭足,不得主時也。
%%   帝曰:脾與胃以膜相連耳,而能為之行其津液何也?岐伯曰:足太陰者三陰也,其脈貫胃屬脾絡嗌,故太陰為之行氣於三陰。陽明者表也,五藏六府之海也,亦為之行氣於三陽。藏府各因其經而受氣於陽明,故為胃行其津液,四支不得稟水谷氣,日以益衰,陰道不利,筋骨肌肉無氣以生,故不用焉。


%% \section{陽明脈解篇第三十}

%%   黃帝問曰:足陽明之脈病,惡人與火,聞木音則惕然而驚,鐘鼓不為動,聞木音而驚,何也?願聞其故。岐伯對曰:陽明者胃脈也,胃者,土也,故聞木音而驚者,土惡木也。帝曰:善。其惡火何也?岐伯曰:陽明主肉,其脈血氣盛,邪客之則熱,熱甚則惡火。
%%   帝曰:其惡人何也?岐伯曰:陽明厥則喘而惋,惋則惡人。帝曰:或喘而死者,或喘而生者,何也?岐伯曰:厥逆連藏則死,連經則生。
%%   帝曰:善。病甚則棄衣而走,登高而歌,或至不食數日,逾垣上屋,所上之處,皆非其素所能也,病反能者何也?岐伯曰:四支者,諸陽之本也,陽盛則四支實,實則能登高也。
%%   帝曰:其棄衣而走者,何也?岐伯曰:熱盛於身,故棄衣欲走也。帝曰:其妄言罵詈,不避親疏而歌者,何也?岐伯曰:陽盛則使人妄言罵詈不避親疏,而不欲食,不欲食,故妄走也。


%% \section{熱論篇第三十一}

%%   黃帝問曰:今夫熱病者,皆傷寒之類也,或愈或死,其死皆以六七日之間,其愈皆以十日以上者,何也?不知其解,願聞其故。
%%   岐伯對曰:巨陽者,諸陽之屬也,其脈連於風府,故為諸陽主氣也。人之傷於寒也,則為病熱,熱雖甚不死;其兩感於寒而病者,必不免於死。
%%   帝曰:願聞其狀。岐伯曰:傷寒一日,巨陽受之,故頭項痛腰脊強。二日陽明受之,陽明主肉,其脈俠鼻絡於目,故身熱目疼而鼻干,不得臥也。三日少陽受之,少陽主膽,其脈循脅絡於耳,故胸脅痛而耳聾。三陽經絡皆受其病,而未入於藏者,故可汗而已。四日太陰受之,太陰脈布胃中絡於嗌,故腹滿而嗌干。五日少陰受之,少陰脈貫腎絡於肺,系舌本,故口燥舌干而喝。六日厥陰受之,厥陰脈循陰器而絡於肝,故煩滿而囊縮。三陰三陽,五藏六府皆受病,榮衛不行,五藏不通則死矣。
%%   其不兩感於寒者,七日巨陽病衰,頭痛少愈;八日陽明病衰,身熱少愈;九日少陽病衰,耳聾微聞;十日太陰病衰,腹減如故,則思飲食;十一日少陰病衰,渴止不滿,舌干已而嚏;十二日厥陰病衰,囊縱少腹微下,大氣皆去,病日已矣。帝曰:治之奈何?岐伯曰:治之各通其藏脈,病日衰已矣。其未滿三日者,可汗而已;其滿三日者,可洩而已。
%%   帝曰:熱病已癒,時有所遺者,何也?岐伯曰:諸遺者,熱甚而強食之,故有所遺也。若此者,皆病已衰,而熱有所藏,因其谷氣相薄,兩熱相合,故有所遺也。帝曰:善。治遺奈何?岐伯曰:視其虛實,調其逆從,可使必已矣。帝曰:病熱當何禁之?岐伯曰:病熱少愈,食肉則復,多食則遺,此其禁也。
%%   帝曰:其病兩感於寒者,其脈應與其病形何如?岐伯曰:兩感於寒者,病一日則巨陽與少陰俱病,則頭痛口乾而煩滿;二日則陽明與太陰俱病,則腹滿身熱,不欲食譫言;三日則少陽與厥陰俱病,則耳聾囊縮而厥,水漿不入,不知人,六日死。帝曰:五藏已傷,六府不通,榮衛不行,如是之後,三日乃死,何也?岐伯曰:陽明者,十二經脈之長也,其血氣盛,故不知人,三日其氣乃盡,故死矣。
%%   凡病傷寒而成溫者,先夏至日者為病溫,後夏至日者為病暑,暑當與汗皆出,勿止。


%% \section{刺熱篇第三十二}

%%   肝熱病者,小便先黃,腹痛多臥身熱,熱爭,則狂言及驚,脅滿痛,手足躁,不得安臥;庚辛甚,甲乙大汗,氣逆則庚辛死。刺足厥陰少陽。其逆則頭痛員員,脈引沖頭也。
%%   心熱病者,先不樂,數日乃熱,熱爭則卒心痛,煩悶善嘔,頭痛面赤,無汗;壬癸甚,丙丁大汗,氣逆則壬癸死。刺手少陰太陽。
%%   脾熱病者,先頭重頰痛,煩心顏青,欲嘔身熱,熱爭則腰痛不可用俛仰,腹滿洩,兩頷痛;甲乙甚,戊己大汗,氣逆則甲乙死。刺足太陰陽明。
%%   肺熱病者,先淅然厥,起毫毛,惡風寒,舌上黃,身熱。熱爭則喘欬,痛走胸膺背,不得大息,頭痛不堪,汗出而寒;丙丁甚,庚辛大汗,氣逆則丙丁死。刺手太陰陽明,出血如大豆,立已。
%%   腎熱病者,先腰痛(骨行)痠,苦喝數飲,身熱,熱爭則項痛而強,(骨行)寒且痠,足下熱,不欲言,其逆則項痛員員澹澹然;戊己甚,壬癸大汗,氣逆則戊己死。刺足少陰太陽。諸汗者,至其所勝日汗出也。
%%   肝熱病者,左頰先赤;心熱病者,顏先赤;脾熱病者,鼻先赤;肺熱病者,右頰先赤;腎熱病者,頤先赤。病雖未發,見赤色者刺之,名曰治未病。熱病從部所起者,至期而已;其刺之反者,三週而已;重逆則死。諸當汗者,至其所勝日,汗大出也。
%%   諸治熱病,以飲之寒水,乃刺之;必寒衣之,居止寒處,身寒而止也。
%%   熱病先胸脅痛,手足躁,刺足少陽,補足太陰,病甚者為五十九刺。熱病始手臂痛者,刺手陽明太陰而汗出止。熱病始於頭首者,刺項太陽而汗出止。熱病始於足脛者,刺足陽明而汗出止。熱病先身重骨痛,耳聾好瞑,刺足少陰,病甚為五十九刺。熱病先眩冒而熱,胸脅滿,刺足少陰少陽。
%%   太陽之脈,色榮顴骨,熱病也,榮未交,曰今且得汗,待時而已。與厥陰脈爭見者,死期不過三日。其熱病內連腎,少陽之脈色也。少陽之脈,色榮頰前,熱病也,榮未交,曰今且得汗,待時而已,與少陰脈爭見者,死期不過三日。
%%   熱病氣穴:三椎下間主胸中熱,四椎下間主鬲中熱,五椎下間主肝熱,六椎下間主脾熱,七椎下間主腎熱,榮在骶也,項上三椎陷者中也。頰下逆顴為大瘕,下牙車為腹滿,顴後為脅痛。頰上者,鬲上也。


%% \section{評熱病論篇第三十三}

%%   黃帝問曰:有病溫者,汗出輒復熱,而脈躁疾不為汗衰,狂言不能食,病名為何?岐伯對曰:病名陰陽交,交者死也。帝曰:願聞其說。岐伯曰:人所以汗出者,皆生於谷,谷生於精。今邪氣交爭於骨肉而得汗者,是邪卻而精勝也。精勝,則當能食而不復熱,復熱者邪氣也,汗者精氣也;今汗出而輒復熱者,是邪勝也,不能食者,精無俾也,病而留者,其壽可立而傾也。且夫《熱論》曰:汗出而脈尚躁盛者死。今脈不與汗相應,此不勝其病也,其死明矣。狂言者是失志,失志者死。今見三死,不見一生,雖愈必死也。
%%   帝曰:有病身熱汗出煩滿,煩滿不為汗解,此為何病?岐伯曰:汗出而身熱者,風也;汗出而煩滿不解者,厥也,病名曰風厥。帝曰:願卒聞之。岐伯曰:巨陽主氣,故先受邪;少陰與其為表裡也,得熱則上從之,從之則厥也。帝曰:治之奈何?岐伯曰:表裡刺之,飲之服湯。
%%   帝曰:勞風為病何如?岐伯曰:勞風法在肺下,其為病也,使人強上冥視,唾出若涕,惡風而振寒,此為勞風之病。帝曰:治之奈何?岐伯曰:以救俛仰。巨陽引。精者三日,中年者五日,不精者七日,咳出青黃涕,其狀如膿,大如彈丸,從口中若鼻中出,不出則傷肺,傷肺則死也。
%%   帝曰:有病腎風者,面胕然(疒龍)壅,害於言,可刺不?岐伯曰:虛不當刺,不當刺而刺,後五日其氣必至。帝曰:其至何如?岐伯曰:至必少氣時熱,時熱從胸背上至頭,汗出,手熱,口乾苦渴,小便黃,目下腫,腹中鳴,身重難以行,月事不來,煩而不能食,不能正偃,正偃則欬,病名曰風水,論在《刺法》中。
%%   帝曰:願聞其說。岐伯曰:邪之所湊,其氣必虛,陰虛者,陽必湊之,故少氣時熱而汗出也。小便黃者,少腹中有熱也。不能正偃者,胃中不和也。正偃則咳甚,上迫肺也。諸有水氣者,微腫先見於目下也。帝曰:何以言?岐伯曰:水者陰也,目下亦陰也,腹者至陰之所居,故水在腹者,必使目下腫也。真氣上逆,故口苦舌干,臥不得正偃,正偃則咳出清水也。諸水病者,故不得臥,臥則驚,驚則咳甚也。腹中鳴者,病本於胃也。薄脾則煩不能食,食不下者,胃脘隔也。身重難以行者,胃脈在足也。月事不來者,胞脈閉也,胞脈者屬心而絡於胞中,今氣上迫肺,心氣不得下通,故月事不來也。帝曰:善。


%% \section{逆調論篇第三十四}

%%   黃帝問曰:人身非常溫也,非常熱也,為之熱而煩滿者何也?岐伯對曰:陰氣少而陽氣勝,故熱而煩滿也。
%%   帝曰:人身非衣寒也,中非有寒氣也,寒從中生者何?岐伯曰:是人多痹氣也,陽氣少,陰氣多,故身寒如從水中出。
%%   帝曰:人有四支熱,逢風寒如炙如火者,何也?岐伯曰:是人者,陰氣虛,陽氣盛,四支者陽也,兩陽相得,而陰氣虛少,少水不能滅盛火,而陽獨治,獨治者,不能生長也,獨勝而止耳,逢風而如炙如火者,是人當肉爍也。
%%   帝曰:人有身寒,湯火不能熱,厚衣不能溫,然不凍栗,是為何病?岐伯曰:是人者,素腎氣勝,以水為事;太陽氣衰,腎脂枯不長;一水不能勝兩火,腎者水也,而生於骨,腎不生,則髓不能滿,故寒甚至骨也。所以不能凍栗者,肝一陽也,心二陽也,腎孤藏也,一水不能勝二火,故不能凍栗,病名曰骨痹,是人當攣節也。
%%   帝曰:人之肉苛者,雖近衣絮,猶尚苛也,是謂何疾?岐伯曰:榮氣虛衛氣實也,榮氣虛則不仁,衛氣虛則不用,榮衛俱虛,則不仁且不用,肉如故也,人身與志不相有,曰死。
%%   帝曰:人有逆氣不得臥而息有音者;有不得臥而息無音者;有起居如故而息有音者;有得臥,行而喘者;有不得臥,不能行而喘者;有不得臥,臥而喘者;皆何藏使然?願聞其故。岐伯曰:不得臥而息有音者,是陽明之逆也,足三陽者下行,今逆而上行,故息有音也。陽明者,胃脈也,胃者六府之海,其氣亦下行,陽明逆不得從其道,故不得臥也。《下經》曰:胃不和則臥不安。此之謂也。夫起居如故而息有音者,此肺之絡脈逆也。絡脈不得隨經上下,故留經而不行,絡脈之病人也微,故起居如故而息有音也。夫不得臥,臥則喘者,是水氣之客也;夫水者,循津液而流也,腎者,水藏,主津液,主臥與喘也。帝曰:善。


%% \section{瘧論篇第三十五}

%% 黃帝問曰:夫痎瘧皆生於風,其蓄作有時者何也?岐伯對曰:瘧之始發也,先起於毫毛,伸欠乃作,寒慄鼓頷,腰脊俱痛,寒去則內外皆熱,頭痛如破,渴欲冷飲。
%%   帝曰:何氣使然?願聞其道。岐伯曰:陰陽上下交爭,虛實更作,陰陽相移也。陽並於陰,則陰實而陽虛,陽明虛,則寒慄鼓頷也;巨陽虛,則腰背頭項痛;三陽俱虛,則陰氣勝,陰氣勝則骨寒而痛;寒生於內,故中外皆寒;陽盛則外熱,陰虛則內熱,外內皆熱則喘而渴,故欲冷飲也。
%%   此皆得之夏傷於暑,熱氣盛,藏於皮膚之內,腸胃之外,此榮氣之所舍也。此令人汗空疏,腠理開,因得秋氣,汗出遇風,及得之以浴,水氣舍於皮膚之內,與衛氣並居。衛氣者,晝日行於陽,夜行於陰,此氣得陽而外出,得陰而內搏,內外相薄,是以日作。
%%   帝曰:其間日而作者何也?岐伯曰:其氣之舍深,內薄於陰,陽氣獨發,陰邪內著,陰與陽爭不得出,是以間日而作也。
%%   帝曰:善。其作日晏與其日早者,何氣使然?岐伯曰:邪氣客於風府,循膂而下,衛氣一日一夜大會於風府,其明日日下一節,故其作也晏,此先客於脊背也。每至於風府則腠理開,腠理開則邪氣入,邪氣入則病作,以此日作稍益晏也。其出於風府,日下一節,二十五日下至骶骨,二十六日入於脊內,注於伏膂之脈;其氣上行,九日出於缺盆之中,其氣日高,故作日益早也。其間日發者,由邪氣內薄於五藏,橫連募原也。其道遠,其氣深,其行遲,不能與衛氣俱行,不得皆出,故間日乃作也。
%%   帝曰:夫子言衛氣每至於風府,腠理乃發,發則邪氣入,入則病作。今衛氣日下一節,其氣之發也,不當風府,其日作者奈何?岐伯曰:此邪氣客於頭項循膂而下者也,故虛實不同,邪中異所,則不得當其風府也。故邪中於頭項者,氣至頭項而病;中於背者,氣至背而病;中於腰脊者,氣至腰脊而病;中於手足者,氣至手足而病。衛氣之所在,與邪氣相合,則病作。故風無常府,衛氣之所發,必開其腠理,邪氣之所合,則其府也。
%%   帝曰:善。夫風之與瘧也,相似同類,而風獨常在,瘧得有時而休者何也?岐伯曰:風氣留其處,故常在,瘧氣隨經絡沉以內薄,故衛氣應乃作。
%%   帝曰:瘧先寒而後熱者,何也?岐伯曰:夏傷於大暑,其汗大出,腠理開發,因遇夏氣淒滄之水寒,藏於腠理皮膚之中,秋傷於風,則病成矣,夫寒者,陰氣也,風者,陽氣也,先傷於寒而後傷於風,故先寒而後熱也,病以時作,名曰寒瘧。
%%   帝曰:先熱而後寒者,何也?岐伯曰:此先傷於風而後傷於寒,故先熱而後寒也,亦以時作,名曰溫瘧。
%%   其但熱而不寒者,陰氣先絕,陽氣獨發,則少氣煩冤,手足熱而欲嘔,名曰癉瘧。
%%   帝曰:夫經言有餘者寫之,不足者補之。今熱為有餘,寒為不足。夫瘧者之寒,湯火不能溫也,及其熱,冰水不能寒也,此皆有餘不足之類。當此之時,良工不能止,必須其自衰,乃刺之,其故何也?願聞其說。
%%   岐伯曰:經言無刺熇熇之熱,無刺渾渾之脈,無刺漉漉之汗,故為其病逆,未可治也。夫瘧之始發也,陽氣並於陰,當是之時,陽虛而陰盛,外無氣,故先寒慄也。陰氣逆極,則復出之陽,陽與陰復並於外,則陰虛而陽實,故先熱而渴。夫瘧氣者,並於陽則陽勝,並於陰則陰勝,陰勝則寒,陽勝則熱。瘧者,風寒之氣不常也,病極則復,至病之發也,如火之熱,如風雨不可當也。故經言曰:方其盛時必毀,因其衰也,事必大昌,此之謂也。夫瘧之未發也,陰未並陽,陽未並陰,因而調之,真氣得安,邪氣乃亡,故工不能治其已發,為其氣逆也。
%%   帝曰:善。攻之奈何?早晏何如?岐伯曰:瘧之且發也,陰陽之且移也,必從四末始也。陽已傷,陰從之,故先其時堅束其處,令邪氣不得入,陰氣不得出,審候見之,在孫絡盛堅而血者皆取之,此真往而未得並者也。
%%   帝曰:瘧不發,其應何如?岐伯曰:瘧氣者,必更盛更虛,當氣之所在也,病在陽,則熱而脈躁;在陰,則寒而脈靜;極則陰陽俱衰,衛氣相離,故病得休;衛氣集,則復病也。
%%   帝曰:時有間二日或至數日發,或渴或不渴,其故何也?岐伯曰:其間日者,邪氣與衛氣客於六府,而有時相失,不能相得,故休數日乃作也。瘧者,陰陽更勝也,或甚或不甚,故或渴或不渴。
%%   帝曰:論言夏傷於暑,秋必病瘧。今瘧不必應者,何也?岐伯曰:此應四時者也。其病異形者,反四時也。其以秋病者寒甚,以冬病者寒不甚,以春病者惡風,以夏病者多汗。
%%   帝曰:夫病溫瘧與寒瘧而皆安舍,舍於何藏?岐伯曰:溫瘧者,得之冬中於風,寒氣藏於骨髓之中,至春則陽氣大發,邪氣不能自出,因遇大暑,腦髓爍,肌肉消,腠理髮洩,或有所用力,邪氣與汗皆出,此病藏於腎,其氣先從內出之於外也。如是者,陰虛而陽盛,陽盛則熱矣,衰則氣復反入,入則陽虛,陽虛則寒矣,故先熱而後寒,名曰溫瘧。
%%   帝曰:癉瘧何如?岐伯曰:癉瘧者,肺素有熱。氣盛於身,厥逆上衝,中氣實而不外洩,因有所用力,腠理開,風寒舍於皮膚之內、分肉之間而發,發則陽氣盛,陽氣盛而不衰則病矣。其氣不及於陰,故但熱而不寒,氣內藏於心,而外舍於分肉之間,令人消爍脫肉,故命曰癉瘧。帝曰:善。

%% \section{刺瘧篇第三十六}

%% 足太陽之瘧,令人腰痛頭重,寒從背起,先寒後熱,熇熇暍暍然,熱止汗出,難已,刺隙中出血。
%%   足少陽之瘧,令人身體解(亻亦),寒不甚,熱不甚,惡見人,見人心惕惕然,熱多汗出甚,刺足少陽。
%%   足陽明之瘧,令人先寒,灑淅灑淅,寒甚久乃熱,熱去汗出,喜見日月光火氣,乃快然,刺足陽明跗上。
%%   足太陰之瘧,令人不樂,好太息,不嗜食,多寒熱汗出,病至則善嘔,嘔已乃衰,即取之。
%%   足少陰之瘧,令人嘔吐甚,多寒熱,熱多寒少,欲閉戶牖而處,其病難已。
%%   足厥陰之瘧,令人腰痛少腹滿,小便不利,如癃狀,非癃也,數便,意恐懼,氣不足,腹中悒悒,刺足厥陰。
%%   肺瘧者,令人心寒,寒甚熱,熱間善驚,如有所見者,刺手太陰陽明。
%%   心瘧者,令人煩心甚,欲得清水,反寒多,不甚熱,刺手少陰。
%%   肝瘧者,令人色蒼蒼然,太息,其狀若死者,刺足厥陰見血。
%%   脾瘧者,令人寒,腹中痛,熱則腸中鳴,鳴已汗出,刺足太陰。
%%   腎瘧者,令人灑灑然,腰脊痛,宛轉,大便難,目眴眴然,手足寒,刺足太陽少陰。
%%   胃瘧者,令人且病也,善飢而不能食,食而支滿腹大,刺足陽明太陰橫脈出血。
%%   瘧發身方熱,刺跗上動脈,開其空,出其血,立寒;瘧方欲寒,刺手陽明太陰,足陽明太陰。瘧脈滿大急,刺背俞,用中針,傍伍胠俞各一,適肥瘦出其血也。瘧脈小實急,灸脛少陰,刺指井。瘧脈滿大急,刺背俞,用五胠俞背俞各一,適行至於血也。
%%   瘧脈緩大虛,便宜用藥,不宜用針。凡治瘧,先發如食頃乃可以治,過之則失時也。諸瘧而脈不見,刺十指間出血,血去必已,先視身之赤如小豆者盡取之。十二瘧者,其發各不同時,察其病形,以知其何脈之病也。先其發時如食頃而刺之,一刺則衰,二刺則知,三刺則已;不已,刺舌下兩脈出血,不已,刺隙中盛經出血,又刺項已下俠脊者必已。舌下兩脈者,廉泉也。
%%   刺瘧者,必先問其病之所先發者,先刺之。先頭痛及重者,先刺頭上及兩額兩眉間出血。先項背痛者,先刺之。先腰脊痛者,先刺隙中出血。先手臂痛者,先刺手少陰陽明十指間。先足脛痠痛者,先刺足陽明十指間出血。風瘧,瘧發則汗出惡風,刺三陽經背俞之血者。(骨行)痠痛甚,按之不可,名曰胕髓病,以饞針針絕骨出血,立已。身體小痛,刺至陰,諸陰之井無出血,間日一刺。瘧不渴,間日而作,刺足太陽;渴而間日作,刺足少陽;溫瘧汗不出,為五十九刺。

%% \section{氣厥論篇第三十七}

%%   黃帝問曰:五藏六府,寒熱相移者何?岐伯曰:腎移寒於肝,癰腫少氣。脾移寒於肝,癰腫筋攣。肝移寒於心,狂隔中。心移寒於肺,肺消,肺消者飲一溲二,死不治。肺移寒於腎,為湧水,湧水者,按腹不堅,水氣客於大腸,疾行則鳴濯濯如囊裹漿,水之病也。
%%   脾移熱於肝,則為驚衄。肝移熱於心,則死。心移熱於肺,傳為鬲消。肺移熱於腎,傳為柔痓。腎移熱於脾,傳為虛,腸澼,死,不可治。
%%   胞移熱於膀胱,則癃溺血。膀胱移熱於小腸,鬲腸不便,上為口糜。小腸移熱於大腸,為虙瘕,為沉。大腸移熱於胃,善食而瘦入,謂之食亦。胃移熱於膽,亦曰食亦。膽移熱於腦,則辛頞鼻淵,鼻淵者,濁涕下不止也,傳為衄蔑瞑目,故得之氣厥也。


%% \section{咳論篇第三十八}

%%   黃帝問曰:肺之令人咳,何也?岐伯對曰:五藏六府皆令人咳,非獨肺也。帝曰:願聞其狀。岐伯曰:皮毛者,肺之合也,皮毛先受邪氣,邪氣以從其合也。其寒飲食入胃,從肺脈上至於肺,則肺寒,肺寒則外內合邪,因而客之,則為肺咳。五藏各以其時受病,非其時,各傳以與之。人與天地相參,故五藏各以治時,感於寒則受病,微則為咳,甚者為洩為痛。乘秋則肺先受邪,乘春則肝先受之,乘夏則心先受之,乘至陰則脾先受之,乘冬則腎先受之。
%%   帝曰:何以異之?岐伯曰:肺咳之狀,而喘息有音,甚則唾血。心咳之狀,則心痛,喉中介介如梗狀,甚則咽腫喉痹。肝咳之狀,咳則兩脅下痛,甚則不可以轉,轉則兩胠下滿。脾咳之狀,咳則右脅下下痛,陰陰引肩背,甚則不可以動,動則咳劇。腎咳之狀,咳則腰背相引而痛,甚則咳涎。
%%   帝曰:六府之咳奈何?安所受病?岐伯曰:五藏之久咳,乃移於六府。脾咳不已,則胃受之,胃咳之狀,咳而嘔,嘔甚則長蟲出。肝咳不已,則膽受之,膽咳之狀,咳嘔膽汁,肺咳不已,則大腸受之,大腸咳狀,咳而遺失。心咳不已,則小腸受之,小腸咳狀,咳而失氣,氣與咳俱失。腎咳不已,則膀胱受之,膀胱咳狀,咳而遺溺。久咳不已,則三焦受之,三焦咳狀,咳而腹滿,不欲食飲,此皆聚於胃,關於肺,使人多涕唾而面浮腫氣逆也。
%%   帝曰:治之奈何?岐伯曰:治藏者治其俞,治府者治其合,浮腫者治其經。帝曰:善。


%% \section{舉痛論篇第三十九}

%%   黃帝問曰:余聞善言天者,必有驗於人;善言古者,必有合於今;善言人者,必有厭於己。如此,則道不惑而要數極,所謂明也。今余問於夫子,令言而可知,視而可見,捫而可得,令驗於己而發蒙解惑,可得而聞乎?岐伯再拜稽首對曰:何道之問也?
%%   帝曰:願聞人之五藏卒痛,何氣使然?岐伯對曰:經脈流行不止、環周不休,寒氣入經而稽遲,泣而不行,客於脈外則血少,客於脈中則氣不通,故卒然而痛。
%%   帝曰:其痛或卒然而止者,或痛甚不休者,或痛甚不可按者,或按之而痛止者,或按之無益者,或喘動應手者,或心與背相引而痛者,或脅肋與少腹相引而痛者,或腹痛引陰股者,或痛宿昔而成積者,或卒然痛死不知人,有少間復生者,或痛而嘔者,或腹痛而後洩者,或痛而閉不通者,凡此諸痛,各不同形,別之奈何?
%%   岐伯曰:寒氣客於脈外則脈寒,脈寒則縮踡,縮踡則脈絀急,絀急則外引小絡,故卒然而痛,得炅則痛立止;因重中於寒,則痛久矣。
%%   寒氣客於經脈之中,與炅氣相薄則脈滿,滿則痛而不可按也。寒氣稽留,炅氣從上,則脈充大而血氣亂,故痛甚不可按也。
%%   寒氣客於腸胃之間,膜原之下,血不得散,小絡急引故痛,按之則血氣散,故按之痛止。
%%   寒氣客於俠脊之脈,則深按之不能及,故按之無益也。
%%   寒氣客於衝脈,衝脈起於關元,隨腹直上,寒氣客則脈不通,脈不通則氣因之,故揣動應手矣。
%%   寒氣客於背俞之脈則脈泣,脈泣則血虛,血虛則痛,其俞注於心,故相引而痛,按之則熱氣至,熱氣至則痛止矣。
%%   寒氣客於厥陰之脈,厥陰之脈者,絡陰器繫於肝,寒氣客於脈中,則血泣脈急,故脅肋與少腹相引痛矣。
%%   厥氣客於陰股,寒氣上及少腹,血泣在下相引,故腹痛引陰股。
%%   寒氣客於小腸膜原之間,絡血之中,血泣不得注於大經,血氣稽留不得行,故宿昔而成積矣。
%%   寒氣客於五藏,厥逆上洩,陰氣竭,陽氣未入,故卒然痛死不知人,氣復反則生矣。
%%   寒氣客於腸胃,厥逆上出,故痛而嘔也。
%%   寒氣客於小腸,小腸不得成聚,故後洩腹痛矣。
%%   熱氣留於小腸,腸中痛,癉熱焦喝,則堅幹不得出,故痛而閉不通矣。
%%   帝曰:所謂言而可知者也。視而可見奈何?岐伯曰:五藏六府,固盡有部,視其五色,黃赤為熱,白為寒,青黑為痛,此所謂視而可見者也。
%%   帝曰:捫而可得奈何?岐伯曰:視其主病之脈,堅而血及陷下者,皆可捫而得也。
%%   帝曰:善。余知百病生於氣也。怒則氣上,喜則氣緩,悲則氣消,恐則氣下,寒則氣收,炅則氣洩,驚則氣亂,勞則氣耗,思則氣結,九氣不同,何病之生?岐伯曰:怒則氣逆,甚則嘔血及飧洩,故氣上矣。喜則氣和志達,榮衛通利,故氣緩矣。悲則心繫急,肺布葉舉,而上焦不通,榮衛不散,熱氣在中,故氣消矣。恐則精卻,卻則上焦閉,閉則氣還,還則下焦脹,故氣不行矣。寒則腠理閉,氣不行,故氣收矣。炅則腠理開,榮衛通,汗大洩,故氣洩。驚則心無所倚,神無所歸,慮無所定,故氣亂矣。勞則喘息汗出,外內皆越,故氣耗矣。思則心有所存,神有所歸,正氣留而不行,故氣結矣。


%% \section{腹中論篇第四十}

%%   黃帝問曰:有病心腹滿,旦食則不能暮食,此為何病?岐伯對曰:名為鼓脹。帝曰:治之奈何?岐伯曰:治之以雞矢醴,一劑知,二劑已。帝曰:其時有復發者何也?岐伯曰:此飲食不節,故時有病也。雖然其病且已,時故當病,氣聚於腹也。
%%   帝曰:有病胸脅支滿者,妨於食,病至則先聞腥臊臭,出清液,先唾血,四支清,目眩,時時前後血,病名為何?何以得之?岐伯曰:病名血枯。此得之年少時,有所大脫血:若醉入房中,氣竭肝傷,故月事衰少不來也。帝曰:治之奈何?復以何術?岐伯曰:以四烏骨一藘茹二物併合之,丸以雀卵,大如小豆,以五丸為後飯,飲以鮑魚汁,利腸中及傷肝也。
%%   帝曰:病有少腹盛,上下左右皆有根,此為何病?可治不?岐伯曰:病名曰伏梁。帝曰:伏梁何因而得之?岐伯曰:裹大膿血,居腸胃之外,不可治,治之每切,按之致死。帝曰:何以然?岐伯曰:此下則因陰,必下膿血,上則迫胃脘,生鬲,俠胃脘內癰,此久病也,難治。居齊上為逆,居齊下為從,勿動亟奪,論在《刺法》中。
%%   帝曰:人有身體髀股(骨行)皆腫,環齊而痛,是為何病?岐伯曰:病名伏梁,此風根也。其氣溢於大腸而著於肓,肓之原在齊下,故環齊而痛也,不可動之,動之為水溺澀之病。
%%   帝曰:夫子數言熱中消中,不可服高梁芳草石藥,石藥發瘨,芳草發狂。夫熱中消中者,皆富貴人也,今禁高梁,是不合其心,禁芳草石藥,是病不愈,願聞其說。岐伯曰:夫芳草之氣美,石藥之氣悍,二者其氣急疾堅勁,故非緩心和人,不可以服此二者。帝曰:不可以服此二者,何以然?岐伯曰:夫熱氣慓悍,藥氣亦然,二者相遇,恐內傷脾,脾者土也而惡木,服此藥者,至甲乙日更論。
%%   帝曰:善。有病膺腫頸痛胸滿腹脹,此為何病?何以得之?岐伯曰:名厥逆。帝曰:治之奈何?岐伯曰:灸之則瘖,石之則狂,須其氣並,乃可治也。帝曰:何以然?岐伯曰:陽氣重上,有餘於上,灸之則陽氣入陰,入則瘖,石之則陽氣虛,虛則狂;須其氣並而治之,可使全也。
%%   帝曰:善。何以知懷子之且生也?岐伯曰:身有病而無邪脈也。
%%   帝曰:病熱而有所痛者何也?岐伯曰:病熱者,陽脈也,以三陽之動也,人迎一盛少陽,二盛太陽,三盛陽明,入陰也。夫陽入於陰,故病在頭與腹,乃(月真)脹而頭痛也。帝曰:善。

%% \section{刺腰痛篇第四十一}

%%   足太陽脈令人腰痛,引項脊尻背如重狀;刺其隙中太陽正經出血,春無見血。
%%   少陽令人腰痛,如以針刺其皮中,循循然不可以俯仰,不可以顧,刺少陽成骨之端出血,成骨在膝外廉之骨獨起者,夏無見血。
%%   陽明令人腰痛,不可以顧,顧如有見者,善悲,刺陽明於(骨行)前三痏,上下和之出血,秋無見血。
%%   足少陰令人腰痛,痛引脊內廉,刺少陰於內踝上二痏,春無見血,出血太多,不可復也。
%%   厥陰之脈,令人腰痛,腰中如張弓弩弦;刺厥陰之脈,在腨踵魚腹之外,循之纍纍然,乃刺之,其病令人善言,默默然不慧,刺之三痏。
%%   解脈令人腰痛,痛引肩,目然,時遺溲,刺解脈,在膝筋肉分間隙外廉之橫脈出血,血變而止。
%%   解脈令人腰痛如引帶,常如折腰狀,善恐,刺解脈在隙中結絡如黍米,刺之血射以黑,見赤血而已。
%%   同陰之脈,令人腰痛,痛如小錘居其中,怫然腫;刺同陰之脈,在外踝上絕骨之端,為三痏。
%%   陽維之脈,令人腰痛,痛上怫然腫;刺陽維之脈,脈與太陽合腨下間,去地一尺所。
%%   衡絡之脈,令人腰痛,不可以俛仰,仰則恐僕,得之舉重傷腰,衡絡絕,惡血歸之,刺之在隙陽筋之間,上隙數寸,衡居為二痏出血。
%%   會陰之脈,令人腰痛,痛上漯漯然汗出,汗干令人欲飲,飲已欲走,刺直陽之脈上三痏,在蹻上隙下五寸橫居,視其盛者出血。
%%   飛陽之脈,令人腰痛,痛上怫怫然,甚則悲以恐;刺飛陽之脈,在內踝上五寸,少陰之前,與陰維之會。
%%   昌陽之脈,令人腰痛,痛引膺,目(目巟)(目巟)然,甚則反折,舌卷不能言;刺內筋為二痏,在內踝上大筋前,太陰後,上踝二寸所。
%%   散脈,令人腰痛而熱,熱甚生煩,腰下如有橫木居其中,甚則遺溲;刺散脈,在膝前骨肉分間,絡外廉束脈,為三痏。
%%   肉裡之脈,令人腰痛,不可以咳,咳則筋縮急;刺肉裡之脈為二痏,在太陽之外,少陽絕骨之後。
%%   腰痛俠脊而痛至頭,幾幾然,目(目巟)(目巟)欲僵仆,刺足太陽隙中出血。腰痛上寒,刺足太陽陽明;上熱,刺足厥陰;不可以俛仰,刺足少陽;中熱而喘,刺足少陰,刺隙中出血。
%%   腰痛上寒,不可顧,刺足陽明;上熱,刺足太陰;中熱而喘,刺足少陰。大便難,刺足少陰。少腹滿,刺足厥陰。如折,不可以俛仰,不可舉,刺足太陽,引脊內廉,刺足少陰。
%%   腰痛引少腹控(月少),不可以仰。刺腰尻交者,兩髁胂上。以月生死為痏數,髮針立已。左取右,右取左。


%% \section{風論篇第四十二}

%%   黃帝問曰:風之傷人也,或為寒熱,或為熱中,或為寒中,或為癘風,或為偏枯,或為風也,其病各異,其名不同,或內至五藏六府,不知其解,願聞其說。
%%   岐伯對曰:風氣藏於皮膚之間,內不得通,外不得洩;風者,善行而數變,腠理開則灑然寒,閉則熱而悶,其寒也則衰食飲,其熱也則消肌肉,故使人怢慄而不能食,名曰寒熱。
%%   風氣與陽明入胃,循脈而上至目內眥,其人肥則風氣不得外洩,則為熱中而目黃;人瘦則外洩而寒,則為寒中而泣出。
%%   風氣與太陽俱入,行諸脈俞,散於分肉之間,與衛氣相干,其道不利,故使肌肉憤(月真)而有瘍,衛氣有所凝而不行,故其肉有不仁也。癘者,有榮氣熱府,其氣不清,故使其鼻柱壞而色敗,皮膚瘍潰,風寒客於脈而不去,名曰癘風,或名曰寒熱。
%%   以春甲乙傷於風者為肝風,以夏丙丁傷於風者為心風,以季夏戊己傷於邪者為脾風,以秋庚辛中於邪者為肺風,以冬壬癸中於邪者為腎風。
%%   風中五藏六府之俞,亦為藏府之風,各入其門戶所中,則為偏風。風氣循風府而上,則為腦風;風入系頭,則為目風,眼寒;飲酒中風,則為漏風;入房汗出中風,則為內風;新沐中風,則為首風;久風入中,則為腸風飧洩;外在腠理,則為洩風。故風者百病之長也,至其變化,乃為他病也,無常方,然致有風氣也。
%%   帝曰:五藏風之形狀不同者何?願聞其診及其病能。
%%   岐伯曰:肺風之狀,多汗惡風,色皏然白,時咳短氣,晝日則差,暮則甚,診在眉上,其色白。
%%   心風之狀,多汗惡風,焦絕,善怒嚇,赤色,病甚則言不可快,診在口,其色赤。
%%   肝風之狀,多汗惡風,善悲,色微蒼,嗌干善怒,時憎女子,診在目下,其色青。
%%   脾風之狀,多汗惡風,身體怠惰,四支不欲動,色薄微黃,不嗜食,診在鼻上,其色黃。
%%   腎風之狀,多汗惡風,面(疒龍)然浮腫,脊痛不能正立,其色炲,隱曲不利,診在肌上,其色黑。
%%   胃風之狀,頸多汗惡風,食飲不下,鬲塞不通,腹善滿,失衣則(月真)脹,食寒則洩,診形瘦而腹大。
%%   首風之狀,頭面多汗,惡風,當先風一日則病甚,頭痛不可以出內,至其風日,則病少愈。
%%   漏風之狀,或多汗,常不可單衣,食則汗出,甚則身汗,喘息惡風,衣常濡,口乾善渴,不能勞事。
%%   洩風之狀,多汗,汗出洩衣上,口中干,上漬其風,不能勞事,身體盡痛則寒。帝曰:善。


%% \section{痹論篇第四十三}

%%   黃帝問曰:痹之安生?岐伯對曰:風寒濕三氣雜至,合而為痹也。其風氣勝者為行痹,寒氣勝者為痛痹,濕氣勝者為著痹也。
%%   帝曰:其有五者何也?岐伯曰:以冬遇此者為骨痹,以春遇此者為筋痹,以夏遇此者為脈痹,以至陰遇此者為肌痹,以秋遇此者為皮痹。
%%   帝曰:內舍五藏六府,何氣使然?岐伯曰:五藏皆有合,病久而不去者,內舍於其合也。故骨痹不已,復感於邪,內舍於腎;筋痹不已,復感於邪,內舍於肝;脈痹不已,復感於邪,內舍於心;肌痹不已,復感於邪,內舍於脾;皮痹不已,復感於邪,內舍於肺。所謂痹者,各以其時,重感於風寒濕之氣也。
%%   凡痹之客五藏者,肺痹者,煩滿喘而嘔;心痹者,脈不通,煩則心下鼓,暴上氣而喘,嗌干善噫,厥氣上則恐;肝痹者,夜臥則驚,多飲數小便,上為引如懷;腎痹者,善脹,尻以代踵,脊以代頭;脾痹者,四支懈惰,發咳嘔汁,上為大塞;腸痹者,數飲而出不得,中氣喘爭,時發飧洩;胞痹者,少腹膀胱,按之內痛,若沃以湯,澀於小便,上為清涕。
%%   陰氣者,靜則神藏,躁則消亡,飲食自倍,腸胃乃傷。淫氣喘息,痹聚在肺;淫氣憂思,痹聚在心;淫氣遺溺,痹聚在腎;淫氣乏竭,痹聚在肝;淫氣肌絕,痹聚在脾。
%%   諸痹不巳,亦益內也,其風氣勝者,其人易已也。
%%   帝曰:痹,其時有死者,或疼久者,或易已者,其故何也?岐伯曰:其入藏者死,其留連筋骨間者疼久,其留皮膚間者易已。
%%   帝曰:其客於六府者何也?岐伯曰:此亦其食飲居處,為其病本也。六府亦各有俞,風寒濕氣中其俞,而食飲應之,循俞而入,各舍其府也。
%%   帝曰:以針治之奈何?岐伯曰:五藏有俞,六府有合,循脈之分,各有所發,各隨其過,則病瘳也。
%%   帝曰:榮衛之氣,亦令人痹乎?岐伯曰:榮者,水谷之精氣也,和調於五藏,灑陳於六府,乃能入於脈也。故循脈上下,貫五藏,絡六府也。衛者,水谷之悍氣也,其氣慓疾滑利,不能入於脈也,故循皮膚之中,分肉之間,熏於肓膜,散於胸腹,逆其氣則病,從其氣則愈,不與風寒濕氣合,故不為痹。
%%   帝曰:善。痹或痛,或不痛,或不仁,或寒,或熱,或燥,或濕,其故何也?岐伯曰:痛者,寒氣多也,有寒故痛也。其不痛不仁者,病久入深,榮衛之行澀,經絡時疏,故不通,皮膚不營,故為不仁。其寒者,陽氣少,陰氣多,與病相益,故寒也。其熱者,陽氣多,陰氣少,病氣勝,陽遭陰,故為痹熱。其多汗而濡者,此其逢濕甚也,陽氣少,陰氣盛,兩氣相感,故汗出而濡也。
%%   帝曰:夫痹之為病,不痛何也?岐伯曰:痹在於骨則重,在於脈則血凝而不流,在於筋則屈不伸,在於肉則不仁,在於皮則寒,故具此五者則不痛也。凡痹之類,逢寒則蟲,逢熱則縱。帝曰:善。


%% \section{痿論篇第四十四}

%%   黃帝問曰:五藏使人痿何也?岐伯對曰:肺主身之皮毛,心主身之血脈,肝主身之筋膜,脾主身之肌肉,腎主身之骨髓。故肺熱葉焦,則皮毛虛弱急薄,著則生痿躄也;心氣熱,則下脈厥而上,上則下脈虛,虛則生脈痿,樞折挈,脛縱而不任地也;肝氣熱,則膽洩口苦筋膜干,筋膜干則筋急而攣,發為筋痿;脾氣熱,則胃干而渴,肌肉不仁,發為肉痿;腎氣熱,則腰脊不舉,骨枯而髓減,發為骨痿。
%%   帝曰:何以得之?岐伯曰:肺者,藏之長也,為心之蓋也;有所失亡,所求不得,則發肺鳴,鳴則肺熱葉焦,故曰,五藏因肺熱葉焦發為痿躄,此之謂也。悲哀太甚,則胞絡絕,胞絡絕,則陽氣內動,發則心下崩,數溲血也。故《本病》曰:大經空虛,發為肌痹,傳為脈痿。思想無窮,所願不得,意淫於外,入房太甚,宗筋弛縱,發為筋痿,及為白淫,故《下經》曰:筋痿者,生於肝使內也。有漸於濕,以水為事,若有所留,居處相濕,肌肉濡漬,痹而不仁,發為肉痿。故《下經》曰:肉痿者,得之濕地也。有所遠行勞倦,逢大熱而渴,渴則陽氣內伐,內伐則熱舍於腎,腎者水藏也,今水不勝火,則骨枯而髓虛,故足不任身,發為骨痿。故《下經》曰:骨痿者,生於大熱也。
%%   帝曰:何以別之?岐伯曰:肺熱者色白而毛敗,心熱者色赤而絡脈溢,肝熱者色蒼而爪枯,脾熱者色黃而肉蠕動;腎熱者色黑而齒槁。
%%   帝曰:如夫子言可矣,論言治痿者獨取陽明,何也?岐伯曰:陽明者,五藏六府之海,主潤宗筋,宗筋主骨而利機關也。衝脈者,經脈之海也,主滲灌谿谷,與陽明合於宗筋,陰陽總宗筋之會,會於氣街,而陽明為之長,皆屬於帶脈,而絡於督脈。故陽明虛則宗筋縱,帶脈不引,故足痿不用也。
%%   帝曰:治之奈何?岐伯曰:各補其滎而通其俞,調其虛實,和其逆順,筋、脈、骨、肉各以其時受月,則病已矣。帝曰:善。


%% \section{厥論篇第四十五}

%%   黃帝問曰:厥之寒熱者何也?岐伯對曰:陽氣衰於下,則為寒厥;陰氣衰於下,則為熱厥。
%%   帝曰:熱厥之為熱也,必起於足下者何也?岐伯曰:陽氣起於足五指之表,陰脈者集於足下,而聚於足心,故陽氣盛則足下熱也。
%%   帝曰:寒厥之為寒也,必從五指而上於膝者何也?岐伯曰:陰氣起於五指之裡,集於膝下而聚於膝上,故陰氣盛,則從五指至膝上寒,其寒也,不從外,皆從內也。
%%   帝曰:寒厥何失而然也?岐伯曰:前陰者,宗筋之所聚,太陰陽明之所合也。春夏則陽氣多而陰氣少,秋冬則陰氣盛而陽氣衰。此人者質壯,以秋冬奪於所用,下氣上爭不能復,精氣溢下,邪氣因從之而上也;氣因於中,陽氣衰,不能滲營其經絡,陽氣日損,陰氣獨在,故手足為之寒也。
%%   帝曰:熱厥何如而然也?岐伯曰;灑入於胃,則絡脈滿而經脈虛;脾主為胃行其津液者也,陰氣虛則陽氣入,陽氣入則胃不和,胃不和則精氣竭,精氣竭則不營其四支也。此人必數醉若飽以入房,氣聚於脾中不得散,酒氣與谷氣相薄,熱盛於中,故熱偏於身內熱而溺赤也。夫酒氣盛而慓悍,腎氣有衰,陽氣獨盛,故手足為之熱也。
%%   帝曰:厥或令人腹滿,或令人暴不知人,或至半日遠至一日乃知人者何也?岐伯曰:陰氣盛於上則下虛,下虛則腹脹滿;陽氣盛於上,則下氣重上,而邪氣逆,逆則陽氣亂,陽氣亂則不知人也。
%%   帝曰:善。願聞六經脈之厥狀病能也。岐伯曰:巨陽之厥,則腫首頭重,足不能行,發為眴僕;陽明之厥,則癲疾欲走呼,腹滿不得臥,面赤而熱,妄見而妄言;少陽之厥,則暴聾頰腫而熱,脅痛,(骨行)不可以運;太陰之厥,則腹滿(月真)脹,後不利,不欲食,食則嘔,不得臥;少陰之厥,則口乾溺赤,腹滿心痛;厥陰之厥,則少腹腫痛,腹脹,涇溲不利,好臥屈膝,陰縮腫,(骨行)內熱。盛則寫之,虛則補之,不盛不虛,以經取之。
%%   太陰厥逆,(骨行)急攣,心痛引腹,治主病者;少陰厥逆,虛滿嘔變,下洩清,治主病者;厥陰厥逆,攣、腰痛,虛滿前閉,譫言,治主病者;三陰俱逆,不得前後,使人手足寒,三日死。太陽厥逆,僵仆,嘔血善衄,治主病者;少陽厥逆,機關不利,機關不利者,腰不可以行,項不可以顧,發腸癰不可治,驚者死;陽明厥逆,喘咳身熱,善驚,衄,嘔血。
%%   手太陰厥逆,虛滿而咳,善嘔沫,治主病者;手心主、少陰厥逆,心痛引喉,身熱死,不可治。手太陽厥逆,耳聾泣出,項不可以顧,腰不可以俛仰,治主病者;手陽明、少陽厥逆,發喉痹、嗌腫,治主病者。

%% \section{病能論篇第四十六}

%%   黃帝問曰:人病胃脘癰者,診當何如?岐伯對曰:診此者當候胃脈,其脈當沉細,沉細者氣逆,逆者人迎甚盛,甚盛則熱;人迎者胃脈也,逆而盛,則熱聚於胃口而不行,故胃脘為癰也。
%%   帝曰:善。人有臥而有所不安者何也?岐伯曰:藏有所傷,及精有所之寄則安,故人不能懸其病也。
%%   帝曰:人之不得偃臥者何也?岐伯曰:肺者藏之蓋也,肺氣盛則脈大,脈大則不得偃臥,論在《奇恆陰陽》中。
%%   帝曰:有病厥者,診右脈沉而緊,左脈浮而遲,不然病主安在?岐伯曰:冬診之,右脈固當沉緊,此應四時,左脈浮而遲,此逆四時,在左當主病在腎,頗關在肺,當腰痛也。帝曰:何以言之?岐伯曰:少陰脈貫腎絡肺,今得肺脈,腎為之病,故腎為腰痛之病也。
%%   帝曰:善。有病頸癰者,或石治之,或針灸治之,而皆已,其真安在?岐伯曰:此同名異等者也。夫癰氣之息者,宜以針開除去之;夫氣盛血聚者,宜石而寫之。此所謂同病異治也。
%%   帝曰:有病怒狂者,此病安生?岐伯曰:生於陽也,帝曰:陽何以使人狂?岐伯曰:陽氣者,因暴折而難決,故善怒也,病名曰陽厥。帝曰:何以知之?岐伯曰:陽明者常動,巨陽少陽不動,不動而動大疾,此其候也。帝曰:治之奈何?岐伯曰:奪其食即已。夫食入於陰,長氣於陽,故奪其食即已。使之服以生鐵洛為飲,夫生鐵洛者,下氣疾也。
%%   帝曰:善。有病身熱解墯,汗出如浴,惡風少氣,此為何病?岐伯曰:病名曰酒風。帝曰:治之奈何?岐伯曰:以澤瀉,術各十分,麋銜五分,合,以三指撮,為後飯。
%%   所謂深之細者,其中手如針也,摩之切之,聚者堅也,博者大也。《上經》者,言氣之通天也;《下經》者,言病之變化也;《金匱》者,決死生也;《揆度》者,切度之也;《奇恆》者,言奇病也。所謂奇者,使奇病不得以四時死也;恆者,得以四時死也。所謂揆者,方切求之也,言切求其脈理也;度者,得其病處,以四時度之也。


%% \section{奇病論篇第四十七}

%% 黃帝問曰:人有重身,九月而瘖,此為何也?岐伯對曰:胞之絡脈絕也。帝曰:何以言之?岐伯曰:胞絡者係於腎,少陰之脈,貫腎系舌本,故不能言。帝曰:治之奈何?岐伯曰:無治也,當十月復。《刺法》曰:無損不足,益有餘,以成其疹,然後調之。所謂無損不足者,身羸瘦,無用鑱石也;無益其有餘者,腹中有形而洩之,洩之則精出而病獨擅中,故曰疹成也。
%%   帝曰:病脅下滿氣逆,二三歲不已,是為何病?岐伯曰:病名曰息積,此不妨於食,不可灸刺,積為導引服藥,藥不能獨治也。
%%   帝曰:人有身體髀股(骨行)皆腫,環齊而痛,是為何病?岐伯曰:病名曰伏梁。此風根也,其氣溢於大腸,而著於肓,肓之原在齊下,故環齊而痛也。不可動之,動之為水溺濇之病也。
%%   帝曰:人有尺脈數甚,筋急而見,此為何病?岐伯曰:此所謂疹筋,是人腹必急,白色黑色見,則病甚。
%%   帝曰:人有病頭痛以數歲不已,此安得之?名為何病?岐伯曰:當有所犯大寒,內至骨髓,髓者以腦為主,腦逆故令頭痛,齒亦痛,病名曰厥逆。帝曰:善。
%%   帝曰:有病口甘者,病名為何?何以得之?岐伯曰:此五氣之溢也,名曰脾癉。夫五味入口,藏於胃,脾為之行其精氣,津液在脾,故令人口甘也;此肥美之所發也,此人必數食甘美而多肥也,肥者令人內熱,甘者令人中滿,故其氣上溢,轉為消渴。治之以蘭,除陳氣也。
%%   帝曰:有病口苦,取陽陵泉,口苦者病名為何?何以得之?岐伯曰:病名曰膽癉。夫肝者中之將也,取決於膽,咽為之使。此人者,數謀慮不決,故膽虛氣上溢,而口為之苦。治之以膽募俞,治在《陰陽十二官相使》中。
%%   帝曰:有癃者,一日數十溲,此不足也。身熱如炭,頸膺如格,人迎躁盛,喘息氣逆,此有餘也。太陰脈微細如發者,此不足也。其病安在?名為何病?岐伯曰:病在太陰,其盛在胃,頗在肺,病名曰厥,死不治。此所謂得五有餘二不足也。帝曰:何謂五有餘二不足?岐伯曰:所謂五有餘者,五病之氣有餘也;二不足者,亦病氣之不足也。今外得五有餘,內得二不足,此其身不表不裡,亦正死明矣。
%%   帝曰:人生而有病巔疾者,病名曰何?安所得之?岐伯曰:病名為胎病。此得之在母腹中時,其母有所大驚,氣上而不下,精氣並居,故令子發為巔疾也。
%%   帝曰:有病(疒龍)然如有水狀,切其脈大緊,身無痛者,形不瘦,不能食,食少,名為何病?岐伯曰:病生在腎,名為腎風。腎風而不能食,善驚,驚已,心氣痿者死。帝曰:善。

%% \section{大奇論篇第四十八}

%%   肝滿腎滿肺滿皆實,即為腫。肺之雍,喘而兩胠滿;肝雍,兩胠滿,臥則驚,不得小便;腎雍,腳下至少腹滿,脛有大小,髀(骨行)大跛,易偏枯。
%%   心脈滿大,癇瘛筋攣;肝脈小急,癇瘛筋攣;肝脈騖,暴有所驚駭,脈不至若瘖,不治自已。
%%   腎脈小急,肝脈小急,心脈小急,不鼓皆為瘕。
%%   腎肝並沉為石水,並浮為風水,並虛為死,並小弦欲驚。
%%   腎脈大急沉,肝脈大急沉,皆為疝。
%%   心脈搏滑急為心疝,肺脈沉搏為肺疝。
%%   三陽急為瘕,三陰急為疝,二陰急為癇厥,二陽急為驚。
%%   脾脈外鼓,沉為腸澼,久自已。肝脈小緩為腸澼,易治。腎脈小搏沉,為腸澼下血,血溫身熱者死。心肝澼亦下血,二藏同病者可治。其脈小沉濇為腸澼,其身熱者死,熱見七日死。
%%   胃脈沉鼓濇,胃外鼓大,心脈小堅急,皆鬲偏枯。男子發左,女子發右,不瘖舌轉,可治,三十日起,其從者,瘖,三歲起。年不滿二十者,三歲死。
%%   脈至而搏,血衄身熱者死,脈來懸鉤浮為常脈。
%%   脈至如喘,名曰暴厥。暴厥者,不知與人言。脈至如數,使人暴驚,三四日自已。
%%   脈至浮合,浮合如數,一息十至以上,是經氣予不足也,微見九十日死;脈至如火薪然,是心精之予奪也,草干而死;脈至如散葉,是肝氣予虛也,木葉落而死;脈至如省客,省客者,脈塞而鼓,是腎氣予不足也,懸去棗華而死;脈至如丸泥,是胃精予不足也,榆莢落而死;脈至如橫格,是膽氣予不足也,禾熟而死;脈至如弦縷,是胞精予不足也,病善言,下霜而死,不言可治;脈至如交漆,交漆者,左右傍至也,微見三十日死;脈至如湧泉,浮鼓肌中,太陽氣予不足也,少氣味,韭英而死;脈至如頹土之狀,按之不得,是肌氣予不足也,五色先見,黑白壘發死;脈至如懸雍,懸雍者,浮揣切之益大,是十二俞之予不足也,水凝而死;脈至如偃刀,偃刀者,浮之小急,按之堅大急,五藏菀熟,寒熱獨並於腎也,如此其人不得坐,立春而死;脈至如丸滑不直手,不直手者,按之不可得也,是大腸氣予不足也,棗葉生而死;脈至如華者,令人善恐,不欲坐臥,行立常聽,是小腸氣予不足也,季秋而死。


%% \section{脈解篇第四十九}

%%   太陽所謂腫腰脽痛者,正月太陽寅,寅太陽也,正月陽氣出在上,而陰氣盛,陽未得自次也,故腫腰脽痛也。病偏虛為跛者,正月陽氣凍解地氣而出也,所謂偏虛者,冬寒頗有不足者,故偏虛為跛也。所謂強上引背者,陽氣大上而爭,故強上也。所謂耳鳴者,陽氣萬物盛上而躍,故耳鳴也。所謂甚則狂巔疾者,陽盡在上,而陰氣從下,下虛上實,故狂巔疾也,所謂浮為聾者,皆在氣也。所謂入中為瘖者,陽盛已衰,故為瘖也。內奪而厥,則為瘖俳,此腎虛也。少陰不至者,厥也。
%%   少陽謂心脅痛者,言少陽盛也,盛者心之所表也。九月陽氣盡而陰氣盛,故心脅痛也。所謂不可反側者,陰氣藏物也,物藏則不動,故不可反側也。所謂甚則躍者,九月萬物盡衰,草木畢落而墮,則氣去陽而之陰,氣盛而陽之下長,故謂躍。
%%   陽明所謂灑灑振寒者,陽明者午也,五月盛陽之陰也,陽盛而陰氣加之,故灑灑振寒也。所謂脛腫而股不收者,是五月盛陽之陰也,陽者衰於五月,而一陰氣上,與陽始爭,故脛腫而股不收也。所謂上喘而為水者,陰氣下而復上,上則邪客於藏府間,故為水也。所謂胸痛少氣者,水氣在藏府也,水者,陰氣也,陰氣在中,故胸痛少氣也。所謂甚則厥,惡人與火,聞木音則惕然而驚者,陽氣與陰氣相薄,水火相惡,故惕然而驚也。所謂欲獨閉戶牖而處者,陰陽相薄也,陽盡而陰盛,故欲獨閉戶牖而居。所謂病至則欲乘高而歌,棄衣而走者,陰陽復爭,而外並於陽,故使之棄衣而走也。所謂客孫脈則頭痛鼻鼽腹腫者,陽明並於上,上者則其孫絡太陰也,故頭痛鼻鼽腹腫也。
%%   太陰所謂病脹者,太陰子也,十一月萬物氣皆藏於中,故曰病脹;所謂上走心為噫者,陰盛而上走於陽明,陽明絡屬心,故曰上走心為噫也;所謂食則嘔者,物盛滿而上溢,故嘔也;所謂得後與氣則快然如衰者,十二月陰氣下衰,而陽氣且出,故曰得後與氣則快然如衰也。
%%   少陰所謂腰痛者,少陰者,腎也,十月萬物陽氣皆傷,故腰痛也。所謂嘔咳上氣喘者,陰氣在下,陽氣在上,諸陽氣浮,無所依從,故嘔咳上氣喘也。所謂色色不能久立久坐,起則目(目巟)(目巟)無所見者,萬物陰陽不定未有主也,秋氣始至,微霜始下,而方殺萬物,陰陽內奪,故目(目巟)(目巟)無所見也。所謂少氣善怒者,陽氣不治,陽氣不治,則陽氣不得出,肝氣當治而未得,故善怒,善怒者,名曰煎厥。所謂恐如人將捕之者,秋氣萬物未有畢去,陰氣少,陽氣入,陰陽相薄,故恐也。所謂惡聞食臭者,胃無氣,故惡聞食臭也。所謂面黑如地色者,秋氣內奪,故變於色也。所謂咳則有血者,陽脈傷也,陽氣未盛於上而脈滿,滿則咳,故血見於鼻也。
%%   厥陰所謂頹疝,婦人少腹腫者,厥陰者辰也,三月陽中之陰,邪在中,故曰頹疝少腹腫也。所謂腰脊痛不可以俯仰者,三月一振榮華,萬物一俯而不仰也。所謂頹癃疝膚脹者,曰陰亦盛而脈脹不通,故曰頹癃疝也。所謂甚則嗌乾熱中者,陰陽相薄而熱,故嗌干也。


%% \section{刺要論篇第五十}

%%   黃帝問曰:願聞刺要。岐伯對曰:病有浮沉,刺有淺深,各至其理,無過其道,過之則內傷,不及則生外壅,壅則邪從之,淺深不得,反為大賊,內動五藏,後生大病。故曰:病有在毫毛腠理者,有在皮膚者,有在肌肉者,有在脈者,有在筋者,有在骨者,有在髓者。
%%   是故刺毫毛腠理無傷皮,皮傷則內動肺,肺動則秋病溫瘧,淅淅然寒慄。
%%   刺皮無傷肉,肉傷則內動脾,脾動則七十二日四季之月,病腹脹煩,不嗜食。
%%   刺肉無傷脈,脈傷則內動心,心動則夏病心痛。
%%   刺脈無傷筋,筋傷則內動肝,肝動則春病熱而筋弛。
%%   刺筋無傷骨,骨傷則內動腎,腎動則冬病脹腰痛。
%%    刺骨無傷髓,髓傷則銷鑠(骨行)酸,體解(亻亦)然不去矣。


%% \section{刺齊論篇第五十一}

%%   黃帝問曰:願聞刺淺深之分。岐伯對曰:刺骨者無傷筋,刺筋者無傷肉,刺肉者無傷脈,刺脈者無傷皮,刺皮者無傷肉,刺肉者無傷筋,刺筋者無傷骨。
%%   帝曰:余未知其所謂,願聞其解。岐伯曰:刺骨無傷筋者,針至筋而去,不及骨也。刺筋無傷肉者,至肉而去,不及筋也。刺肉無傷脈者,至脈而去,不及肉也。刺脈無傷皮者,至皮而去,不及脈也。
%%   所謂刺皮無傷肉者,病在皮中,針入皮中,無傷肉也。刺肉無傷筋者,過肉中筋也。刺筋無傷骨者,過筋中骨也。此之謂反也。


%% \section{刺禁論篇第五十二}
%%  
%%   黃帝問曰:願聞禁數。岐伯對曰:藏有要害,不可不察,肝生於左,肺藏於右,心部於表,腎治於裡,脾為之使,胃為之市。鬲肓之上,中有父母,七節之傍,中有小心,從之有福,逆之有咎。
%%   刺中心,一日死,其動為噫。刺中肝,五日死,其動為語。刺中腎,六日死,其動為嚏。刺中肺,三日死,其動為咳。刺中脾,十日死,其動為吞。刺中膽,一日半死,其動為嘔。
%%   刺跗上,中大脈,血出不止死。刺面,中溜脈,不幸為盲。刺頭,中腦戶,入腦立死。刺舌下,中脈太過,血出不止為瘖。刺足下布絡中脈,血不出為腫。刺隙中大脈,令人僕脫色。刺氣街中脈,血不出為腫,鼠僕。刺脊間中髓,為傴。刺乳上,中乳房,為腫,根蝕。刺缺盆中內陷,氣洩,令人喘咳逆。刺手魚腹內陷,為腫。
%%   無刺大醉,令人氣亂。無刺大怒,令人氣逆。無刺大勞人,無刺新飽人,無刺大飢人,無刺大渴人,無刺大驚人。
%%   刺陰股中大脈,血出不止死。刺客主人內陷中脈,為內漏、為聾。刺膝髕出液,為跛。刺臂太陰脈,出血多立死。刺足少陰脈,重虛出血,為舌難以言。刺膺中陷,中肺,為喘逆仰息。刺肘中內陷,氣歸之,為不屈伸。刺陰股下三寸內陷,令人遺溺。刺掖下脅間內陷,令人咳。刺少腹,中膀胱,溺出,令人少腹滿。刺腨腸內陷為腫。刺匡上陷骨中脈,為漏、為盲。刺關節中液出,不得屈伸。


%% \section{刺志論篇第五十三}

%%   黃帝問曰:願聞虛實之要。岐伯對曰:氣實形實,氣虛形虛,此其常也,反此者病。谷盛氣盛,谷虛氣虛,此其常也,反此者病。脈實血實,脈虛血虛,此其常也,反此者病。
%%   帝曰:如何而反?岐伯曰:氣虛身熱,此謂反也;谷入多而氣少,此謂反也;谷不入而氣多,此謂反也;脈盛血少,此謂反也;脈少血多,此謂反也。
%%   氣盛身寒,得之傷寒。氣虛身熱,得之傷暑。谷入多而氣少者,得之有所脫血,濕居下也。谷入少而氣多者,邪在胃及與肺也。脈小血多者,飲中熱也。脈大血少者,脈有風氣,水漿不入,此之謂也。
%%   夫實者,氣入也;虛者,氣出也;氣實者,熱也;氣虛者,寒也。入實者,左手開針空也;入虛者,左手閉針空也。


%% \section{針解篇第五十四}

%%   黃帝問曰:願聞九針之解,虛實之道。岐伯對曰:刺虛則實之者,針下熱也,氣實乃熱也。滿而洩之者,針下寒也,氣虛乃寒也。菀陳則除之者,出惡血也。邪勝則虛之者,出針勿按;徐而疾則實者,徐出針而疾按之;疾而徐則虛者,疾出針而徐按之;言實與虛者,寒溫氣多少也。若無若有者,疾不可知也。察後與先者,知病先後也。為虛與實者,工勿失其法。若得若失者,離其法也。虛實之要,九針最妙者,為其各有所宜也。補寫之時者,與氣開闔相合也。九針之名,各不同形者,針窮其所當補寫也。
%%   刺實須其虛者,留針陰氣隆至,乃去針也;刺虛須其實者,陽氣隆至,針下熱乃去針也。經氣已至,慎守勿失者,勿變更也。深淺在志者,知病之內外也;近遠如一者,深淺其候等也。如臨深淵者,不敢墮也。手如握虎者,欲其壯也。神無營於眾物者,靜志觀病人,無左右視也;義無邪下者,欲端以正也;必正其神者,欲瞻病人目制其神,令氣易行也。所謂三里者,下膝三寸也;所謂跗之者,舉膝分易見也;巨虛者,蹻足(骨行)獨陷者;下廉者,陷下者也。
%%   帝曰:余聞九針,上應天地四時陰陽,願聞其方,令可傳於後世以為常也。岐伯曰:夫一天、二地、三人、四時、五音、六律、七星、八風、九野,身形亦應之,針各有所宜,故曰九針。人皮應天,人肉應地,人脈應人,人筋應時,人聲應音,人陰陽合氣應律,人齒面目應星,人出入氣應風,人九竅三百六十五絡應野,故一針皮,二針肉,三針脈,四針筋,五針骨,六針調陰陽,七針益精,八針除風,九針通九竅,除三百六十五節氣,此之謂各有所主也。人心意應八風,人氣應天,人發齒耳目五聲應五音六律,人陰陽脈血氣應地,人肝目應之九。九竅三百六十五。人一以觀動靜天二以候五色七星應之,以候發毋澤五音一,以候宮商角徵羽六律有餘,不足應之二地一,以候高下有餘九野一節俞應之,以候閉節,三人變一分人,候齒洩多血少十分角之變,五分以候緩急,六分不足三分寒關節第九,分四時人寒溫燥濕四時,一應之以候相反,一四方各作解。


%% \section{長刺節論篇第五十五}

%%   刺家不診,聽病者言,在頭,頭疾痛,為藏針之,刺至骨,病已上,無傷骨肉及皮,皮者道也。
%%   陰刺,入一傍四處。治寒熱。深專者,刺大藏,迫藏刺背,背俞也。刺之迫藏,藏會,腹中寒熱去而止。與刺之要,髮針而淺出血。
%%   治腐腫者刺腐上,視癰小大深淺刺,刺大者多血,小者深之,必端內針為故止。
%%   病在少腹有積,刺皮(骨盾)以下,至少腹而止;刺俠脊兩傍四椎間,刺兩髂季脅肋間,導腹中氣熱下已。
%%   病在少腹,腹痛不得大小便,病名曰疝,得之寒;刺少腹兩股間,刺腰髁骨間,刺而多之,盡炅病已。
%%   病在筋,筋攣節痛,不可以行,名曰筋痹。刺筋上為故,刺分肉間,不可中骨也;病起筋炅,病已止。
%%   病在肌膚,肌膚盡痛,名曰肌痹,傷於寒濕。刺大分、小分,多髮針而深之,以熱為故;無傷筋骨,傷筋骨,癰發若變;諸分盡熱,病已止。
%%   病在骨,骨重不可舉,骨髓痠痛,寒氣至,名曰骨痹。深者刺,無傷脈肉為故,其道大分小分,骨熱病已止。
%%   病在諸陽脈,且寒且熱,諸分且寒且熱,名曰狂。刺之虛脈,視分盡熱,病已止。
%%   病初發,歲一發,不治月一發,不治,月四五發,名曰癲病。刺諸分諸脈,其無寒者以針調之,病已止。
%%   病風且寒且熱,炅汗出,一日數過,先刺諸分理絡脈;汗出且寒且熱,三日一刺,百日而已。
%%   病大風,骨節重,鬚眉墮,名曰大風,刺肌肉為故,汗出百日,刺骨髓,汗出百日,凡二百日,鬚眉生而止針。


%% \section{皮部論篇第五十六}

%%   黃帝問曰:余聞皮有分部,脈有經紀,筋有結絡,骨有度量。其所生病各異,別其分部,左右上下,陰陽所在,病之始終,願聞其道。
%%   岐伯對曰:欲知皮部以經脈為紀者,諸經皆然。陽明之陽,名曰害蜚,上下同法。視其部中有浮絡者,皆陽明之絡也。其色多青則痛,多黑則痹,黃赤則熱,多白則寒,五色皆見,則寒熱也。絡盛則入客於經,陽主外,陰主內。
%%   少陽之陽,名曰樞持,上下同法。視其部中有浮絡者,皆少陽之絡也,絡盛則入客於經,故在陽者主內,在陰者主出,以滲於內,諸經皆然。
%%   太陽之陽,名曰關樞,上下同法。視其部中有浮絡者,皆太陽之絡也。絡盛則入客於經。
%%   少陰之陰,名曰樞儒,上下同法。視其部中有浮絡者,皆少陰之絡也。絡盛則入客於經,其入經也,從陽部注於經;其出者,從陰內注於骨。
%%   心主之陰,名曰害肩,上下同法。視其部中有浮絡者,皆心主之絡也。絡盛則入客於經。
%%   太陰之陰,名曰關蟄,上下同法。視其部中有浮絡者,皆太陰之絡也。絡盛則入客於經。凡十二經絡脈者,皮之部也。
%%   是故百病之始生也,必先於皮毛,邪中之則腠理開,開則入客於絡脈,留而不去,傳入於經,留而不去,傳入於府,廩於腸胃。邪之始入於皮毛也,晰然起毫毛,開腠理;其入於絡也,則絡脈盛色變;其入客於經也,則感虛乃陷下。其留於筋骨之間,寒多則筋攣骨痛,熱多則筋弛骨消,肉爍(月囷)破,毛直而敗。
%%   帝曰:夫子言皮之十二部,其生病皆何如?岐伯曰:皮者脈之部也,邪客於皮則腠理開,開則邪入客於絡脈,絡脈滿則注於經脈,經脈滿則入舍於府藏也,故皮者有分部,不與而生大病也。


%% \section{經絡論篇第五十七}

%%   黃帝問曰:夫絡脈之見也,其五色各異,青黃赤白黑不同,其故何也?岐伯對曰:經有常色而絡無常變也。
%%   帝曰:經之常色何如?岐伯曰:心赤,肺白、肝青、脾黃、腎黑,皆亦應其經脈之色也。
%%   帝曰:絡之陰陽,亦應其經乎?岐伯曰:陰絡之色應其經,陽絡之色變無常,隨四時而行也。寒多則凝泣,凝泣則青黑;熱多則淖澤,淖澤則黃赤;此皆常色,謂之無病,五色具見者,謂之寒熱。帝曰:善。


%% \section{氣穴論篇第五十八}

%%   黃帝問曰:余聞氣穴三百六十五,以應一歲,未知其所,願卒聞之。岐伯稽首再拜對曰:窘乎哉問也!其非聖帝,孰能窮其道焉!因請溢意盡言其處。帝捧手逡巡而卻曰:夫子之開余道也,目未見其處,耳未聞其數,而目以明,耳以聰矣。岐伯曰:此所謂聖人易語,良馬易御也。帝曰:余非聖人之易語也,世言真數開人意,今余所訪問者真數,發蒙解惑,未足以論也。然余願聞夫子溢志盡言其處,令解其意,請藏之金匱,不敢復出。
%%   岐伯再拜而起曰:臣請言之,背與心相控而痛,所治天突與十椎及上紀,上紀者,胃脘也,下紀者,關元也。背胸邪系陰陽左右,如此其病前後痛濇,胸脅痛而不得息,不得臥,上氣短氣偏痛,脈滿起,斜出尻脈,絡胸脅支心貫鬲,上肩加天突,斜下肩交十椎下。
%%   藏俞五十穴,府俞七十二穴,熱俞五十九穴,水俞五十七穴,頭上五行行五,五五二十五穴,中兩傍各五,凡十穴,大椎上兩傍各一,凡二穴,目瞳子浮白二穴,兩髀厭分中二穴,犢鼻二穴,耳中多所聞二穴,眉本二穴,完骨二穴,頂中央一穴,枕骨二穴,上關二穴,大迎二穴,下關二穴,天柱二穴,巨虛上下廉四穴,曲牙二穴,天突一穴,天府二穴,天牖二穴,扶突二穴,天窗二穴,肩解二穴,關元一穴,委陽二穴,肩貞二穴,瘖門一穴,齊一穴,胸俞十二穴,背俞二穴,膺俞十二穴,分肉二穴,踝上橫二穴,陰陽蹻四穴,水俞在諸分,熱俞在氣穴,寒熱俞在兩骸厭中二穴,大禁二十五,在天府下五寸,凡三百六十五穴,針之所由行也。
%%   帝曰:余已知氣穴之處,游針之居,願聞孫絡谿谷,亦有所應乎?岐伯曰:孫絡三百六十五穴會,亦以應一歲,以溢奇邪,以通榮衛,榮衛稽留,衛散榮溢,氣竭血著,外為發熱,內為少氣,疾寫無怠,以通榮衛,見而寫之,無問所會。
%%   帝曰:善。願聞谿谷之會也。岐伯曰:肉之大會為谷,肉之小會為谿,肉分之間,谿谷之會,以行榮衛,以會大氣。邪溢氣壅,脈熱肉敗榮衛不行,必將為膿,內銷骨髓,外破大膕,留於節湊,必將為敗。積寒留舍,榮衛不居,卷肉縮筋,肋肘不得伸,內為骨痹,外為不仁,命曰不足,大寒留於谿谷也。谿谷三百六十五穴會,亦應一歲,其小痹淫溢,循脈往來,微針所及,與法相同。
%%   帝乃辟左右而起,再拜曰:今日發蒙解惑,藏之金匱,不敢復出,乃藏之金蘭之室,署曰氣穴所在。岐伯曰:孫絡之脈別經者,其血盛而當寫者,亦三百六十五脈,並注於絡,傳注十二絡脈,非獨十四絡脈也,內解寫於中者十脈。

%% \section{氣府論篇第五十九}

%%   足太陽脈氣所發者七十八穴:兩眉頭各一,入發至項三寸半,傍五,相去三寸,其浮氣在皮中者凡五行,行五,五五二十五,項中大筋兩傍各一,風府兩傍各一,俠背以下至尻尾二十一節,十五間各一,五藏之俞各五,六府之俞各六,委中以下至足小指傍各六俞。
%%   足少陽脈氣所發者六十二穴:兩角上各二,直目上髮際內各五,耳前角上各一,耳前角下各一,銳發下各一,客主人各一,耳後陷中各一,下關各一,耳下牙車之後各一,缺盆各一,掖下三寸,脅下至胠,八間各一,髀樞中傍各一,膝以下至足小指次指各六俞。
%%   足陽明脈氣所發者六十八穴:額顱髮際傍各三,面鼽骨空各一,大迎之骨空各一,人迎各一,缺盆外骨空各一,膺中骨間各一,俠鳩尾之外,當乳下三寸,俠胃脘各五,俠齊廣三寸各三,下齊二寸俠之各三。氣街動脈各一,伏菟上各一,三里以下至足中指各八俞,分之所在穴空。
%%   手太陽脈氣所發者三十六穴:目內眥各一,目外眥各一,鼽骨下各一,耳郭上各一,耳中各一,巨骨穴各一,曲掖上骨穴各一,柱骨上陷者各一,上天窗四寸各一,肩解各一,肩解下三寸各一,肘以下至手小指本各六俞。
%%   手陽明脈氣所發者二十二穴:鼻空外廉、項上各二,大迎骨空各一,柱骨之會各一,髃骨之會各一,肘以下至手大指次指本各六俞。
%%   手少陽脈氣所發者三十二穴:鼽骨下各一,眉後各一,角上各一,下完骨後各一,項中足太陽之前各一,俠扶突各一,肩貞各一,肩貞下三寸分間各一,肘以下至手小指次指本各六俞。
%%   督脈氣所發者二十八穴:項中央二,髮際後中八,面中三,大椎以下至尻尾及傍十五穴,至骶下凡二十一節,脊椎法也。
%%   任脈之氣所發者二十八穴:喉中央二,膺中骨陷中各一,鳩尾下三寸,胃脘五寸,胃脘以下至橫骨六寸半一,腹脈法也。下陰別一,目下各一,下唇一,齗交一。
%%   衝脈氣所發者二十二穴:俠鳩尾外各半寸至齊寸一,俠齊下傍各五分至橫骨寸一,腹脈法也。
%%   足少陰舌下,厥陰毛中急脈各一,手少陰各一,陰陽蹻各一,手足諸魚際脈氣所發者,凡三百六十五穴也。


%% \section{骨空論篇第六十}

%%   黃帝問曰:余聞風者百病之始也,以針治之奈何?岐伯對曰:風從外入,令人振寒,汗出頭痛,身重惡寒,治在風府,調其陰陽,不足則補,有餘則寫。
%%   大風頸項痛,刺風府,風府在上椎。大風汗出,灸譩譆,譩譆在背下俠脊傍三寸所,厭之,令病者呼譩譆,譩譆應手。
%%   從風憎風,刺眉頭。失枕,在肩上橫骨間。折,使榆臂,齊肘正,灸脊中。
%%   絡季脅引少腹而痛脹,刺譩譆。
%%   腰痛不可以轉搖,急引陰卵,刺八髎與痛上,八髎在腰尻分間。
%%   鼠瘻,寒熱,還刺寒府,寒府在附膝外解營。取膝上外者使之拜,取足心者使之跪。
%%   任脈者,起於中極之下,以上毛際,循腹裡上關元,至咽喉,上頤循面入目。衝脈者,起於氣街,並少陰之經,俠齊上行,至胸中而散。任脈為病,男子內結七疝,女子帶下瘕聚。衝脈為病,逆氣裡急。
%%   督脈為病,脊強反折。督脈者,起於少腹以下骨中央,女子入系廷孔,其孔,溺孔之端也。其絡循陰器合篡間,繞篡後,別繞臀,至少陰與巨陽中絡者合,少陰上股內後廉,貫脊屬腎,與太陽起於目內眥,上額交巔,上入絡腦,還出別下項,循肩髆,內俠脊抵腰中,入循膂絡腎。其男子循莖下至篡,與女子等。其少腹直上者,貫齊中央,上貫心入喉,上頤環唇,上繫兩目之下中央。此生病,從少腹上衝心而痛,不得前後,為沖疝;其女子不孕,癃痔遺溺嗌干。督脈生病治督脈,治在骨上,甚者在齊下營。
%%   其上氣有音者,治其喉中央,在缺盆中者,其病上衝喉者治其漸,漸者,上俠頤也。
%%   蹇,膝伸不屈,治其楗。坐而膝痛,治其機。立而暑解,治其骸關。膝痛,痛及拇指治其膕。坐而膝痛如物隱者,治其關。膝痛不可屈伸,治其背內。連(骨行)若折,治陽明中俞髎。若別,治巨陽少陰滎。淫濼脛痠,不能久立,治少陽之維,在外上五寸。
%%   輔骨上,橫骨下為楗,俠髖為機,膝解為骸關,俠膝之骨為連骸,骸下為輔,輔上為膕,膕上為關,頭橫骨為枕。
%%   水俞五十七穴者,尻上五行,行五;伏菟上兩行,行五,左右各一行,行五;踝上各一行,行六穴,髓空在腦後三分,在顱際銳骨之下,一在齗基下,一在項後中復骨下,一在脊骨上空在風府上。脊骨下空,在尻骨下空。數髓空在面俠鼻,或骨空在口下當兩肩。兩髆骨空,在髆中之陽。臂骨空在臂陽,去踝四寸兩骨空之間。股骨上空在股陽,出上膝四寸。(骨行)骨空在輔骨之上端,股際骨空在毛中動下。尻骨空在髀骨之後,相去四寸。扁骨有滲理湊,無髓孔,易髓無孔。
%%   灸寒熱之法,先灸項大椎,以年為壯數,次灸橛骨,以年為壯數,視背俞陷者灸之,舉臂肩上陷者灸之,兩季脅之間灸之,外踝上絕骨之端灸之,足小指次指間灸之,腨下陷脈灸之,外踝後灸之,缺盆骨上切之堅痛如筋者灸之,膺中陷骨間灸之,掌束骨下灸之,齊下關元三寸灸之,毛際動脈灸之,膝下三寸分間灸之,足陽明跗上動脈灸之,巔上一灸之。犬所齧之處灸之三壯,即以犬傷病法灸之。凡當灸二十九處,傷食灸之,不已者,必視其經之過於陽者,數刺其俞而藥之。


%% \section{水熱穴論篇第六十一}

%%   黃帝問曰:少陰何以主腎?腎何以主水?岐伯對曰:腎者,至陰也,至陰者,盛水也。肺者,太陰也,少陰者,冬脈也,故其本在腎,其末在肺,皆積水也。
%%   帝曰:腎何以能聚水而生病?岐伯曰:腎者,胃之關也,關門不利,故聚水而從其類也。上下溢於皮膚,故為胕腫,胕腫者,聚水而生病也。
%%   帝曰:諸水皆生於腎乎?岐伯曰:腎者,牝藏也,地氣上者屬於腎,而生水液也,故曰至陰。勇而勞甚則腎汗出,腎汗出逢於風,內不得入於藏府,外不得越於皮膚,客於玄府,行於皮裡,傳為胕腫,本之於腎,名曰風水。所謂玄府者,汗空也。
%%   帝曰:水俞五十七處者,是何主也?岐伯曰:腎俞五十七穴,積陰之所聚也,水所從出入也。尻上五行行五者,此腎俞,故水病下為胕腫大腹,上為喘呼,不得臥者,標本俱病,故肺為喘呼,腎為水腫,肺為逆不得臥,分為相輸俱受者,水氣之所留也。伏菟上各二行行五者,此腎之街也,三陰之所交結於腳也。踝上各一行行六者,此腎脈之下行也,名曰太沖。凡五十七穴者,皆藏之陰絡,水之所客也。
%%   帝曰:春取絡脈分肉,何也?岐伯曰:春者木始治,肝氣始生,肝氣急,其風疾,經脈常深,其氣少,不能深入,故取絡脈分肉間。
%%   帝曰:夏取盛經分腠,何也?岐伯曰:夏者火始治,心氣始長,脈瘦氣弱,陽氣留溢,熱熏分腠,內至於經,故取盛經分腠,絕膚而病去者,邪居淺也。所謂盛經者,陽脈也。
%%   帝曰:秋取經俞,何也?岐伯曰:秋者金始治,肺將收殺,金將勝火,陽氣在合,陰氣初勝,濕氣及體,陰氣未盛,未能深入,故取俞以寫陰邪,取合以虛陽邪,陽氣始衰,故取於合。
%%   帝曰:冬取井榮,何也?岐伯曰:冬者水始治,腎方閉,陽氣衰少,陰氣堅盛,巨陽伏沉,陽脈乃去,故取井以下陰逆,取榮以陽氣。故曰:冬取井榮,春不鼽衄,此之謂也。
%%   帝曰:夫子言治熱病五十九俞,余論其意,未能領別其處,願聞其處,因聞其意。岐伯曰:頭上五行行五者,以越諸陽之熱逆也;大杼、膺俞、缺盆、背俞,此八者,以寫胸中之熱也;氣街、三里、巨虛上下廉,此八者,以寫胃中之熱也;雲門、髃骨、委中、髓空,此八者,以寫四支之熱也;五藏俞傍五,此十者,以寫五藏之熱也。凡此五十九穴者,皆熱之左右也。
%%   帝曰:人傷於寒而傳為熱,何也?岐伯曰:夫寒盛,則生熱也。

%% \section{調經論篇第六十二}

%%   黃帝問曰:余聞刺法言,有餘寫之,不足補之,何謂有餘?何謂不足?岐伯對曰:有餘有五,不足亦有五,帝欲何問?帝曰:願盡聞之。岐伯曰:神有餘有不足,氣有餘有不足,血有餘有不足,形有餘有不足,志有餘有不足,凡此十者,其氣不等也。
%%   帝曰:人有精氣津液,四支、九竅、五藏十六部、三百六十五節,乃生百病,百病之生,皆有虛實。今夫子乃言有餘有五,不足亦有五,何以生之乎?岐伯曰:皆生於五藏也。夫心藏神,肺藏氣,肝藏血,脾藏肉,腎藏志,而此成形。志意通,內連骨髓,而成身形五藏。五藏之道,皆出於經隧,以行血氣,血氣不和,百病乃變化而生,是故守經隧焉。
%%   帝曰:神有餘不足何如?岐伯曰:神有餘則笑不休,神不足則悲。血氣未並,五藏安定,邪客於形,灑淅起於毫毛,未入於經絡也,故命曰神之微。帝曰:補寫奈何?岐伯曰:神有餘,則寫其小絡之血,出血勿之深斥,無中其大經,神氣乃平。神不足者,視其虛絡,按而致之,刺而利之,無出其血,無洩其氣,以通其經,神氣乃平。帝曰:刺微奈何?岐伯曰:按摩勿釋,著針勿斥,移氣於不足,神氣乃得復。
%%   帝曰:善。有餘不足奈何?岐伯曰:氣有餘則喘咳上氣,不足則息利少氣。血氣未並,五藏安定,皮膚微病,命曰白氣微洩。帝曰:補寫奈何?岐伯曰:氣有餘,則寫其經隧,無傷其經,無出其血,無洩其氣。不足,則補其經隧,無出其氣。帝曰:刺微奈何?岐伯曰:按摩勿釋,出針視之,曰我將深之,適人必革,精氣自伏,邪氣散亂,無所休息,氣洩腠理,真氣乃相得。
%%   帝曰:善。血有餘不足奈何?岐伯曰:血有餘則怒,不足則恐。血氣未並,五藏安定,孫絡水溢,則經有留血。帝曰:補寫奈何?岐伯曰:血有餘,則寫其盛經出其血。不足,則視其虛經內針其脈中,久留而視;脈大,疾出其針,無令血洩。帝曰:刺留血,奈何?岐伯曰:視其血絡,刺出其血,無令惡血得入於經,以成其疾。
%%   帝曰:善。形有餘不足奈何?岐伯曰:形有餘則腹脹、涇溲不利,不足則四支不用。血氣未並,五藏安定,肌肉蠕動,命曰微風。帝曰:補寫奈何?岐伯曰:形有餘則寫其陽經,不足則補其陽絡。帝曰:刺微奈何?岐伯曰:取分肉間,無中其經,無傷其絡,衛氣得復,邪氣乃索。
%%   帝曰:善。志有餘不足奈何?岐伯曰:志有餘則腹脹飧洩,不足則厥。血氣未並,五藏安定,骨節有動。帝曰:補寫奈何?岐伯曰:志有餘則寫然筋血者,不足則補其復溜。帝曰:刺未並奈何?岐伯曰:即取之,無中其經,邪所乃能立虛。
%%   帝曰:善。余已聞虛之形,不知其何以生!岐伯曰:氣血以並,陰陽相頃,氣亂於衛,血逆於經,血氣離居,一實一虛。血並於陰,氣並於陽,故為驚狂;血並於陽,氣並於陰,乃為炅中;血並於上,氣並於下,心煩惋善怒;血並於下,氣並於上,亂而喜忘。帝曰:血並於陰,氣並於陽,如是血氣離居,何者為實?何者為虛?岐伯曰:血氣者,喜溫而惡寒,寒則泣不能流,溫則消而去之,是故氣之所並為血虛,血之所並為氣虛。
%%   帝曰:人之所有者,血與氣耳。今夫子乃言血並為虛,氣並為虛,是無實乎?岐伯曰:有者為實,無者為虛,故氣並則無血,血並則無氣,今血與氣相失,故為虛焉。絡之與孫脈俱輸於經,血與氣並,則為實焉。血之與氣並走於上,則為大厥,厥則暴死,氣復反則生,不反則死。
%%   帝曰:實者何道從來?虛者何道從去?虛實之要,願聞其故。岐伯曰:夫陰與陽,皆有俞會,陽注於陰,陰滿之外,陰陽勻平,以充其形,九候若一,命曰平人。夫邪之生也,或生於陰,或生於陽。其生於陽者,得之風雨寒暑;其生於陰者,得之飲食居處,陰陽喜怒。
%%   帝曰:風雨之傷人奈何?岐伯曰:風雨之傷人也,先客於皮膚,傳入於孫脈,孫脈滿則傳入於絡脈,絡脈滿則輸於大經脈,血氣與邪並客於分腠之間,其脈堅大,故曰實。實者外堅充滿,不可按之,按之則痛。帝曰:寒濕之傷人奈何?岐伯曰:寒濕之中人也,皮膚不收,肌肉堅緊,榮血泣,衛氣去,故曰虛。虛者聶辟,氣不足,按之則氣足以溫之,故快然而不痛。
%%   帝曰:善。陰之生實奈何?岐伯曰:喜怒不節,則陰氣上逆,上逆則下虛,下虛則陽氣走之,故曰實矣。帝曰:陰之生虛奈何?岐伯曰:喜則氣下,悲則氣消,消則脈虛空,因寒飲食,寒氣熏滿,則血泣氣去,故曰虛矣。
%%   帝曰:經言陽虛則外寒,陰虛則內熱,陽盛則外熱,陰盛則內寒,余已聞之矣,不知其所由然也。岐伯曰:陽受氣於上焦,以溫皮膚分肉之間。令寒氣在外,則上焦不通,上焦不通,則寒氣獨留於外,故寒慄。帝曰:陰虛生內熱奈何?岐伯曰:有所勞倦,形氣衰少,谷氣不盛,上焦不行,下脘不通,胃氣熱,熱氣熏胸中,故內熱。帝曰:陽盛生外熱奈何?岐伯曰:上焦不通利,則皮膚緻密,腠理閉塞,玄府不通,衛氣不得洩越,故外熱。帝曰:陰盛生內寒奈何?岐伯曰:厥氣上逆,寒氣積於胸中而不寫,不寫則溫氣去,寒獨留,則血凝泣,凝則脈不通,其脈盛大以濇,故中寒。
%%   帝曰:陰與陽並,血氣以並,病形以成,刺之奈何?岐伯曰:刺此者,取之經隧,取血於營,取氣於衛,用形哉,因四時多少高下。帝曰:血氣以並,病形以成,陰陽相頃,補寫奈何?岐伯曰:寫實者氣盛乃內針,針與氣俱內,以開其門,如利其戶;針與氣俱出,精氣不傷,邪氣乃下,外門不閉,以出其疾;搖大其道,如利其路,是謂大寫,必切而出,大氣乃屈。帝曰:補虛奈何?岐伯曰:持針勿置,以定其意,候呼內針,氣出針入,針空四塞,精無從去,方實而疾出針,氣入針出,熱不得還,閉塞其門,邪氣布散,精氣乃得存,動氣候時,近氣不失,遠氣乃來,是謂追之。
%%   帝曰:夫子言虛實者有十,生於五藏,五藏五脈耳。夫十二經脈皆生其病,今夫子獨言五藏,夫十二經脈者,皆絡三百六十五節,節有病必被經脈,經脈之病,皆有虛實,何以合之?岐伯曰:五藏者,故得六府與為表裡,經絡支節,各生虛實,其病所居,隨而調之。病在脈,調之血;病在血,調之絡;病在氣,調之衛;病在肉,調之分肉;病在筋,調之筋;病在骨,調之骨;燔針劫刺其下及與急者;病在骨,焠針藥熨;病不知所痛,兩蹻為上;身形有痛,九候莫病,則繆刺之;痛在於左而右脈病者,巨刺之。必謹察其九候,針道備矣。


%% \section{繆刺論篇第六十三}

%%   黃帝問曰:余聞繆刺,未得其意,何謂繆刺?岐伯對曰:夫邪之客於形也,必先舍於皮毛,留而不去,入舍於孫脈,留而不去,入舍於絡脈,留而不去,入舍於經脈,內連五藏,散於腸胃,陰陽俱感,五藏乃傷,此邪之從皮毛而入,極於五藏之次也,如此則治其經焉。今邪客於皮毛,入舍於孫絡,留而不去,閉塞不通,不得入於經,流溢於大絡,而生奇病也。夫邪客大絡者,左注右,右注左,上下左右,與經相干,而佈於四末,其氣無常處,不入於經俞,命曰繆刺。
%%   帝曰:願聞繆刺,以左取右以右取左,奈何?其與巨刺何以別之?岐伯曰:邪客於經,左盛則右病,右盛則左病,亦有移易者,左痛未已而右脈先病,如此者,必巨刺之,必中其經,非絡脈也。故絡病者,其痛與經脈繆處,故命曰繆刺。
%%   帝曰:願聞繆刺奈何?取之何如?岐伯曰:邪客於足少陰之絡,令人卒心痛,暴脹,胸脅支滿,無積者,刺然骨之前出血,如食頃而已。不已,左取右,右取左。病新發者,取五日,已。
%%   邪客於手少陽之絡,令人喉痹舌卷,口乾心煩,臂外廉痛,手不及頭,刺手中指次指爪甲上,去端如韭葉各一痏,壯者立已,老者有頃已,左取右,右取左,此新病數日已。
%%   邪客於足厥陰之絡,令人卒疝暴痛,刺足大指爪甲上,與肉交者各一痏,男子立已,女子有頃已,左取右,右取左。
%%   邪客於足太陽之絡,令人頭項肩痛,刺足小指爪甲上,與肉交者各一痏,立已,不已,刺外踝下三痏,左取右,右取左,如食頃已。
%%   邪客於手陽明之絡,令人氣滿胸中,喘息而支胠,胸中熱,刺手大指、次指爪甲上,去端如韭葉各一痏,左取右,右取左,如食頃已。
%%   邪客於臂掌之間,不可得屈,刺其踝後,先以指按之痛,乃刺之,以月死生為數,月生一日一痏,二日二痏,十五日十五痏,十六日十四痏。
%%   邪客於足陽蹻之脈,令人目痛從內眥始,刺外踝之下半寸所各二痏,左刺右,右刺左,如行十里頃而已。
%%   人有所墮墜,惡血留內,腹中滿脹,不得前後,先飲利藥,此上傷厥陰之脈,下傷少陰之絡,刺足內踝之下,然骨之前,血脈出血,刺足跗上動脈,不已,刺三毛上各一痏,見血立已,左刺右,右刺左。善悲驚不樂,刺如右方。
%%   邪客於手陽明之絡,令人耳聾,時不聞音,刺手大指次指爪甲上,去端如韭葉各一痏,立聞,不已,刺中指爪甲上與肉交者,立聞,其不時聞者,不可刺也。耳中生風者,亦刺之如此數,左刺右,右刺左。
%%   凡痹往來行無常處者,在分肉間痛而刺之,以月死生為數,用針者隨氣盛衰,以為痏數,針過其日數則脫氣,不及日數則氣不寫,左刺右,右刺左,病已,止,不已,復刺之如法,月生一日一痏,二日二痏,漸多之;十五日十五痏,十六日十四,漸少之。
%%   邪客於足陽明之經,令人鼽衄上齒寒,足中指次指爪甲上,與肉交者各一痏,左刺右,右刺左。
%%   邪客於足少陽之絡,令人脅痛不得息,咳而汗出,刺足小指次指爪甲上,與肉交者各一痏,不得息立已,汗出立止,咳者溫衣飲食,一日已。左刺右,右刺左,病立已,不已,復刺如法。
%%   邪客於足少陰之絡,令人嗌痛,不可內食,無故善怒,氣上走賁上,刺足下中央之脈各三痏,凡六刺,立已,左刺右,右刺左。嗌中腫,不能內唾,時不能出唾者,刺然骨之前,出血立已,左刺右,右刺左。
%%   邪客於足太陰之絡,令人腰痛,引少腹控(月少),不可以仰息,刺腰尻之解,兩胂之上,是腰俞,以月死生為痏數,髮針立已,左刺右,右刺左。
%%   邪客於足太陽之絡,令人拘攣背急,引脅而痛,刺之從項始,數脊椎俠脊,疾按之應手如痛,刺之傍三痏,立已。
%%   邪客於足少陽之絡,令人留於樞中痛,髀不可舉,刺樞中以毫針,寒則久留針,以月死生為數,立已。
%%   治諸經刺之,所過者不病,則繆刺之。
%%   耳聾,刺手陽明,不已,刺其通脈出耳前者。
%%   齒齲,刺手陽明,不已,刺其脈入齒中,立已。
%%   邪客於五藏之間,其病也,脈引而痛,時來時止,視其病,繆刺之於手足爪甲上,視其脈,出其血,間日一刺,一刺不已,五刺已。
%%   繆傳引上齒,齒唇寒痛,視其手背脈血者去之,足陽明中指爪甲上一痏,手大指次指爪甲上各一痏,立已,左取右,右取左。
%%   邪客於手足少陰太陰足陽明之絡,此五絡,皆會於耳中,上絡左角,五絡俱竭,令人身脈皆動,而形無知也,其狀若屍,或曰屍厥,刺其足大指內側爪甲上,去端如韭葉,後刺足心,後刺足中指爪甲上各一痏,後刺手大指內側,去端如韭葉,後刺手心主,少陰銳骨之端各一痏,立已。不已,以竹管吹其兩耳,鬄其左角之發方一寸,燔治,飲以美酒一杯,不能飲者灌之,立已。
%%   凡刺之數,先視其經脈,切而從之,審其虛而調之,不調者經刺之,有痛而經不病者繆刺之,因視其皮部有血絡者盡取之,此繆刺之數也。


%% \section{四時刺逆從論篇第六十四}

%%   厥陰有餘,病陰痹;不足病生熱痹;滑則病狐疝風;濇則病少腹積氣。
%%   少陰有餘,病皮痹隱軫;不足病肺痹;滑則病肺風疝;濇則病積溲血。
%%   太陰有餘,病肉痹寒中;不足病脾痹;滑則病脾風疝;濇則病積心腹時滿。
%%   陽明有餘,病脈痹,身時熱;不足病心痹;滑則病心風疝;濇則病積時善驚 。
%%   太陽有餘,病骨痹身重;不足病腎痹;滑則病腎風疝;濇則病積時善巔疾。
%%   少陽有餘,病筋痹脅滿;不足病肝痹;滑則病肝風疝;濇則病積時筋急目痛。
%%   是故春氣在經脈,夏氣在孫絡,長夏氣在肌肉,秋氣在皮膚,冬氣在骨髓中。帝曰:余願聞其故。岐伯曰:春者,天氣始開,地氣始洩,凍解冰釋,水行經通,故人氣在脈。夏者,經滿氣溢,入孫絡受血,皮膚充實。長夏者,經絡皆盛,內溢肌中。秋者,天氣始收,腠理閉塞,皮膚引急。冬者蓋藏,血氣在中,內著骨髓,通於五藏。是故邪氣者,常隨四時之氣血而入客也,至其變化不可為度,然必從其經氣,辟除其邪,除其邪,則亂氣不生。
%%   帝曰:逆四時而生亂氣奈何?岐伯曰:春刺絡脈,血氣外溢,令人少氣;春刺肌肉,血氣環逆,令人上氣;春刺筋骨,血氣內著,令人腹脹。夏刺經脈,血氣乃竭,令人解(亻亦);夏刺肌肉,血氣內卻,令人善恐;夏刺筋骨,血氣上逆,令人善怒。秋刺經脈,血氣上逆,令人善忘;秋刺絡脈,氣不外行,令人臥不欲動;秋刺筋骨,血氣內散,令人寒慄。冬刺經脈,血氣皆脫,令人目不明;冬刺絡脈,內氣外洩,留為大痹;冬刺肌肉,陽氣竭絕,令人善忘。凡此四時刺者,大逆之病,不可不從也,反之,則生亂氣相淫病焉。故刺不知四時之經,病之所生,以從為逆,正氣內亂,與精相薄。必審九候,正氣不亂,精氣不轉。
%%   帝曰:善。刺五藏,中心一日死,其動為噫;中肝五日死,其動為語;中肺三日死,其動為咳;中腎六日死,其動為嚏欠;中脾十日死,其動為吞。刺傷人五藏必死,其動則依其藏之所變候知其死也。


%% \section{標本病傳論篇第六十五}

%%   黃帝問曰:病有標本,刺有逆從,奈何?岐伯對曰:凡刺之方,必別陰陽,前後相應,逆從得施,標本相移。故曰:有其在標而求之於標,有其在本而求之於本,有其在本而求之於標,有其在標而求之於本,故治有取標而得者,有取本而得者,有逆取而得者,有從取而得者。故知逆與從,正行無問,知標本者,萬舉萬當,不知標本,是謂妄行。
%%   夫陰陽逆從,標本之為道也,小而大,言一而知百病之害。少而多,淺而博,可以言一而知百也。以淺而知深,察近而知遠,言標與本,易而勿及。治反為逆,治得為從。先病而後逆者治其本,先逆而後病者治其本,先寒而後生病者治其本,先病而後生寒者治其本,先熱而後生病者治其本,先熱而後生中滿者治其標,先病而後洩者治其本,先洩而後生他病者治其本,必且調之,乃治其他病,先病而後生中滿者治其標,先中滿而後煩心者治其本。人有客氣,有同氣。小大不利治其標,小大利治其本。病發而有餘,本而標之,先治其本,後治其標;病發而不足,標而本之,先治其標,後治其本。謹察間甚,以意調之,間者並行,甚者獨行。先小大不利而後生病者治其本。
%%   夫病傳者,心病先心痛,一日而咳,三日脅支痛,五日閉塞不通,身痛體重;三日不已,死。冬夜半,夏日中。
%%   肺病喘咳,三日而脅支滿痛,一日身重體痛,五日而脹,十日不已,死。冬日入,夏日出。
%%   肝病頭目眩脅支滿,三日體重身痛,五日而脹,三日腰脊少腹痛脛,三日不已,死。冬日入,夏早食。
%%   脾病身痛體重,一日而脹,二日少腹腰脊痛脛酸,三日背(月呂)筋痛,小便閉,十日不已,死。冬人定,夏晏食。
%%   腎病少腹腰脊痛,(骨行)酸,三日背(月呂)筋痛,小便閉;三日腹脹;三日兩脅支痛,三日不已,死。冬大晨,夏晏晡。
%%   胃病脹滿,五日少腹腰脊痛,(骨行)酸;三日背(月呂)筋痛,小便閉;五日身體重;六日不已,死。冬夜半後,夏日昳。
%%   膀胱病小便閉,五日少腹脹,腰脊痛,(骨行)酸;一日腹脹;一日身體痛;二日不已,死。冬雞鳴,夏下晡。
%%   諸病以次是相傳,如是者,皆有死期,不可刺。間一藏止,及至三四藏者,乃可刺也。


%% \section{天元紀大論篇第六十六}

%%   黃帝問曰:天有五行,御五位,以生寒暑燥濕風;人有五藏,化五氣,以生喜怒思憂恐。論言五運相襲而皆治之,終期之日,週而復始,余已知之矣,願聞其與三陰三陽之候,奈何合之?
%%   鬼臾區稽首再拜對曰:昭乎哉問也。夫五運陰陽者,天地之道也,萬物之綱紀,變化之父母,生殺之本始,神明之府也,可不通乎!故物生謂之化,物極謂之變,陰陽不測謂之神,神用無方謂之聖。夫變化之為用也,在天為玄,在人為道,在地為化,化生五味,道生智,玄生神。神在天為風,在地為木;在天為熱,在地為火;在天為濕,在地為土;在天為燥,在地為金;在天為寒,在地為水;故在天為氣,在地成形,形氣相感而化生萬物矣。然天地者,萬物之上下也;左右者,陰陽之道路也;水火者,陰陽之徵兆也;金木者,生成之終始也。氣有多少,形有盛衰,上下相召,而損益彰矣。
%%   帝曰:願聞五運之主時也何如?鬼臾區曰:五氣運行,各終期日,非獨主時也。帝曰:請聞其所謂也。鬼臾區曰:臣積考《太始天元冊》文曰:太虛寥廓,肇基化元,萬物資始,五運終天,布氣真靈,總統坤元,九星懸朗,七曜周旋,曰陰曰陽,曰柔曰剛,幽顯既位,寒暑弛張,生生化化,品物咸章。臣斯十世,此之謂也。
%%   帝曰:善。何謂氣有多少,形有盛衰?鬼臾區曰:陰陽之氣各有多少,故曰三陰三陽也。形有盛衰,謂五行之治,各有太過不及也。故其始也,有餘而往,不足隨之,不足而往,有餘從之,知迎知隨,氣可與期。應天為天符,承歲為歲直,三合為治。
%%   帝曰:上下相召奈何?鬼臾區曰:寒暑燥濕風火,天之陰陽也,三陰三陽上奉之。木火土金水火,地之陰陽也,生長化收藏下應之。天以陽生陰長,地以陽殺陰藏。天有陰陽,地亦有陰陽。木火土金水火,地之陰陽也,生長化收藏。故陽中有陰,陰中有陽。所以欲知天地之陰陽者,應天之氣,動而不息,故五歲而右遷,應地之氣,靜而守位,故六期而環會,動靜相召,上下相臨,陰陽相錯,而變由生也。
%%   帝曰:上下週紀,其有數乎?鬼臾區曰:天以六為節,地以五為制,周天氣者,六期為一備;終地紀者,五歲為一週。君火以明,相火以位,五六相合而七百二十氣為一紀,凡三十歲;千四百四十氣,凡六十歲而為一週,不及太過,斯皆見矣。
%%   帝曰:夫子之言,上終天氣,下畢地紀,可謂悉矣。余願聞而藏之,上以治民,下以治身,使百姓昭著,上下和親,德澤下流,子孫無憂,傳之後世,無有終時,可得聞乎?鬼臾區曰:至數之機,迫迮以微,其來可見,其往可追,敬之者昌,慢之者亡。無道行私,必得天殃,謹奉天道,請言真要。
%%   帝曰:善言始者,必會於終,善言近者,必知其遠,是則至數極而道不惑,所謂明矣,願夫子推而次之,令有條理,簡而不匱,久而不絕,易用難忘,為之綱紀,至數之要,願盡聞之。鬼臾區曰:昭乎哉問!明乎哉道!如鼓之應桴,響之應聲也。臣聞之:甲己之歲,土運統之;乙庚之歲,金運統之;丙辛之歲,水運統之;丁壬之歲,木運統之;戊癸之歲,火運統之。
%%   帝曰:其於三陰三陽,合之奈何?鬼臾區曰:子午之歲,上見少陰;丑未之歲,上見太陰;寅申之歲,上見少陽;卯酉之歲,上見陽明;辰戌之歲,上見太陽;巳亥之歲,上見厥陰。少陰,所謂標也,厥陰,所謂終也。厥陰之上,風氣主之;少陰之上,熱氣主之;太陰之上,濕氣主之;少陽之上,相火主之;陽明之上,燥氣主之;太陽之上,寒氣主之。所謂本也,是謂六元。帝曰:光乎哉道!明乎哉論!請著之玉版,藏之金匱,署曰《天元紀》。


%% \section{五運行大論篇第六十七}

%%   黃帝坐明堂,始正天綱,臨觀八極,考建五常,請天師而問之曰:論言天地之動靜,神明為之紀;陰陽之升降,寒暑彰其兆。余聞五運之數於夫子,夫子之所言,正五氣之各主歲爾,首甲定運,余因論之。鬼臾區曰:土主甲己,金主乙庚,水主丙辛,木主丁壬,火主戊癸。子午之上,少陰主之;丑未之上,太陰主之;寅申之上,少陽主之;卯酉之上,陽明主之;辰戌之上,太陽主之;巳亥之上,厥陰主之。不合陰陽,其故何也?
%%   岐伯曰:是明道也,此天地之陰陽也。夫數之可數者,人中之陰陽也,然所合,數之可得者也。夫陰陽者,數之可十,推之可百,數之可千,推之可萬。天地陰陽者,不以數推,以象之謂也。
%%   帝曰:願聞其所始也。岐伯曰:昭乎哉問也!臣覽《太始天元冊》文,丹天之氣,經於牛女戊分;黅天之氣,經於心尾已分;蒼天之氣,經於危室柳鬼;素天之氣,經於亢氐昴畢;玄天之氣,經於張翼婁胃。所謂戊己分者,奎璧角軫,則天地之門戶也。夫候之所始,道之所生,不可不通也。
%%   帝曰:善。論言天地者,萬物之上下,左右者,陰陽之道路,未知其所謂也。岐伯曰:所謂上下者,歲上下見陰陽之所在也。左右者,諸上見厥陰,左少陰,右太陽;見少陰,左太陰,右厥陰;見太陰,左少陽,右少陰;見少陽,左陽明,右太陰;見陽明,左太陽,右少陽;見太陽,左厥陰,右陽明。所謂面北而命其位,言其見也。
%%   帝曰:何謂下?岐伯曰:厥陰在上,則少陽在下,左陽明右太陰。少陰在上則陽明在下,左太陽右少陽。太陰在上則太陽在下,左厥陰右陽明。少陽在上則厥陰在下,左少陰右太陽。陽明在上則少陰在下,左太陰右厥陰。太陽在上則太陰在下,左少陽右少陰。所謂面南而命其位,言其見也。上下相遘,寒暑相臨,氣相得則和,不相得則疾。帝曰:氣相得而病者,何也?岐伯曰:以下臨上,不當位也。帝曰:動靜何如?岐伯曰:上者右行,下者左行,左右周天,余而復會也。帝曰:余聞鬼臾區曰,應地者靜。今夫子乃言下者左行,不知其所謂也,願聞何以生之乎?岐伯曰:天地動靜,五行遷復,雖鬼臾區其上候而巳,猶不能遍明。夫變化之用,天垂象,地成形,七曜緯虛,五行麗地。地者,所以載生成之形類也。虛者,所以列應天之精氣也。形精之動,猶根本之與枝葉也,仰觀其象,雖遠可知也。
%%   帝曰:地之為下,否乎?岐伯曰:地為人之下,太虛之中者也。帝曰:馮乎?岐伯曰:大氣舉之也。燥以干之,暑以蒸之,風以動之,濕以潤之,寒以堅之,火以溫之。故風寒在下,燥熱在上,濕氣在中,火遊行其間,寒暑六入,故令虛而生化也。故燥勝則地幹,暑勝則地熱,風勝則地動,濕勝則地泥,寒勝則地裂,火勝則地固矣。
%%   帝曰:天地之氣,何以候之?岐伯曰:天地之氣,勝復之作,不形於診也。《脈法》曰:天地之變,無以脈診,此之謂也。
%%   帝曰:間氣何如?岐伯曰:隨氣所在,期於左右。帝曰:期之奈何?岐伯曰:從其氣則和,違其氣則病,不當其位者病,迭移其位者病,失守其位者危,尺寸反者死,陰陽交者死。先立其年,以知其氣,左右應見,然後乃可以言死生之逆順也。
%%   帝曰:寒暑燥濕風火,在人合之奈何?其於萬物何以生化?岐伯曰:東方生風,風生木,木生酸,酸生肝,肝生筋,筋生心。其在天為玄,在人為道,在地為化。化生五味,道生智,玄生神,化生氣。神在天為風,在地為木,在體為筋,在氣為柔,在藏為肝。其性為暄,其德為和,其用為動,其色為蒼,其化為榮,其蟲毛,其政為散,其令宣發,其變摧拉,其眚為隕,其味為酸,其志為怒。怒傷肝,悲勝怒;風傷肝,燥勝風;酸傷筋,辛勝酸。
%%   南方生熱,熱生火,火生苦,苦生心,心生血,血生脾。其在天為熱,在地為火,在體為脈,在氣為息,在藏為心。其性為暑,其德為顯,其用為躁,其色為赤,其化為茂,其蟲羽,其政為明,其令郁蒸,其變炎爍,其眚燔焫,其味為苦,其志為喜。喜傷心,恐勝喜;熱傷氣,寒勝熱;苦傷氣,咸勝苦。
%%   中央生濕,濕生土,土生甘,甘生脾,脾生肉,肉生肺。其在天為濕,在地為土,在體為肉,在氣為充,在藏為脾。其性靜兼,其德為濡,其用為化,其色為黃,其化為盈,其蟲裸,其政為謐,其令雲雨,其變動注,其眚淫潰,其味為甘,其志為思。思傷脾,怒勝思;濕傷肉,風勝濕;甘傷脾,酸勝甘。
%%   西方生燥,燥生金,金生辛,辛生肺,肺生皮毛,皮毛生腎。其在天為燥,在地為金,在體為皮毛,在氣為成,在藏為肺。其性為涼,其德為清,其用為固,其色為白,其化為斂,其蟲介,其政為勁,其令霧露,其變肅殺,其眚蒼落,其味為辛,其志為憂。憂傷肺,喜勝憂;,熱傷皮毛,寒勝熱;辛傷皮毛,苦勝辛。
%%   北方生寒,寒生水,水生咸,咸生腎,腎生骨髓,髓生肝。其在天為寒,在地為水,在體為骨,在氣為堅,在藏為腎。其性為凜,其德為寒,其用為藏,其色為黑,其化為肅,其蟲鱗,其政為靜,其令霰雪,其變凝冽,其眚冰雹,其味為咸,其志為恐。恐傷腎,思勝恐;寒傷血,燥勝寒;咸傷血,甘勝咸。正氣更立,各有所先,非其位則邪,當其位則正。
%%   帝曰:病生之變何如?岐伯曰:氣相得則微,不相得則甚。帝曰:主歲何如?岐伯曰:氣有餘,則制己所勝而侮所不勝;其不及,則己所不勝侮而乘之,己所勝輕而侮之。侮反受邪。侮而受邪,寡於畏也。帝曰:善。


%% \section{六微旨大論篇第六十八}

%%   黃帝問曰:嗚呼!遠哉,天之道也,如迎浮雲,若視深淵,視深淵尚可測,迎浮雲莫知其極。夫子數言謹奉天道,余聞而藏之,心私異之,不知其所謂也。願夫子溢志盡言其事,令終不滅,久而不絕,天之道可得聞乎?岐伯稽首再拜對曰:明乎哉問,天之道也!此因天之序,盛衰之時也。
%%   帝曰:願聞天道六六之節盛衰何也?岐伯曰:上下有位,左右有紀。故少陽之右,陽明治之;陽明之右,太陽治之;太陽之右,厥陰治之;厥陰之右,少陰治之;少陰之右,太陰治之;太陰之右,少陽治之。此所謂氣之標,蓋南面而待之也。故曰:因天之序,盛衰之時,移光定位,正立而待之,此之謂也。
%%   少陽之上,火氣治之,中見厥陰;陽明之上,燥氣治之,中見太陰;太陽之上,寒氣治之,中見少陰;厥陰之上,風氣治之,中見少陽;少陰之上,熱氣治之,中見太陽;太陰之上,濕氣治之,中見陽明。所謂本也,本之下,中之見也,見之下,氣之標也。本標不同,氣應異象。
%%   帝曰:其有至而至,有至而不至,有至而太過,何也?岐伯曰:至而至者和;至而不至,來氣不及也;未至而至,來氣有餘也。帝曰:至而不至,未至而至如何?岐伯曰:應則順,否則逆,逆則變生,變則病。帝曰:善。請言其應。岐伯曰:物,生其應也。氣,脈其應也。
%%   帝曰:善。願聞地理之應六節氣位何如?岐伯曰:顯明之右,君火之位也;君火之右,退行一步,相火治之;復行一步,土氣治之;復行一步,金氣治之;復行一步,水氣治之;復行一步,木氣治之;復行一步,君火治之。
%%   相火之下,水氣承之;水位之下,土氣承之;土位之下,風氣承之;風位之下,金氣承之;金位之下,火氣承之;君火之下,陰精承之。帝曰:何也?岐伯曰:亢則害,承乃制,制則生化,外列盛衰,害則敗亂,生化大病。
%%   帝曰:盛衰何如?岐伯曰:非其位則邪,當其位則正,邪則變甚,正則微。帝曰:何謂當位?岐伯曰:木運臨卯,火運臨午,土運臨四季,金運臨酉,水運臨子,所謂歲會,氣之平也。帝曰:非位何如?岐伯曰:歲不與會也。
%%   帝曰:土運之歲,上見太陰;火運之歲,上見少陽少陰;金運之歲,上見陽明;木運之歲,上見厥陰;水運之歲,上見太陽,奈何?岐伯曰:天之與會也。故《天元冊》曰天符。
%%   天符歲會何如?岐伯曰:太一天符之會也。
%%   帝曰:其貴賤何如?岐伯曰:天符為執法,歲位為行令,太一天符為貴人。帝曰:邪之中也奈何?岐伯曰:中執法者,其病速而危;中行令者,其病徐而持;中貴人者,其病暴而死。帝曰:位之易也何如?岐伯曰:君位臣則順,臣位君則逆,逆則其病近,其害速;順則其病遠,其害微。所謂二火也。
%%   帝曰:善。願聞其步何如?岐伯曰:所謂步者,六十度而有奇,故二十四步積盈百刻而成日也。
%%   帝曰:六氣應五行之變何如?岐伯曰:位有終始,氣有初中,上下不同,求之亦異也。帝曰:求之奈何?岐伯曰:天氣始於甲,地氣始於子,子甲相合,命曰歲立,謹候其時,氣可與期。
%%   帝曰:願聞其歲,六氣始終,早晏何如?岐伯曰:明乎哉問也!甲子之歲,初之氣,天數始於水下一刻,終於八十七刻半;二之氣始於八十七刻六分,終於七十五刻;三之氣,始於七十六刻,終於六十二刻半;四之氣,始於六十二刻六分,終於五十刻;五之氣,始於五十一刻,終於三十七刻半;六之氣,始於三十七刻六分,終於二十五刻。所謂初六,天之數也。
%%   乙丑歲,初之氣,天數始於二十六刻,終於一十二刻半;二之氣,始於一十二刻六分,終於水下百刻;三之氣,始於一刻,終於八十七刻半;四之氣,始於八十七刻六分,終於七十五刻;五之氣,始於七十六刻,終於六十二刻半;六之氣,始於六十二刻六分,終於五十刻。所謂六二,天之數也。
%%   丙寅歲,初之氣,天數始於五十一刻,終於三十七刻半;二之氣,始於三十七刻六分,終於二十五刻;三之氣,始於二十六刻,終於一十二刻半;四之氣,始於一十二刻六分,終於水下百刻;五之氣,始於一刻,終於八十七刻半;六之氣,始於八十七刻六分,終於七十五刻。所謂六三,天之數也。
%%   丁卯歲,初之氣,天數始於七十六刻,終於六十二刻半;二之氣,始於六十二刻六分,終於五十刻;三之氣,始於五十一刻,終於三十七刻半;四之氣,始於三十七刻六分,終於二十五刻;五之氣,始於二十六刻,終於一十二刻半;六之氣,始於一十二刻六分,終於水下百刻。所謂六四,天之數也。次戊辰歲,初之氣復始於一刻,常如是無已,週而復始。
%%   帝曰:願聞其歲候何如?岐伯曰:悉乎哉問也!日行一週,天氣始於一刻,日行再周,天氣始於二十六刻,日行三週,天氣始於五十一刻,日行四周,天氣始於七十六刻,日行五週,天氣復始於一刻,所謂一紀也。是故寅午戌歲氣會同,卯未亥歲氣會同,辰申子歲氣會同,巳酉丑歲氣會同,終而復始。
%%   帝曰:願聞其用也。岐伯曰:言天者求之本,言地者求之位,言人者求之氣交。帝曰:何謂氣交?岐伯曰:上下之位,氣交之中,人之居也。故曰:天樞之上,天氣主之;天樞之下,地氣主之;氣交之分,人氣從之,萬物由之,此之謂也。
%%   帝曰:何謂初中?岐伯曰:初凡三十度而有奇,中氣同法。帝曰:初中何也?岐伯曰:所以分天地也。帝曰:願卒聞之。岐伯曰:初者地氣也,中者天氣也。
%%   帝曰:其升降何如?岐伯曰:氣之升降,天地之更用也。帝曰:願聞其用何如?岐伯曰:升已而降,降者謂天;降已而升,升者謂地。天氣下降,氣流於地;地氣上升,氣騰於天。故高下相召,升降相因,而變作矣。
%%   帝曰:善。寒濕相遘,燥熱相臨,風火相值,其有聞乎?岐伯曰:氣有勝復,勝復之作,有德有化,有用有變,變則邪氣居之。帝曰:何謂邪乎?岐伯曰:夫物之生從於化,物之極由乎變,變化之相薄,成敗之所由也。故氣有往復,用有遲速,四者之有,而化而變,風之來也。帝曰:遲速往復,風所由生,而化而變,故因盛衰之變耳。成敗倚伏游乎中,何也?岐伯曰:成敗倚伏生乎動,動而不已,則變作矣。
%%   帝曰:有期乎?岐伯曰:不生不化,靜之期也。帝曰:不生化乎?岐伯曰:出入廢則神機化滅,升降息則氣立孤危。故非出入,則無以生長壯老已;非升降,則無以生長化收藏。是以升降出入,無器不有。故器者生化之宇,器散則分之,生化息矣。故無不出入,無不升降,化有小大,期有近遠,四者之有而貴常守,反常則災害至矣。故曰無形無患,此之謂也。帝曰:善。有不生不化乎?岐伯曰:悉乎哉問也!與道合同,惟真人也。帝曰:善。

%% \section{氣交變大論篇第六十九}

%%   黃帝問曰:五運更治,上應天期,陰陽往復,寒暑迎隨,真邪相薄,內外分離,六經波蕩,五氣頃移,太過不及,專勝兼併,願言其始,而有常名,可得聞乎?岐伯稽首再拜對曰:昭乎哉問也!是明道也。此上帝所貴,先師傳之,臣雖不敏,往聞其旨。帝曰:余聞得其人不教,是謂失道,傳非其人,慢洩天寶。余誠菲德,未足以受至道,然而眾子哀其不終,願夫子保於無窮,流於無極,余司其事,則而行之奈何?岐伯曰:請遂言之也。《上經》曰:夫道者上知天文,下知地理,中知人事,可以長久,此之謂也。帝曰:何謂也?岐伯曰:本氣位也,位天者,天文也;位地者,地理也;通於人氣之變化者,人事也。故太過者先天,不及者後天,所謂治化而人應之也。
%%   帝曰:五運之化,太過何如?岐伯曰:歲木太過,風氣流行,脾土受邪。民病飧洩,食減,體重,煩冤,腸鳴腹支滿,上應歲星。甚則忽忽善怒,眩冒巔疾。化氣不政,生氣獨治,雲物飛動,草木不寧,甚而搖落,反脅痛而吐甚,沖陽絕者死不治,上應太白星。
%%   歲火太過,炎暑流行,金肺受邪。民病瘧,少氣咳喘,血溢血洩注下,嗌燥耳聾,中熱肩背熱,上應熒惑星。甚則胸中痛,脅支滿脅痛,膺背肩胛間痛,兩臂內痛,身熱骨痛而為浸淫。收氣不行,長氣獨明,雨水霜寒,上應辰星。上臨少陰少陽,火燔焫,冰泉涸,物焦槁,病反譫妄狂越,咳喘息鳴,下甚血溢洩不已,太淵絕者死不治,上應熒惑星。
%%   歲土太過,雨濕流行,腎水受邪。民病腹痛,清厥意不樂,體重煩冤,上應鎮星。甚則肌肉萎,足痿不收,行善瘈,腳下痛,飲發中滿食減,四支不舉。變生得位,藏氣伏,化氣獨治之,泉湧河衍,涸澤生魚,風雨大至,土崩潰,鱗見於陸,病腹滿溏洩腸鳴,反下甚而太谿絕者,死不治,上應歲星。
%%   歲金太過,燥氣流行,肝木受邪。民病兩脅下少腹痛,目赤痛眥瘍,耳無所聞。肅殺而甚,則體重煩冤,胸痛引背,兩脅滿且痛引少腹,上應太白星。甚則喘咳逆氣,肩背痛,尻陰股膝髀腨(骨行)足皆病,上應熒惑星。收氣峻,生氣下,草木斂,蒼干凋隕,病反暴痛,脅不可反側,咳逆甚而血溢,太沖絕者,死不治,上應太白星。
%%   歲水太過,寒氣流行,邪害心火。民病身熱煩心,躁悸,陰厥上下中寒,譫妄心痛,寒氣早至,上應辰星。甚則腹大脛腫,喘咳,寢汗出憎風,大雨至,埃霧朦郁,上應鎮星。上臨太陽,雨冰雪,霜不時降,濕氣變物,病反腹滿腸鳴溏洩,食不化,渴而妄冒,神門絕者,死不治,上應熒惑辰星。
%%   帝曰:善。其不及何如?岐伯曰:悉乎哉問也!歲木不及,燥乃大行,生氣失應,草木晚榮,肅殺而甚,則剛木辟著,悉萎蒼干,上應太白星,民病中清,胠脅痛,少腹痛,腸鳴溏洩,涼雨時至,上應太白星,其谷蒼。上臨陽明,生氣失政,草木再榮,化氣乃急,上應太白鎮星,其主蒼早。復則炎暑流火,濕性燥,柔脆草木焦槁,下體再生,華實齊化,病寒熱瘡瘍疿胗癰痤,上應熒惑太白,其谷白堅。白露早降,收殺氣行,寒雨害物,蟲食甘黃,脾土受邪,赤氣後化,心氣晚治,上勝肺金,白氣乃屈,其谷不成,咳而鼽,上應熒惑太白星。
%%   歲火不及,寒乃大行,長政不用,物榮而下,凝慘而甚,則陽氣不化,乃折榮美,上應辰星,民病胸中痛,脅支滿,兩脅痛,膺背肩胛間及兩臂內痛,郁冒朦昧,心痛暴瘖,胸腹大,脅下與腰背相引而痛,甚則屈不能伸,髖髀如別,上應熒惑辰星,其谷丹。復則埃郁,大雨且至,黑氣乃辱,病溏腹滿,食飲不下,寒中腸鳴,洩注腹痛,暴攣痿痹,足不任身,上應鎮星辰星,玄谷不成。
%%   歲土不及,風乃大行,化氣不令,草木茂榮,飄揚而甚,秀而不實,上應歲星,民病飧洩霍亂,體重腹痛,筋骨繇復,肌肉瞤酸,善怒,藏氣舉事,蟄蟲早附,咸病寒中,上應歲星鎮星,其谷黅。復則收政嚴峻,名木蒼凋,胸脅暴痛,下引少腹,善太息,蟲食甘黃,氣客於脾,黅谷乃減,民食少失味,蒼谷乃損,上應太白歲星。上臨厥陰,流水不冰,蟄蟲來見,藏氣不用,白乃不復,上應歲星,民乃康。
%%   歲金不及,炎火乃行,生氣乃用,長氣專勝,庶物以茂,燥爍以行,上應熒惑星,民病肩背瞀重,鼽嚏血便注下,收氣乃後,上應太白星,其谷堅芒。復則寒雨暴至,乃零冰雹霜雪殺物,陰厥且格,陽反上行,頭腦戶痛,延及囟頂發熱,上應辰星,丹谷不成,民病口瘡,甚則心痛。
%%   歲水不及,濕乃大行,長氣反用,其化乃速,暑雨數至,上應鎮星,民病腹滿身重,濡洩寒瘍流水,腰股痛發,膕腨股膝不便,煩冤,足痿,清厥,腳下痛,甚則跗腫,藏氣不政,腎氣不衡,上應辰星,其谷秬。上臨太陰,則大寒數舉,蟄蟲早藏,地積堅冰,陽光不治,民病寒疾於下,甚則腹滿浮腫,上應鎮星,其主黅谷。復則大風暴發,草偃木零,生長不鮮,面色時變,筋骨並辟,肉瞤瘛,目視(目巟)(目巟),物疏璺,肌肉胗發,氣並鬲中,痛於心腹,黃氣乃損,其谷不登,上應歲星。
%%   帝曰:善。願聞其時也。岐伯曰:悉乎哉問也!木不及,春有鳴條律暢之化,則秋有霧露清涼之政。春有慘淒殘賊之勝,則夏有炎暑燔爍之復。其眚東,其藏肝,其病內舍胠脅,外在關節。
%%   火不及,夏有炳明光顯之化,則冬有嚴肅霜寒之政。夏有慘淒凝冽之勝,則不時有埃昏大雨之復。其眚南,其藏心,其病內舍膺脅,外在經絡。
%%   土不及,四維有埃雲潤澤之化,則春有鳴條鼓拆之政。四維發振拉飄騰之變,則秋有肅殺霖霪之復。其眚四維,其藏脾,其病內舍心腹,外在肌肉四支。
%%   金不及,夏有光顯郁蒸之令,則冬有嚴凝整肅之應。夏有炎爍燔燎之變,則秋有冰雹霜雪之復。其眚西,其藏肺,其病內舍膺脅肩背,外在皮毛。
%%   水不及,四維有湍潤埃雲之化,則不時有和風生發之應。四維發埃驟注之變,則不時有飄蕩振拉之復。其眚北,其藏腎,其病內舍腰脊骨髓,外在谿谷腨膝。夫五運之政,猶權衡也,高者抑之,下者舉之,化者應之,變者復之,此生長化成收藏之理,氣之常也,失常則天地四塞矣。故曰:天地之動靜,神明為之紀,陰陽之往復,寒暑彰其兆,此之謂也。
%%   帝曰:夫子之言五氣之變,四時之應,可謂悉矣。夫氣之動亂,觸遇而作,發無常會,卒然災合,何以期之?岐伯曰:夫氣之動變,固不常在,而德化政令災變,不同其候也。帝曰:何謂也?岐伯曰:東方生風,風生木,其德敷和,其化生榮,其政舒啟,其令風,其變振發,其災散落。南方生熱,熱生火,其德彰顯,其化蕃茂,其政明曜,其令熱,其變銷爍,其災燔焫。中央生濕,濕生土,其德溽蒸,其化豐備,其政安靜,其令濕,其變驟注,其災霖潰。西方生燥,燥生金,其德清潔,其化緊斂,其政勁切,其令燥,其變肅殺,其災蒼隕。北方生寒,寒生水,其德淒滄,其化清謐,其政凝肅,其令寒,其變凓冽,其災冰雪霜雹。是以察其動也,有德有化,有政有令,有變有災,而物由之,而人應之也。
%%   帝曰:夫子之言歲候,不及其太過,而上應五星。今夫德化政令,災眚變易,非常而有也,卒然而動,其亦為之變乎?岐伯曰:承天而行之,故無妄動,無不應也。卒然而動者,氣之交變也,其不應焉。故曰:應常不應卒,此之謂也。帝曰:其應奈何?岐伯曰:各從其氣化也。
%%   帝曰:其行之徐疾逆順何如?岐伯曰:以道留久,逆守而小,是謂省下;以道而去,去而速來,曲而過之,是謂省遺過也;久留而環,或離或附,是謂議災與其德也;應近則小,應遠則大。芒而大倍常之一,其化甚;大常之二,其眚即也;小常之一,其化減;小常之二,是謂臨視,省下之過與其德也。德者福之,過者伐之。是以象之見也,高而遠則小,下而近則大,故大則喜怒邇,小則禍福遠。歲運太過,則運星北越,運氣相得,則各行以道。故歲運太過,畏星失色而兼其母,不及則色兼其所不勝。肖者瞿瞿,莫知其妙,閔閔之當,孰者為良,妄行無徵,是畏候王。
%%   帝曰:其災應何如?岐伯曰:亦各從其化也。故時至有盛衰,凌犯有逆順,留守有多少,形見有善惡,宿屬有勝負,徵應有吉凶矣。
%%   帝曰:其善惡,何謂也?岐伯曰:有喜有怒,有憂有喪,有澤有燥,此象之常也,必謹察之。帝曰:六者高下異乎?岐伯曰:象見高下,其應一也,故人亦應之。
%%   帝曰:善。其德化政令之動靜損益皆何如?岐伯曰:夫德化政令災變,不能相加也。勝復盛衰,不能相多也。往來小大,不能相過也。用之升降,不能相無也。各從其動而復之耳。
%%   帝曰:其病生何如?岐伯曰:德化者氣之祥,政令者氣之章,變易者復之紀,災眚者傷之始,氣相勝者和,不相勝者病,重感於邪則甚也。
%%   帝曰:善。所謂精光之論,大聖之業,宣明大道,通於無窮,究於無極也。余聞之,善言天者,必應於人,善言古者,必驗於今,善言氣者,必彰於物,善言應者,同天地之化,善言化言變者,通神明之理,非夫子孰能言至道歟!乃擇良兆而藏之靈室,每旦讀之,命曰《氣交變》,非齋戒不敢發,慎傳也。


%% \section{五常政大論篇第七十}

%%   黃帝問曰:太虛寥廓,五運迴薄,衰盛不同,損益相從,願聞平氣何如而名?何如而紀也?岐伯對曰:昭乎哉問也!木曰敷和,火曰升明,土曰備化,金曰審平,水曰靜順。
%%   帝曰:其不及奈何?岐伯曰:木曰委和,火曰伏明,土曰卑監,金曰從革,水曰涸流。帝曰:太過何謂?岐伯曰:木曰發生,火曰赫曦,土曰敦阜,金曰堅成,水曰流衍。
%%   帝曰:三氣之紀,願聞其候。岐伯曰:悉乎哉問也!敷和之紀,木德周行,陽舒陰布,五化宣平,其氣端,其性隨,其用曲直,其化生榮,其類草木,其政發散,其候溫和,其令風,其藏肝,肝其畏清,其主目,其谷麻,其果李,其實核,其應春,其蟲毛,其畜犬,其色蒼,其養筋,其病裡急支滿,其味酸,其音角,其物中堅,其數八。
%%   升明之紀,正陽而治,德施周普,五化均衡,其氣高,其性速,其用燔灼,其化蕃茂,其類火,其政明曜,其候炎暑,其令熱,其藏心,心其畏寒,其主舌,其穀麥,其果杏,其實絡,其應夏,其蟲羽,其畜馬,其色赤,其養血,其病瞤瘛,其味苦,其音徵,其物脈,其數七。
%%   備化之紀,氣協天休,德流四政,五化齊修,其氣平,其性順,其用高下,其化豐滿,其類土,其政安靜,其候溽蒸,其令濕,其藏脾,脾其畏風,其主口,其谷稷,其果棗,其實肉,其應長夏,其蟲裸,其畜牛,其色黃,其養肉,其病否,其味甘,其音宮,其物膚,其數五。
%%   審平之紀,收而不爭,殺而無犯,五化宣明,其氣潔,其性剛,其用散落,其化堅斂,其類金,其政勁肅,其候清切,其令燥,其藏肺,肺其畏熱,其主鼻,其穀稻,其果桃,其實殼,其應秋,其蟲介,其畜雞,其色白,其養皮毛,其病咳,其味辛,其音商,其物外堅,其數九。
%%   靜順之紀,藏而勿害,治而善下,五化咸整,其氣明,其性下,其用沃衍,其化凝堅,其類水,其政流演,其候凝肅,其令寒,其藏腎,腎其畏濕,其主二陰,其谷豆,其果栗,其實濡,其應冬,其蟲鱗,其畜彘,其色黑,其養骨髓,其病厥,其味咸,其音羽,其物濡,其數六。
%%   故生而勿殺,長而勿罰,化而勿制,收而勿害,藏而勿抑,是謂平氣。
%%   委和之紀,是謂勝生。生氣不政,化氣乃揚,長氣自平,收令乃早。涼雨時降,風雲並興,草木晚榮,蒼干凋落,物秀而實,膚肉內充。其氣斂,其用聚,其動緛戾拘緩,其發驚駭,其藏肝,其果棗李,其實核殼,其谷稷稻,其味酸辛,其色白蒼,其畜犬雞,其蟲毛介,其主霧露淒滄,其聲角商。其病搖動注恐,從金化也,少角與判商同,上角與正角同,上商與正商同;其病支廢腫瘡瘍,其甘蟲,邪傷肝也,上宮與正宮同。蕭飋肅殺,則炎赫沸騰,眚於三,所謂復也。其主飛蠹蛆雉,乃為雷霆。
%%   伏明之紀,是謂勝長。長氣不宣,藏氣反布,收氣自政,化令乃衡,寒清數舉,暑令乃薄。承化物生,生而不長,成實而稚,遇化已老,陽氣屈伏,蟄蟲早藏。其氣鬱,其用暴,其動彰伏變易,其發痛,其藏心,其果栗桃,其實絡濡,其谷豆稻,其味苦咸,其色玄丹,其畜馬彘,其蟲羽鱗,其主冰雪霜寒,其聲徵羽。其病昏惑悲忘,從水化也,少徵與少羽同,上商與正商同,邪傷心也。凝慘凜冽,則暴雨霖霪,眚於九,其主驟注雷霆震驚,沉(蕓去草頭令)淫雨。
%%   卑監之紀,是謂減化。化氣不令,生政獨彰,長氣整,雨乃愆,收氣平,風寒並興,草木榮美,秀而不實,成而秕也。其氣散,其用靜定,其動瘍湧分潰癰腫。其發濡滯,其藏脾,其果李栗,其實濡核,其谷豆麻,其味酸甘,其色蒼黃,其畜牛犬,其蟲裸毛,其主飄怒振發,其聲宮角,其病留滿否塞,從木化也,少宮與少角同,上宮與正宮同,上角與正角同,其病飧洩,邪傷脾也。振拉飄揚,則蒼干散落,其眚四維,其主敗折虎狼,清氣乃用,生政乃辱。
%%   從革之紀,是謂折收。收氣乃後,生氣乃揚,長化合德,火政乃宣,庶類以蕃。其氣揚,其用躁切,其動鏗禁瞀厥,其發咳喘,其藏肺,其果李杏,其實殼絡,其谷麻麥,其味苦辛,其色白丹,其畜雞羊,其蟲介羽,其主明曜炎爍,其聲商徵,其病嚏咳鼽衄,從火化也,少商與少徵同,上商與正商同,上角與正角同,邪傷肺也。炎光赫烈,則冰雪霜雹,眚於七,其主鱗伏彘鼠,歲氣早至,乃生大寒。
%%   涸流之紀,是謂反陽,藏令不舉,化氣乃昌,長氣宣佈,蟄蟲不藏,土潤水泉減,草木條茂,榮秀滿盛。其氣滯,其用滲洩,其動堅止,其發燥槁,其藏腎,其果棗杏,其實濡肉,其谷黍稷,其味甘咸,其色黅玄,甚畜彘牛,其蟲鱗裸,其主埃郁昏翳,其聲羽宮,其病痿厥堅下,從土化也,少羽與少宮同,上宮與正宮同,其病癃閟,邪傷腎也,埃昏驟雨,則振拉摧拔,眚於一,其主毛顯狐貉,變化不藏。
%%   故乘危而行,不速而至,暴虐無德,災反及之,微者復微,甚者復甚,氣之常也。
%%   發生之紀,是謂啟陳,土疏洩,蒼氣達,陽和布化,陰氣乃隨,生氣淳化,萬物以榮。其化生,其氣美,其政散,其令條舒,其動掉眩巔疾,其德鳴靡啟坼,其變振拉摧拔,其谷麻稻,其畜雞犬,其果李桃,其色青黃白,其味酸甘辛,其象春,其經足厥陰少陽,其藏肝脾,其蟲毛介,其物中堅外堅,其病怒,太角與上商同,上徵則其氣逆,其病吐利。不務其德,則收氣復,秋氣勁切,甚則肅殺,清氣大至,草木凋零,邪乃傷肝。
%%   赫曦之紀,是謂蕃茂,陰氣內化,陽氣外榮,炎暑施化,物得以昌。其化長,其氣高,其政動,其令鳴顯,其動炎灼妄擾,其德暄暑郁蒸,其變炎烈沸騰,其穀麥豆,其畜羊彘,其果杏栗,其色赤白玄,其味苦辛咸,其象夏,其經手少陰太陽,手厥陰少陽,其藏心肺,其蟲羽鱗,其物脈濡,其病笑瘧瘡瘍血流狂妄目赤,上羽與正徵同,其收齊,其病痓,上徵而收氣後也。暴烈其政,藏氣乃復,時見凝慘,甚則雨水霜雹切寒,邪傷心也。
%%   敦阜之紀,是謂廣化,厚德清靜,順長以盈,至陰內實,物化充成,煙埃朦郁,見於厚土,大雨時行,濕氣乃用,燥政乃辟,其化員,其氣豐,其政靜,其令周備,其動濡積並稸,其德柔潤重淖,其變震驚飄驟崩潰,其谷稷麻,其畜牛犬,其果棗李,其色黅玄蒼,其味甘咸酸,其象長夏,其經足太陰陽明,其藏脾腎,其蟲裸毛,其物肌核,其病腹滿,四支不舉,大風迅至,邪傷脾也。
%%   堅成之紀,謂收引,天氣潔,地氣明,陽氣隨,陰治化,燥行其政,物以司成,收氣繁布,化洽不終。其化成,其氣削,其政肅,其令銳切,其動暴折瘍疰,其德霧露蕭飋,其變肅殺凋零,其穀稻黍,其畜雞馬,其果桃杏,其色白青丹,其味辛酸苦,其象秋,其經手太陰陽明,其藏肺肝,其蟲介羽,其物殼絡,其病喘喝,胸憑仰息。上徵與正商同,其生齊,其病咳,政暴變,則名木不營,柔脆焦首,長氣斯救,大火流,炎爍且至,蔓將槁,邪傷肺也。
%%   流衍之紀,是謂封藏,寒司物化,天地嚴凝,藏政以布,長令不揚。其化凜,其氣堅,其政謐,其令流注,其動漂洩沃湧,其德凝慘寒雰,其變冰雪霜雹,其谷豆稷,其畜彘牛,其果栗棗,其色黑丹黅,其味咸苦甘,其象冬,其經足少陰太陽,其藏腎心,其蟲鱗裸,其物濡滿,其病脹,上羽而長氣不化也。政過則化氣大舉,而埃昏氣交,大雨時降,邪傷腎也。故曰:不恆其德,則所勝來復,政恆其理,則所勝同化,此之謂也。
%%   帝曰:天不足西北,左寒而右涼;地不滿東南,右熱而左溫,其故何也?岐伯曰:陰陽之氣,高下之理,太少之異也。東南方,陽也,陽者其精降於下,故右熱而左溫。西北方,陰也,陰者其精奉於上,故左寒而右涼。是以地有高下,氣有溫涼,高者氣寒,下者氣熱。故適寒涼者脹之,之溫熱者瘡,下之則脹已,汗之則瘡已,此湊理開閉之常,太少之異耳。
%%   帝曰:其於壽夭何如?岐伯曰:陰精所奉其人壽,陽精所降其人夭。帝曰:善。其病也,治之奈何?岐伯曰:西北之氣散而寒之,東南之氣收而溫之,所謂同病異治也。故曰:氣寒氣涼,治以寒涼,行水漬之。氣溫氣熱,治以溫熱,強其內守。必同其氣,可使平也,假者反之。
%%   帝曰:善。一州之氣生化壽夭不同,其故何也?岐伯曰:高下之理,地勢使然也。崇高則陰氣治之,污下則陽氣治之,陽勝者先天,陰勝者後天,此地理之常,生化之道也。帝曰:其有壽夭乎?岐伯曰:高者其氣壽,下者其氣夭,地之小大異也,小者小異,大者大異。故治病者,必明天道地理,陰陽更勝,氣之先後,人之壽夭,生化之期,乃可以知人之形氣矣。
%%   帝曰:善。其歲有不病,而藏氣不應不用者,何也?岐伯曰:天氣制之,氣有所從也。帝曰:願卒聞之。岐伯曰:少陽司天,火氣下臨,肺氣上從,白起金用,草木眚,火見燔焫,革金且耗,大暑以行,咳嚏鼽衄鼻窒,曰瘍,寒熱胕腫。風行於地,塵沙飛揚,心痛胃脘痛,厥逆鬲不通,其主暴速。
%%   陽明司天,燥氣下臨,肝氣上從,蒼起木用而立,土乃眚,淒滄數至,木伐草萎,脅痛目赤,掉振鼓慄,筋痿不能久立。暴熱至,土乃暑,陽氣鬱發,小便變,寒熱如瘧,甚則心痛,火行於槁,流水不冰,蟄蟲乃見。
%%   太陽司天,寒氣下臨,心氣上從,而火且明,丹起金乃眚,寒清時舉,勝則水冰,火氣高明,心熱煩,嗌干善渴,鼽嚏,喜悲數欠,熱氣妄行,寒乃復,霜不時降,善忘,甚則心痛。土乃潤,水豐衍,寒客至,沉陰化,濕氣變物,水飲內稸,中滿不食,皮(疒帬)肉苛,筋脈不利,甚則胕腫,身後癰。
%%   厥陰司天,風氣下臨,脾氣上從,而土且隆,黃起,水乃眚,土用革,體重肌肉萎,食減口爽,風行太虛,雲物搖動,目轉耳鳴。火縱其暴,地乃暑,大熱消爍,赤沃下,蟄蟲數見,流水不冰,其發機速。
%%   少陰司天,熱氣下臨,肺氣上從,白起金用,草木眚,喘嘔寒熱,嚏鼽衄鼻窒,大暑流行,甚則瘡瘍燔灼,金爍石流。地乃燥清,淒滄數至,脅痛善太息,肅殺行,草木變。
%%   太陰司天,濕氣下臨,腎氣上從,黑起水變,埃冒雲雨,胸中不利,陰痿,氣大衰,而不起不用。當其時,反腰脽痛,動轉不便也,厥逆。地乃藏陰,大寒且至,蟄蟲早附,心下否痛,地裂冰堅,少腹痛,時害於食,乘金則止水增,味乃咸,行水減也。
%%   帝曰:歲有胎孕不育,治之不全,何氣使然?岐伯曰:六氣五類,有相勝制也,同者盛之,異者衰之,此天地之道,生化之常也。故厥陰司天,毛蟲靜,羽蟲育,介蟲不成;在泉,毛蟲育,裸蟲耗,羽蟲不育。少陰司天,羽蟲靜,介蟲育,毛蟲不成;在泉,羽蟲育,介蟲耗不育。太陰司天,裸蟲靜,鱗蟲育,羽蟲不成;在泉,裸蟲育,鱗蟲不成。少陽司天,羽蟲靜,毛蟲育,裸蟲不成;在泉,羽蟲育,介蟲耗,毛蟲不育。陽明司天,介蟲靜,羽蟲育,介蟲不成;在泉,介蟲育,毛蟲耗,羽蟲不成。太陽司天,鱗蟲靜,裸蟲育;在泉,鱗蟲耗,裸蟲不育。諸乘所不成之運,則甚也。故氣主有所製,歲立有所生,地氣制己勝,天氣制勝己,天制色,地制形,五類衰盛,各隨其氣之所宜也。故有胎孕不育,治之不全,此氣之常也,所謂中根也。根於外者亦五,故生化之別,有五氣五味五色五類五宜也。帝曰:何謂也?岐伯曰:根於中者,命曰神機,神去則機息。根於外者,命曰氣立,氣止則化絕。故各有制,各有勝,各有生,各有成。故曰:不知年之所加,氣之同異,不足以言生化,此之謂也。
%%   帝曰:氣始而生化,氣散而有形,氣布而蕃育,氣終而像變,其致一也。然而五味所資,生化有薄,成熟有多少,終始不同,其故何也?岐伯曰:地氣制之也,非天不生,地不長也。帝曰:願聞其道。岐伯曰:寒熱燥濕,不同其化也。故少陽在泉,寒毒不生,其味辛,其治苦酸,其谷蒼丹。陽明在泉,濕毒不生,其味酸,其氣濕,其治辛苦甘,其谷丹素。太陽在泉,熱毒不生,其味苦,其治淡咸,其谷黅秬。厥陰在泉,清毒不生,其味甘,其治酸苦,其谷蒼赤,其氣專,其味正。少陰在泉,寒毒不生,其味辛,其治辛苦甘,其谷白丹。太陰在泉,燥毒不生,其味咸,其氣熱,其治甘咸,其谷黅秬。化淳則咸守,氣專則辛化而俱治。
%%   故曰:補上下者從之,治上下者逆之,以所在寒熱盛衰而調之。故曰:上取下取,內取外取,以求其過。能毒者以厚藥,不勝毒者以薄藥,此之謂也。氣反者,病在上,取之下;病在下,取之上;病在中,傍取之。治熱以寒,溫而行之;治寒以熱,涼而行之;治溫以清,冷而行之;治清以溫,熱而行之。故消之削之,吐之下之,補之寫之,久新同法。
%%   帝曰:病在中而不實不堅,且聚且散,奈何?岐伯曰:悉乎哉問也!無積者求其藏,虛則補之,藥以袪之,食以隨之,行水漬之,和其中外,可使畢已。
%%   帝曰:有毒無毒,服有約乎?岐伯曰:病有久新,方有大小,有毒無毒,固宜常制矣。大毒治病,十去其六;常毒治病,十去其;,小毒治病,十去其八;無毒治病,十去其九。谷肉果菜,食養盡之,無使過之,傷其正也。不盡,行復如法,必先歲氣,無伐天和,無盛盛,無虛虛,而遺人天殃,無致邪,無失正,絕人長命。帝曰:其久病者,有氣從不康,病去而瘠,奈何?岐伯曰:昭乎哉聖人之問也!化不可代,時不可違。夫經絡以通,血氣以從,復其不足,與眾齊同,養之和之,靜以待時,謹守其氣,無使頃移,其形乃彰,生氣以長,命曰聖王。故《大要》曰:無代化,無違時,必養必和,待其來復,此之謂也。帝曰:善。


%% \section{六元正紀大論篇第七十一}

%%   黃帝問曰:六化六變,勝復淫治,甘苦辛咸酸淡先後,余知之矣。夫五運之化,或從天氣,或逆天氣,或從天氣而逆地氣,或從地氣而逆天氣,或相得,或不相得,余未能明其事。欲通天之紀,從地之理,和其運,調其化,使上下合德,無相奪倫,天地升降,不失其宜,五運宣行,勿乖其政,調之正味,從逆奈何?岐伯稽首再拜對曰:昭乎哉問也。此天地之綱紀,變化之淵源,非聖帝孰能窮其至理歟!臣雖不敏,請陳其道,令終不滅,久而不易。
%%   帝曰:願夫子推而次之,從其類序,分其部主,別其宗司,昭其氣數,明其正化,可得聞乎?岐伯曰:先立其年以明其氣,金木水火土運行之數,寒暑燥濕風火臨御之化,則天道可見,民氣可調,陰陽卷舒,近而無惑,數之可數者,請遂言之。
%%   帝曰:太陽之政奈何?岐伯曰:辰戌之紀也。太陽太角太陰壬辰壬戌,其運風,其化鳴紊啟拆,其變振拉摧拔,其病眩掉目瞑。
%%   太角少徵太宮少商太羽太陽太徵太陰戊辰戊戌同正徵,其運熱,其化暄暑鬱燠,其變炎烈沸騰,其病熱郁。
%%   太徵少宮太商少羽少角太陽太宮太陰甲辰歲會甲戌歲會,其運陰埃,其化柔潤重澤,其變震驚飄驟,其病濕下重。
%%   太宮少商太羽太角少徵太陽太商太陰庚辰庚戌,其運涼,其化霧露蕭,其變肅殺凋零,其病燥背瞀胸滿。
%%   太商少羽少角太徵少宮太陽太羽太陰丙辰天符丙戌天符,其運寒,其化凝慘凓冽,其變冰雪霜雹,其病大寒留於谿谷。
%%   太羽太角少徵太宮少商
%%   凡此太陽司天之政,氣化運行先天,天氣肅,地氣靜,寒凝太虛,陽氣不令,水土合德,上應辰星鎮星。其谷玄黅,其政肅,其令徐。寒政大舉,澤無陽焰,則火發待時。少陽中治,時雨乃涯,止極雨散,還於太陰,雲朝北極,濕化乃布,澤流萬物,寒敷於上,雷動於下,寒濕之氣,持於氣交。民病寒濕,發肌肉萎,足痿不收,濡寫血熱。初之氣,地氣遷,氣乃大溫,草乃早榮,民乃厲,溫病乃作,身熱頭痛嘔吐,肌腠瘡瘍。二之氣,大涼反至,民乃慘,草乃遇寒,火氣遂抑,民病氣鬱中滿,寒乃始。三之氣,天政布,寒氣行,雨乃降,民病寒,反熱中,癰疽注下,心熱瞀悶,不治者死。四之氣,風濕交爭,風化為雨,乃長乃化乃成,民病大熱少氣,肌肉萎,足痿,注下赤白。五之氣,陽復化,草乃長,乃化乃成,民乃舒。終之氣,地氣正,濕令行,陰凝太虛,埃昬郊野,民乃慘淒,寒風以至,反者孕乃死。故歲宜苦以燥之溫之,必折其郁氣,先資其化源,抑其運氣,扶其不勝,無使暴過而生其疾,食歲谷以全其真,避虛邪以安其正。適氣同異,多少制之,同寒濕者燥熱化,異寒濕者燥濕化,故同者多之,異者少之,用寒遠寒,用涼遠涼,用溫遠溫,用熱遠熱,食宜同法。有假者反常,反是者病,所謂時也。
%%   帝曰:善。陽明之政奈何?岐伯曰:卯酉之紀也。陽明少角少陰,清熱勝復同,同正商。丁卯歲會丁酉,其運風清熱。少角太徵少宮太商少羽陽明少徵少陰,寒雨勝復同,同正商。癸卯癸酉,其運熱寒雨。少徵太宮少商太羽太角陽明少宮少陰,風涼勝復同。己卯己酉,其運雨風涼。少宮太商少羽少角太徵陽明少商少陰,熱寒勝復同,同正商。乙卯天符,乙酉歲會,太一天符,其運涼熱寒。少商太羽太角少徵太宮陽明少羽少陰,雨風勝復同,辛卯少宮同。辛酉辛卯其運寒雨風。少羽少角太徵太宮太商
%%   凡此陽明司天之政,氣化運行後天,天氣急,地氣明,陽專其令,炎暑大行,物燥以堅,淳風乃治,風燥橫運,流於氣交,多陽少陰,雲趨雨府,濕化乃敷。燥極而澤,其谷白丹,間谷命太者,其耗白甲品羽,金火合德,上應太白熒惑。其政切,其令暴,蟄蟲乃見,流水不冰,民病咳嗌塞,寒熱發,暴振凓癃閟,清先而勁,毛蟲乃死,熱後而暴,介蟲乃殃,其發躁,勝復之作,擾而大亂,清熱之氣,持於氣交。初之氣,地氣遷,陰始凝,氣始肅,水乃冰,寒雨化。其病中熱脹面目浮腫,善眠,鼽衄,嚏欠,嘔,小便黃赤,甚則淋。二之氣,陽乃布,民乃舒,物乃生榮。厲大至,民善暴死。三之氣,天政布,涼乃行,燥熱交合,燥極而澤,民病寒熱。四之氣,寒雨降,病暴僕,振慄譫妄,少氣,嗌干引飲,及為心痛癰腫瘡瘍瘧寒之疾,骨痿血便。五之氣,春令反行,草乃生榮,民氣和。終之氣,陽氣布,候反溫,蟄蟲來見,流水不冰,民乃康平,其病溫。故食歲谷以安其氣,食間谷以去其邪,歲宜以咸以苦以辛,汗之、清之、散之,安其運氣,無使受邪,折其郁氣,資其化源。以寒熱輕重少多其制,同熱者多天化,同清者多地化,用涼遠涼,用熱遠熱,用寒遠寒,用溫遠溫,食宜同法。有假者反之,此其道也。反是者,亂天地之經,擾陰陽之紀也。
%%   帝曰:善。少陽之政奈何?岐伯曰:寅申之紀也。少陽太角厥陰壬寅壬申,其運風鼓,其化鳴紊啟坼,其變振拉摧拔,其病掉眩,支脅,驚駭。太角少徵太宮少商太羽少陽太徵厥陰戊寅天符戊申天符,其運暑,其化暄囂鬱燠,其變炎烈沸騰,其病上熱郁,血溢血洩心痛。太徵少宮太商少羽少角少陽太商厥陰甲寅甲申,其運陰雨,其化柔潤重澤,其變震驚飄驟,其病體重,胕腫痞飲。太宮少商太羽太角少徵少陽太商厥陰庚寅庚申同正商,其運涼,其化霧露清切,其變肅殺凋零,其病肩背胸中。太商少羽少角太徵少宮少陽太羽厥陰丙寅丙申,其運寒肅,其化凝慘凓冽,其變冰雪霜雹,其病寒浮腫。太羽太角少徵太宮少商.
%%   凡此少陽司天之政,氣化運行先天,天氣正,地氣擾,風乃暴舉,木偃沙飛,炎火乃流,陰行陽化,雨乃時應,火木同德,上應熒惑歲星。其谷丹蒼,其政嚴,其令擾。故風熱參布,雲物沸騰,太陰橫流,寒乃時至,涼雨並起。民病寒中,外發瘡瘍,內為洩滿。故聖人遇之,和而不爭。往復之作,民病寒熱瘧洩,聾瞑嘔吐,上怫腫色變。初之氣,地氣遷,風勝乃搖,寒乃去,候乃大溫,草木早榮。寒來不殺,溫病乃起,其病氣怫於上,血溢目赤,咳逆頭痛,血崩脅滿,膚腠中瘡。二之氣,火反郁,白埃四起,雲趨雨府,風不勝濕,雨乃零,民乃康。其病熱鬱於上,咳逆嘔吐,瘡發於中,胸嗌不利,頭痛身熱,昬憒膿瘡。三之氣,天政布,炎暑至,少陽臨上,雨乃涯。民病熱中,聾瞑血溢,膿瘡咳嘔,鼽衄渴嚏欠,喉痹目赤,善暴死。四之氣,涼乃至,炎暑間化,白露降,民氣和平,其病滿身重。五之氣,陽乃去,寒乃來,雨乃降,氣門乃閉,剛木早凋,民避寒邪,君子周密。終之氣,地氣正,風乃至,萬物反生,霿霧以行。其病關閉不禁,心痛,陽氣不藏而咳。抑其運氣,贊所不勝,必折其郁氣,先取化源,暴過不生,苛疾不起。故歲宜咸辛宜酸,滲之洩之,漬之發之,觀氣寒溫以調其過,同風熱者多寒化,異風熱者少寒化,用熱遠熱,用溫遠溫,用寒遠寒,用涼遠涼,食宜同法,此其道也。有假者反之,反是者,病之階也。
%%   帝曰:善。太陰之政奈何?岐伯曰:丑未之紀也。太陰少角太陽,清熱勝復同,同正宮,丁丑丁未,其運風清熱。少角太徵少宮太商少羽太陰少徵太陽,寒雨勝復同,癸丑癸未,其運熱寒雨。少徵太宮少商太羽太角太陰少宮太陽,風清勝復同,同正宮,己丑太一天符,己未太一天符,其運雨風清。少宮太商少羽少角太徵太陰少商太陽,熱寒勝復同,乙丑乙未,其運涼熱寒。少商太羽太角少徵太宮太陰少羽太陽,雨風勝復同,同正宮。辛丑辛未,其運寒雨風。少羽少角太徵少宮太商凡此太陰司天之政,氣化運行後天,陰專其政,陽氣退辟,大風時起,天氣下降,地氣上騰,原野昏霿,白埃四起,雲奔南極,寒雨數至,物成於差夏。民病寒濕,腹滿,身(月真)憤,胕腫,痞逆寒厥拘急。濕寒合德,黃黑埃昏,流行氣交,上應鎮星辰星。其政肅,其令寂,其谷黅玄。故陰凝於上,寒積於下,寒水勝火,則為冰雹,陽光不治,殺氣乃行。故有餘宜高,不及宜下,有餘宜晚,不及宜早,土之利,氣之化也,民氣亦從之,間谷命其太也。初之氣,地氣遷,寒乃去,春氣正,風乃來,生布萬物以榮,民氣條舒,風濕相薄,雨乃後。民病血溢,筋絡拘強,關節不利,身重筋痿。二之氣,大火正,物承化,民乃和,其病溫厲大行,遠近咸若,濕蒸相薄,雨乃時降。三之氣,天政布,濕氣降,地氣騰,雨乃時降,寒乃隨之。感於寒濕,則民病身重胕腫,胸腹滿。四之氣,畏火臨,溽蒸化,地氣騰,天氣否隔,寒風曉暮,蒸熱相薄,草木凝煙,濕化不流,則白露陰布,以成秋令。民病腠理熱,血暴溢瘧,心腹滿熱,臚脹,甚則胕腫。五之氣,慘令已行,寒露下,霜乃早降,草木黃落,寒氣及體,君子周密,民病皮腠。終之氣,寒大舉,濕大化,霜乃積,陰乃凝,水堅冰,陽光不治。感於寒則病人關節禁固,腰脽痛,寒濕推於氣交而為疾也。必折其郁氣,而取化源,益其歲氣,無使邪勝,食歲谷以全其真,食間谷以保其精。故歲宜以苦燥之溫之,甚者發之洩之。不發不洩,則濕氣外溢,肉潰皮拆而水血交流。必贊其陽火,令御甚寒,從氣異同,少多其判也,同寒者以熱化,同濕者以燥化,異者少之,同者多之,用涼遠涼,用寒遠寒,用溫遠溫,用熱遠熱,食宜同法。假者反之,此其道也,反是者病也。
%%   帝曰:善,少陰之政奈何?岐伯曰:子午之紀也。少陰太角陽明壬子壬午,其運風鼓,其化鳴紊啟折,其變振拉摧拔,其病支滿。太角少徵太宮少商太羽少陰太徵陽明戊子天符戊午太一天符,其運炎暑,其化暄曜鬱燠,其變炎烈沸騰,其病上熱血溢。太徵少宮太商少羽少角少陰太宮陽明甲子甲午,其運陰雨,其化柔潤時雨,其變震驚飄驟,其病中滿身重。太宮少商太羽太角少徵少陰太商陽明庚子庚午同正商。其運涼勁,其化霧露蕭飋,其變肅殺凋零,其病下清。太商少羽少角太徵少宮少陰太羽陽明丙子歲會丙午,其運寒,其化凝慘凓冽,其變冰雪霜雹,其病寒下。太羽太角少徵太宮少商。
%%   凡此少陰司天之政,氣化運行先天,地氣肅,天氣明,寒交暑,熱加燥,雲馳雨府,濕化乃行,時雨乃降,金火合德,上應熒惑太白。其政明,其令切,其谷丹白。水火寒熱持於氣交而為病始也。熱病生於上,清病生於下,寒熱凌犯而爭於中,民病咳喘,血溢血洩,鼽嚏,目赤,眥瘍,寒厥入胃,心痛,腰痛,腹大,嗌干腫上。初之氣,地氣遷,燥將去,寒乃始,蟄復藏,水乃冰,霜復降,風乃至,陽氣鬱,民反周密,關節禁固,腰脽痛,炎暑將起,中外瘡瘍。二之氣,陽氣布,風乃行,春氣以正,萬物應榮,寒氣時至,民乃和,其病淋,目瞑目赤,氣鬱於上而熱。三之氣,天政布,大火行,庶類蕃鮮,寒氣時至。民病氣厥心痛,寒熱更作,咳喘目赤。四之氣,溽暑至,大雨時行,寒熱互至。民病寒熱,嗌干,黃癉,鼽衄,飲發。五之氣,畏火臨,暑反至,陽乃化,萬物乃生乃長榮,民乃康,其病溫。終之氣,燥令行,余火內格,腫於上,咳喘,甚則血溢。寒氣數舉,則霿霧翳,病生皮腠,內舍於脅,下連少腹而作寒中,地將易也。必抑其運氣,資其歲勝,折其郁發,先取化源,無使暴過而生其病也。食歲谷以全真氣,食間谷以辟虛邪。歲宜咸以之,而調其上,甚則以苦發之,以酸收之,而安其下,甚則以苦洩之。適氣同異而多少之,同天氣者以寒清化,同地氣者以溫熱化,用熱遠熱,用涼遠涼,用溫遠溫,用寒遠寒,食宜同法。有假則反,此其道也,反是者病作矣。
%%   帝曰:善。厥陰之政奈何?岐伯曰:巳亥之紀也。厥陰少角少陽,清熱勝復同,同正角。丁巳天符,丁亥天符,其運風清熱。少角太徵少宮太商少羽厥陰少徵少陽,寒雨勝復同,癸巳癸亥,其運熱寒雨。少徵太宮少商太羽太角厥陰少宮少陽,風清勝復同,同正角。己巳己亥,其運雨風清。少宮太商少羽少角太徵厥陰少商少陽,寒熱勝復同,同正角。乙巳乙亥,其運涼熱寒。少商太羽太角少徵太宮厥陰少羽少陽,雨風勝復同,辛巳辛亥,其運寒雨風。少羽少角太徵少宮太商
%%   凡此厥陰司天之政,氣化運行後天,諸同正歲,氣化運行同天,天氣擾,地氣正,風生高遠,炎熱從之,雲趨雨府,濕化乃行,風火同德,上應歲星熒惑。其政撓,其令速,其谷蒼丹,間谷言太者,其耗文角品羽。風燥火熱,勝復更作,蟄蟲來見,流水不冰,熱病行於下,風病行於上,風燥勝復形於中。初之氣,寒始肅,殺氣方至,民病寒於右之下。二之氣,寒不去,華雪水冰,殺氣施化,霜乃降,名草上焦,寒雨數至,陽復化,民病熱於中。三之氣,天政布,風乃時舉,民病泣出耳鳴掉眩。四之氣,溽暑濕熱相薄,爭於左之上,民病黃疸而為胕腫。五之氣,燥濕更勝,沉陰乃布,寒氣及體,風雨乃行。終之氣,畏火司令,陽乃大化,蟄蟲出見,流水不冰,地氣大發,草乃生,人乃舒,其病溫厲,必折其郁氣,資其化源,贊其運氣,無使邪勝,歲宜以辛調上,以咸調下,畏火之氣,無妄犯之,用溫遠溫,用熱遠熱,用涼遠涼,用寒遠寒,食宜同法。有假反常,此之道也,反是者病。
%%   帝曰:善。夫子言可謂悉矣,然何以明其應乎?岐伯曰:昭乎哉問也!夫六氣者,行有次,止有位,故常以正月朔日平旦視之,睹其位而知其所在矣。運有餘,其至先,運不及,其至後,此天之道,氣之常也。運非有餘非不足,是謂正歲,其至當其時也。帝曰:勝復之氣,其常在也,災眚時至,候也奈何?岐伯曰:非氣化者,是謂災也。
%%   帝曰:天地之數,終始奈何?岐伯曰:悉乎哉問也!是明道也。數之始,起於上而終於下,歲半之前,天氣主之,歲半之後,地氣主之,上下交互,氣交主之,歲紀畢矣。故曰位明,氣月可知乎,所謂氣也。帝曰:余司其事,則而行之,不合其數,何也?岐伯曰:氣用有多少,化洽有盛衰,衰盛多少,同其化也。帝曰:願聞同化何如?岐伯曰:風溫春化同,熱曛昏火夏化同,勝與復同,燥清煙露秋化同,雲雨昏暝埃長夏化同,寒氣霜雪冰冬化同,此天地五運六氣之化,更用盛衰之常也。
%%   帝曰:五運行同天化者,命曰天符,余知之矣。願聞同地化者何謂也?岐伯曰:太過而同天化者三,不及而同天化者亦三,太過而同地化者三,不及而同地化者亦三,此凡二十四歲也。帝曰:願聞其所謂也。岐伯曰:甲辰甲戌太宮下加太陰,壬寅壬申太角下加厥陰,庚子庚午太商下加陽明,如是者三。癸巳癸亥少徵下加少陽,辛丑辛未少羽下加太陽,癸卯癸酉少徵下加少陰,如是者三。戊子戊午太徵上臨少陰,戊寅戊申太徵上臨少陽,丙辰丙戌太羽上臨太陽,如是者三。丁巳丁亥少角上臨厥陰,乙卯乙酉少商上臨陽明,己丑己未少宮上臨太陰,如是者三。除此二十四歲,則不加不臨也。帝曰:加者何謂?岐伯曰:太過而加同天符,不及而加同歲會也。帝曰:臨者何謂?岐伯曰:太過不及,皆曰天符,而變行有多少,病形有微甚,生死有早晏耳。
%%   帝曰:夫子言用寒遠寒,用熱遠熱,余未知其然也,願聞何謂遠?岐伯曰:熱無犯熱,寒無犯寒,從者和,逆者病,不可不敬畏而遠之,所謂時與六位也。帝曰:溫涼何如?岐伯曰:司氣以熱,用熱無犯,司氣以寒,用寒無犯,司氣以涼,用涼無犯,司氣以溫,用溫無犯,間氣同其主無犯,異其主則小犯之,是謂四畏,必謹察之。帝曰:善。其犯者何如?岐伯曰:天氣反時,則可依時,及勝其主則可犯,以平為期,而不可過,是謂邪氣反勝者。故曰:無失天信,無逆氣宜,無翼其勝,無贊其復,是謂至治。
%%   帝曰:善。五運氣行主歲之紀,其有常數乎?岐伯曰:臣請次之。甲子甲午歲,上少陰火,中太宮土運,下陽明金,熱化二,雨化五,燥化四,所謂正化日也。其化上咸寒,中苦熱,下酸熱,所謂藥食宜也。
%%   乙丑乙未歲,上太陰土,中少商金運,下太陽水,熱化寒化勝復同,所謂邪氣化日也。災七宮。濕化五,清化四,寒化六,所謂正化日也。其化上苦熱,中酸和,下甘熱,所謂藥食宜也。
%%   丙寅丙申歲,上少陽相火,中太羽水運,下厥陰木。火化二,寒化六,風化三,所謂正化日也。其化上咸寒,中咸溫下辛溫,所謂藥食宜也。
%%   丁卯丁酉歲,上陽明金,中少角木運,下少陰火,清化熱化勝復同,所謂邪氣化日也。災三宮。燥化九,風化三,熱化七,所謂正化日也。其化上苦小溫,中辛和,下咸寒,所謂藥食宜也。
%%   戊辰戊戌歲,上太陽水,中太徵火運,下太陰土。寒化六,熱化七,濕化五,所謂正化日也。其化上苦溫,中甘和,下甘溫,所謂藥食宜也。
%%   己巳己亥歲,上厥陰木,中少宮土運,下少陽相火,風化清化勝復同,所謂邪氣化日也。災五宮。風化三,濕化五,火化七,所謂正化日也。其化上辛涼,中甘和,下咸寒,所謂藥食宜也。
%%   庚午庚子歲,上少陰火,中太商金運,下陽明金,熱化七,清化九,燥化九,所謂正化日也。其化上咸寒,中辛溫,下酸溫,所謂藥食宜也。
%%   辛未辛丑歲,上太陰土,中少羽水運,下太陽水,雨化風化勝復同,所謂邪氣化日也。災一宮。雨化五,寒化一,所謂正化日也。其化上苦熱,中苦和,下苦熱,所謂藥食宜也。
%%   壬申壬寅歲,上少陽相火,中太角木運,下厥陰木,火化二,風化八,所謂正化日也。其化土咸寒,中酸和,下辛涼,所謂藥食宜也。
%%   癸酉癸卯歲,上陽明金,中少徵火運,下少陰火,寒化雨化勝復同,所謂邪氣化日也。災九宮。燥化九,熱化二,所謂正化日也。其化上苦小溫,中咸溫,下咸寒,所謂藥食宜也。
%%   甲戌甲辰歲,上太陽水,中太宮土運,下太陰土,寒化六,濕化五,正化日也。其化上苦熱,中苦溫,下苦溫,藥食宜也。
%%   乙亥乙巳歲,上厥陰木,中少商金運,下少陽相火,熱化寒化勝復同,邪氣化日也。災七宮。風化八,清化四,火化二,正化度也。其化上辛涼,中酸和,下咸寒,藥食宜也。
%%   丙子丙午歲,上少陰火,中太羽水運,下陽明金,熱化二,寒化六,清化四,正化度也。其化上咸寒,中咸熱,下酸溫,藥食宜也。
%%   丁丑丁未歲,上太陰土,中少角木運,下太陽水,清化熱化勝復同,邪氣化度也。災三宮。雨化五,風化三,寒化一,正化度也。其化上苦溫,中辛溫,下甘熱,藥食宜也。
%%   戊寅戊申歲,上少陽相火,中太徵火運,下厥陰木,火化七,風化三,正化度也。其化上咸寒,中甘和,下辛涼,藥食宜也。
%%   己卯己酉歲,上陽明金,中少宮土運,下少陰火,風化清化勝復同,邪氣化度也。災五宮。清化九,雨化五,熱化七,正化度也,其化上苦小溫,中甘和,下咸寒,藥食宜也。
%%   庚辰庚戌歲,上太陽水,中太商金運,下太陰土。寒化一,清化九,雨化五,正化度也。其化上苦熱,中辛溫,下甘熱,藥食宜也。
%%   辛巳辛亥歲,上厥陰木,中少羽水運,下少陽相火,雨化風化勝復同,邪氣化度也。災一宮。風化三,寒化一,火化七,正化度也。其化上辛涼,中苦和,下咸寒,藥食宜也。
%%   壬午壬子歲,上少陰火,中太角木運,下陽明金。熱化二,風化八,清化四,正化度也。其化上咸寒,中酸涼,下酸溫,藥食宜也。
%%   癸未癸丑歲,上太陰土,中少徵火運,下太陽水,寒化雨化勝復同,邪氣化度也。災九宮。雨化五,火化二,寒化一,正化度也。其化上苦溫,中咸溫,下甘熱,藥食宜也。
%%   甲申甲寅歲,上少陽相火,中太宮土運,下厥陰木。火化二,雨化五,風化八,正化度也。其化上咸寒,中咸和,下辛涼,藥食宜也。
%%   乙酉乙卯歲,上陽明金,中少商金運,下少陰火,熱化寒化勝復同,邪氣化度也。災七宮。燥化四,清化四,熱化二,正化度也。其化上苦小溫,中苦和,下咸寒,藥食宜也。
%%   丙戌丙辰歲,上太陽水,中太羽水運,下太陰土。寒化六,雨化五,正化度也。其化上苦熱,中咸溫,下甘熱,藥食宜也。
%%   丁亥丁巳歲,上厥陰木,中少角木運,下少陽相火,清化熱化勝復同,邪氣化度也。災三宮。風化三,火化七,正化度也。其化上辛涼,中辛和,下咸寒,藥食宜也。
%%   戊子戊午歲,上少陰火,中太徵火運,下陽明金。熱化七,清化九,正化度也。其化上咸寒,中甘寒,下酸溫,藥食宜也。
%%   己丑己未歲,上太陰土,中少宮土運,下太陽水,風化清化勝復同,邪氣化度也。災五宮。雨化五,寒化一,正化度也。其化上苦熱,中甘和,下甘熱,藥食宜也。
%%   庚寅庚申歲,上少陽相火,中太商金運,下厥陰木。火化七,清化九,風化三,正化度也。其化上咸寒,中辛溫,下辛涼,藥食宜也。
%%   辛卯辛酉歲,上陽明金,中少羽水運,下少陰火,雨化風化勝復同,邪氣化度也。災一宮。清化九,寒化一,熱化七,正化度也。其化上苦小溫,中苦和,下咸寒,藥食宜也。
%%   壬辰壬戌歲,上太陽水,中太角木運,下太陰土。寒化六,風化八,雨化五,正化度也。其化上苦溫,中酸和,下甘溫,藥食宜也。
%%   癸巳癸亥歲,上厥陰木,中少徵火運,下少陽相火,寒化雨化勝復同,邪氣化度也。災九宮。風化八,火化二,正化度也。其化上辛涼,中咸和,下咸寒,藥食宜也。
%%   凡此定期之紀,勝復正化,皆有常數,不可不察。故知其要者一言而終,不知其要,流散無窮,此之謂也。
%%   帝曰:善。五運之氣,亦復歲乎?岐伯曰:郁極乃發,待時而作者也。帝曰:請問其所謂也?岐伯曰:五常之氣,太過不及,其發異也。帝曰:願卒聞之。岐伯曰:太過者暴,不及者徐,暴者為病甚,徐者為病持。帝曰:太過不及,其數何如?岐伯曰:太過者其數成,不及者其數生,土常以生也。
%%   帝曰:其發也何如?岐伯曰:土郁之發,岩谷震驚,雷殷氣交,埃昏黃黑,化為白氣,飄驟高深,擊石飛空,洪水乃從,川流漫衍,田牧土駒。化氣乃敷,善為時雨,始生始長,始化始成。故民病心腹脹,腸鳴而為數後,甚則心痛脅(月真),嘔吐霍亂,飲發注下,胕腫身重。雲奔雨府,霞擁朝陽,山澤埃昏。其乃發也,以其四氣。雲橫天山,浮游生滅,怫之先兆。
%%   金郁之發,天潔地明,風清氣切,大涼乃舉,草樹浮煙,燥氣以行,霿霧數起,殺氣來至,草木蒼干,金乃有聲。故民病咳逆,心脅滿,引少腹善暴痛,不可反側,嗌乾麵塵色惡。山澤焦枯,土凝霜鹵,怫乃發也,其氣五。夜零白露,林莽聲淒,怫之兆也。
%%   水郁之發,陽氣乃辟,陰氣暴舉,大寒乃至,川澤嚴凝,寒雰結為霜雪,甚則黃黑昏翳,流行氣交,乃為霜殺,水乃見祥。故民病寒客心痛,腰脽痛,大關節不利,屈伸不便,善厥逆,痞堅腹滿。陽光不治,空積沉陰,白埃昏暝,而乃發也,其氣二火前後。太虛深玄,氣猶麻散,微見而隱,色黑微黃,怫之先兆也。
%%   木郁之發,太虛埃昏,雲物以擾,大風乃至,屋發折木,木有變。故民病胃脘當心而痛,上支兩脅,鬲咽不通,食飲不下,甚則耳鳴眩轉,目不識人,善暴僵仆。太虛蒼埃,天山一色,或氣濁色,黃黑郁若,橫雲不起,雨而乃發也,其氣無常。長川草偃,柔葉呈陰,松吟高山,虎嘯岩岫,怫之先兆也。
%%   火郁之發,太虛腫翳,大明不彰,炎火行,大暑至,山澤燔燎,材木流津,廣廈騰煙,土浮霜鹵,止水乃減,蔓草焦黃,風行惑言,濕化乃後。故民病少氣,瘡瘍癰腫,脅腹胸背,面首四支(月真)憤,臚脹,瘍痱,嘔逆,瘛瘲骨痛,節乃有動,注下溫瘧,腹中暴痛,血溢流注,精液乃少,目赤心熱,甚則瞀悶懊憹,善暴死。刻終大溫,汗濡玄府,其乃發也,其氣四。動復則靜,陽極反陰,濕令乃化乃成。華髮水凝,山川冰雪,焰陽午澤,怫之先兆也。有怫之應而後報也,皆觀其極而乃發也,木發無時,水隨火也。謹候其時,病可與期,失時反歲,五氣不行,生化收藏,政無恆也。
%%   帝曰:水發而雹雪,土發而飄驟,木發而毀折,金髮而清明,火發而曛昧,何氣使然?岐伯曰:氣有多少,發有微甚,微者當其氣,甚者兼其下,徵其下氣而見可知也。
%%   帝曰:善。五氣之發,不當位者何也?岐伯曰:命其差。帝曰:差有數乎?岐伯曰:後皆三十度而有奇也。
%%   帝曰:氣至而先後者何?岐伯曰:運太過則其至先。運不及則其至後,此候之常也。帝曰:當時而至者何也?岐伯曰:非太過,非不及,則至當時,非是者眚也。
%%   帝曰:善。氣有非時而化者何也?岐伯曰:太過者當其時,不及者歸其己勝也。
%%   帝曰:四時之氣,至有早晏高下左右,其候何如?岐伯曰:行有逆順,至有遲速,故太過者化先天,不及者化後天。
%%   帝曰:願聞其行何謂也?岐伯曰:春氣西行,夏氣北行,秋氣東行,冬氣南行。故春氣始於下,秋氣始於上,夏氣始於中,冬氣始於標,春氣始於左,秋氣始於右,冬氣始於後,夏氣始於前,此四時正化之常。故至高之地,冬氣常在,至下之地,春氣常在。必謹察之。帝曰:善。
%%   黃帝問曰:五運六氣之應見,六化之正,六變之紀,何如?岐伯對曰:夫六氣正紀,有化有變,有勝有復,有用有病,不同其候,帝欲何乎?帝曰:願盡聞之。岐伯曰:請遂言之。夫氣之所至也,厥陰所至為和平,少陰所至為暄,太陰所至為埃溽,少陽所至為炎暑,陽明所至為清勁,太陽所至為寒雰,時化之常也。
%%   厥陰所至為風府,為璺啟;少陰所至為火府,為舒榮;太陰所至為雨府,為員盈;少陽所至為熱府,為行出;陽明所至為司殺府,為庚蒼;太陽所至為寒府,為歸藏;司化之常也。
%%   厥陰所至為生,為風搖;少陰所至為榮,為形見;太陰所至為化,為雲雨;少陽所至為長,為蕃鮮;陽明所至為收,為霧露;太陽所至為藏,為周密;氣化之常也。
%%   厥陰所至為風生,終為肅;少陰所至為熱生,中為寒;太陰所至為濕生,終為注雨;少陽所至為火生,終為蒸溽;陽明所至為燥生,終為涼;太陽所至為寒生,中為溫;德化之常也。
%%   厥陰所至為毛化,少陰所至為羽化,太陰所至為裸化,少陽所至羽化,陽明所至為介化,太陽所至為鱗化,德化之常也。
%%   厥陰所至為生化,少陰所至為榮化,太陰所至為濡化,少陽所至為茂化,陽明所至為堅化,太陽所至為藏化,布政之常也。
%%   厥陰所至為飄怒大涼,少陰所至為大暄寒,太陰所至為雷霆驟雨烈風,少陽所至為飄風燔燎霜凝,陽明所至為散落溫,太陽所至為寒雪冰雹白埃,氣變之常也。
%%   厥陰所至為撓動,為迎隨;少陰所至為高明,焰為曛;太陰所至為沉陰,為白埃,為晦暝;少陽所至為光顯,為彤雲,為曛;陽明所至為煙埃,為霜,為勁切,為淒鳴;太陽所至為剛固,為堅芒,為立;令行之常也。
%%   厥陰所至為裡急;少陰所至為瘍胗身熱;太陰所至為積飲否隔;少陽所至為嚏嘔,為瘡瘍;陽明所至為浮虛;太陽所至為屈伸不利;病之常也。
%%   厥陰所至為支痛;少陰所至為驚惑,惡寒,戰慄,譫妄;太陰所至為稸滿,少陽所至為驚躁,瞀昧,暴病;陽明所至為鼽,尻陰膝髀腨(骨行)足病;太陽所至為腰痛;病之常也。
%%   厥陰所至為緛戾;少陰所至為悲妄衄衊;太陰所至為中滿霍亂吐下;少陽所至為喉痹,耳鳴嘔湧;陽明所至皴揭;太陽所至為寢汗,痙;病之常也。
%%   厥陰所至為脅痛嘔洩,少陰所至為語笑,太陰所至為重胕腫,少陽所至為暴注、瞤瘛、暴死,陽明所至為鼽嚏,太陽所至為流洩禁止,病之常也。
%%   凡此十二變者,報德以德,報化以化,報政以政,報令以令,氣高則高,氣下則下,氣後則後,氣前則前,氣中則中,氣外則外,位之常也。故風勝則動,熱勝則腫,燥勝則干,寒勝則浮,濕勝則濡洩,甚則水閉胕腫,隨氣所在,以言其變耳。
%%   帝曰:願聞其用也。岐伯曰:夫六氣之用,各歸不勝而為化。故太陰雨化,施於太陽;太陽寒化,施於少陰;少陰熱化,施於陽明;陽明燥化,施於厥陰;厥陰風化,施於太陰。各命其所在以徵之也。帝曰:自得其位何如?岐伯曰:自得其位,常化也。帝曰:願聞所在也。岐伯曰:命其位而方月可知也。
%%   帝曰:六位之氣盈虛何如?岐伯曰:太少異也,太者之至徐而常,少者暴而亡。帝曰:天地之氣盈虛何如?岐伯曰:天氣不足,地氣隨之,地氣不足,天氣從之,運居其中而常先也。惡所不勝,歸所同和,隨運歸從而生其病也。故上勝則天氣降而下,下勝則地氣遷而上,多少而差其分,微者小差,甚者大差,甚則位易氣交易,則大變生而病作矣。《大要》曰:甚紀五分,微紀七分,其差可見,此之謂也。
%%   帝曰:善。論言熱無犯熱,寒無犯寒。余欲不遠熱,不遠熱奈何?岐伯曰:悉乎哉問也!發表不遠熱,攻裡不遠寒。帝曰:不發不攻而犯寒犯熱,何如?岐伯曰:寒熱內賊,其病益甚。帝曰:願聞無病者何如?岐伯曰:無者生之,有者甚之。帝曰:生者何如?岐伯曰:不遠熱則熱至,不遠寒則寒至。寒至則堅否腹滿,痛急下利之病生矣。熱至則身熱,吐下霍亂,癰疽瘡瘍,瞀郁注下,瞤瘛腫脹,嘔,鼽衄頭痛,骨節變,肉痛,血溢血洩,淋閟之病生矣。帝曰:治之奈何?岐伯曰:時必順之,犯者治以勝也。
%%   黃帝問曰:婦人重身,毒之何如?岐伯曰:有故無殞,亦無殞也。帝曰:願聞其故何謂也?岐伯曰:大積大聚,其可犯也,衰其大半而止,過者死。
%%   帝曰:善。郁之甚者治之奈何?岐伯曰:木郁達之,火郁發之,土郁奪之,金郁洩之,水郁折之,然調其氣,過者折之,以其畏也,所謂寫之。帝曰:假者何如?岐伯曰:有假其氣,則無禁也。所謂主氣不足,客氣勝也。帝曰:至哉聖人之道!天地大化運行之節,臨御之紀,陰陽之政,寒暑之令,非夫子孰能通之!請藏之靈蘭之室,署曰《六元正紀》,非齋戒不敢示,慎傳也。


%% 刺法論七十二(亡)

%% 本病論七十三(亡)

%% \section{至真要大論篇第七十四}

%%   黃帝問曰:五氣交合,盈虛更作,余知之矣。六氣分治,司天地者,其至何如?岐伯再拜對曰:明乎哉問也!天地之大紀,人神之通應也。帝曰:願聞上合昭昭,下合冥冥奈何?岐伯曰:此道之所主,工之所疑也。
%%   帝曰:願聞其道也。岐伯曰:厥陰司天,其化以風;少陰司天,其化以熱;太陰司天,其化以濕;少陽司天,其化以火;陽明司天,其化以燥;陽司天,其化以寒。以所臨藏位,命其病者也。
%%   帝曰:地化奈何?岐伯曰:司天同候,間氣皆然。帝曰:間氣何謂?岐伯曰:司左右者,是謂間氣也。帝曰:何以異之?岐伯曰:主歲者紀歲,間氣者紀步也。帝曰:善。歲主奈何?岐伯曰:厥陰司天為風化,在泉為酸化,司氣為蒼化,間氣為動化。少陰司天為熱化,在泉為苦化,不司氣化,居氣為灼化。太陰司天為濕化,在泉為甘化,司氣為黅化,間氣為柔化。少陽司天為火化,在泉苦化,司氣為丹化,間氣為明化。陽明司天為燥化,在泉為辛化,司氣為素化,間氣為清化。太陽司天為寒化,在泉為咸化,司氣為玄化,間氣為藏化。故治病者,必明六化分治,五味五色所生,五藏所宜,乃可以言盈虛病生之緒也。
%%   帝曰:厥陰在泉而酸化先,余知之矣。風化之行也,何如?岐伯曰:風行於地,所謂本也,餘氣同法。本乎天者,天之氣也,本乎地者,地之氣也,天地合氣,六節分而萬物化生矣。故曰:謹候氣宜,無失病機,此之謂也。
%%   帝曰:其主病何如?岐伯曰:司歲備物,則無遺主矣。帝曰:先歲物何也?岐伯曰:天地之專精也。帝曰:司氣者何如?岐伯曰:司氣者主歲同,然有餘不足也。帝曰:非司歲物何謂也?岐伯曰:散也,故質同而異等也,氣味有薄厚,性用有躁靜,治保有多少,力化有淺深,此之謂也。
%%   帝曰:歲主藏害何謂?岐伯曰:以所不勝命之,則其要也。帝曰:治之奈何?岐伯曰:上淫於下,所勝平之,外淫於內,所勝治之。帝曰:善。平氣何如?岐伯曰:謹察陰陽所在而調之,以平為期,正者正治,反者反治。
%%   帝曰:夫子言察陰陽所在而調之,論言人迎與寸口相應,若引繩小大齊等,命曰平,陰之所在寸口何如?岐伯曰:視歲南北,可知之矣。帝曰:願卒聞之。岐伯曰:北政之歲,少陰在泉,則寸口不應;厥陰在泉,則右不應;太陰在泉,則左不應。南政之歲,少陰司天,則寸口不應;厥陰司天,則右不應;太陰司天,則左不應。諸不應者,反其診則見矣。帝曰:尺候何如?岐伯曰:北政之歲,三陰在下,則寸不應;三陰在上,則尺不應。南政之歲,三陰在天,則寸不應;三陰在泉,則尺不應,左右同。故曰:知其要者,一言而終,不知其要,流散無窮,此之謂也。
%%   帝曰:善。天地之氣,內淫而病何如?岐伯曰:歲厥陰在泉,風淫所勝,則地氣不明,平野昧,草乃早秀。民病灑灑振寒,善伸數欠,心痛支滿,兩脅裡急,飲食不下,鬲咽不通,食則嘔,腹脹善噫,得後與氣,則快然如衰,身體皆重。
%%   歲少陰在泉,熱淫所勝,則焰浮川澤,陰處反明。民病腹中常鳴,氣上衝胸,喘不能久立,寒熱皮膚痛,目瞑齒痛(出頁)腫,惡寒發熱如瘧,少腹中痛,腹大,蟄蟲不藏。
%%   歲太陰在泉,草乃早榮,濕淫所勝,則埃昏岩谷,黃反見黑,至陰之交。民病飲積,心痛,耳聾,渾渾焞焞,嗌腫喉痹,陰病血見,少腹痛腫,不得小便,病沖頭痛,目似脫,項似拔,腰似折,髀不可以回,膕如結,腨如別。
%%   歲少陽在泉,火淫所勝,則焰明郊野,寒熱更至。民病注洩赤白,少腹痛溺赤,甚則血便,少陰同候。
%%   歲陽明在泉,燥淫所勝,則霿霧清瞑。民病喜嘔,嘔有苦,善太息,心脅痛不能反側,甚則嗌乾麵塵,身無膏澤,足外反熱。
%%   歲太陽在泉,寒淫所勝,則凝肅慘慄。民病少腹控睾,引腰脊,上衝心痛,血見,嗌痛頷腫。
%%   帝曰:善。治之奈何?岐伯曰:諸氣在泉,風淫於內,治以辛涼,佐以苦,以甘緩之,以辛散之。熱淫於內,治以咸寒,佐以甘苦,以酸收之,以苦發之。濕淫於內,治以苦熱,佐以酸淡,以苦燥之,以淡洩之。火淫於內,治以咸冷,佐以苦辛,以酸收之,以苦發之。燥淫於內,治以苦溫,佐以甘辛,以苦下之。寒淫於內,治以甘熱,佐以苦辛,以咸寫之,以辛潤之,以苦堅之。
%%   帝曰:善。天氣之變何如?岐伯曰:厥陰司天,風淫所勝,則太虛埃昏,雲物以擾,寒生春氣,流水不冰,民病胃脘當心而痛,上支兩脅,鬲咽不通,飲食不下,舌本強,食則嘔,冷洩腹脹,溏洩,瘕水閉,蟄蟲不去,病本於脾。沖陽絕,死不治。
%%   少陰司天,熱淫所勝,怫熱至,火行其政,民病胸中煩熱,嗌干,右胠滿,皮膚痛,寒熱咳喘,大雨且至,唾血血洩,鼽衄嚏嘔,溺色變,甚則瘡瘍胕腫,肩背臂臑及缺盆中痛,心痛肺(月真),腹大滿,膨膨而喘咳,病本於肺。尺澤絕,死不治。
%%   太陰司天,濕淫所勝,則沉陰且布,雨變枯槁,胕腫骨痛,陰痹,陰痹者,按之不得,腰脊頭項痛,時眩,大便難,陰氣不用,飢不欲食,咳唾則有血,心如懸,病本於腎。太谿絕,死不治。
%%   少陽司天,火淫所勝,則溫氣流行,金政不平,民病頭痛,發熱惡寒而瘧,熱上皮膚痛,色變黃赤,傳而為水,身面胕腫,腹滿仰息,洩注赤白,瘡瘍咳唾血,煩心,胸中熱,甚則鼽衄,病本於肺。天府絕,死不治。
%%   陽明司天,燥淫所勝,則木乃晚榮,草乃晚生,筋骨內變,民病左胠脅痛,寒清於中,感而瘧,大涼革候,咳,腹中鳴,注洩鶩溏,名木斂,生菀於下,草焦上首,心脅暴痛,不可反側,嗌乾麵塵,腰痛,丈夫頹疝,婦人少腹痛,目昧眥,瘍瘡痤癰,蟄蟲來見,病本於肝。太沖絕,死不治。
%%   太陽司天,寒淫所勝,則寒氣反至,水且冰,血變於中,發為癰瘍,民病厥心痛,嘔血血洩鼽衄,善悲,時眩僕,運火炎烈,雨暴乃雹,胸腹滿,手熱肘攣,掖腫,心澹澹大動,胸脅胃脘不安,面赤目黃,善噫嗌干,甚則色炲,渴而欲飲,病本於心。神門絕,死不治。所謂動氣知其藏也。
%%   帝曰:善。治之奈何?岐伯曰:司天之氣,風淫所勝,平以辛涼,佐以苦甘,以甘緩之,以酸寫之。熱淫所勝,平以咸寒,佐以苦甘,以酸收之。濕淫所勝,平以苦熱,佐以酸辛,以苦燥之,以淡洩之。濕上甚而熱,治以苦溫,佐以甘辛,以汗為故而止。火淫所勝,平以酸冷,佐以苦甘,以酸收之,以苦發之,以酸復之,熱淫同。燥淫所勝,平以苦濕,佐以酸辛,以苦下之。寒淫所勝,平以辛熱,佐以甘苦,以咸寫之。
%%   帝曰:善。邪氣反勝,治之奈何?岐伯曰:風司於地,清反勝之,治以酸溫,佐以苦甘,以辛平之。熱司於地,寒反勝之,治以甘熱,佐以苦辛,以咸平之。濕司於地,熱反勝之,治以苦冷,佐以咸甘,以苦平之。火司於地,寒反勝之,治以甘熱,佐以苦辛,以咸平之。燥司於地,熱反勝之,治以平寒,佐以苦甘,以酸平之,以和為利。寒司於地,熱反勝之,治以咸冷,佐以甘辛,以苦平之。
%%   帝曰:其司天邪勝何如?岐伯曰:風化於天,清反勝之,治以酸溫,佐以甘苦。熱化於天,寒反勝之,治以甘溫,佐以苦酸辛。濕化於天,熱反勝之,治以苦寒,佐以苦酸。火化於天,寒反勝之,治以甘熱,佐以苦辛。燥火於天,熱反勝之,治以辛寒,佐以苦甘。寒化於天,熱反勝之,治以咸冷,佐以苦辛。
%%   帝曰:六氣相勝奈何?岐伯曰:厥陰之勝,耳鳴頭眩,憒憒欲吐,胃鬲如寒,大風數舉,裸蟲不滋,胠脅氣並,化而為熱,小便黃赤,胃脘當心而痛,上支兩脅,腸鳴飧洩,少腹痛,注下赤白,甚則嘔吐,鬲咽不通。
%%   少陰之勝,心下熱,善飢,齊下反動,氣游三焦,炎暑至,木乃津,草乃萎,嘔逆躁煩,腹滿痛,溏洩,傳為赤沃。
%%   太陰之勝,火氣內郁,瘡瘍於中,流散於外,病在胠脅,甚則心痛,熱格,頭痛喉痹項強,獨勝則濕氣內郁,寒迫下焦,痛留頂,互引眉間,胃滿,雨數至,燥化乃見,少腹滿,腰脽重強,內不便,善注洩,足下溫,頭重,足脛胕腫,飲發於中,胕腫於上。
%%   少陽之勝,熱客於胃,煩心心痛,目赤欲嘔,嘔酸善飢,耳痛溺赤,善驚譫妄,暴熱消爍,草萎水涸,介蟲乃屈,少腹痛,下沃赤白。
%%   陽明之勝,清發於中,左胠脅痛,溏洩,內為嗌塞,外發頹疝,大涼肅殺,華英改容,毛蟲乃殃,胸中不便,嗌塞而咳。
%%   太陽之勝,凝凓且至,非時水冰,羽乃後化,痔瘧發,寒厥入胃,則內生心痛,陰中乃瘍,隱曲不利,互引陰股,筋肉拘苛,血脈凝泣,絡滿色變,或為血洩,皮膚否腫,腹滿食減,熱反上行,頭項囟頂腦戶中痛,目如脫,寒入下焦,傳為濡寫。
%%   帝曰:治之奈何?岐伯曰:厥陰之勝,治以甘清,佐以苦辛,以酸寫之。少陰之勝,治以辛寒,佐以苦咸,以甘寫之。太陰之勝,治以咸熱,佐以辛甘,以苦寫之。少陽之勝,治以辛寒,佐以甘咸,以甘寫之。陽明之勝,治以酸溫,佐以辛甘,以苦洩之。太陽之勝,治以甘熱,佐以辛酸,以咸寫之。
%%   帝曰:六氣之復何如?岐伯曰:悉乎哉問也!厥陰之復,少腹堅滿,裡急暴痛,偃木飛沙,裸蟲不榮,厥心痛,汗發嘔吐,飲食不入,入而復出,筋骨掉眩,清厥,甚則入脾,食痹而吐。沖陽絕,死不治。
%%   少陰之復,燠熱內作,煩躁鼽嚏,少腹絞痛,火見燔焫,嗌燥,分注時止,氣動於左,上行於右,咳,皮膚痛,暴瘖心痛,郁冒不知人,乃灑淅惡寒,振慄譫妄,寒已而熱,渴而欲飲,少氣骨痿,隔腸不便,外為浮腫,噦噫,赤氣後化,流水不冰,熱氣大行,介蟲不復,病疿胗瘡瘍,癰疽痤痔,甚則入肺,咳而鼻淵。天府絕,死不治。
%%   太陰之復,濕變乃舉,體重中滿,食飲不化,陰氣上厥,胸中不便,飲發於中,咳喘有聲,大雨時行,鱗見於陸,頭頂痛重,而掉瘛尤甚,嘔而密默,唾吐清液,甚則入腎竅,寫無度。太谿絕,死不治。
%%   少陽之復,大熱將至,枯燥燔爇,介蟲乃耗,驚瘛咳衄,心熱煩躁,便數憎風,厥氣上行,面如浮埃,目乃瞤瘛,火氣內發,上為口麋嘔逆,血溢血洩,發而為瘧,惡寒鼓慄,寒極反熱,嗌絡焦槁,渴引水漿,色變黃赤,少氣脈萎,化而為水,傳為胕腫,甚則入肺,咳而血洩。尺澤絕,死不治。
%%   陽明之復,清氣大舉,森木蒼干,毛蟲乃厲,病生胠脅,氣歸於左,善太息,甚則心痛否滿,腹脹而洩,嘔苦咳噦,煩心,病在鬲中,頭痛,甚則入肝,驚駭筋攣。太沖絕,死不治。
%%   太陽之復,厥氣上行,水凝雨冰,羽蟲乃死。心胃生寒,胸膈不利,心痛否滿,頭痛善悲,時眩僕,食減,腰脽反痛,屈伸不便,地裂冰堅,陽光不治,少腹控睾,引腰脊,上衝心,唾出清水,及為噦噫,甚則入心,善忘善悲。神門絕,死不治。
%%   帝曰:善,治之奈何?岐伯曰:厥陰之復,治以酸寒,佐以甘辛,以酸寫之,以甘緩之。少陰之復,治以咸寒,佐以苦辛,以甘寫之,以酸收之,辛苦發之,以咸軟之。太陰之復,治以苦熱,佐以酸辛,以苦寫之,燥之,洩之。少陽之復,治以咸冷,佐以苦辛,以咸軟之,以酸收之,辛苦發之,發不遠熱,無犯溫涼,少陰同法。陽明之復,治以辛溫,佐以苦甘,以苦洩之,以苦下之,以酸補之。太陽之復,治以咸熱,佐以甘辛,以苦堅之。治諸勝復,寒者熱之,熱者寒之,溫者清之,清者溫之,散者收之,抑者散之,燥者潤之,急者緩之,堅者耎之,脆者堅之,衰者補之,強者寫之,各安其氣,必清必靜,則病氣衰去,歸其所宗,此治之大體也。
%%   帝曰:善。氣之上下,何謂也?岐伯曰:身半以上,其氣三矣,天之分也,天氣主之。身半以下,其氣三矣,地之分也,地氣主之。以名命氣,以氣命處,而言其病。半,所謂天樞也。故上勝而下俱病者,以地名之,下勝而上俱病者,以天名之。所謂勝至,報氣屈伏而未發也,復至則不以天地異名,皆如復氣為法也。
%%   帝曰:勝復之動,時有常乎?氣有必乎?岐伯曰:時有常位,而氣無必也。帝曰:願聞其道也。岐伯曰:初氣終三氣,天氣主之,勝之常也。四氣盡終氣,地氣主之,復之常也。有勝則復,無勝則否。帝曰:善。復已而勝何如?岐伯曰:勝至則復,無常數也,衰乃止耳。復已而勝,不復則害,此傷生也。帝曰:復而反病何也?岐伯曰:居非其位,不相得也,大復其勝則主勝之,故反病也,所謂火燥熱也。帝曰:治之何如?岐伯曰:夫氣之勝也,微者隨之,甚者制之。氣之復也,和者平之,暴者奪之,皆隨勝氣,安其屈伏,無問其數,以平為期,此其道也。
%%   帝曰:善。客主之勝復奈何?岐伯曰:客主之氣,勝而無復也。帝曰:其逆從何如?岐伯曰:主勝逆,客勝從,天之道也。
%%   帝曰:其生病何如?岐伯曰:厥陰司天,客勝則耳鳴掉眩,甚則咳;主勝則胸脅痛,舌難以言。少陰司天,客勝則鼽嚏頸項強,肩背瞀熱,頭痛少氣,發熱耳聾目暝,甚則胕腫血溢,瘡瘍咳喘;主勝則心熱煩躁,甚則脅痛支滿。太陰司天,客勝則首面胕腫,呼吸氣喘;主勝則胸腹滿,食已而瞀。少陽司天,客勝則丹胗外發,及為丹熛瘡瘍,嘔逆喉痹,頭痛嗌腫,耳聾血溢,內為瘛瘲;主勝則胸滿咳仰息,甚而有血,手熱。陽明司天,清復內余,則咳衄嗌塞,心鬲中熱,咳不止而白血出者死。太陽司天,客勝則胸中不利,出清涕,感寒則咳;主勝則喉嗌中鳴。
%%   厥陰在泉,客勝則大關節不利,內為痙強拘瘛,外為不便;主勝則筋骨繇並,腰腹時痛。少陰在泉,客勝則腰痛,尻股膝髀腨(骨行)足病,瞀熱以酸,胕腫不能久立,溲便變;主勝則厥氣上行,心痛發熱,鬲中,眾痹皆作,發於胠脅,魄汗不藏,四逆而起。太陰在泉,客勝則足痿下重,便溲不時,濕客下焦,發而濡寫,及為腫,隱曲之疾;主勝則寒氣逆滿,食飲不下,甚則為疝。少陽在泉,客勝則腰腹痛而反惡寒,甚則下白溺白;主勝則熱反上行而客於心,心痛發熱,格中而嘔。少陰同候。陽明在泉,客勝則清氣動下,少腹堅滿而數便寫;主勝則腰重腹痛,少腹生寒,下為鶩溏,則寒厥於腸,上衝胸中,甚則喘,不能久立。太陽在泉,寒復內余,則腰尻痛,屈伸不利,股脛足膝中痛。
%%   帝曰:善,治之奈何?岐伯曰:高者抑之,下者舉之,有餘折之,不足補之,佐以所利,和以所宜,必安其主客,適其寒溫,同者逆之,異者從之。
%%   帝曰:治寒以熱,治熱以寒,氣相得者逆之,不相得者從之,余己知之矣。其於正味何如?岐伯曰:木位之主,其寫以酸,其補以辛。火位之主,其寫以甘,其補以咸。土位之主,其寫以苦,其補以甘。金位之主,其寫以辛,其補以酸。水位之主,其寫以咸,其補以苦。厥陰之客,以辛補之,以酸寫之,以甘緩之。少陰之客,以咸補之,以甘寫之,以咸收之。太陰之客,以甘補之,以苦寫之,以甘緩之。少陽之客,以咸補之,以甘寫之,以咸軟之。陽明之客,以酸補之。以辛寫之,以苦洩之。太陽之客,以苦補之,以咸寫之,以苦堅之,以辛潤之。開發腠理,致津液通氣也。
%%   帝曰:善。願聞陰陽之三也何謂?岐伯曰:氣有多少,異用也。帝曰:陽明何謂也?岐伯曰:兩陽合明也。帝曰:厥陰何也?岐伯曰:兩陰交盡也。
%%   帝曰:氣有多少,病有盛衰,治有緩急,方有大小,願聞約奈何?岐伯曰:氣有高下,病有遠近,證有中外,治有輕重,適其至所為故也。《大要》曰:君一臣二,奇之制也;君二臣四,偶之制也;君二臣三,奇之制也;君三臣六,偶之制也。故曰:近者奇之,遠者偶之,汗者不以奇,下者不以偶,補上治上制以緩,補下治下制以急,急則氣味厚,緩則氣味薄,適其至所,此之謂也。病所遠而中道氣味之者,食而過之,無越其制度也。是故平氣之道,近而奇偶,制小其服也。遠而奇偶,制大其服也。大則數少,小則數多。多則九之,少則二之。奇之不去則偶之,是謂重方。偶之不去,則反佐以取之,所謂寒熱溫涼,反從其病也。
%%   帝曰:善。病生於本,余知之矣。生於標者,治之奈何?岐伯曰:病反其本,得標之病,治反其本,得標之方。
%%   帝曰:善。六氣之勝,何以候之?岐伯曰:乘其至也。清氣大來,燥之勝也,風木受邪,肝病生焉。熱氣大來,火之勝也,金燥受邪,肺病生焉。寒氣大來,水之勝也,火熱受邪,心病生焉。濕氣大來,土之勝也,寒水受邪,腎病生焉。風氣大來,木之勝也,土濕受邪,脾病生焉。所謂感邪而生病也。乘年之虛,則邪甚也。失時之和,亦邪甚也。遇月之空,亦邪甚也。重感於邪,則病危矣。有勝之氣,其必來復也。
%%   帝曰:其脈至何如?岐伯曰:厥陰之至,其脈弦,少陰之至,其脈鉤,太陰之至,其脈沉,少陽之至,大而浮,陽明之至,短而濇,太陽之至,大而長。至而和則平,至而甚則病,至而反者病,至而不至者病,未至而至者病,陰陽易者危。
%%   帝曰:六氣標本,所從不同,奈何?岐伯曰:氣有從本者,有從標本者,有不從標本者也。帝曰:願卒聞之。岐伯曰:少陽太陰從本,少陰太陽從本從標,陽明厥陰,不從標本,從乎中也。故從本者,化生於本,從標本者,有標本之化,從中者,以中氣為化也。帝曰:脈從而病反者,其診何如?岐伯曰:脈至而從,按之不鼓,諸陽皆然。帝曰:諸陰之反,其脈何如?岐伯曰:脈至而從,按之鼓甚而盛也。
%%   是故百病之起,有生於本者,有生於標者,有生於中氣者,有取本而得者,有取標而得者,有取中氣而得者,有取標本而得者,有逆取而得者,有從取而得者。逆,正順也。若順,逆也。故曰:知標與本,用之不殆,明知逆順,正行無問。此之謂也。不知是者,不足以言診,足以亂經。故《大要》曰:粗工嘻嘻,以為可知,言熱未已,寒病復始,同氣異形,迷診亂經,此之謂也,夫標本之道,要而博,小而大,可以言一而知百病之害,言標與本,易而勿損,察本與標,氣可令調,明知勝復,為萬民式,天之道畢矣。
%%   帝曰:勝復之變,早晏何如?岐伯曰:夫所勝者,勝至已病,病已慍慍,而復已萌也。夫所復者,勝盡而起,得位而甚,勝有微甚,復有少多,勝和而和,勝虛而虛,天之常也。帝曰:勝復之作,動不當位,或後時而至,其故何也?岐伯曰:夫氣之生,與其化衰盛異也。寒暑溫涼盛衰之用,其在四維。故陽之動,始於溫,盛於暑;陰之動,始於清,盛於寒。春夏秋冬,各差其分。故《大要》曰:彼春之暖,為夏之暑,彼秋之忿,為冬之怒,謹按四維,斥候皆歸,其終可見,其始可知。此之謂也。帝曰:差有數乎?岐伯曰:又凡三十度也。帝曰:其脈應皆何如?岐伯曰:差同正法,待時而去也。《脈要》曰:春不沉,夏不弦,冬不濇,秋不數,是謂四塞。沉甚曰病,弦甚曰病,澀甚曰病,數其曰病,參見曰病,復見曰病,未去而去曰病,去而不去曰病,反者死。故曰:氣之相守司也,如權衡之不得相失也。夫陰陽之氣,清靜則生化治,動則苛疾起,此之謂也。
%%   帝曰:幽明何如?岐伯曰:兩陰交盡故曰幽,兩陽合明故曰明,幽明之配,寒暑之異也。帝曰:分至何如?岐伯曰:氣至之謂至,氣分之謂分,至則氣同,分則氣異,所謂天地之正紀也。
%%   帝曰:夫子言春秋氣始於前,冬夏氣始於後,余已知之矣。然六氣往復,主歲不常也,其補寫奈何?岐伯曰:上下所主,隨其攸利,正其味,則其要也,左右同法。《大要》曰:少陽之主,先甘後咸;陽明之主,先辛後酸;太陽之主,先咸後苦;厥陰之主,先酸後辛;少陰之主,先甘後咸;太陰之主,先苦後甘。佐以所利,資以所生,是謂得氣。
%%   帝曰:善。夫百病之生也,皆生於風寒暑濕燥火,以之化之變也。經言盛者寫之,虛者補之,余錫以方士,而方士用之,尚未能十全,余欲令要道必行,桴鼓相應,猶拔刺雪汙,工巧神聖,可得聞乎?岐伯曰:審察病機,無失氣宜,此之謂也。帝曰:願聞病機何如?岐伯曰:諸風掉眩,皆屬於肝。諸寒收引,皆屬於腎。諸氣膹郁,皆屬於肺。諸濕腫滿,皆屬於脾。諸熱瞀瘈,皆屬於火。諸痛癢瘡,皆屬於心。諸厥固洩,皆屬於下。諸痿喘嘔,皆屬於上。諸禁鼓慄,如喪神守,皆屬於火。諸痙項強,皆屬於濕。諸逆衝上,皆屬於火。諸脹腹大,皆屬於熱。諸躁狂越,皆屬於火。諸暴強直,皆屬於風。諸病有聲,鼓之如鼓,皆屬於熱。諸病胕腫,痛酸驚駭,皆屬於火。諸轉反戾,水液渾濁,皆屬於熱。諸病水液,澄澈清冷,皆屬於寒。諸嘔吐酸,暴注下迫,皆屬於熱。故《大要》曰:謹守病機,各司其屬,有者求之,無者求之,盛者責之,虛者責之,必先五勝,疏其血氣,令其調達,而致和平,此之謂也。
%%   帝曰:善,五味陰陽之用何如?岐伯曰:辛甘發散為陽,酸苦湧洩為陰,鹹味湧洩為陰,淡味滲洩為陽。六者或收或散,或緩或急,或燥或潤,或軟或堅,以所利而行之,調其氣,使其平也。帝曰:非調氣而得者,治之奈何?有毒無毒,何先何後?願聞其道。岐伯曰:有毒無毒,所治為主,適大小為制也。帝曰:請言其制。岐伯曰:君一臣二,制之小也;君一臣三佐五,制之中也;君一臣三佐九,制之大也。寒者熱之,熱者寒之,微者逆之,甚者從之,堅者削之,客者除之,勞者溫之,結者散之,留者政之,燥者濡之,急者緩之,散者收之,損者溫之,逸者行之,驚者平之,上之下之,摩之浴之,薄之劫之,開之發之,適事為故。
%%   帝曰:何謂逆從?岐伯曰:逆者正治,從者反治,從少從多,觀其事也。帝曰:反治何謂?岐伯曰:熱因寒用,寒因熱用,塞因塞用,通因通用,必伏其所主,而先其所因,其始則同,其終則異,可使破積,可使潰堅,可使氣和,可使必已。帝曰:善。氣調而得者何如?岐伯曰:逆之從之,逆而從之,從而逆之,疏氣令調,則其道也。
%%   帝曰:善。病之中外何如?岐伯曰:從內之外者調其內;從外之內者治其外;從內之外而盛於外者,先調其內而後治其外;從外之內而盛於內者,先治其外,而後調其內;中外不相及,則治主病。
%%   帝曰:善。火熱復,惡寒發熱,有如瘧狀,或一日發,或間數日發,其故何也?岐伯曰:勝復之氣,會遇之時,有多少也。陰氣多而陽氣少,則其發日遠;陽氣多而陰氣少,則其發日近。此勝復相薄,盛衰之節,瘧亦同法。
%%   帝曰:論言治寒以熱,治熱以寒,而方士不能廢繩墨而更其道也。有病熱者,寒之而熱,有病寒者,熱之而寒,二者皆在,新病復起,奈何治?岐伯曰:諸寒之而熱者取之陰,熱之而寒者取之陽,所謂求其屬也。帝曰:善。服寒而反熱,服熱而反寒,其故何也?岐伯曰:治其王氣,是以反也。帝曰:不治王而然者何也?岐伯曰:悉乎哉問也!不治五味屬也。夫五味入胃,各歸所喜,攻酸先入肝,苦先入心,甘先入脾,辛先入肺,咸先入腎,久而增氣,物化之常也。氣增而久,夭之由也。
%%   帝曰:善。方制君臣何謂也?岐伯曰:主病之謂君,佐君之謂臣,應臣之謂使,非上下三品之謂也。帝曰:三品何謂/岐伯曰:所以明善惡之殊貫也。
%%   帝曰:善。病之中外何如?岐伯曰:調氣之方,必別陰陽,定其中外,各守其鄉。內者內治,外者外治,微者調之,其次平之,盛者奪之,汗之下之,寒熱溫涼,衰之以屬,隨其攸利,謹道如法,萬舉萬全,氣血正平,長有天命。帝曰:善。

%% \section{著至教論篇第七十五}

%%   黃帝坐明堂,召雷公而問之曰:子知醫之道乎?雷公對曰:誦而頗能解,解而未能別,別而未能明,明而未能彰,足以治群僚,不足至侯王。願得受樹天之度,四時陰陽合之,別星辰與日月光,以彰經術,後世益明,上通神農,著至教,疑於二皇。帝曰:善!無失之,此皆陰陽表裡上下雌雄相輸應也,而道上知天文,下知地理,中知人事,可以長久,以教眾庶,亦不疑殆,醫道論篇,可傳後世,可以為寶。
%%   雷公曰:請受道,諷誦用解。帝曰:子不聞《陰陽傳》乎!曰:不知。曰:夫三陽天為業,上下無常,合而病至,偏害陰陽。雷公曰:三陽莫當,請聞其解。帝曰:三陽獨至者,是三陽並至,並至如風雨,上為巔疾,下為漏病,外無期,內無正,不中經紀,診無上下,以書別。雷公曰:臣治疏愈,說意而已。帝曰:三陽者,至陽也,積並則為驚,病起疾風,至如礔礪,九竅皆塞,陽氣滂溢,干嗌喉塞,並於陰,則上下無常,薄為腸澼,此謂三陽直心,坐不得起,臥者便身全。三陽之病,且以知天下,何以別陰陽,應四時,合之五行。
%%   雷公曰:陽言不別,陰言不理,請起受解,以為至道。帝曰:子若受傳,不知合至道以惑師教,語子至道之要。病傷五藏,筋骨以消,子言不明不別,是世主學盡矣。腎且絕,惋惋日暮,從容不出,人事不殷。


%% \section{示從容論篇第七十六}

%%   黃帝燕坐,召雷公而問之曰:汝受術誦書者,若能覽觀雜學,及於比類,通合道理,為余言子所長,五藏六府,膽胃大小腸,脾胞膀胱,腦髓涕唾,哭泣悲哀,水所從行,此皆人之所生,治之過失,子務明之,可以十全,即不能知,為世所怨。雷公曰:臣請誦《脈經上下篇》,甚眾多矣,別異比類,猶未能以十全,又安足以明之。
%%   帝曰:子別試通五藏之過,六府之所不和,針石所敗,毒藥所宜,湯液滋味,具言其狀,悉言以對,請問不知。雷公曰:肝虛腎虛脾虛,皆令人體重煩冤,當投毒藥刺灸砭石湯液,或已,或不已,願聞其解。帝曰:公何年之長而問之少,余真問以自謬也。吾問子窈冥,子言《上下篇》以對,何也?夫脾虛浮似肺,腎小浮似脾,肝急沉散似腎,此皆工之所時亂也,然從容得之。若夫三藏土木水參居,此童子之所知,問之何也?
%%   雷公曰:於此有人,頭痛,筋攣骨重,怯然少氣,噦噫腹滿,時驚,不嗜臥,此何藏之發也?脈浮而弦,切之石堅,不知其解,復問所以三藏者,以知其比類也。帝曰:夫從容之謂也。夫年長則求之於府,年少則求之於經,年壯則求之於藏。今子所言皆失,八風菀熟,五藏消爍,傳邪相受。夫浮而弦者,是腎不足也。沉而石者,是腎氣內著也。怯然少氣者,是水道不行,形氣消索也。咳嗽煩冤者,是腎氣之逆也。一人之氣,病在一藏也。若言三藏俱行,不在法也。
%%   雷公曰:於此有人,四支解墯,咳喘血洩,而愚診之,以為傷肺,切脈浮大而緊,愚不敢治,粗工下砭石,病癒多出血,血止身輕,此何物也?帝曰:子所能治,知亦眾多,與此病失矣。譬以鴻飛,亦沖於天。夫聖人之治病,循法守度,援物比類,化之冥冥,循上及下,何必守經。今夫脈浮大虛者,是脾氣之外絕,去胃外歸陽明也。夫二火不勝三水,是以脈亂而無常也。四支解墯,此脾精之不行也。咳喘者,是水氣並陽明也。血洩者,脈急血無所行也。若夫以為傷肺者,由失以狂也。不引比類,是知不明也。夫傷肺者,脾氣不守,胃氣不清,經氣不為使,真藏壞決,經脈傍絕,五藏漏洩,不衄則嘔,此二者不相類也。譬如天之無形,地之無理,白與黑相去遠矣。是失,吾過矣。以子知之,故不告子,明引比類從容,是以名曰診輕,是謂至道也。


%% \section{疏五過論篇第七十七}

%%   黃帝曰:嗚呼遠哉!閔閔乎若視深淵,若迎浮雲,視深淵尚可測,迎浮雲莫知其際。聖人之術,為萬民式,論裁志意,必有法則,循經守數,接循醫事,為萬民副,故事有五過四德,汝知之乎?雷公避席再拜曰:臣年幼小,蒙愚以惑,不聞五過與四德,比類形名,虛引其經,心無所對。
%%   帝曰:凡未診病者,必問嘗貴後賤,雖不中邪,病從內生,名曰脫營。嘗富後貧,名曰失精,五氣留連,病有所並。醫工診之,不在藏府,不變軀形,診之而疑,不知病名。身體日減,氣虛無精,病深無氣,灑灑然時驚,病深者,以其外耗於衛,內奪於榮。良工所失,不知病情,此亦治之一過也。
%%   凡欲診病者,必問飲食居處,暴樂暴苦,始樂後苦,皆傷精氣,精氣竭絕,形體毀沮。暴怒傷陰,暴喜傷陽,厥氣上行,滿脈去形。愚醫治之,不知補寫,不知病情,精華日脫,邪氣乃並,此治之二過也。
%%   善為脈者,必以比類奇恆,從容知之,為工而不知道,此診之不足貴,此治之三過也。
%%   診有三常,必問貴賤,封君敗傷,及欲侯王。故貴脫勢,雖不中邪,精神內傷,身必敗亡。始富後貧,雖不傷邪,皮焦筋屈,痿躄為攣。醫不能嚴,不能動神,外為柔弱,亂至失常,病不能移,則醫事不行,此治之四過也。
%%   凡診者必知終始,有知余緒,切脈問名,當合男女。離絕菀結,憂恐喜怒,五藏空虛,血氣離守,工不能知,何術之語。嘗富大傷,斬筋絕脈,身體復行,令澤不息。故傷敗結積,留薄歸陽,膿積寒炅。粗工治之,亟刺陰陽,身體解散,四支轉筋,死日有期,醫不能明,不問所發,唯言死日,亦為粗工,此治之五過也。
%%   凡此五者,皆受術不通,人事不明也。故曰:聖人之治病也,必知天地陰陽,四時經紀,五藏六府,雌雄表裡,刺灸砭石,毒藥所主,從容人事,以明經道,貴賤貧富,各異品理,問年少長,勇怯之理,審於分部,知病本始,八正九候,診必副矣。
%%   治病之道,氣內為寶,循求其理,求之不得,過在表裡。守數據治,無失俞理,能行此術,終身不殆。不知俞理,五藏菀熟,癰發六府,診病不審,是謂失常。謹守此治,與經相明,《上經》《下經》,揆度陰陽,奇恆五中,決以明堂,審於終始,可以橫行。


%% \section{徵四失論篇第七十八}

%%   黃帝在明堂,雷公侍坐,黃帝曰:夫子所通書受事眾多矣,試言得失之意,所以得之,所以失之。雷公對曰:循經受業,皆言十全,其時有過失者,請聞其事解也。
%%   帝曰:子年少,智未及邪,將言以雜合耶?夫經脈十二,絡脈三百六十五,此皆人之所明知,工之所循用也。所以不十全者,精神不專,志意不理,外內相失,故時疑殆。診不知陰陽逆從之理,此治之一失矣。
%%   受師不卒,妄作雜術,謬言為道,更名自功,妄用砭石,後遺身咎,此治之二失也。
%%   不適貧富貴賤之居,坐之薄厚,形之寒溫,不適飲食之宜,不別人之勇怯,不知比類,足以自亂,不足以自明,此治之三失也。
%%   診病不問其始,憂患飲食之失節,起居之過度,或傷於毒,不先言此,卒持寸口,何病能中,妄言作名,為所窮,此治之四失也。
%%   是以世人之語者,馳千里之外,不明尺寸之論,診無人事。治數之道,從容之葆,坐持寸口,診不中五脈,百病所起,始以自怨,遺師其咎。是故治不能循理,棄術於市,妄治時愈,愚心自得。嗚呼!窈窈冥冥,熟知其道?道之大者,擬於天地,配於四海,汝不知道之諭,受以明為晦。

%% \section{陰陽類論篇第七十九}

%%   孟春始至,黃帝燕坐,臨觀八極,正八風之氣,而問雷公曰:陰陽之類,經脈之道,五中所主,何藏最貴?雷公對曰:春甲乙青,中主肝,治七十二日,是脈之主時,臣以其藏最貴。帝曰:卻念上下經,陰陽從容,子所言最貴,其下也。
%%   雷公致齋七日,旦復侍坐。帝曰:三陽為經,二陽為維,一陽為游部,此知五藏終始。三陽為表,二陰為裡,一陰至絕,作朔晦,卻具合以正其理。雷公曰:受業未能明。
%%   帝曰:所謂三陽者,太陽為經,三陽脈,至手太陰,弦浮而不沉,決以度,察以心,合之陰陽之論。所謂二陽者,陽明也,至手太陰,弦而沉急不鼓,炅至以病皆死。一陽者,少陽也,至手太陰,上連人迎,弦急懸不絕,此少陽之病也,專陰則死。
%%   三陰者,六經之所主也,交於太陰,伏鼓不浮,上空志心。二陰至肺,其氣歸膀胱,外連脾胃。一陰獨至,經絕,氣浮不鼓,鉤而滑。此六脈者,乍陰乍陽,交屬相併,繆通五藏,合於陰陽,先至為主,後至為客。
%%   雷公曰:臣悉盡意,受傳經脈,頌得從容之道,以合《從容》,不知陰陽,不知雌雄。帝曰:三陽為父,二陽為衛,一陽為紀。三陰為母,二陰為雌,一陰為獨使。
%%   二陽一陰,陽明主病,不勝一陰,軟而動,九竅皆沉。三陽一陰,太陽脈勝,一陰不能止,內亂五藏,外為驚駭。二陰二陽,病在肺,少陰脈沉,勝肺傷脾,外傷四支。二陰二陽皆交至,病在腎,罵詈妄行,巔疾為狂。二陰一陽,病出於腎,陰氣客遊於心脘,下空竅堤,閉塞不通,四支別離。一陰一陽代絕,此陰氣至心,上下無常,出入不知,喉咽乾燥,病在土脾。二陽三陰,至陰皆在,陰不過陽,陽氣不能止陰,陰陽並絕,浮為血瘕,沉為膿胕。陰陽皆壯,下至陰陽。上合昭昭,下合冥冥,診決生死之期,遂合歲首。
%%   雷公曰:請問短期。黃帝不應。雷公復問。黃帝曰:在經論中。雷公曰:請聞短期。黃帝曰:冬三月之病,病合於陽者,至春正月脈有死徵,皆歸出春。冬三月之病,在理已盡,草與柳葉皆殺,春陰陽皆絕,期在孟春。春三月之病,曰陽殺,陰陽皆絕,期在草干。夏三月之病,至陰不過十日,陰陽交,期在溓水。秋三月之病,三陽俱起,不治自已。陰陽交合者,立不能坐,坐不能起。三陽獨至,期在石水。二陰獨至,期在盛水。

%% \section{方盛衰論篇第八十}

%%   雷公請問氣之多少,何者為逆,何者為從。黃帝答曰:陽從左,陰從右,老從上,少從下。是以春夏歸陽為生,歸秋冬為死,反之則歸秋冬為生,是以氣多少,逆皆為厥。
%%   問曰:有餘者厥耶?答曰:一上不下,寒厥到膝,少者秋冬死,老者秋冬生。氣上不下,頭痛巔疾,求陽不得,求陰不審,五部隔無徵,若居曠野,若伏空室,綿綿乎屬不滿日。
%%   是以少氣之厥,令人妄夢,其極至迷。三陽絕,三陰微,是為少氣。
%%   是以肺氣虛,則使人夢見白物,見人斬血藉藉,得其時,則夢見兵戰。腎氣虛,則使人夢見舟船溺人,得其時,則夢伏水中,若有畏恐。肝氣虛,則夢見菌香生草,得其時,則夢伏樹下不敢起。心氣虛,則夢救火陽物,得其時,則夢燔灼。脾氣虛,則夢飲食不足,得其時,則夢築垣蓋屋。此皆五藏氣虛,陽氣有餘,陰氣不足。合之五診,調之陰陽,以在經脈。
%%   診有十度度人:脈度,藏度,肉度,筋度,俞度。陰陽氣盡,人病自具。脈動無常,散陰頗陽,脈脫不具,診無常行,診必上下,度民君卿。受師不卒,使術不明,不察逆從,是為妄行,持雌失雄,棄陰附陽,不知併合,診故不明,傳之後世,反論自章。
%%   至陰虛,天氣絕;至陽盛,地氣不足。陰陽並交,至人之所行。陰陽交並者,陽氣先至,陰氣後至。是以聖人持診之道,先後陰陽而持之,《奇恆之勢》乃六十首,診合微之事,追陰陽之變,章五中之情,其中之論,取虛實之要,定五度之事,知此乃足以診。是以切陰不得陽,診消亡,得陽不得陰,守學不湛,知左不知右,知右不知左,知上不知下,知先不知後,故治不久。知醜知善,知病知不病,知高知下,知坐知起,知行知止,用之有紀,診道乃具,萬世不殆。起所有餘,知所不足。度事上下,脈事因格。是以形弱氣虛,死;形氣有餘,脈氣不足,死。脈氣有餘,形氣不足,生。
%%   是以診有大方,坐起有常,出入有行,以轉神明,必清必淨,上觀下觀,司八正邪,別五中部,按脈動靜,循尺滑濇,寒溫之意,視其大小,合之病能,逆從以得,復知病名,診可十全,不失人情。故診之,或視息視意,故不失條理,道甚明察,故能長久。不知此道,失經絕理,亡言妄期,此謂失道。

%% \section{解精微論篇第八十一}

%%   黃帝在明堂,雷公請曰:臣授業,傳之行教以經論,從容形法,陰陽刺灸,湯藥所滋,行治有賢不肖,未必能十全。若先言悲哀喜怒,燥濕寒暑,陰陽婦女,請問其所以然者,卑賤富貴,人之形體,所從群下,通使臨事以適道術,謹聞命矣。請問有毚愚僕漏之問,不在經者,欲聞其狀。帝曰:大矣。
%%   公請問:哭泣而淚不出者,若出而少涕,其故何也?帝曰:在經有也。復問:不知水所從生,涕所出也。帝曰:若問此者,無益於治也,工之所知,道之所生也。
%%   夫心者,五藏之專精也,目者,其竅也,華色者,其榮也,是以人有德也,則氣和於目,有亡,憂知於色。是以悲哀則泣下,泣下水所由生。水宗者,積水也,積水者,至陰也,至陰者,腎之精也。宗精之水所以不出者,是精持之也。輔者裹之,故水不行也。夫水之精為志,火之精為神,水火相感,神志俱悲,是以目之水生也。故諺言曰:心悲名曰志悲,志與心精共湊於目也。是以俱悲則神氣傳於心,精上不傳於志,而志獨悲,故泣出也。泣涕者,腦也,腦者,陰也,髓者,骨之充也,故腦滲為涕。志者骨之主也,是以水流而涕從之者,其行類也。夫涕之與泣者,譬如人之兄弟,急則俱死,生則俱生,其志以神悲,是以涕泣俱出而橫行也。夫人涕泣俱出而相從者,所屬之類也。
%%   雷公曰:大矣。請問人哭泣而淚不出者,若出而少,涕不從之何也?帝曰:夫泣不出者,哭不悲也。不泣者,神不慈也。神不慈則志不悲,陰陽相持,泣安能獨來。夫志悲者惋,惋則沖陰,沖陰則志去目,志去則神不守精,精神去目,涕泣出也。
%%   且子獨不誦不念夫經言乎,厥則目無所見。夫人厥則陽氣並於上,陰氣並於下。陽並於上,則火獨光也;陰並於下則足寒,足寒則脹也。夫一水不勝五火,故目盲。是以沖風,泣下而不止。夫風之中目也,陽氣內守於精,是火氣燔目,故見風則泣下也。有以比之,夫火疾風生乃能雨,此之類也。




%% \section{刺法論篇第七十二(遺篇)}

%% 黃帝問曰:升降不前,氣交有變,即成暴郁,余已知之。何如預救生靈,可得卻乎?岐伯稽首再拜對曰:昭乎哉問!臣聞夫子言,既明天元,須窮刺法,可以折郁扶運,補弱全真,寫盛蠲余,令除斯苦。

%% 帝曰:願卒聞之。岐伯曰:升之不前,即有期凶也。木欲升而天柱窒抑之,木欲發郁,亦須待時,當刺足厥陰之井。火欲升而天蓬窒抑之,火欲發郁,亦須待時,君火相火同刺包絡之熒。土欲升而天沖窒抑之,土欲發郁,亦須待時,當刺足太陰之俞。金欲升而天英窒抑之,金欲發郁,亦須待時,當刺手太陰之經。水欲升而天芮窒抑之,水欲發郁,亦須待時,當刺足少陰之合。

%% 帝曰:升之不前,可以預備,願聞其降,可能先防。岐伯曰:既明其升。必達其降也,升降之道,皆可先治也。木欲降而地晶窒抑之,降而不入,抑之郁發,散而可得位,降而郁發,暴如天間之待時也。降而不下,郁可速矣,降可折其所勝也,當刺手太陰之所出,刺手陽明之所入。火欲降,而地玄窒抑之,降而不入,抑之郁發,散而可矣。當折其所勝,可散其郁,當刺足少陰之所出,刺足太陽之所入。土欲降而地蒼窒抑之,降而不下,抑之郁發,散而可入,當折其勝,可散其郁,當刺足厥陰之所出,刺足少陽之所入,金欲降而地彤窒抑,降而不下,抑之郁發,散而可入,當折其勝,可散其郁,當刺心包絡所出,制手少陽所入也。水欲降而地阜窒抑之,降而不下,抑之郁發,散而可入,當折其土,可散其郁,當刺足太陰之所出,刺足陽明之所入。

%% 帝曰:五運之至有前後,與升降往來,有所承抑之,可得聞乎刺法?岐伯曰:當取其化源也。是故太過取之,不及資之,太過取之,次抑其郁,取其運之化源,令折郁氣;不及扶資,以扶運氣,以避虛邪也。資取之法,令出《密語》。黃帝問曰:升降之刺,以知其要。願聞司天未得遷正,使司化之失其常政,即萬化之或其皆妄,然與民為病,可得先除,欲濟群生,願聞其說。岐伯稽首再拜曰:悉乎哉問!言其至理,聖念慈憫,欲濟群生,臣乃盡陳斯道,可申洞微。太陽復布,即厥陰不遷正,不遷正,氣塞於止,當寫足厥陰之所流。厥陰復布,少陰不遷正,不遷正,即氣塞於上,當刺心包絡脈之所流。少陰復布,太陰不遷正,不遷正,即氣留於上,當刺足太陰之所流。太陰復布,少陽不遷正,不遷正,則氣塞未通,當刺手少陽之所流。少陽復布,則陽明不遷正,不遷正,則氣未通上,當刺手太陰之所流。陽明復布,太陽遷正,不遷正,則復塞其氣,當刺足少陰之所流。

%% 帝曰:遷正不前,以通其要。願聞不退,欲折其餘,無令過失,可得明乎?岐伯曰:氣過有餘,復作布正,是名不退位也。使地氣不得後化,新司天未可遷正,故復布化令如故也。巳亥之歲,天數有餘,故厥陰不退位也,風行於上,木化布天,當刺足厥陰之所入。子午之歲,天數有餘,故少陰不退位也,熱行於上,火余化布天,當刺手厥陰之所入。丑未之歲,天數有餘,故太陰不退位也,濕行於上,雨化布天,當刺足太陰之所入。寅申之歲,天數有餘,故少陽不退位也,熱行於上,火化布天,當刺手少陽所入。卯酉之歲,天數有餘,故陽明不退位也,金行於上,燥化布天,當刺手太陰之所入。辰戌之歲,天數有餘,故太陽不退位也,寒行於上,凜水化布天,當刺足少陰之所入。故天地氣逆,化成民病,以法刺之,預可平痾。

%% 黃帝問曰:剛柔二干,失守其位,使天運之氣皆虛乎?與民為病,可得平乎?岐伯曰:深乎哉問!明其奧旨,天地迭移,三年化疫,是謂根之可見,必有逃門。

%% 假令甲子剛柔失守,剛未正,柔孤而有虧,時序不令,即音律非從,如此三年,變大疫也。詳其微甚。察其淺深,欲至而可刺,刺之當先補腎俞,次三日,可刺足太陰之所注。又有下位已卯不至,而甲子孤立者,次三年作土癘,其法補寫,一如甲子同法也。其刺以畢,又不須夜行及遠行,令七日潔,清靜齋戒,所有自來。腎有久痛者,可以寅時面向南,淨神不亂思,閉氣不息七遍,以引頸嚥氣順之,如咽甚硬物,如此七遍後,餌舌下津令無數。

%% 假令丙寅剛柔失守,上剛干失守,下柔不可獨主之,中水運非太過,不可執法而定之。布天有餘,而失守上正,天地不合,即律呂音異,如此即天運失序,後三年變疫。詳其微甚,差有大小,徐至即後三年,至甚即首三年,當先補心俞,次五日,可刺腎之所入。又有下位地甲子辛已柔不附剛,亦名失守,即地運皆虛,後三年變水癘,即刺法皆如此矣。其刺如華,慎其大喜欲情於中,如不忌,即其氣復散也,令靜七日,心欲實,令少思。

%% 假令庚辰剛柔失守,上位失守,下位無合,乙庚金運,故非相招,布天未退,中運勝來,上下相錯,謂之失守,姑洗林鐘,商音不應也。如此則天運化易,三年變大疫。詳天數,差的微甚,微即微,三年至,甚即甚,三年至,當先補肝俞,次三日,可刺肺之所行。刺畢,可靜神七日,慎勿大怒,怒必真氣卻散之。又或在下地甲子乙未失守者,即乙柔干,即上庚獨治之,亦名失守者,即天運孤主之,三年變癘,名曰金癘,其至待時也。詳其地數之等差,亦推其微甚,可知遲速耳。諸位乙庚失守,刺法同。肝欲平,即勿怒。

%% 假令壬午剛柔失守,上壬未近正,下丁獨然,即雖陽年,虧及不同,上下失守,相招其有期,差之微甚,各有其數也,律呂二角,失而不和,同音有日,微甚如見,三年大疫。當刺脾之俞,次三日,可刺肝之所出也。刺畢,靜神七日,勿大醉歌樂,其氣復散,又勿飽食,勿食生物,欲令脾實,氣無滯飽,無久坐,食無太酸,無食一切生物,宜甘宜淡。又或地下甲子丁酉失守其位,未得中司,即氣不當位,下不與壬奉合者,亦名失守,非名合德,故柔不附剛,即地運不合,三年變癘,其刺法亦如木疫之法。

%% 假令戊申剛柔失守,戊癸雖火運,陽年不太過也,上失其剛,柔地獨主,其氣不正,故有邪干,迭移其位,差有淺深,欲至將合,音律先同,如此天運失時,三年之中,火疫至矣,當刺肺之俞。刺畢,靜神七日,勿大悲傷也,悲傷即肺動,而其氣復散也,人欲實肺者,要在息氣也。又或地下甲子癸亥失守者,即柔失守位也,即上失其剛也。即亦名戊癸不相合德者也,即運與地虛,後三年變癘,即名火癘。

%% 是故立地五年,以明失守,以窮法刺,於是疫之與癘,即是上下剛柔之名也,窮歸一體也。即刺疫法,只有五法,即總其諸位失守,故只歸五行而統之也。

%% 黃帝曰:余聞五疫之至,皆相梁易,無問大小,病狀相似,不施救療,如何可得不相移易者?岐伯曰:不相染者,正氣存內,邪氣可幹,避其毒氣,天牝從來,復得其往,氣出於腦,即不邪干。氣出於腦,即室先想心如日,欲將入於疫室,先想青氣自肝而出,左行於東,化作林木;次想白氣自肺而出,右行於西,化作戈甲;次想赤氣自心而出,南行於上,化作焰明;次想黑氣自腎而出,北行於下,化作水;次想黃氣自脾而出,存於中央,化作土。五氣護身之畢,以想頭上如北斗之煌煌,然後可入於疫室。又一法,於春分之日,日未出而吐之。又一法,於雨水日後,三浴以藥洩汗。又一法,小金丹方:辰砂二兩,水磨雄黃一兩,葉子雌黃一兩,紫金半兩,同入合中,外固,了地一尺築地實,不用爐,不須藥製,用火二十斤鍛了也;七日終,候冷七日取,次日出合子埋藥地中,七日取出,順日研之三日,煉白沙蜜為丸,如梧桐子大,每日望東吸日華氣一口,冰水一下丸,和氣咽之,服十粒,無疫干也。

%% 黃帝問曰:人虛即神遊失守位,使鬼神外干,是致夭亡,何以全真?願聞刺法。岐伯稽首再拜曰:昭乎哉問!謂神移失守,雖在其體,然不致死,或有邪干,故令夭壽。只如厥陰失守,天以虛,人氣肝虛,感天重虛。即魂遊於上,邪干,厥大氣,身溫猶可刺之,制其足少陽之所過,次刺肝之俞。人病心虛,又遇群相二火司天失守,感而三虛,遇火不及,黑屍鬼犯之,令人暴亡,可刺手少陽之所過,復刺心俞。人脾病,又遇太陰司天失守,感而三虛,又遇土不及,青屍鬼邪,犯之於人,令人暴亡,可刺足陽明之所過,復刺脾之俞。人肺病,遇陽明司天失守,感而三虛,又遇金不及,有赤屍鬼犯人,令人暴亡,可刺手陽明之所過,復刺肺俞。人腎病,又遇太陽司天失守,感而三虛,又遇水運不及之年,有黃屍鬼,干犯人正氣,吸人神魂,致暴亡,可刺足太陽之所過,復刺腎俞。

%% 黃帝問曰:十二藏之相使,神失位,使神彩之不圓,恐邪干犯,治之可刺?願聞其要。岐伯稽首再拜曰:悉乎哉問!至理道真宗,此非聖帝,焉窮斯源,是謂氣神合道,契符上天。心者,君主之官,神明出焉,可刺手少陰之源。肺者,相傅之官,治節出焉,可刺手太陰之源。肝者,將軍之官,謀虛出焉,可刺足厥陰之源。膽者,中正不官,決斷出焉,可刺足少陽之源。羶中者,臣使之官,喜樂出焉,可刺心包絡所流。脾為諫議之官,知周出焉,可刺脾之源。胃為倉廩之官,五味出焉,可刺胃之源。大腸者,傳道之官,變化出焉,可刺大腸之源。小腸者,受盛之官,化物出焉,可刺小腸之源。腎者,作強之官,伎巧出焉,刺其腎之源。三焦者,決瀆之官,水道出焉,刺三焦之源。膀胱者,州都之官,津液藏焉,氣化則能出矣,刺膀胱之源。凡此十二官者,不得相失也。是故刺法有全神養真之旨,亦法有修真之道,非治疾也。故要修養和神也,道貴常存,補神固根,精氣不散,神守不分,然即神守而雖不去,亦能全真,人神不守,非達至真,至真之要,在乎天玄,神守天息,復入本元,命曰歸宗。

%% \section{本病論篇第七十三(遺篇)}

%% 黃帝問曰:天元九窒,余已知之,願聞氣交,何名失守?岐伯曰:謂其上下升降,遷正退位,各有經論,上下各有不前,故名失守也。是故氣交失易位,氣交乃變,變易非常,即四失序,萬化不安,變民病也。

%% 帝曰:升降不前,願聞其故,氣交有變,何以明知?岐伯曰:昭乎哉問,明乎道矣?氣交有變,是謂天地機,但欲降而不得降者,地窒刑之。又有五運太過,而先天而至者,即交不前,但欲升而不得其升,中運抑之,但欲降而不得其降,中運抑之。於是有升之不前,降之不下者,有降之不下,升而至天者,有升降俱不前,作如此之分別,即氣交之變。變之有異,常各各不同,災有微甚者也。

%% 帝曰:願聞氣交遇會勝抑之由,變成民病,輕重何如?岐伯曰:勝相會,抑伏使然。是故辰戌之歲,木氣升之,主逢天柱,勝而不前;又遇庚戌,金運先天,中運勝之忽然不前,木運升天,金乃抑之,升而不前,即清生風少,肅殺於春,露霜復降,草木乃萎。民病溫疫早發,咽嗌乃干,四肢滿,肢節皆痛;久而化郁,即大風摧拉,折隕鳴紊。民病卒中偏痹,手足不仁。

%% 是故巳亥之歲,君火升天,主窒天蓬,勝之不前;又厥陰未遷正,則少陰未得升天,水運以至其中者,君火欲升,而中水運抑之,升之不前,即清寒復作,冷生旦暮。民病伏陽,而內生煩熱,心神驚悸,寒熱間作;日久成郁,即暴熱乃至,赤風瞳翳,化疫,溫癘暖作,赤氣彰而化火疫,皆煩而燥渴,渴甚,治之以洩之可止。

%% 是故子午之歲,太陰升天,主窒天沖,勝之不前;又或遇壬子,木運先天而至者,中木運抑之也,升天不前,即風埃四起,時舉埃昏,雨濕不化。民病風厥涎潮,偏痹不隨,脹滿;久而伏郁,即黃埃化疫也。民病夭亡,臉肢府黃疸滿閉。濕令弗布,雨化乃微。

%% 是故丑未之年,少陽升天,主窒天蓬,勝之不前;又或遇太陰未遷正者,即少陰未升天也,水運以至者,升天不前,即寒冰反布,凜冽如冬,水復涸,冰再結,暄暖乍作,冷夏布之,寒暄不時。民病伏陽在內,煩熱生中,心神驚駭,寒熱間爭;以久成郁,即暴熱乃生,赤風氣腫翳,化成疫癘,乃化作伏熱內煩,痹而生厥,甚則血溢。

%% 是故寅申之年,陽明升天,主窒天英,勝之不前;又或遇戊申戊寅,火運先天而至;金欲升天,火運抑之,升之不前。即時雨不降,西風數舉,鹹鹵燥生。民病上熱喘嗽,血溢;久而化郁,即白埃翳霧,清生殺氣,民病脅滿,悲傷,寒鼽嚏,嗌干,手坼皮膚燥。

%% 是故卯酉之年,太陽升天,主窒天芮,勝之不前;又遇陽明未遷正者,即太陽未升天也,土運以至,水欲升天,土運抑之,升之不前,即濕而熱蒸,寒生兩間。民病注下,食不及化;久而成郁,冷來客熱,冰雹卒至。民病厥逆而噦,熱生於內,氣痹於外,足脛痠疼,反生心悸,懊熱,暴煩而復厥。

%% 黃帝曰:升之不前,余已盡知其旨,願聞降之不下,可得明乎?岐伯曰:悉乎哉問也!是之謂天地微旨,可以盡陳斯道。所謂升已必降也,至天三年,次歲必降,降而入地,始為左間也。如此升降往來,命之六紀也。

%% 是故丑未之歲,厥陰降地,主窒地晶,勝而不前;又或遇少陰未退位,即厥陰未降下,金運以至中,金運承之,降之未下,抑之變郁,木欲降下,金運承之,降而不下,蒼埃遠見,白氣承之,風舉埃昏,清燥行殺,霜露復下,肅殺布令。久而不降,抑之化郁,即作風燥相伏,暄而反清,草木萌動,殺霜乃下,蟄蟲未見,懼清傷藏。

%% 是故寅申之歲,少陰降地,主窒地玄,勝之不入;又或遇丙申丙寅,水運太過,先天而至,君火欲降,水運承之,降而不下,即彤雲才見,黑氣反生,暄暖如舒,寒常布雪,凜冽復作,天雲慘淒。久而不降,伏之化郁,寒勝復熱,赤風化疫,民病面赤、心煩、頭痛、目眩也,赤氣彰而溫病欲作也。

%% 是故卯酉之歲,太陰降地,主窒地蒼,勝之不入;又或少陽未退位者,即太陰未得降也;或木運以至,木運承之,降而不下,即黃雲見而青霞彰,郁蒸作而大風,霧翳埃勝,折隕乃作。久而不降也,伏之化郁,天埃黃氣,地布濕蒸。民病四肢不舉、昏眩、肢節痛、腹滿填臆。

%% 是故辰戌之歲,少陽降地,主窒地玄,勝之不入;又或遇水運太過,先天而至也,水運承之,降而不下,即彤雲才見,黑氣反生,暄暖欲生,冷氣卒至,甚則冰雹也。久而不降,伏之化郁,冰氣復熱,赤風化疫,民病面赤、心煩、頭痛、目眩也,赤氣彰而熱病欲作也。

%% 是故巳亥之歲,陽明降地,主窒地彤,用而不入;又或遇太陽未退位,即陽明未得降;即火運以至之,火運承之不下,即天清而肅,赤氣乃彰,暄熱反作。民皆錯倦,夜臥不安,咽乾引飲,懊熱內煩,天清朝暮,暄還復作;久而不降,伏之化郁,天清薄寒,遠生白氣。民病掉眩,手足直而不仁,兩脅作痛,滿目 然。

%% 是故子午之年,太陽降地,主窒地阜勝之,降而不入;又或遇土運太過,先天而至,土運承之,降而不入,即天彰黑氣,暝暗淒慘,才施黃埃而布濕,寒化令氣,蒸濕復令。久而不降,伏之化郁,民病大厥,四肢重怠,陰痿少力,天布沉陰,蒸濕間作。

%% 帝曰:升降不前,晰知其宗,願聞遷正,可得明乎?岐伯曰:正司中位,是謂遷正位,司天不得其遷正者,即前司天,以過交司之日,即遇司天太過有餘日也,即仍舊治天數,新司天未得遷正也。

%% 厥陰不遷正,即風暄不時,花卉萎瘁。民病淋溲,目系轉,轉筋,喜怒,小便赤。風欲令而寒由不去,溫暄不正,春正失時。

%% 少陰不遷正,即冷氣不退,春冷後寒,暄暖不時。民病寒熱,四肢煩痛,腰脊強直。木氣雖有餘,而位不過於君火也。

%% 太陰不遷正,即雲雨失令,萬物枯焦,當生不發。民病手足肢節腫滿,大腹水腫,填臆不食,飧洩脅滿,四肢不舉。雨化欲令,熱猶治之,溫煦於氣,亢而不澤。

%% 少陽不遷正,即炎灼弗令,苗莠不榮,酷暑於秋,肅殺晚至,霜露不時。民病痎瘧,骨熱,心悸,驚駭;甚時血溢。

%% 陽明不遷正,則暑化於前,肅殺於後,草木反榮。民病寒熱,鼽嚏,皮毛折,爪甲枯焦;甚則喘嗽息高,悲傷不樂。熱化乃布,燥化未令,即清勁未行,肺金復病。

%% 陽明不遷正,即冬清反寒,易令於春,殺霜在前,寒冰於後,陽光復治,凜冽不作,民病溫癘至,喉閉嗌干,煩躁而渴,喘息而有音也。寒化待燥,猶治天氣,過失序,與民作災。

%% 帝曰:遷正早晚,以命其旨,願聞退位,可得明哉?岐伯曰:所謂不退者,即天數未終,即天數有餘,名曰復布政,故名曰再治天也。即天令如故,而不退位也。

%% 厥陰不退位,即大風早舉,時雨不降,濕令不化,民病溫疫,疵廢,風生,皆肢節痛,頭目痛,伏熱內煩,咽喉乾引飲。

%% 少陰不退位,即溫生春冬,蟄蟲早至,草木發生,民病膈熱,咽干,血溢,驚駭,小便赤澀,丹瘤,瘡瘍留毒。

%% 太陰不退位,而取寒暑不時,埃昏布作,濕令不去,民病四肢少力,食飲不下,洩注淋滿,足脛寒,陰痿,閉塞,失溺,小便數。

%% 少陽不退位,即熱生於春,暑乃後化,冬溫不凍,流水不冰,蟄蟲出見,民病少氣,寒熱更作,便血,上熱,小腹堅滿,小便赤沃,甚則血溢。

%% 陽明不退位,即春生清冷,草木晚榮,寒熱間作。民病嘔吐,暴注,食飲不下,大便乾燥,四肢不舉,目瞑掉眩。

%% 太陽不退位,即春寒夏作,冷雹乃降,沉陰昏翳,二之氣寒猶不去。民病痹厥,陰痿,失溺,腰膝皆痛,溫癘晚發。

%% 帝曰:天歲早晚,余已知之,願聞地數,可得聞乎?岐伯曰:地下遷正、升天及退位不前之法,即地土產化,萬物失時之化也。

%% 帝曰:余聞天地二甲子,十干十二支,上下經緯天地,數有迭移,失守其位,可得昭乎?岐伯曰:失之迭位者,謂雖得歲正,未得正位之司,即四時不節,即生大疫。注《玄珠密語》云:陽年三十年,除六年天刑,計有太過二十四年,除此六年,皆作太過之用。令不然之旨,今言迭支迭位,皆可作其不及也。

%% 假令甲子陽年,土運太窒,如癸亥天數有餘者,年雖交得甲子,厥陰猶尚治天,地已遷正,陽明在泉,去歲少陽以作右間,即厥陰之地陽明,故不相和奉者也。癸巳相會,土運太過,虛反受木勝,故非太過也,何以言土運太過,況黃鍾不應太窒,木即勝而金還復,金既復而少陰如至,即木勝如火而金復微,如此則甲已失守,後三年化成土疫,晚至丁卯,早至丙寅,土疫至也,大小善惡,推其天地,詳乎太乙。又只如甲子年,如甲至子而合,應交司而治天,即下己卯未遷正,而戊寅少陽未退位者,亦甲已下有合也,即土運非太過,而木乃乘虛而勝土也,金次又行復勝之,即反邪化也。陰陽天地殊異爾,故其大小善惡,一如天地之法旨也。

%% 假令丙寅陽年太過,如乙丑天數有餘者,雖交得丙寅,太陰尚治天也。地已遷正,厥陰司地,去歲太陽以作右間,即天太陰而地厥陰,故地不奉天化也。乙辛相會,水運太虛,反受土勝,故非太過,即太簇之管,太羽不應,土勝而雨化,木復即風,此者丙辛失守其會,後三年化成水疫,晚至己巳,早至戊辰,甚即速,微即徐,水疫至也,大小善惡,推其天地數乃太乙游宮。又只如丙寅年,丙至寅且合,應交司而治天,即辛巳未得遷正,而庚辰太陽未退位者,亦丙辛不合德也,即水運亦小虛而小勝,或有復,後三年化癘,名曰水癘,其狀如水疫。治法如前。假令庚辰陽年太過,如己卯天數有餘者,雖交得庚辰年也,陽明猶尚治天,地已遷正,太陰司地,去歲少陰以作右間,即天陽明而地太陰也,故地不奉天也。乙巳相會,金運太虛,反受火勝,故非太過也,即姑洗之管,太商不應,火勝熱化,水復寒刑,此乙庚失守,其後三年化成金疫也,速至壬午,徐至癸未,金疫至也,大小善惡,推本年天數及太乙也。又只如庚辰,如庚至辰,且應交司而治天,即下乙未得遷正者,即地甲午少陰未退位者,且乙良不合德也,即下乙未柔干失剛,亦金運小虛也,有小勝或無復,且三年化癘,名曰金癘,其狀如金疫也。治法如前。

%% 假令壬午陽年太過,如辛巳天數有餘者,雖交得壬午年也,厥陰猶尚治天,地已遷正,陽明在泉,去歲丙申少陽以作右間,即天厥陰而地陽明,故地不奉天者也。丁辛相合會,木運太虛,反受金勝,故非太過也,即蕤賓之管,太角不應,金行燥勝,火化熱復,甚即速,微即徐。疫至大小善惡,推疫至之年天數及太乙。又只如壬至午,且應交司而治之,即下丁酉未得遷正者,即地下丙申少陽未得退位者,見丁壬不合德也,即丁柔干失賜,亦木運小虛也,有小勝小復。後三年化癘,名曰木癘,其狀如風疫也。治法如前。

%% 假令戊申陽年太過,如丁未天數太過者,雖交得戊申年也。太陰猶尚司天,地已遷正,厥陰在泉,去歲壬戌太陽以退位作右間,即天丁未,地癸亥,故地不奉天化也。丁癸相會,火運太虛,反受水勝,故非太過也,即夷則之管,上太徵不應,此戊癸失守其會,後三年化疫也,速至庚戌,大小善惡,推疫至之年天數及太乙。又只如戊申,如戊至申,且應交司治天,即下癸亥未得遷正者,即地下壬戌太陽未退者,見戊癸亥未合德也,即下癸柔干失剛,見火運小虛,有小勝或無復也,後三年化癘,名曰火癘也。治法如前;治之法,可寒之洩之。

%% 黃帝曰:人氣不足,天氣如虛,人神失守,神光不聚,邪鬼干人,致有夭亡,可得聞乎?岐伯曰:人之五藏,一藏不足,又會天虛,感邪之至也。人憂愁思慮即傷心,又或遇少陰司天,天數不及,太陰作接間至,即謂天虛也,此即人氣天氣同虛也。又遇驚而奪精,汗出於心,因而三虛,神明失守。心為群主之官,神明出焉,神失守位,即神遊上丹田,在帝太一帝群泥丸宮一下。神既失守,神光不聚,卻遇火不及之歲,有黑屍鬼見之,令人暴亡。

%% 人飲食、勞倦即傷脾,又或遇太陰司天,天數不及,即少陽作接間至,即謂之虛也,此即人氣虛而天氣虛也。又遇飲食飽甚,汗出於胃,醉飽行房,汗出於脾,因而三虛,脾神失守,脾為諫議之官,智周出焉。神既失守,神光失位而不聚也,卻遇土不及之年,或已年或甲年失守,或太陰天虛,青屍鬼見之,令人卒亡。

%% 人久坐濕地,強力入水即傷腎,腎為作強之官,伎巧出焉。因而三虛,腎神失守,神志失位,神光不聚,卻遇水不及之年,或辛不會符,或丙年失守,或太陽司天虛,有黃屍鬼至,見之令人暴亡。

%% 人或恚怒,氣逆上而不下,即傷肝也。又遇厥陰司天,天數不及,即少陰作接間至,是謂天虛也,此謂天虛人虛也。又遇疾走恐懼,汗出於肝。肝為將軍之官,謀慮出焉。神位失守,神光不聚,又遇木不及年,或丁年不符,或壬年失守,或厥陰司天虛也,有白屍鬼見之,令人暴亡也。

%% 已上五失守者,天虛而人虛也,神遊失守其位,即有五屍鬼干人,令人暴亡也,謂之曰屍厥。人犯五神易位,即神光不圓也。非但屍鬼,即一切邪犯者,皆是神失守位故也。此謂得守者生,失守者死。得神者昌,失神者亡。

\part{黃帝內經·靈樞}
\section{經脈第十}
雷公問於黃帝曰:禁脈之言,凡刺之理,經脈為始,營其所行,制其度量,內次五藏,外別六府,願盡聞其道。黃帝曰:人始生,先成精,精成而腦髓生,骨為干,脈為營,筋為剛,肉為牆,皮膚堅而毛髮長,谷入於胃,脈道以通,血氣乃行。雷公曰:願卒聞經脈之始生。黃帝曰:經脈者,所以能決死生,處百病,調虛實,不可不通。

肺手太陰之脈,起於中焦,下絡大腸,還循胃口,上膈屬肺,從肺系橫出腋下,下循臑內,行少陰心主之前,下肘中,循臂內上骨下廉,入寸口,上魚,循魚際,出大指之端;其支者,從腕後直出次指內廉,出其端。是動則病肺脹滿膨膨而喘咳,缺盆中痛,甚則交兩手而瞀,此為臂厥。是主肺所生病者,咳,上氣喘渴,煩心胸滿,臑臂內前廉痛厥,掌中熱。氣盛有餘,則肩背痛風寒,汗出中風,小便數而欠。氣虛則肩背痛寒,少氣不足以息,溺色變。為此諸病,盛則瀉之,虛則補之,熱則疾之,寒則留之,陷下則灸之,不盛不虛,以經取之。盛者寸口大三倍於人迎,虛者則寸口反小於人迎也。

大腸手陽明之脈,起於大指次指之端,循指上廉,出合谷兩骨之間,上入兩筋之中,循臂上廉,入肘外廉,上臑外前廉,上肩,出\ytz{髃}骨之前廉,上出於柱骨之會上,下入缺盆絡肺,下膈屬大腸;其支者,從缺盆上頸貫頰,入下齒中,還出挾口,交人中,左之右,右之左,上挾鼻孔。是動則病齒痛頸腫。是主津液所生病者,目黃口乾,鼽衄,喉痹,肩前臑痛,大指次指痛不用。氣有餘則當脈所過者熱腫,虛則寒慄不復。為此諸病,盛則瀉之,虛則補之,熱則疾之,寒則留之,陷下則灸之,不盛不虛,以經取之。盛者人迎大三倍於寸口,虛者人迎反小於寸口也。

胃足陽明之脈,起於鼻之交頞中,旁納太陽之脈,下循鼻外,入上齒中,還出挾口環唇,下交承漿,卻循頤後下廉,出大迎,循頰車,上耳前,過客主人,循髮際,至額顱;其支者,從大迎前下人迎,循喉嚨,入缺盆,下膈屬胃絡脾;其直者,從缺盆下乳內廉,下挾臍,入氣街中;其支者,起於胃口,下循腹裡,下至氣街中而合,以下髀關,抵伏兔,下膝臏中,下循脛外廉,下足跗,入中指內間;其支者,下廉三寸而別,下入中指外間;其支者,別跗上,入大指間,出其端。是動則病灑灑振寒,善呻數欠顏黑,病至則惡人與火,聞木聲則惕然而驚,心欲動,獨閉戶塞牖而處,甚則欲上高而歌,棄衣而走,賁響腹脹,是為骭厥。是主血所生病者,狂瘧溫淫汗出,鼽衄,口喎唇胗,頸腫喉痹,大腹水腫,膝臏腫痛,循膺、乳、氣街、股、伏兔、骭外廉、足跗上皆痛,中指不用。氣盛則身以前皆熱,其有餘於胃,則消谷善飢,溺色黃。氣不足則身以前皆寒慄,胃中寒則脹滿。為此諸病,盛則瀉之,虛則補之,熱則疾之,寒則留之,陷下則灸之,不盛不虛,以經取之。盛者人迎大三倍於寸口,虛者人迎反小於寸口也。
脾足太陰之脈,起於大指之端,循指內側白肉際,過核骨後,上內踝前廉,上踹內,循脛骨後,交出厥陰之前,上膝股內前廉,入腹屬脾絡胃,上膈,挾咽,連舌本,散舌下;其支者,復從胃,別上膈,注心中。是動則病舌本強,食則嘔,胃脘痛,腹脹善噫,得後與氣則快然如衰,身體皆重。是主脾所生病者,舌本痛,體不能動搖,食不下,煩心,心下急痛,溏、瘕、洩,水閉、黃疸,不能臥,強立股膝內腫厥,足大指不用。為此諸病,盛則瀉之,虛則補之,熱則疾之,寒則留之,陷下則灸之,不盛不虛,以經取之。盛者,寸口大三倍於人迎,虛者,寸口反小於人迎也。

心手少陰之脈,起於心中,出屬心繫,下膈絡小腸;其支者,從心繫上挾咽,系目系;其直者,復從心繫卻上肺,下出腋下,下循臑內後廉,行太陰心主之後,下肘內,循臂內後廉,抵掌後銳骨之端,入掌內後廉,循小指之內出其端。是動則病嗌干心痛,渴而欲飲,是為臂厥。是主心所生病者,目黃脅痛,臑臂內後廉痛厥,掌中熱痛。為此諸病,盛則瀉之,虛則補之,熱則疾之,寒則留之,陷下則灸之,不盛不虛,以經取之。盛者寸口大再倍於人迎,虛者寸口反小於人迎也。

小腸手太陽之脈,起於小指之端,循手外側上腕,出踝中,直上循臂骨下廉,出肘內側兩筋之間,上循臑外後廉,出肩解,繞肩胛,交肩上,入缺盆絡心,循嚥下膈,抵胃屬小腸;其支者,從缺盆循頸上頰,至目銳眥,卻入耳中;其支者,別頰上\ytzn{䪼}抵鼻,至目內眥,斜絡於顴。是動則病嗌痛頷腫,不可以顧,肩似拔,臑似折。是主液所生病者,耳聾目黃頰腫,頸頷肩臑肘臂外後廉痛。為此諸病,盛則瀉之,虛則補之,熱則疾之,寒則留之,陷下則灸之,不盛不虛,以經取之。盛者人迎大再倍於寸口,虛者人迎反小於寸口也。

膀胱足太陽之脈,起於目內眥,上額交巔;其支者,從巔至耳上角;其直者,從巔入絡腦,還出別下項,循肩髆內,挾脊抵腰中,入循膂,絡腎屬膀胱;其支者,從腰中下挾脊貫臀,入膕中;其支者,從髆內左右,別下貫胛,挾脊內,過髀樞,循髀外從後廉下合膕中,以下貫踹內,出外踝之後,循京骨,至小指外側。是動則病沖頭痛,目似脫,項如拔,脊痛,腰似折,髀不可以曲,膕如結,踹如裂,是為踝厥。是主筋所生病者,痔瘧狂顛疾,頭囟項痛,目黃淚出鼽衄,項背腰尻膕踹腳皆痛,小指不用。為此諸病,盛則瀉之,虛則補之,熱則疾之,寒則留之,陷下則灸之,不盛不虛,以經取之。盛者人迎大再倍於寸口,虛者人迎反小於寸口也。

腎足少陰之脈,起於小指之下,邪走足心,出於然谷之下,循內踝之後,別入跟中,以上踹內,出膕內廉,上股內後廉,貫脊屬腎絡膀胱;其直者,從腎上貫肝膈,入肺中,循喉嚨,挾舌本;其支者,從肺出絡心,注胸中。是動則病飢不欲食,面如漆柴,咳唾則有血,喝喝而喘,坐而欲起,目\ytz{䀮}\ytz{䀮}如無所見,心如懸若飢狀,氣不足則善恐,心惕惕如人將捕之,是為骨厥。是主腎所生病者,口熱舌干,咽腫上氣,嗌干及痛,煩心心痛,黃疸腸澼,脊股內後廉痛,痿厥嗜臥,足下熱而痛。為此諸病,盛則瀉之,虛則補之,熱則疾之,寒則留之,陷下則灸之,不盛不虛,以經取之。灸則強食生肉,緩帶披髮,大杖重履而步。盛者寸口大再倍於人迎,虛者寸口反小於人迎也。

心主手厥陰心包絡之脈,起於胸中,出屬心包絡,下膈,歷絡三膲;其支者,循胸出脅,下腋三寸,上抵腋,下循臑內,行太陰少陰之間,入肘中,下臂行兩筋之間,入掌中,循中指出其端;其支者,別掌中,循小指次指出其端。是動則病手心熱,臂肘攣急,腋腫,甚則胸脅支滿,心中憺憺大動,面赤目黃,喜笑不休。是主脈所生病者,煩心心痛,掌中熱。為此諸病,盛則瀉之,虛則補之,熱則疾之,寒則留之,陷下則灸之,不盛不虛,以經取之。盛者寸口大一倍於人迎,虛者寸口反小於人迎也。

三焦手少陽之脈,起於小指次指之端,上出兩指之間,循手錶腕,出臂外兩骨之間,上貫肘,循臑外上肩,而交出足少陽之後,入缺盆,布羶中,散落心包,下膈,循屬三焦;其支者,從羶中上出缺盆,上項,系耳後,直上出耳上角,以屈下頰至\ytzn{䪼};其支者,從耳後入耳中,出走耳前,過客主人前,交頰,至目銳眥。是動則病耳聾渾渾焞焞,嗌腫喉痹。是主氣所生病者,汗出,目銳眥痛,頰痛,耳後肩臑肘臂外皆痛,小指次指不用。為此諸病,盛則瀉之,虛則補之,熱則疾之,寒則留之,陷下則灸之,不盛不虛,以經取之。盛者人迎大一倍於寸口,虛者人迎反小於寸口也。

膽足少陽之脈,起於目銳眥,上抵頭角,下耳後,循頸行手少陽之前,至肩上,卻交出手少陽之後,入缺盆;其支者,從耳後入耳中,出走耳前,至目銳眥後;其支者,別銳眥,下大迎,合於手少陽,抵於\ytzn{䪼},下加頰車,下頸合缺盆以下胸中,貫膈絡肝屬膽,循脅裡,出氣街,繞毛際,橫入髀厭中;其直者,從缺盆下腋,循胸過季脅,下合髀厭中,以下循髀陽,出膝外廉,下外輔骨之前,直下抵絕骨之端,下出外踝之前,循足跗上,入小指次指之間;其支者,別跗上,入大指之間,循大指岐骨內出其端,還貫爪甲,出三毛。是動則病口苦,善太息,心脅痛不能轉側,甚則面微有塵,體無膏澤,足外反熱,是為陽厥。是主骨所生病者,頭痛頷痛,目銳眥痛,缺盆中腫痛,腋下腫,馬刀俠癭,汗出振寒,瘧,胸脅肋髀膝外至脛絕骨外髁前及諸節皆痛,小指次指不用。為此諸病,盛則瀉之,虛則補之,熱則疾之,寒則留之,陷下則灸之,不盛不虛,以經取之。盛者人迎大一倍於寸口,虛者人迎反小於寸口也。

肝足厥陰之脈,起於大指叢毛之際,上循足跗上廉,去內踝一寸,上踝八寸,交出太陰之後,上膕內廉,循股陰入毛中,過陰器,抵小腹,挾胃屬肝絡膽,上貫膈,布脅肋,循喉嚨之後,上入頏顙,連目系,上出額,與督脈會於巔;其支者,從目系下頰裡,環唇內;其支者,復從肝別貫膈,上注肺。是動則病腰痛不可以俯仰,丈夫\ytz{㿉}疝,婦人少腹腫,甚則嗌干,面塵脫色。是主肝所生病者,胸滿嘔逆飧洩,狐疝遺溺閉癃。為此諸病,盛則瀉之,虛則補之,熱則疾之,寒則留之,陷下則灸之,不盛不虛,以經取之。盛者寸口大一倍於人迎,虛者寸口反小於人迎也。

手太陰氣絕則皮毛焦,太陰者行氣溫於皮毛者也,故氣不榮則皮毛焦,皮毛焦則津液去皮節,津液去皮節者則爪枯毛折,毛折者則毛先死,丙篤丁死,火勝金也。手少陰氣絕則脈不通,脈不通則血不流,血不流則髦色不澤,故其面黑如漆柴者,血先死,壬篤癸死,水勝火也。足太陰氣絕者則脈不榮肌肉,唇舌者肌肉之本也,脈不榮則肌肉軟,肌肉軟則舌萎人中滿,人中滿則唇反,唇反者肉先死,甲篤乙死,木勝土也。足少陰氣絕則骨枯,少陰者冬脈也,伏行而濡骨髓者也,故骨不濡則肉不能著也,骨肉不相親則肉軟卻,肉軟卻故齒長而垢發無澤,發無澤者骨先死,戊篤己死,土勝水也。足厥陰氣絕則筋絕,厥陰者肝脈也,肝者筋之合也,筋者聚於陰氣,而脈絡於舌本也,故脈弗榮則筋急,筋急則引舌與卵,故唇青舌卷卵縮則筋先死,庚篤辛死,金勝木也。五陰氣俱絕則目系轉,轉則目運,目運者為志先死,志先死則遠一日半死矣。六陽氣絕,則陰與陽相離,離則腠理髮洩,絕汗乃出,故旦佔夕死,夕佔旦死。

經脈十二者,伏行分肉之間,深而不見;其常見者,足太陰過於外踝之上,無所隱故也。諸脈之浮而常見者,皆絡脈也。六經絡手陽明少陽之大絡,起於五指間,上合肘中。飲酒者,衛氣先行皮膚,先充絡脈,絡脈先盛,故衛氣已平,營氣乃滿,而經脈大盛。脈之卒然動者,皆邪氣居之,留於本末;不動則熱,不堅則陷且空,不與眾同,是以知其何脈之動也。雷公曰:何以知經脈之與絡脈異也?黃帝曰:經脈者常不可見也,其虛實也以氣口知之,脈之見者皆絡脈也。雷公曰:細子無以明其然也。黃帝曰:諸絡脈皆不能經大節之間,必行絕道而出,入復合於皮中,其會皆見於外。故諸刺絡脈者,必刺其結上,甚血者雖無結,急取之以瀉其邪而出其血,留之發為痹也。凡診絡脈,脈色青則寒且痛,赤則有熱。胃中寒,手魚之絡多青矣;胃中有熱,魚際絡赤;其暴黑者,留久痹也;其有赤有黑有青者,寒熱氣也;其青短者,少氣也。凡刺寒熱者皆多血絡,必間日而一取之,血盡乃止,乃調其虛實;其小而短者少氣,甚者瀉之則悶,悶甚則僕不得言,悶則急坐之也。

手太陰之別,名曰列缺,起於腕上分間,並太陰之經直入掌中,散入於魚際。其病實則手銳掌熱,虛則欠\ytz{㰦},小便遺數,取之去腕半寸,別走陽明也。手少陰之別,名曰通裡,去腕一寸半,別而上行,循經入於心中,系舌本,屬目系。其實則支膈,虛則不能言,取之掌後一寸,別走太陽也。手心主之別,名曰內關,去腕二寸,出於兩筋之間,循經以上,繫於心包絡。心繫實則心痛,虛則為頭強,取之兩筋間也。手太陽之別,名曰支正,上腕五寸,內注少陰;其別者,上走肘,絡肩\ytz{髃}。實則節弛肘廢,虛則生\ytz{肬},小者如指痂疥,取之所別也。手陽明之別,名曰偏歷,去腕三寸,別入太陰;其別者,上循臂,乘肩\ytz{髃},上曲頰偏齒;其別者,入耳合於宗脈。實則齲聾,虛則齒寒痹隔,取之所別也。手少陽之別,名曰外關,去腕二寸,外遶臂,注胸中,合心主。病實則肘攣,虛則不收,取之所別也。足太陽之別,名曰飛揚,去踝七寸,別走少陰。實則鼽窒頭背痛,虛則鼽衄,取之所別也。足少陽之別,名曰光明,去踝五寸,別走厥陰,下絡足跗。實則厥,虛則痿躄,坐不能起,取之所別也。足陽明之別,名曰豐隆,去踝八寸,別走太陰;其別者,循脛骨外廉,上絡頭項,合諸經之氣,下絡喉嗌。其病氣逆則喉痹瘁瘖,實則狂巔,虛則足不收脛枯,取之所別也。足太陰之別,名曰公孫,去本節之後一寸,別走陽明;其別者,入絡腸胃。厥氣上逆則霍亂,實則腸中切痛,虛則鼓脹,取之所別也。足少陰之別,名曰大鍾,當踝後繞跟,別走太陽;其別者,並經上走於心包,下外貫腰脊。其病氣逆則煩悶,實則閉癃,虛則腰痛,取之所別者也。足厥陰之別,名曰蠡溝,去內踝五寸,別走少陽;其別者,徑脛上睾,結於莖。其病氣逆則睾腫卒疝,實則挺長,虛則暴癢,取之所別也。任脈之別,名曰尾翳,下鳩尾,散於腹。實則腹皮痛,虛則癢搔,取之所別也。督脈之別,名曰長強,挾膂上項,散頭上,下當肩胛左右,別走太陽,入貫膂。實則脊強,虛則頭重,高搖之,挾脊之有過者,取之所別也。脾之大絡,名曰大包,出淵腋下三寸,布胸脅。實則身盡痛,虛則百節盡皆縱,此脈若羅絡之血者,皆取之脾之大絡脈也。凡此十五絡者,實則必見,虛則必下,視之不見,求之上下,人經不同,絡脈異所別也。


\section{异体字注音}

焫:ruo4

鑱:chan2

熇:he4

虙:fu2

皏:peng3

炲:tai2

脽:shui2

齗:yin2

\ytz{鬄}:ti4

稸:xu4

黅:jin1

憹:nao2

爇:ruo4

吤:jie4

\ytz{蛕}:hui2,蛔

腄:chui2

晬:zui4

\ytz{肬}:you2

\ytz{髃}:yu2

覩:du3,睹

\ytz{昬}:hun1,昏

\ytz{顀}:chui2,椎

頄:qiu2,颧骨

\ytz{䐜}(月真):chen1,胸膈或上腹部脹滿不適

\yt{𩅞}(雩重):zhong1,氣之往來不息

㾓(疒肙):juan1

㑊(亻亦):yi4

\ytz{䐃}(月囷):jun4,肌肉的突起部位

\ytz{㶼}(火矣):ai1

\ytz{䀮}(目巟):huang1,目昏暗,視物不清

\ytz{䯒}(骨行):heng2,脛腓骨的統稱,小腿部,腳脛部

\ytza{𤼃}(疒龍):

\ytz{䏚}(月少):miao3,季脅下挾脊兩旁空軟處

\ytz{㶼}(火矣):ai1,火燒,火熨、灸焫等治法

\ytza{𩩻}(骨盾):tu2:皮肉肥厚之處

\ytza{𦛗}(月呂):lv2

雲(蕓去草頭令)

\ytza{𤸷}(疒帬):wan2,痹,麻木

\ytz{䪼}(出頁):zhuo1,眼眶下面的骨

痠(疒峻-山):suan1,同“酸”

\ytz{㿉}(疒貴):tui2

\ytz{㒤}(亻聶):che4,懾

\ytz{㕮}(口父):fu3,用嘴咀嚼

\ytz{㰦}(去欠):qu4,呿,张口

\ytza{𩩲}(骨曷):he2,肩骨

\ytza{𩨗}(骨亏)

\ytz{㽷}(疒水):shui4

\end{document}
