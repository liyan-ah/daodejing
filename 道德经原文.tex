\documentclass[a5paper,zihao=-4,oneside,UTF8]{ctexart}

%页面旋转
\usepackage{everypage}
\AddEverypageHook{\special{pdf: put @thispage <</Rotate 90>>}}

%% font文字旋转
\defaultCJKfontfeatures{RawFeature={vertical:+vert}}

%% 默认字体为源雲明體
\setCJKmainfont[BoldFont=源雲明體 TTF SemiBold,ItalicFont=思源宋体 Light]{源雲明體 TTF Medium}

% 页码编号为汉字
\renewcommand{\thepage}{\Chinese{page}}


\usepackage{geometry}
\newgeometry{top=50pt, bottom=80pt, left=60pt, right=60pt,headsep=0pt}

%章节标题样式
\usepackage{titlesec}
\titleformat{\section}[hang]{\Large\bfseries}{}{0pt}{}
\titlespacing{\section}{4em}{0pt}{0pt}
\titleformat{\subsection}[hang]{\large\bfseries}{第 \Chinese{subsection}\hspace{1em}}{0pt}{}

\ctexset{
	today = big,
	punct = kaiming, 
	autoindent = 0pt,
	space = true	
}

%去掉页眉
\usepackage{fancyhdr}
\renewcommand{\headrulewidth}{0.0pt}%
\fancypagestyle{plain}{% Redefine plain pages tyle  
  % Clear header/footer
  \fancyhf{}
  \renewcommand{\headrulewidth}{0.0pt}%
  \fancyfoot[L]{\thepage}
}
\pagestyle{plain}

%%% 命令设置
\usepackage{xcolor}
\definecolor{gray}{gray}{0.3}
\newcommand{\zhushi}[1]{\scriptsize{\textit{\textcolor{gray}{#1}}}\normalsize}

%全文颜色修改
\definecolor{gray1}{gray}{0.2}
\color{gray1}




\usepackage[page]{pagenote}%[page] Because of how TeX writes information to files, when the page option is used there must be no notes on the page where \printnotes or \printnotes* closes the ent file. If necessary, a \clearpage or similar must be used before the \print (標記注釋所在頁碼必須讓printnotes另起一頁,否則可能吃掉與\printnotes同頁的注釋^_^)

% 正文中的注释编号样式调整:圓圈數字
\renewcommand\notenumintext[1]{\rotatebox[origin=cc]{-270}{\textsuperscript{\footnotesize\ncircrds{#1}}}}
% 章节末尾的注释编号样式调整:方形黑底數字
\renewcommand\notenuminnotes[1]{\rotatebox[origin=cc]{-270}{\ncircbs{#1}}}
%尾註標記該註釋在正文出現的頁碼
\renewcommand{\pagename}{{P.}}

\makepagenote


\title{\textbf{老子}\textit{ \hspace{5em}帛書校勘版}\hfil}
\author{Colin Yang \\Email: yangjianlin@gmail.com}
\date{\normalsize\today 版}

\begin{document}

\ziju{0.1}
\large

\maketitle

\section*{简介}

马王堆三号汉墓出土帛书《老子》共有甲、乙两本,甲本用篆书抄写,乙本用隶书抄写,据所避汉朝名讳,知甲本年代稍早。特别是隶书乙本的发现,再次弥补了从秦上篆向东汉隶书演进过程中长期以来所存在的实物作品之匮乏。

老子是春秋时代著名思想家,也是道家的创始人,提出“清静无为”的道家学说,其朴素的辩证法思想,对中国古代哲学思想史的发展产生过很大影响。帛书版《老子》与传本《老子》在章节排列上以及文字表述上有所不同,帛书版《老子》第一至第四十四章为上篇德经,第四十五章至第八十一章为下篇道经。

%\tableofcontents
\newpage

\section{德 篇}
道,可道也,非恒道也。名,可名也,非恒名也。无名,万物之始也;有名,万物之母也。故恒无欲也,以观其妙;恒有欲也,以观其所徼。两者同出,异名同谓,玄之又玄,众妙之门。

天下皆知美之为美,恶已;皆知善,斯不善矣。有无之相生也,难易之相成也,长短之相形也,高下之相盈也,音声之相和也,先后之相随,恒也。是以圣人居无为之事,行不言之教。万物作而弗始也,为而弗恃也,成功而弗居也,夫唯弗居,是以弗去。

不上贤,使民不争。不贵难得之货,使民不为盗。不见可欲,使民不乱。是以圣人之治也,虚其心,实其腹,弱其志,强其骨。恒使民无知无欲也,使夫智不敢,弗为而已,则无不治矣。
04道盅,而用之又弗盈也。渊呵,似万物之宗。銼其锐,解其纷,和其光,同其尘。湛呵似或存。吾不知其谁之子也,象帝之先。
05天地不仁,以万物为刍狗,圣人不仁,以百姓为刍狗。天地之间,其犹橐龠与,虚而不屈,动而愈出。多闻数穷,不若守于中。
06谷神不死,是谓玄牝,玄牝之门,是谓天地之根。緜緜呵若存,用之不勤。
07天长,地久。天地之所以能长且久者,以其不自生也,故能长生。是以圣人退其身而身先,外其身而身存。不以其无私与?故能成其私。
08上善似水。水善利万物而有静,居众人之所恶,故几于道矣。居善地,心善渊,予善天,言善信,政善治,事善能,动善时。夫唯不争,故无尤。
09持而盈之,不若其已。揣而锐之,不可长保也。金玉盈室,莫之守也。贵富而骄,自遗咎也。功遂身退,天之道也。
10戴营魄抱一,能毋离乎?搏气至柔,能婴儿乎?滌除玄鉴,能毋疵乎?爱民治国,能毋以智乎?天门启阖,能为雌乎?明白四达,能毋以知乎?生之畜之。生而弗有。长而弗宰也。是谓玄德。
11卅辐同一毂,当其无,有车之用也。埏埴为器,当其无,有埴器之用也。凿户牗,当其无,有室之用也。故有之以为利。无之以为用。
12五色使人目盲。驰骋田猎使人心发狂。难得之货使人之行妨。五味使人之口爽。五音使人之耳聋。是以圣人之治也,为腹不为目,故去彼而取此。
13宠辱若惊,贵大患若身。何谓宠辱若惊?宠之为下,得之若惊,失之若惊,是谓宠辱若惊。何谓贵大患若身?吾所以有大患者,为吾有身也,及吾无身,有何患?故贵为身于为天下,若可以托天下矣。爱以身为天下,如可以寄天下矣。
14视之而弗见,名之曰微。听之而弗闻,名之曰希。捪之而弗得,名之曰夷。三者不可致诘,故混而为一。一者,其上不皦,其下不昧,绳绳不可名也,复归于无物。是谓无状之状,无物之象,是谓忽恍。随而不见其后,迎而不见其首。执今之道。以御今之有。以知古始。是谓道纪。
15古之善为道者,微妙玄达深不可识。夫唯不可识,故强为之容。曰:豫呵其若冬涉水,猷呵其若畏四邻,严呵其若客,涣呵其若淩释,敦呵其若朴,混呵其若浊,旷呵其若谷。浊而静之徐清,安以动之徐生。葆此道不欲盈,夫唯不欲盈,是以能敝而不成。
16致虚极也,守静笃也,万物并作,吾以观其复也。夫物云云,各复归于其根。归根曰静。静,是谓复命。复命常也。知常明也。不知常妄。妄作。凶。知常容。容乃公。公乃王。王乃天。天乃道。道乃久。没身不殆。
17太上。下知有之。其次。亲誉之。其次。畏之。其下。侮之。信不足。案有不信。猷呵。其贵言也。成功遂事。而百姓谓我自然。
18故大道废。案有仁义。智慧出。案有大伪。六亲不和。案有孝慈。邦家昏乱。案有贞臣。
19绝圣弃智。民利百倍。绝仁弃义。民复孝慈。绝巧弃利。盗贼无有。此三言也。以为文未足。故令之有所属。见素抱朴。少私而寡欲。绝学无忧。
20唯与诃。其相去几何。美与恶。其相去何若。人之所畏。亦不可以不畏人。望呵。其未央哉。众人熙熙。若饗于大牢。而春登台。我泊焉未兆。若婴儿未咳。累呵如无所归。众人皆有余。我独匮。我愚人之心也。沌沌呵。俗人昭昭。我独若昏呵。俗人察察。我独闷闷呵。忽呵。其若海。恍呵。其若无所止。众人皆有以。我独顽以俚。吾欲独异于人。而贵食母。
21孔德之容。唯道是从。道之物。唯恍唯忽。忽呵恍呵。中有象呵。恍呵忽呵。中有物呵。幽呵冥呵。中有情呵。其情甚真。其中有信。自今及古。其名不去。以顺众父。吾何以知众父之然也。以此。
22企者不立。自视者不彰。自见者不明。自伐者无功。自矜者不长。其在道。曰余食赘行。物或恶之。故有裕者弗居。
23曲则全。枉则正。洼则盈。敝则新。少则得。多则惑。是以圣人执一。以为天下牧。不自视故彰。不自见故明。不自伐故有功。弗矜故能长。夫唯不争。故莫能与之争。古之所谓曲全者。岂语哉。诚全归之。
24希言自然。飘风不终朝。暴雨不终日。孰为此。天地而弗能久。又况于人乎。故从事而道者同于道。德者同于德。失者同于失。同于德者。道亦德之。同于失者。道亦失之。
25有物混成。先天地生。寂呵寥呵。独立而不改。可以为天地母。吾未知其名。字之曰道。吾强为之名曰大。大曰逝。逝曰远。远曰返。道大。天大。地大。王亦大。国中有四大。而王居其一焉。人法地。地法天。天法道。道法自然。
26重为轻根。静为躁君。是以君子终日行。不离其辎重。虽有营官。燕处则超若。若何万乘之王。而以身轻于天下。轻则失本。躁则失君。
27善行者无辙迹。善言者无瑕谪。善数者不以筹策。善闭者无关鑰而不可启也。善结者无绳约而不可解也。是以圣人恒善救人。而无弃人。物无弃财。是谓袭明。故善人。善人之师。不善人。善人之资也。不贵其师。不爱其资。虽智乎大迷。是谓妙要。
28知其雄。守其雌。为天下溪。为天下溪。恒德不离。恒德不离。复归于婴儿。知其荣。守其辱。为天下谷。为天下谷。恒德乃足。恒德乃足。复归于朴。知其白。守其黑。为天下式。为天下式。恒德不忒。恒德不忒。复归于无极。朴散则为器。圣人用则为官长。夫大制无割。
29将欲取天下而为之。吾见其弗得已。夫天下神器也。非可为者也。为者败之。执者失之。物或行或随。或嘘或吹。或强或羸。或培或墮。是以圣人去甚。去泰。去奢。
30以道佐人主。不以兵强于天下。其事好还。师之所居。楚棘生之。善者果而已矣。毋以取强焉。果而毋骄。果而勿矜。果而勿伐。果而毋得已居。是谓果而不强。物壮而老。是谓之不道。不道早已。
31夫兵者。不祥之器也。物或恶之。故有裕者弗居。君子居则贵左。用兵则贵右。故兵者非君子之器也。兵者不祥之器也。不得已而用之。恬淡为上。勿美也。若美之。是乐杀人也。夫乐杀人。不可以得志于天下矣。是以吉事上左。丧事上右。是以偏将军居左。上将军居右。言以丧礼居之也。杀人众以悲哀()之。战胜以丧礼处之。
32道恒无名。朴虽小。而天下弗敢臣。侯王若能守之。万物将自宾。天地相合以雨甘露。民莫之令而自均焉。始制有名。名亦既有。夫亦将知止。知止所以不殆。譬道之在天下也。猷小谷之与江海也。
33知人者智也。自知者明也。胜人者有力也。自胜者强也。知足者富也。强行者有志也。不失其所者久也。死而不亡者寿也。
34道汜呵。其可左右也。成功遂事而弗名有也。万物归焉而弗为主。则恒无欲也。可名于小。万物归焉而弗为主。可名於大。是以圣人之能成大也。以其不为大也。故能成大。
35执大象。天下往。往而不害。安平太。乐与饵。过客止。故道之出言也。曰淡呵其无味也。视之不足见也。听之不足闻也。用之不可既也。
36将欲翕之。必固张之。将欲弱之。必固强之。将欲去之。必固举之。将欲夺之。必固予之。是谓微明。柔弱胜强。鱼不可脱于渊。邦利器不可以示人。
37道恒无名。侯王若能守之。万物将自化。化而欲作。吾将镇之以无名之朴。镇之以无名之朴。夫将不欲。不欲以静。天地将自正。
38上德不德。是以有德。下德不失德。是以无德。上德无为而无以为也。上仁为之而无以为也。上义为之而有以为也。上礼为之而莫之应也。则攘臂而扔之。故失道而后德。失德而后仁。失仁而后义。失义而后礼。夫礼者。忠信之薄也而乱之首也。前识者。道之华也而愚之首也。是以大丈夫居其厚而不居其薄。居其实而不居其华。故去彼取此。
39昔之得一者。天得一以清。地得一以宁。神得一以灵。谷得一以盈。侯王得一而以为天下正。其诫之也。谓天毋已清将恐裂。谓地毋已宁将恐发。谓神毋已灵将恐歇。谓谷毋已盈将恐竭。谓侯王毋已贵以高将恐蹶。故必贵而以贱为本。必高矣而以下为基。夫是以侯王自谓孤寡不谷。此其贱之本与。非也。故致数誉无誉。是故不欲禄禄若玉。硌硌若石。
40上士闻道。勤能行之。中士闻道。若存若亡。下士闻道。大笑之。弗笑。不足以为道。是以建言有之曰。明道如昧。进道如退。夷道如类。上德如谷。大白如辱。广德如不足。建德如偷。质真如渝。大方如隅。大器免成。大音希声。大象无形。道褒无名。夫唯道。善始且善成。
41反也者。道之动也。弱也者。道之用也。天下之物生于有。有生于无。
42道生一。一生二。二生三。三生万物。万物负阴而抱阳。沖气以为和。天下之所恶。唯孤寡不谷。而王公以自名也。物或损之而益。益之而损。古人之所教。亦我而教人。故强梁者不得其死。我将以为学父。
43天下之至柔。驰骋于天下之至坚。无有入于无间。吾是以知无为之有益也。不言之教。无为之益。天下希能及之矣。
44名与身孰亲。身与货孰多。得与亡孰病。甚爱必大费。多藏必厚亡。故知足不辱。知止不殆。可以长久。
45大成若缺。其用不敝。大盈若盅。其用不穷。大直如诎。大巧如拙。大赢如肭。(大辩如讷)。躁胜寒。静胜热。清静可以为天下正。
46天下有道。却走马以粪。天下无道。戎马生于郊。罪莫大于可欲。祸莫大于不知足。咎莫憯于欲得。故知足之足。恒足矣。
47不出于户。以知天下。不窥于牗。以知天道。其出弥远。其知弥少。是以圣人不行而知。不见而明。弗为而成。
48为学者日益。为道者日损。损之又损。以至于无为。无为而无以为。取天下也。恒无事。及其有事也。又不足以取天下。
49圣人恒无心。以百姓之心为心。善者善之。不善者亦善之。德善也。信者信之。不信者亦信之。德信也。圣人之在天下。翕翕焉。为天下浑心。百姓皆属耳目焉。圣人皆孩之。
50出生。入死。生之徒十有三。死之徒十有三。而民生生。动皆之死地之十有三。夫何故也。以其生生也。盖闻善攝生者。陵行不避兕虎。入军不被甲兵。兕无所投其角。虎无所措其爪。兵无所容其刃。夫何故也。以其无死地焉。
51道生之而德畜之。物形之而器成之。是以万物尊道而贵德。道之尊。德之贵也。夫莫之爵。而恒自然也。道生之。畜之。长之。育之。亭之。毒之。养之。覆之。生而弗有也。为而弗恃也。长而弗宰也。此之谓玄德。
52天下有始。以为天下母。既得其母。以知其子。既知其子。复守其母。没身不殆。塞其兑。闭其门。终身不勤。启其兑。济其事。终身不救。见小曰明。守柔曰强。用其光。复归其明。毋遗身殃。是谓袭常。
53使我絜有知。行于大道。唯迆是畏。大道甚夷。民甚好径。朝甚除。田甚芜。仓甚虚。服文采。带利剑。猒饮食。资财有余。是谓盗竽。非道也哉。
54善建者不拔。善抱者不脱。子孙以祭祀不绝。修之身。其德乃真。修之家。其德有余。修之乡。其德乃长。修之国。其德乃丰。修之天下。其德乃博。以身观身。以家观家。以乡观乡。以邦观邦。以天下观天下。吾何以知天下之然哉。以此。
55含德之厚者。比于赤子。蜂虿虺蛇弗螫。攫鸟猛兽弗搏。骨弱筋柔而握固。未知牝牡之会而朘怒。精之至也。终日号而不嘎。和之至也。知和曰常。知常曰明。益生曰祥。心使气曰强。物壮则老。谓之不道。不道早已。
56知者弗言。言者弗知。塞其兑。闭其门。知其光。同其尘。挫其锐。解其纷。是谓玄同。故不可得而亲。亦不可德而疏。不可得而利。亦不可得而害。不可得而贵。亦不可得而贱。故为天下贵。
57以正治邦。以奇用兵。以无事取天下。
吾何以知其然也哉。夫天下多忌讳。而民弥贫。民多利器而邦家滋昏。人多智巧。而奇物滋起。法物滋彰。而盗贼多有。是以圣人之言曰。我无为而民自化。我好静而民自正。我无事而民自富。我欲不欲而民自朴。
58其政闷闷。其民惇惇。其政察察。其民缺缺。祸。福之所倚。福。祸之所伏。孰知其极。其无正也。正复为奇。善复为妖。人之迷也。其日固久矣。是以方而不割。廉而不刺。直而不肆。光而不燿。
59治人事天莫若啬。夫唯啬。是以早服。早服是谓重积德。重积德则无不克。无不克则莫知其极。莫知其极。可以有国。有国之母。可以长久。是谓深根固柢。长生久视之道也。
60治大国若烹小鲜。以道莅天下。其鬼不神。非其鬼不神也。其神不伤人也。非其神不伤人也。圣人亦弗伤也。夫两不相伤。故德交归焉。
61大邦者。下流也。天下之牝。天下之交也。牝恒以静胜牡。为其静也。故宜为下。故大邦以下小邦。则取小邦。小邦以下大邦。则取于大邦。故或下以取。或下而取。故大邦者。不过欲兼畜人。小邦者。不过欲入事人。夫皆得其欲。则大者宜为下。
62道者万物之主也。善人之宝也。不善人之所保也。美言可以市。尊行可以贺人。人之不善也。何弃之有。故立天子。置三卿。虽有拱之璧以駪駟马。不若坐而进此。古之所以贵此者何也。不谓求以得。有罪以免与。故为天下贵。
63为无为。事无事。味无味。大小。多少。图难乎其易也。为大乎其细也。天下之难作于易。天下之大作于细。是以圣人终不为大。故能成其大。夫轻诺必寡信。多易必多难。是以圣人犹难之。故终于无难。
64其安也。易持也。其未兆也。易谋也。其脆也。易破也。其微也。易散也。为之於其未有也。治之於其未乱也。合抱之木。生于毫末。九层之台。作于蔂土。百仞之高。始于足下。为之者败之。执之者失之。是以圣人无为也。故无败也。无执也。故无失也。民之从事也。恒于几成而败之。故慎终若始。则无败事矣。是以圣人欲不欲。而不贵难得之货。学不学。而复众人之所过。能辅万物之自然。而弗敢为。
65故曰。为道者非以明民也。将以愚之也。民之难治也。以其智也。故以智治邦。邦之贼也。以不智治邦。邦之德也。恒知此两者。亦稽式也。恒知稽式。此谓玄德。玄德深矣。远矣。与物反矣。乃至大顺。
66江海之所以能为百谷王者。以其善下之。是以能为百谷王。是以圣人之欲上民也。必以其言下之。其欲先民也。必以其身后之。故居前而民弗害也。居上而民弗重也。天下乐推而弗猒也。非以其无争与。故天下莫能与争。
67小邦寡民。使有什佰人之器而毋用。使民重死而远徙。有舟车无所乘之。有甲兵无所陈之。使民复结绳而用之。甘其食。美其服。乐其俗。安其居。邻邦相望。鸡犬之声相闻。民至老死不相往来。
68信言不美。美言不信。知者不博。博者不知。善者不多。多者不善。圣人无积。既以为人。已愈有。既以予人矣。已愈多。故天之道。利而不害。人之道。为而弗争。
69天下皆谓我大。大而不肖。夫唯不肖。故能大。若肖。久矣其细也夫。我恒有三宝。持而宝之。一曰慈。二曰俭。三曰不敢为天下先。夫慈。故能勇。俭。故能广。不敢为天下先。故能为成事长。今舍其慈。且勇。舍其俭。且广。舍其后。且先。则必死矣。夫慈。以战则胜。以守则固。天将建之。如以慈垣之。
70故善为士者不武。善战者不怒。善胜敌者弗与。善用人者为之下。是谓不争之德。是谓用人。是谓配天。古之极也。
71用兵有言曰。吾不敢为主而为客。吾不敢进寸而退尺。是谓行无行。攘无臂。执无兵。乃无敌矣。祸莫大于无敌。无敌近亡吾宝矣。故称兵相若。则哀者胜矣。
72吾言甚易知也。甚易行也。而人莫之能知也。而莫之能行也。言有宗。事有君。夫唯无知也。是以不我知。知我者希。则我贵矣。是以圣人被褐而怀玉。
73知不知。尚矣。不知知。病矣。是以圣人之不病。以其病病。是以不病。
74民之不畏威则大威将至矣。毋狭其所居。毋圧其所生。夫唯弗圧。是以不厭。是以圣人自知而不自见也。自爱而不自贵也。故去彼取此。
75勇于敢者则杀。勇于不敢者则活。此两者或利或害。天之所恶。孰知其故。天之道。不战而善胜。不言而善应。不召而自来。坦而善谋。天网恢恢。疏而不失。
76若民恒且不畏死。奈何以杀惧之也。若民恒且畏死。而为奇者吾得而杀之。夫孰敢矣。若民恒且必畏死。则恒有司杀者。夫代司杀者杀。是代大匠斫也。夫代大匠斫者。则希不伤其手矣。
77人之饥也。以其取食税之多也。是以饥。百姓之不治也。以其上有以为也。是以不治。民之轻死。以其求生之厚也。是以轻死。夫唯无以生为者。是贤贵生。
78人之生也柔弱。其死也筋仞坚强。万物草木之生也柔脆。其死也枯槁。故曰。坚强者死之徒也。柔弱者生之徒也。兵强则不胜。木强则烘。强大居下。柔弱居上。
79天之道。犹张弓者也。高者抑之。下者举之。有余者损之。不足者补之。故天之道。损有余而补不足。人之道则不然。损不足而奉有余。孰能有余而有以取奉于天者乎。唯有道者乎。是以圣人为而弗有。成功而弗居也。若此其不欲见贤也。
80天下莫柔弱于水。而攻坚强者莫之能胜也。以其无以易之也。柔之胜刚。弱之胜强。天下莫弗知也。而莫之能行也。故圣人之言云。曰。受邦之诟。是谓社稷之主。受邦之不祥。是谓天下之王。正言若反。
81和大怨。必有余怨。焉可以为善。是以执右契。而不以责于人。故有德司契。无德司徹。夫天道无亲。恒与善人。
 
\subsection{論德}



上德不德 是以有德 下德不失德 是以无德 
上德无爲 而无以爲也 上仁爲之 而无以爲也 上義爲之 而有以爲也 上禮爲之 而莫之應也 則攘臂而乃之 
故失道而後德 失德而後仁 失仁而後義 失義而後禮 夫禮者 忠信之泊也 而亂之首也 
前識者 道之華也 而愚之首也 
是以大丈夫居其厚 而不居其泊 居其實 而不居其華 故去皮取此 



\subsection{得一}



昔之得一者 天得一以清 地得一以寧 神得一以霝 浴得一以盈 侯王得一以爲天下正

其至之也 謂天毋已清將恐裂 謂地毋已寧將恐發 謂神毋已靈將恐歇 謂浴毋已盈將恐竭 謂侯王毋已貴以高將恐蹶 
故必貴而以賤爲本 必高矣而以下爲基 
夫是以侯王自謂孤 寡 不榖 此其賤之爲本欤 非也 
故致數與无與 是故不欲祿祿若玉 硌硌若石




\subsection{聞道}



上士聞道 堇能行之 中士聞道 若存若亡 下士聞道 大笑之 弗笑 不足以爲道 
是以建言有之曰 明道如費 進道如退 夷道如類 上德如浴 大白如辱 廣德如不足 建德如輸 質真如渝 大方无隅 大器免成 大音希聲 天象无刑 道隱无名 
夫唯道 善始且善成




\subsection{反復}



反也者 道之動也 弱也者 道之用也 
天下之物生於有 有生於无 



\subsection{中和}



道生一 一生二 二生三 三生萬物 
萬物負陰而抱陽 中氣以爲和 
天下之所惡 唯孤 寡 不穀 而王公以自名也 
勿或損之而益 益之而損 故人之所教 夕議而教人   
故強良者不得死 我將以爲學父




\subsection{至柔}



天下之至柔 馳騁於天下之致堅 
无有入於无間 吾是以知无爲之有益 
不言之教 无爲之益 天下希能及之矣




\subsection{立戒}



名與身孰親 身與貨孰多 得與亡孰病 
甚愛必大費 多藏必厚亡 
故知足不辱 知止不殆 可以長久  



\subsection{請靚}



大成若缺 其用不幣 大盈若冲 其用不穷 大直若詘 大巧若拙 大贏如炳 
趮勝寒 靚勝炅 請靚可以爲天下正 



\subsection{知足}



天下有道 卻走馬以糞 天下无道 戎馬生於郊 
罪莫大於可欲 禍莫大於不知足 咎莫憯於欲得 
故知足之足 恆足矣




\subsection{知天下}



不出於戶 以知天下 不規於牖 以知天道 
其出也彌遠 其知也彌少 
是以聲(聖)人不行而知 不見而名 弗爲而成




\subsection{无爲}



爲學者日益 聞道者日損 損之又損 以至於无爲 无爲而无不爲 
將欲取天下者恆无事 及其有事也 不足以取天下 



\subsection{德善}



聲(聖)人恆无心 以百姓之心爲心 
善者善之 不善者亦善之 德善也 
信者信之 不信者亦信之 德信也 
聲(聖)人之在天下 翕翕焉 爲天下渾心 百姓皆屬其耳目焉 聲(聖)人皆孩之 



\subsection{生死}



出生 入死 
生之徒十有三 死之徒十有三 而民生生 動皆之死地之十有三 
夫何故也 以其生生也 
蓋聞善執生者 陵行不辟兕虎 入軍不被甲兵 兕无所椯其角 虎无所昔其蚤 兵无所容其刃 
夫何故也 以其无死地焉




\subsection{尊貴}



道生之 而德畜之 物形之 而器成之 
是以萬物尊道而貴德 
道之尊 德之貴也 夫莫之而恆自祭也 
道生之 畜之 長之 遂之 亭之 毒之 養之 覆之 
生而弗有也 爲而弗寺也 長而弗宰也 此之謂玄德 



\subsection{守母}



天下有始 以爲天下母 
既得其母 以知其子 復守其母 沒身不殆 
塞其兑 閉其門 終身不堇 
啓其悶 濟其事 終身不救 
見小曰明 守柔曰強 
用其光 復歸其明

毋遺身殃 是謂襲常




\subsection{盜桍}



使我挈然有知也 行於大道 唯迆是畏 
大道甚夷 民甚好解 
朝甚除 田甚蕪 倉甚虛 服文采 帶利劍 厭食而齎財有餘 是謂盜桍 
盜桍 非道也




\subsection{善觀}



善建者不拔 善抱者不脫 子孫以祭祀不絕 
修之身 其德乃真 修之家 其德有餘 修之鄉 其德乃長 修之邦 其德乃豐 修之天下 其德乃博 
以身觀身 以家觀家 以鄉觀鄉 以邦觀邦 以天下觀天下 
吾何以知天下然茲 以此 



\subsection{含德}



含德之厚者 比於赤子 蜂地弗螫 鳥猛獸弗搏 骨弱筋柔而握固 未知牝牡之合而朘怒 精之至也 終日號而不 和之至也 
和曰常 知和曰明 益生曰祥 心使氣曰強 
物壯即老 謂之不道 不道早已 



\subsection{玄同}



知者弗言 言者弗知 
塞其悶 閉其門 和其光 同其尘 坐其兌而解其紛 是謂玄同 
故不可得而親 亦不可得而踈 不可得而利 亦不可得而害 不可得而貴 亦不可得而淺 故爲天下貴




\subsection{治邦}



以正之邦 以畸用兵 以无事取天下 吾何以知其然也  
夫天下多忌諱 而民彌貧 民多利器 而邦家茲昏 人多知 而奇物茲起 法物茲彰 而盜賊多有 
是以聲(聖)人之言曰 我无爲也 而民自化 我好靜 而民自正 我无事 民自富 我欲不欲 而民自樸 



\subsection{爲正}



其正 其民屯屯 其正察察 其邦夬夬 
禍 福之所倚 福 禍之所伏 孰知其極 
其无正也 正復爲奇 善復爲訞 
人之迷也 其日固久矣 
是以方而不割 兼而不刺 直而不絏 光而不耀 



\subsection{長生}



治人事天 莫若嗇 
夫唯嗇 是以服 服胃之重積德 重積德則无不克 无不克則莫知其極 莫知其極 可以有國 有國之母 可以長久 
是謂深根固柢 長生久視之道也 



\subsection{居位}



治大國 若烹小鮮 
以道蒞天下 其鬼不神 非其鬼不神也 其神不傷人也 非其申不傷人也 聲(聖)人亦弗傷也 
夫兩不相傷 故德交歸焉 



\subsection{處下}



大邦者下流也 天下之牝也 天下之郊也 
牝恆以靚勝牡 爲其靚也 故宜爲下 
大邦以下小邦 則取小邦 小邦以下大邦 則取於大邦 故或下以取 或下而取 
故大邦者 不過欲兼畜人 小邦者 不過欲入事人 
夫皆得其欲 則大者宜爲下 



\subsection{道注}



道者 萬物之注也 善 人之也 不善 人之所也 美言可以市 奠行可以賀人 
人之不善 何棄之有 
故立天子 置三卿 雖有共之璧 以先四馬 不若坐而進此 
古之所以貴此者何 不謂求以得 有罪以免與 故爲天下貴 



\subsection{无難}



爲无爲 事无事 味无未 
大小多少 報怨以德 
圖難於其易也 爲大於其細也 天下之難作於易 天下之大作於細 
是以聲(聖)人終不爲大 故能成其大 
夫輕諾必寡信 多易必多難 
是以聲(聖)人猶難之 故終於无難 



\subsection{輔物}



其安也 易持也 其未兆也 易謀也 其脆也 易判也 其微也 易散也 爲之於其未有也 治之於其未亂也 合抱之木 生於毫末 九成之臺 作於羸土 百仁之高 始於足下 
爲之者敗之 執之者失之 是以聲(聖)人无爲也 故无敗也 无執也 故无失也 民之從事也 恆於其成事而敗之 故慎終若始 則无敗事矣 是以聲(聖)人欲不欲 而不貴難得之 學不學 而復眾人之所過 能輔萬物之自然 而弗敢爲 



\subsection{玄德}



故曰爲道者非以明民也 將以愚之也 民之難治也 以其知也 
故以知知邦 邦之賊也 以不知知邦 邦之德也 
恆知此兩者 亦稽式也 恆知稽式 此謂玄德 
玄德深矣 遠矣 與物反矣 乃至大順 



\subsection{江海}



江海所以能爲百浴王者 以其善下之 是以能爲百浴王 
是以聲(聖)人之欲上民也 必以其言下之 欲先民也 必以其身後之 故居前而民弗害也 居上而民弗重也 天下樂佳而弗猒也 非以其无諍與 故天下莫能與諍 



\subsection{安居}



小邦寡民 
使十百人之器毋用 使民重死而遠徙 有車周无所乘之 有甲兵无所陳之 使民復結繩而用之 
甘其食 美其服 樂其俗 安其居 
鄰邦相望 雞狗之聲相聞 民至老死 不相往來 



\subsection{不積}



信言不美 美言不信 
知者不博 博者不知 
善者不多 多者不善 
聲(聖)人无積 既以爲人己愈有 既以予人己愈多 
故天之道 利而不害 人之道 爲而弗爭 



\subsection{三寶}



天下皆謂我大 不宵 夫唯大 故不宵 若宵 細久矣 
我恆有三寶之 一曰慈 二曰檢 三曰不敢爲天下先 
夫慈 故能勇 檢 故能廣 不敢爲天下先 故能爲成事長 
今捨其慈且勇 捨其檢且廣 捨其後且先 則必死矣 
夫慈 以戰則勝 以守則固 
天將建之 如以慈垣之 



\subsection{不爭}



善爲士者不武 善戰者不怒 善勝敵者弗與 善用人者爲之下 
是謂不爭之德 是謂用人 是謂天 古之極也 



\subsection{用兵}



用兵有言曰 吾不敢爲主而爲客 吾不進寸而芮尺 
是謂行无行 攘无臂 執无兵 乃无敵矣 
莫大於无適 无適近亡吾寶矣 
故稱兵相若 則哀者勝矣 



\subsection{懷玉}



吾言甚易知也 甚易行也 而人莫之能知也 而莫之能行也 
言有君 事有宗 其唯无知也 是以不我知 
知我者希 則我貴矣 是以聲(聖)人被褐而懷玉 



\subsection{知病}



知不知 尚矣 不知不知 病矣 
是以聲(聖)人之不病 以其病病也 是以不病 



\subsection{畏畏}



民之不畏畏 則大畏將至矣 
毋閘其所居 毋猒其所生 夫唯弗猒 是以不猒 
是以聲(聖)人 自知而不自見也 自愛而不自貴也 故去被取此  



\subsection{天網}



勇於敢者則殺 勇於不敢者則栝 此兩者或利或害 
天之所惡 孰知其故 
天之道 不彈而善勝 不言而善應 不召而自來 彈而善謀 
天網恢恢 疏而不失 



\subsection{司殺}



若民恆且不畏死 奈何以殺懼之也 
若民恆是死 則而爲者 吾將得而殺之 夫孰敢矣 
若民恆且必畏死 則恆有司殺者 
夫代司殺者殺 是代大匠斵也 夫代大匠斵者 則希不傷其手矣 



\subsection{貴生}



人之饑也 以其取食之多也 是以饑 
百姓之不治也 以其上有以爲也 是以不治 
民之輕死 以其求生之厚也 是以輕死 
夫唯无以生爲者 是賢貴生 



\subsection{柔弱}



人之生也柔弱 其死也仞賢強 萬物草木之生也柔脆 其死也 
故曰 堅強者 死之徒也 柔弱微細 生之徒也 
兵強則不勝 木強則恆 
強大居下 柔弱微細居上 



\subsection{天道}



天下之道 猶張弓者也 高者抑之 下者舉之 有餘者損之 不足者補之 
故天之道 有餘而益不足 人之道 不足而奉有餘 
孰能有餘而有以取奉於天者 此有道者乎 
是以聲(聖)人爲而弗有 成功而弗居也 若此 其不欲見賢也 



\subsection{水德}



天下莫柔弱於水 而攻堅強者莫之能先也 以其无以易之也 
水之勝剛也 弱之勝強也 天下莫弗知也 而莫之能行之也 
故聲(聖)人之言云曰 受邦之詬 是謂社稷之主 受邦之不祥 是謂天下之王 
正言若反 



\subsection{右介}



和大怨 必有餘怨 焉可以爲善 
是以聖右介而不以責於人 
故有德司介 无德司 
夫天道无親 恆與善人


\section{道 篇}


\subsection{觀眇}

道 可道也 非恆道也 名 可名也 非恆名也 無名 萬物之始也 有名 萬物之母也 【故】恆无欲也 以觀其眇(妙) 恆有欲也 以观其所徼 兩者同出 異名同胃(謂) 玄之有(又)玄 眾眇(妙)之【門】

\subsection{觀噭}

天下皆知美爲美 惡已 皆知善 訾(斯)不善矣 有无之相生也 難易之相成也 長短之相刑(形)也 高下之相盈也 意(音)聲之相和也 先後之相遀(隨) 恆也 是以聲(聖)人居无爲之事 【行不言之教 萬物作而弗始】也 爲而弗志(恃)也 成功而弗居也 夫唯弗居 是以弗去



\subsection{安民}



不上賢 【使民不爭 不貴難得之貨 使】民不爲【盜】 不【見可欲 使】民不亂 是以聲(聖)人之【治也 虛其心 實其腹 弱其志】 強其骨 恆使民无知 无欲也 使【夫知不敢 弗爲而已 則无不治矣



\subsection{道用}



道沖 而用之又弗】盈也 潚(淵)呵 始(似)萬物之宗 
銼(挫)其兌 解其紛 和其光 同【其塵 
湛呵 似】或存 
吾不知【誰】子也 象帝之先 



\subsection{用中}



天地不仁 以萬物爲芻狗 聲(聖)人不仁 以百省(姓)【爲芻】狗 
天地【之】間 【其】猶橐籥舆(與) 虛而不淈(屈) 蹱(動)而俞(愈)出 
多聞數窮 不若守於中 



\subsection{浴神}



浴(谷)神不死 是謂玄牝 
玄牝之門 是胃(謂)【天】地之根 
綿綿呵若存 用之不堇(勤)



\subsection{无私}



天長 地久 天地之所以能【長】且久者 以其不自生也 故能長生 
是以聲(聖)人芮(退)其身而身先 外其身而身存 
不以其无【私】舆(與) 故能成其私 



\subsection{治水}


上善治(似)水 水利萬物而有静(爭) 居眾人之所惡 故幾於道矣

居善地 心善潚(淵) 予善信 正(政)善治 事善能 蹱(動)善時

夫唯不爭 故无尤



\subsection{持盈}



持而盈之 不若其已 
錐而銳之 不可常葆之 
金玉盈室 莫之守也 
貴富而驕 自遺咎也 
功述身芮 天之道也 



\subsection{无不爲}



戴營抱一 能毋離乎  摶氣至柔 能嬰兒乎  
脩除玄藍 能毋疵乎 愛民栝國 能毋以知乎 
天門啓闔 能爲雌乎 明白四達 能毋以知乎 
生之 畜之 生而弗有 長而弗宰也 是謂玄德 



\subsection{玄中}



卅輻同一轂 當其无 有車之用也 
埴而爲器 當其无 有埴器之用也 
鑿戶牖 當其无 有室之用也 
故有之以爲利 无之以爲用 



\subsection{爲腹}



五色使人目盲 馳騁田臘 使人心發狂 難得之貨 使人之行妨 五味使人之口爽 五音使人之耳聾 
是以聲(聖)人之治也 爲腹不爲目 故去疲取此 



\subsection{寵辱}



寵辱若驚 貴大梡若身 
何謂寵辱若驚 寵之爲下 得之若驚 失之若驚 是謂寵辱若驚 
何謂貴大梡若身 吾所以有大患者 爲吾有身也 及吾无身 有何患 
故貴爲身於爲天下 若可以托天下矣 愛以身爲天下 如可以寄天下 



\subsection{道紀}



視之而弗見 名之曰微 聽之而弗聞 名之曰希 捪之而弗得 名之曰夷 三者不可至計 故混而爲一 
一者 其上不攸 其下不忽 尋尋呵 不可名也 復歸於无物 
是謂无狀之狀 无物之象 是謂忽望 隨而不見其後 迎而不見其首 
執今之道 以御今之有 以知古始 是謂道紀 



\subsection{不盈}



古之善爲道者 微眇玄達 深不可志 夫唯不可志 故強爲之容 曰 
與呵 其若冬涉水 猶呵 其若畏四鄰 儼呵 其若客 渙呵 其若淩澤 沌呵 其若樸 湷呵 其若濁 呵 其若浴 
濁而情之余清 女以重之余生 
葆此道不欲盈 夫唯不欲盈 是以能敝而不成 



\subsection{歸根}



至虛 極也 守靜 督也 萬物旁作 吾以觀其復也 天物雲雲 各復歸於其根 
曰靜 靜是謂復命 復命常也 知常明也 不知常 茫茫作兇 知常容 容乃公 
公乃王 王乃天 天乃道 道乃久 歿身不殆




\subsection{知有}



大上 下知有之 其次親譽之 其次畏之 其下母之 
信不足 案有不信 猶呵 其貴言也 
成功遂事 而百省謂我自然




\subsection{四有}



故大道廢 案有仁義 
知快出 案有大僞 
六親不和 案有畜慈 
邦家亂 案有貞臣 
<br> 



\subsection{樸素}



絕聖棄智 民利百負 
絕仁棄義 民復畜慈 
絕巧棄利 盜賊无有 
此三言也 以爲文未足 故令之有所屬 
見素抱樸 少私而寡欲 絕學无憂 

\subsection{食母}



唯與訶 其相去幾何 美與惡 其相去何若 
人之所畏 亦不可以不畏 
望呵 其未央哉 
眾人巸巸 若鄉於大牢 而春登臺 
我泊焉未兆 若嬰兒未咳 纍呵 似无所歸 
眾人皆有餘 我獨遺 我愚人之心也 湷湷呵 鬻人昭昭 我獨呵 鬻人蔡蔡 我獨悶悶呵 
忽呵 其若海 望呵 其若无所止 
眾人皆有以 我獨頑以悝 吾欲獨異於人 而貴食母 



\subsection{從道}



孔德之容 唯道是從 道之物 唯望唯忽 
忽呵 望呵 中有象呵 望呵 忽呵 中有物呵 呵 鳴呵 中有請吔 其請甚真 其中有信 
自今及古 其名不去 以順眾

吾何以知眾父之祭 以此 



\subsection{弗居}



炊者不立 自視不彰 
自見者不明 自伐者无功 自矜者不長 
其在道也 曰食贅行 
物或惡之 故有欲者弗居 



\subsection{執一}



曲則金 枉則定 
洼則盈 敝則新 少則得 多則惑 
是以聲(聖)人執一以爲天下牧 
不自視故明 不自見故彰 不自伐故有功 弗矜故能長 
夫唯不爭 故莫能與之爭 
古之所謂曲則金者 幾語才 誠金歸之 



\subsection{同道}



希言自然 
飄風不終朝 暴雨不終日 
孰爲此 天地而弗能久 又況於人乎 
故從事而道者同於道 德者同於德 失者同於失 
同於德者 道亦德之 同於失者 道亦失之 



\subsection{昆成}



有物昆成 先天地生 
繡呵 繆呵 獨立而不亥 可以爲天地母 
吾未知其名 字之曰道 吾強爲之名曰大 
大曰筮 筮曰遠 遠曰反 
道大 天大 地大 王亦大 
國中有四大 而王居一焉 
人法地 地法天 天法道 道法自然 




\subsection{輜重}



重爲巠根 清爲趮君 
是以君子眾日行 不離其甾重 
唯有環官 燕處則昭 
若若何萬乘之王 而以身巠於天下 
巠則失本 趮則失君 



\subsection{曳明}



善行者无迹 善言者无瑕適 善數者不用檮 善閉者无籥而不可啓也 善結者无纆約而不可解也 
是以聲(聖)人恆善人 而无棄人 物无棄財 是謂明 
故善人 善人之師 不善人 善人之齎也 
不貴其師 不愛其齎 唯知乎大眯 是謂眇要 



\subsection{恆德}



知其雄 守其雌 爲天下谿 爲天下谿 恆德不離 恆德不離 復歸嬰兒 
知其榮 守其辱 爲天下浴 爲天下浴 恆德乃足 德乃足 復歸於樸 
知其白 守其黑 爲天下式 爲天下式 恆德不忒 恆德不忒 復歸於无極 
散則爲器 聲(聖)人用則爲官長 夫大制无割 



\subsection{自然}



將欲取天下而爲之 吾見其弗得已 
天下神器也 非可爲者也 
爲者敗之 執者失之 
物或行或隨 或吹或炅 或強或 或坯或橢 
是以聲(聖)人去甚 去大 去楮 



\subsection{不強}



以道佐人主 不以兵強於天下 其事好還 
師之所居 楚棘生之 
善者果而已矣 毋以取強焉 
果而毋 果而勿矜 果而勿伐 果而毋得已居 是謂果而不強 
物壯而老 是謂之不道 不道蚤已 



\subsection{貴左}



夫兵者 不祥之器也 物或惡之 故有欲者弗居 
君子居則貴左 用兵則貴右 
故兵者 非君子之器也 兵者 不祥之器也 不得已而用之 銛龐爲上 
勿美也 若美之 是樂殺人也 
夫樂殺人 不可以得志於天下矣 
是以吉事尚左 喪事尚右 
是以偏將軍居左 上將軍居右 言以喪禮居之也 
殺人眾 以悲依立之 戰勝 以喪禮處之 



\subsection{知止}



道恆无名 
唯小 而天下弗敢臣 
侯王若能守之 萬物將自賓 天地相合 以俞甘洛 民莫之令而自均焉 
始制有名 名亦既有 夫亦將知止 知止可以不殆 
俾道之在天下也 猶小浴之與江海也 



\subsection{盡己}



知人者 智也 自知者 明也 
勝人者 有力也 自勝者 強也 
知足者 富也 強行者 有志也 
不失其所者 久也 死不忘者 壽也 



\subsection{成大}



道渢呵 其可左右也 
成功遂事 而弗名有也 萬物歸焉而弗爲主 
則恆无欲也 可名於小 萬物歸焉而弗爲主 可名於大 
是以聲(聖)人之能成大也 以其不爲大也 故能成大 



\subsection{大象}



執大象 天下住 住而不害 安平大 
樂與餌 過格止 
故道之出言也 曰 談呵 其无味也 視之不足見也 聽之不足聞也 用之不可既也 



\subsection{微明}



將欲拾之 必故張之 將欲弱之 必故強之 將欲去之 必故與之 將欲奪之 必故予之 是謂微明 
友弱勝強 
魚不可脫於淵 邦利器不可以示人 



\subsection{无名}



道恆无名 
侯王若守之 萬物將自化 
化而欲作 吾將貞之以无名之朴 
貞之以无名之朴 夫將不辱 
不辱以静 天地將自正

\end{document}
