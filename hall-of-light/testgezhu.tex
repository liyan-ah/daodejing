\documentclass{ctexart}
\usepackage{zhspacing}
\zhspacing
\usepackage{gezhu}

\usepackage{geometry}
\newgeometry{
  top=50pt, bottom=50pt, left=86pt, right=86pt,
    headsep=5pt,
}
\savegeometry{mdGeo}
\loadgeometry{mdGeo}

\usepackage{everyshi}
%% \setCJKmainfont{AdobeSongStd-Light.otf} 

\begin{document}


%% \part{黃帝內經·素問}
%% \section{上古天真論篇第一}

  昔在黃帝,生而神靈,弱而能言,幼而徇齊,長而敦敏,成而登天。乃問於天師曰:余聞上古之人,春秋皆度百歲,而動作不衰;今時之人,年半百而動作皆衰者,時世異耶,人將失之耶。岐伯對曰:上古之人,其知道者,法於陰陽,和於術數,食飲有節,起居有常,不妄作勞,故能形與神俱,而盡終其天年,度百歲乃去。今時之人不然也,以酒為漿,以妄為常,醉以入房,以欲竭其精,以耗散其真,不知持滿,不時御神,務快其心,逆於生樂,起居無節,故半百而衰也。
  夫上古聖人之教下也,皆謂之虛邪賊風,避之有時,恬淡虛無,真氣從之,精神內守,病安從來。是以志閒而少欲,心安而不懼,形勞而不倦,氣從以順,各從其欲,皆得所願。故美其食,任其服,樂其俗,高下不相慕,其民故曰朴。是以嗜欲不能勞其目,淫邪不能惑其心,愚智賢不肖不懼於物,故合於道。所以能年皆度百歲,而動作不衰者,以其德全不危也。
  帝曰:人年老而無子者,材力盡耶,將天數然也。岐伯曰:女子七歲。腎氣盛,齒更髮長;二七而天癸至,任脈通,太衝脈盛,月事以時下,故有子;三七,腎氣平均,故真牙生而長極;四七,筋骨堅,髮長極,身體盛壯;五七,陽明脈衰,面始焦,發始墮;六七,三陽脈衰於上,面皆焦,發始白;七七,任脈虛,太衝脈衰少,天癸竭,地道不通,故形壞而無子也。丈夫八歲,腎氣實,髮長齒更;二八,腎氣盛,天癸至,精氣溢寫,陰陽和,故能有子;三八,腎氣平均,筋骨勁強,故真牙生而長極;四八,筋骨隆盛,肌肉滿壯;五八,腎氣衰,發墮齒槁;六八,陽氣衰竭於上,面焦,髮鬢頒白;七八,肝氣衰,筋不能動,天癸竭,精少,腎藏衰,形體皆極;八八,則齒發去。腎者主水,受五藏六府之精而藏之,故五藏盛,乃能寫。今五藏皆衰,筋骨解墮,天癸盡矣。故髮鬢白,身體重,行步不正,而無子耳。
  帝曰:有其年已老而有子者何也。岐伯曰:此其天壽過度,氣脈常通,而腎氣有餘也。此雖有子,男不過盡八八,女不過盡七七,而天地之精氣皆竭矣。
  帝曰:夫道者年皆百數,能有子乎。岐伯曰:夫道者能卻老而全形,身年雖壽,能生子也。
  黃帝曰:余聞上古有真人者,提挈天地,把握陰陽,呼吸精氣,獨立守神,肌肉若一,故能壽敝天地,無有終時,此其道生。中古之時,有至人者,淳德全道,和於陰陽,調於四時,去世離俗,積精全神,遊行天地之間,視聽八達之外,此蓋益其壽命而強者也,亦歸於真人。其次有聖人者,處天地之和,從八風之理,適嗜欲於世俗之間。無恚嗔之心,行不欲離於世,被服章,舉不欲觀於俗,外不勞形於事,內無思想之患,以恬愉為務,以自得為功,形體不敝,精神不散,亦可以百數。其次有賢人者,法則天地,像似日月,辨列星辰,逆從陰陽,分別四時,將從上古合同於道,亦可使益壽而有極時。


\section{四氣調神大論篇第二}

  春三月,此謂發陳,天地俱生,萬物以榮,夜臥早起,廣步於庭,被發緩形,以使志生,生而勿殺,予而勿奪,賞而勿罰,此春氣之應,養生之道也。逆之則傷肝,夏為寒變,奉長者少。
  夏三月,此謂蕃秀,天地氣交,萬物華實,夜臥早起,無厭於日,使志無怒,使華英成秀,使氣得洩,若所愛在外,此夏氣之應,養長之道也。逆之則傷心,秋為痎瘧,奉收者少,冬至重病。
  秋三月,此謂容平,天氣以急,地氣以明,早臥早起,與雞俱興,使志安寧,以緩秋刑,收斂神氣,使秋氣平,無外其志,使肺氣清,此秋氣之應,養收之道也。逆之則傷肺,冬為飧洩,奉藏者少。
  冬三月,此謂閉藏,水冰地坼,無擾乎陽,早臥晚起,必待日光,使志若伏若匿,若有私意,若已有得,去寒就溫,無洩皮膚,使氣亟奪,此冬氣之應,養藏之道也。逆之則傷腎,春為痿厥,奉生者少。
  天氣,清淨光明者也,藏德不止,故不下也。天明則日月不明,邪害空竅,陽氣者閉塞,地氣者冒明,雲霧不精,則上應白露不下。交通不表,萬物命故不施,不施則名木多死。惡氣不發,風雨不節,白露不下,則菀槁不榮。賊風數至,暴雨數起,天地四時不相保,與道相失,則未央絕滅。唯聖人從之,故身無奇病,萬物不失,生氣不竭。逆春氣,則少陽不生,肝氣內變。逆夏氣,則太陽不長,心氣內洞。逆秋氣,則太陰不收,肺氣焦滿。逆冬氣,則少陰不藏,腎氣獨沉。夫四時陰陽者,萬物之根本也。所以聖人春夏養陽,秋冬養陰,以從其根,故與萬物沉浮於生長之門。逆其根,則伐其本,壞其真矣。
  故陰陽四時者,萬物之終始也,死生之本也,逆之則災害生,從之則苛疾不起,是謂得道。道者,聖人行之,愚者佩之。從陰陽則生。逆之則死,從之則治,逆之則亂。反順為逆,是謂內格。
  是故聖人不治已病,治未病,不治已亂,治未亂,此之謂也。夫病已成而後藥之,亂已成而後治之,譬猶渴而穿井,而鑄錐,不亦晚乎。


\section{生氣通天論篇第三}

  黃帝曰:夫自古通天者生之本,本於陰陽。天地之間,六合之內,其氣九州、九竅、五藏、十二節,皆通乎天氣。其生五,其氣三,數犯此者,則邪氣傷人,此壽命之本也。
  蒼天之氣清淨,則志意治,順之則陽氣固,雖有賊邪,弗能害也,此因時之序。故聖人傳精神,服天氣,而通神明。失之則內閉九竅,外壅肌肉,衛氣散解,此謂自傷,氣之削也。
  陽氣者若天與日,失其所,則折壽而不彰,故天運當以日光明。是故陽因而上,衛外者也。因於寒,欲如運樞,起居如驚,神氣乃浮。因於暑,汗煩則喘喝,靜則多言,體若燔炭,汗出而散。因於濕,首如裹,濕熱不攘,大筋短,小筋弛長,短為拘,弛長為痿。因於氣,為腫,四維相代,陽氣乃竭。
  陽氣者,煩勞則張,精絕,辟積於夏,使人煎厥。目盲不可以視,耳閉不可以聽,潰潰乎若壞都,汨汨乎不可止。陽氣者,大怒則形氣絕,而血菀於上,使人薄厥。有傷於筋,縱,其若不容,汗出偏沮,使人偏枯。汗出見濕,乃生痤。高粱之變,足生大丁,受如持虛。勞汗當風,寒薄為,郁乃痤。
  陽氣者,精則養神,柔則養筋。開闔不得,寒氣從之,乃生大僂。陷脈為瘻。留連肉腠,俞氣化薄,傳為善畏,及為驚駭。營氣不從,逆於肉理,乃生癰腫。魄汗未盡,形弱而氣爍,穴俞以閉,發為風瘧。
  故風者,百病之始也,清靜則肉腠閉拒,雖有大風苛毒,弗之能害,此因時之序也。
  故病久則傳化,上下不併,良醫弗為。故陽畜積病死,而陽氣當隔,隔者當寫,不亟正治,粗乃敗之。
  故陽氣者,一日而主外,平旦人氣生,日中而陽氣隆,日西而陽氣已虛,氣門乃閉。是故暮而收拒,無擾筋骨,無見霧露,反此三時,形乃困薄。
  岐伯曰:陰者,藏精而起亟也,陽者,衛外而為固也。陰不勝其陽,則脈流薄疾,並乃狂。陽不勝其陰,則五藏氣爭,九竅不通。是以聖人陳陰陽,筋脈和同,骨髓堅固,氣血皆從。如是則內外調和,邪不能害,耳目聰明,氣立如故。
  風客淫氣,精乃亡,邪傷肝也。因而飽食,筋脈橫解,腸澼為痔。因而大飲,則氣逆。因而強力,腎氣乃傷,高骨乃壞。
  凡陰陽之要,陽密乃固,兩者不和,若春無秋,若冬無夏,因而和之,是謂聖度。故陽強不能密,陰氣乃絕,陰平陽秘,精神乃治,陰陽離決,精氣乃絕。
  因於露風,乃生寒熱。是以春傷於風,邪氣留連,乃為洞洩,夏傷於暑,秋為瘧。秋傷於濕,上逆而咳,發為痿厥。冬傷於寒,春必溫病。四時之氣,更傷五藏。
  陰之所生,本在五味,陰之五宮,傷在五味。是故味過於酸,肝氣以津,脾氣乃絕。味過於咸,大骨氣勞,短肌,心氣抑。味過於甘,心氣喘滿,色黑,腎氣不衡。味過於苦,脾氣不濡,胃氣乃厚。味過於辛,筋脈沮弛,精神乃央。是故謹和五味,骨正筋柔,氣血以流,腠理以密,如是,則骨氣以精,謹道如法,長有天命。


\section{金匱真言論篇第四}

  黃帝問曰:天有八風,經有五風,何謂?岐伯對曰:八風發邪,以為經風,觸五藏,邪氣發病。所謂得四時之勝者,春勝長夏,長夏勝冬,冬勝夏,夏勝秋,秋勝春,所謂四時之勝也。
  東風生於春,病在肝,俞在頸項;南風生於夏,病在心,俞在胸脅;西風生於秋,病在肺,俞在肩背;北風生於冬,病在腎,俞在腰股;中央為土,病在脾,俞在脊。故春氣者病在頭,夏氣者病在藏,秋氣者病在肩背,冬氣者病在四支。
  故春善病鼽衄,仲夏善病胸脅,長夏善病洞洩寒中,秋善病風瘧,冬善病痹厥。故冬不按蹻,春不鼽衄,春不病頸項,仲夏不病胸脅,長夏不病洞洩寒中,秋不病風瘧,冬不病痹厥,飧洩而汗出也。
  夫精者身之本也。故藏於精者春不病溫。夏暑汗不出者,秋成風瘧。此平人脈法也。
  故曰:陰中有陰,陽中有陽。平旦至日中,天之陽,陽中之陽也;日中至黃昏,天之陽,陽中之陰也;合夜至雞鳴,天之陰,陰中之陰也;雞鳴至平旦,天之陰,陰中之陽也。
  故人亦應之。夫言人之陰陽,則外為陽,內為陰。言人身之陰陽,則背為陽,腹為陰。言人身之藏府中陰陽。則藏者為陰,府者為陽。肝心脾肺腎五藏,皆為陰。膽胃大腸小腸膀胱三焦六府,皆為陽。所以欲知陰中之陰陽中之陽者何也,為冬病在陰,夏病在陽,春病在陰,秋病在陽,皆視其所在,為施針石也。故背為陽,陽中之陽,心也;背為陽,陽中之陰,肺也;腹為陰,陰中之陰,腎也;腹為陰,陰中之陽,肝也;腹為陰,陰中之至陰,脾也。此皆陰陽表裡內外雌雄相俞應也,故以應天之陰陽也。
  帝曰:五藏應四時,各有收受乎?岐伯曰:有。東方青色,入通於肝,開竅於目,藏精於肝,其病發驚駭。其味酸,其類草木,其畜雞,其穀麥,其應四時,上為歲星,是以春氣在頭也,其音角,其數八,是以知病之在筋也,其臭臊。
  南方赤色,入通於心,開竅於耳,藏精於心,故病在五藏,其味苦,其類火,其畜羊,其谷黍,其應四時,上為熒惑星,是以知病之在脈也,其音徵,其數七,其臭焦。
  中央黃色,入通於脾,開竅於口,藏精於脾,故病在舌本,其味甘,其類土,其畜牛,其谷稷,其應四時,上為鎮星,是以知病之在肉也,其音宮,其數五,其臭香。
  西方白色,入通於肺,開竅於鼻,藏精於肺,故病在背,其味辛,其類金,其畜馬,其穀稻,其應四時,上為太白星,是以知病之在皮毛也,其音商,其數九,其臭腥。
  北方黑色,入通於腎,開竅於二陰,藏精於腎,故病在谿,其味咸,其類水,其畜彘,其谷豆,其應四時,上為辰星,是以知病之在骨也,其音羽,其數六,其臭腐。故善為脈者,謹察五藏六府,一逆一從,陰陽表裡雌雄之紀,藏之心意,合心於精,非其人勿教,非其真勿授,是謂得道。


\section{陰陽應像大論篇第五}

黃帝曰:陰陽者,天地之道也,萬物之綱紀,變化之父母,生殺之本始,神明之府也。治病必求於本。故積陽為天,積陰為地。陰靜陽躁,陽生陰長,陽殺陰藏。陽化氣,陰成形。寒極生熱,熱極生寒。寒氣生濁,熱氣生清。清氣在下,則生飧洩;濁氣在上,則生(月真)脹。此陰陽反作,病之逆從也。
  故清陽為天,濁陰為地;地氣上為雲,天氣下為雨;雨出地氣,雲出天氣。故清陽出上竅,濁陰出下竅;清陽發腠理,濁陰走五藏;清陽實四支,濁陰歸六府。
  水為陰,火為陽,陽為氣,陰為味。味歸形,形歸氣,氣歸精,精歸化,精食氣,形食味,化生精,氣生形。味傷形,氣傷精,精化為氣,氣傷於味。
  陰味出下竅,陽氣出上竅。味厚者為陰,薄為陰之陽。氣厚者為陽,薄為陽之陰。味厚則洩,薄則通。氣薄則發洩,厚則發熱。壯火之氣衰,少火之氣壯。壯火食氣,氣食少火。壯火散氣,少火生氣。
  氣味辛甘發散為陽,酸苦湧洩為陰。陰勝則陽病,陽勝則陰病。陽勝則熱,陰勝則寒。重寒則熱,重熱則寒。寒傷形,熱傷氣。氣傷痛,形傷腫。故先痛而後腫者,氣傷形也;先腫而後痛者,形傷氣也。
  風勝則動,熱勝則腫,燥勝則干,寒勝則浮,濕勝則濡寫。
  天有四時五行,以生長收藏,以生寒暑燥濕風。人有五藏,化五氣,以生喜怒悲憂恐。故喜怒傷氣,寒暑傷形。暴怒傷陰,暴喜傷陽。厥氣上行,滿脈去形。喜怒不節,寒暑過度,生乃不固。故重陰必陽,重陽必陰。
  故曰:冬傷於寒,春必溫病;春傷於風,夏生飧洩;夏傷於暑,秋必痎瘧;秋傷於濕,冬生咳嗽。
  帝曰:余聞上古聖人,論理人形,列別藏府,端絡經脈,會通六合,各從其經,氣穴所發各有處名,谿谷屬骨皆有所起,分部逆從,各有條理,四時陰陽,盡有經紀,外內之應,皆有表裡,其信然乎?
  岐伯對曰:東方生風,風生木,木生酸,酸生肝,肝生筋,筋生心,肝主目。其在天為玄,在人為道,在地為化。化生五味,道生智,玄生神,神在天為風,在地為木,在體為筋,在藏為肝,在色為蒼,在音為角,在聲為呼,在變動為握,在竅為目,在味為酸,在志為怒。怒傷肝,悲勝怒;風傷筋,燥勝風;酸傷筋,辛勝酸。
  南方生熱,熱生火,火生苦,苦生心,心生血,血生脾,心主舌。其在天為熱,在地為火,在體為脈,在藏為心,在色為赤,在音為徵,在聲為笑,在變動為憂,在竅為舌,在味為苦,在志為喜。喜傷心,恐勝喜;熱傷氣,寒勝熱,苦傷氣,咸勝苦。
  中央生濕,濕生土,土生甘,甘生脾,脾生肉,肉生肺,脾主口。其在天為濕,在地為土,在體為肉,在藏為脾,在色為黃,在音為宮,在聲為歌,在變動為噦,在竅為口,在味為甘,在志為思。思傷脾,怒勝思;濕傷肉,風勝濕;甘傷肉,酸勝甘。
  西方生燥,燥生金,金生辛,辛生肺,肺生皮毛,皮毛生腎,肺主鼻。其在天為燥,在地為金,在體為皮毛,在藏為肺,在色為白,在音為商,在聲為哭,在變動為咳,在竅為鼻,在味為辛,在志為憂。憂傷肺,喜勝憂;熱傷皮毛,寒勝熱;辛傷皮毛,苦勝辛。
  北方生寒,寒生水,水生咸,咸生腎,腎生骨髓,髓生肝,腎主耳。其在天為寒,在地為水,在體為骨,在藏為腎,在色為黑,在音為羽,在聲為呻,在變動為栗,在竅為耳,在味為咸,在志為恐。恐傷腎,思勝恐;寒傷血,燥勝寒;咸傷血,甘勝咸。
  故曰:天地者,萬物之上下也;陰陽者,血氣之男女也;左右者,陰陽之道路也;水火者,陰陽之徵兆也;陰陽者,萬物之能始也。故曰:陰在內,陽之守也;陽在外,陰之使也。
  帝曰:法陰陽奈何?岐伯曰:陽勝則身熱,腠理閉,喘粗為之仰,汗不出而熱,齒干以煩冤,腹滿,死,能冬不能夏。陰勝則身寒,汗出,身常清,數栗而寒,寒則厥,厥則腹滿,死,能夏不能冬。此陰陽更勝之變,病之形能也。
  帝曰:調此二者奈何?岐伯曰:能知七損八益,則二者可調,不知用此,則早衰之節也。年四十,而陰氣自半也,起居衰矣。年五十,體重,耳目不聰明矣。年六十,陰痿,氣大衰,九竅不利,下虛上實,涕泣俱出矣。故曰:知之則強,不知則老,故同出而名異耳。智者察同,愚者察異,愚者不足,智者有餘,有餘則耳目聰明,身體輕強,老者復壯,壯者益治。是以聖人為無為之事,樂恬憺之能,從欲快志於虛無之守,故壽命無窮,與天地終,此聖人之治身也。
  天不足西北,故西北方陰也,而人右耳目不如左明也。地不滿東南,故東南方陽也,而人左手足不如右強也。帝曰:何以然?岐伯曰:東方陽也,陽者其精並於上,並於上,則上明而下虛,故使耳目聰明,而手足不便也。西方陰也,陰者其精並於下,並於下,則下盛而上虛,故其耳目不聰明,而手足便也。故俱感於邪,其在上則右甚,在下則左甚,此天地陰陽所不能全也,故邪居之。
  故天有精,地有形,天有八紀,地有五里,故能為萬物之父母。清陽上天,濁陰歸地,是故天地之動靜,神明為之綱紀,故能以生長收藏,終而復始。惟賢人上配天以養頭,下象地以養足,中傍人事以養五藏。天氣通於肺,地氣通於嗌,風氣通於肝,雷氣通於心,谷氣通於脾,雨氣通於腎。六經為川,腸胃為海,九竅為水注之氣。以天地為之陰陽,陽之汗,以天地之雨名之;陽之氣,以天地之疾風名之。暴氣象雷,逆氣象陽。故治不法天之紀,不用地之理,則災害至矣。
  故邪風之至,疾如風雨,故善治者治皮毛,其次治肌膚,其次治筋脈,其次治六府,其次治五藏。治五藏者,半死半生也。故天之邪氣,感則害人五藏;水谷之寒熱,感則害於六府;地之濕氣,感則害皮肉筋脈。
  故善用針者,從陰引陽,從陽引陰,以右治左,以左治右,以我知彼,以表知裡,以觀過與不及之理,見微得過,用之不殆。善診者,察色按脈,先別陰陽;審清濁,而知部分;視喘息,聽音聲,而知所苦;觀權衡規矩,而知病所主。按尺寸,觀浮沉滑澀,而知病所生;以治無過,以診則不失矣。
  故曰:病之始起也,可刺而已;其盛,可待衰而已。故因其輕而揚之,因其重而減之,因其衰而彰之。形不足者,溫之以氣;精不足者,補之以味。其高者,因而越之;其下者,引而竭之;中滿者,寫之於內;其有邪者,漬形以為汗;其在皮者,汗而發之;其慓悍者,按而收之;其實者,散而寫之。審其陰陽,以別柔剛,陽病治陰,陰病治陽,定其血氣,各守其鄉,血實宜決之,氣虛宜掣引之。

\section{陰陽離合論篇第六}

  黃帝問曰:余聞天為陽,地為陰,日為陽,月為陰,大小月三百六十日成一歲,人亦應之。今三陰三陽,不應陰陽,其故何也?岐伯對曰:陰陽者,數之可十,推之可百,數之可千,推之可萬,萬之大不可勝數,然其要一也。
  天覆地載,萬物方生,未出地者,命曰陰處,名曰陰中之陰;則出地者,命曰陰中之陽。陽予之正,陰為之主。故生因春,長因夏,收因秋,藏因冬,失常則天地四塞。陰陽之變,其在人者,亦數之可數。
  帝曰:願聞三陰三陽之離合也。岐伯曰:聖人南面而立,前曰廣明,後曰太沖,太沖之地,名曰少陰,少陰之上,名曰太陽,太陽根起於至陰,結於命門,名曰陰中之陽。中身而上,名曰廣明,廣明之下,名曰太陰,太陰之前,名曰陽明,陽明根起於厲兌,名曰陰中之陽。厥陰之表,名曰少陽,少陽根起於竅陰,名曰陰中之少陽。是故三陽之離合也,太陽為開,陽明為闔,少陽為樞。三經者,不得相失也,搏而勿浮,命曰一陽。
  帝曰:願聞三陰。岐伯曰:外者為陽,內者為陰,然則中為陰,其衝在下,名曰太陰,太陰根起於隱白,名曰陰中之陰。太陰之後,名曰少陰,少陰根起於湧泉,名曰陰中之少陰。少陰之前,名曰厥陰,厥陰根起於大敦,陰之絕陽,名曰陰之絕陰。是故三陰之離合也,太陰為開,厥陰為闔,少陰為樞。
  三經者不得相失也。搏而勿沉,名曰一陰。陰陽(雩重)(雩重),積傳為一週,氣裡形表而為相成也。


\section{陰陽別論篇第七}

  黃帝問曰:人有四經十二從,何謂?岐伯對曰:四經應四時,十二從應十二月,十二月應十二脈。脈有陰陽,知陽者知陰,知陰者知陽。凡陽有五,五五二十五陽。所謂陰者,真藏也,見則為敗,敗必死也;所謂陽者,胃脘之陽也。別於陽者,知病處也;別於陰者,知死生之期。
  三陽在頭,三陰在手,所謂一也。別於陽者,知病忌時;別於陰者,知死生之期。謹熟陰陽,無與眾謀。
  所謂陰陽者,去者為陰,至者為陽;靜者為陰,動者為陽;遲者為陰,數者為陽。凡持真脈之藏脈者,肝至懸絕急,十八日死;心至懸絕,九日死;肺至懸絕,十二日死;腎至懸絕,七日死;脾至懸絕,四日死。
  曰:二陽之病發心脾,有不得隱曲,女子不月;其傳為風消,其傳為息賁者,死不治。
  曰:三陽為病,發寒熱,下為癰腫,及為痿厥腨(疒肙);其傳為索澤,其傳為頹疝。
  曰:一陽發病,少氣善咳善洩;其傳為心掣,其傳為隔。
  二陽一陰發病,主驚駭背痛,善噫善欠,名曰風厥。
  二陰一陽發病,善脹心滿善氣。
  三陽三陰發病,為偏枯痿易,四支不舉。
  一陽曰鉤,鼓一陰曰毛,鼓陽勝急曰弦,鼓陽至而絕曰石,陰陽相過曰溜。
  陰爭於內,陽擾於外,魄汗未藏,四逆而起,起則熏肺,使人喘鳴。陰之所生,和本曰和。是故剛與剛,陽氣破散,陰氣乃消亡。淖則剛柔不和,經氣乃絕。
  死陰之屬,不過三日而死;生陽之屬,不過四日而死。所謂生陽死陰者,肝之心,謂之生陽。心之肺,謂之死陰。肺之腎,謂之重陰。腎之脾,謂之辟陰,死不治。
  結陽者,腫四支。結陰者便血一升,再結二升,三結三升。陰陽結斜,多陰少陽曰石水,少腹腫。二陽結謂之消,三陽結謂之隔,三陰結謂之水,一陰一陽結謂之喉痹。陰搏陽別謂之有子。陰陽虛腸辟死。陽加於陰謂之汗。陰虛陽搏謂之崩。
  三陰俱搏,二十日夜半死。二陰俱搏,十三日夕時死。一陰俱搏,十日死。三陽俱搏且鼓,三日死。三陰三陽俱搏,心腹滿,發盡不得隱曲,五日死。二陽俱搏,其病溫,死不治,不過十日死。


\section{靈蘭秘典論篇第八}

黃帝問曰:願聞十二藏之相使,貴賤何如?岐伯對曰:悉乎哉問也,請遂言之。心者,君主之官也,神明出焉。肺者,相傅之官,治節出焉。肝者,將軍之官,謀慮出焉。膽者,中正之官,決斷出焉。羶中者,臣使之官,喜樂出焉。脾胃者,倉廩之官,五味出焉。大腸者,傳道之官,變化出焉。小腸者,受盛之官,化物出焉。腎者,作強之官,伎巧出焉。三焦者,決瀆之官,水道出焉。膀胱者,州都之官,津液藏焉,氣化則能出矣。
  凡此十二官者,不得相失也。故主明則下安,以此養生則壽,歿世不殆,以為天下則大昌。主不明則十二官危,使道閉塞而不通,形乃大傷,以此養生則殃,以為天下者,其宗大危,戒之戒之。
  至道在微,變化無窮,孰知其原;窘乎哉,消者瞿瞿,孰知其要;閔閔之當,孰者為良。恍惚之數,生於毫氂,毫氂之數,起於度量,千之萬之,可以益大,推之大之,其形乃制。
  黃帝曰:善哉!余聞精光之道,大聖之業,而宣明大道,非齋戒擇吉日,不敢受也。黃帝乃澤吉日良兆,而藏靈蘭之室,以傳保焉。

\section{六節藏象論篇第九}

  黃帝問曰:余聞天以六六之節,以成一歲,人以九九制會,計人亦有三百六十五節,以為天地久矣,不知其所謂也。岐伯對曰:昭乎哉問也,請遂言之。夫六六之節,九九制會者,所以正天之度、氣之數也。天度者,所以制日月之行也;氣數者,所以紀化生之用也。
  天為陽,地為陰;日為陽,月為陰。行有分紀,周有道理,日行一度,月行十三度而有奇焉,故大小月三百六十五日而成歲,積氣余而盈閏矣。立端於始,表正於中,推余於終,而天度畢矣。
  帝曰:余已聞天度矣,願聞氣數何以合之。岐伯曰:天以六六為節,地以九九制會,天有十日,日六竟而周甲,甲六復而終歲,三百六十日法也。夫自古通天者,生之本,本於陰陽。其氣九州九竅,皆通乎天氣。故其生五,其氣三,三而成天,三而成地,三而成人,三而三之,合則為九,九分為九野,九野為九藏,故形藏四,神藏五,合為九藏以應之也。
  帝曰:余已聞六六九九之會也,夫子言積氣盈閏,願聞何謂氣。請夫子發蒙解惑焉。岐伯曰:此上帝所秘,先師傳之也。帝曰:請遂聞之。岐伯曰:五日謂之候,三候謂之氣,六氣謂之時,四時謂之歲,而各從其主治焉。五運相襲,而皆治之,終期之日,週而復始,時立氣布,如環無端,候亦同法。故曰:不知年之所加,氣之盛衰,虛實之所起,不可以為工矣。
  帝曰:五運之始,如環無端,其太過不及何如?岐伯曰:五氣更立,各有所勝,盛虛之變,此其常也。帝曰:平氣何如?岐伯曰:無過者也。帝曰:太過不及奈何?岐伯曰:在經有也。帝曰:何謂所勝?岐伯曰:春勝長夏,長夏勝冬,冬勝夏,夏勝秋,秋勝春,所謂得五行時之勝,各以氣命其藏。帝曰:何以知其勝?岐伯曰:求其至也,皆歸始春,未至而至,此謂太過,則薄所不勝,而乘所勝也,命曰氣淫。不分邪僻內生,工不能禁。至而不至,此謂不及,則所勝妄行,而所生受病,所不勝薄之也,命曰氣迫。所謂求其至者,氣至之時也。謹候其時,氣可與期,失時反候,五治不分,邪僻內生,工不能禁也。
  帝曰:有不襲乎?岐伯曰:蒼天之氣,不得無常也。氣之不襲,是謂非常,非常則變矣。帝曰:非常而變奈何?岐伯曰:變至則病,所勝則微,所不勝則甚,因而重感於邪,則死矣。故非其時則微,當其時則甚也。
  帝曰:善。余聞氣合而有形,因變以正名。天地之運,陰陽之化,其於萬物,孰少孰多,可得聞乎?岐伯曰:悉哉問也,天至廣不可度,地至大不可量,大神靈問,請陳其方。草生五色,五色之變,不可勝視,草生五味,五味之美,不可勝極,嗜欲不同,各有所通。天食人以五氣,地食人以五味。五氣入鼻,藏於心肺,上使五色修明,音聲能彰。五味入口,藏於腸胃,味有所藏,以養五氣,氣和而生,津液相成,神乃自生。
  帝曰:藏象何如?岐伯曰:心者,生之本,神之變也,其華在面,其充在血脈,為陽中之太陽,通於夏氣。肺者,氣之本,魄之處也,其華在毛,其充在皮,為陽中之太陰,通於秋氣。腎者,主蟄,封藏之本,精之處也,其華在發,其充在骨,為陰中之少陰,通於冬氣。肝者,罷極之本,魂之居也,其華在爪,其充在筋,以生血氣,其味酸,其色蒼,此為陽中之少陽,通於春氣。脾胃大腸小腸三焦膀胱者,倉廩之本,營之居也,名曰器,能化糟粕,轉味而入出者也,其華在唇四白,其充在肌,其味甘,其色黃,此至陰之類,通於土氣。凡十一藏取決於膽也。
  故人迎一盛病在少陽,二盛病在太陽,三盛病在陽明,四盛已上為格陽。寸口一盛,病在厥陰,二盛病在少陰,三盛病在太陰,四盛已上為關陰。人迎與寸口俱盛四倍已上為關格,關格之脈羸,不能極於天地之精氣,則死矣。


\section{五藏生成篇第十}

  心之合脈也,其榮色也,其主腎也。肺之合皮也,其榮毛也,其主心也。肝之合筋也,其榮爪也,其主肺也。脾之合肉也,其榮唇也,其主肝也。腎之合骨也,其榮發也,其主脾也。
  是故多食咸則脈凝泣而變色;多食苦則皮槁而毛拔;多食辛則筋急而爪枯;多食酸,則肉胝而唇揭;多食甘則骨痛而發落,此五味之所傷也。故心欲苦,肺欲辛,肝欲酸,脾欲甘,腎欲咸,此五味之所合也。
  五藏之氣。故色見青如草茲者死,黃如枳實者死,黑如炲者死,赤如衃血者死,白如枯骨者死,此五色之見死也。青如翠羽者生,赤如雞冠者生,黃如蟹腹者生,白如豕膏者生,黑如烏羽者生,此五色之見生也。生於心,如以縞裹朱;生於肺,如以縞裹紅;生於肝,如以縞裹紺;生於脾,如以縞裹栝樓實,生於腎,如以縞裹紫,此五藏所生之外榮也。
  色味當五藏:白當肺,辛,赤當心,苦,青當肝,酸,黃當脾,甘,黑當腎,咸,故白當皮,赤當脈,青當筋,黃當肉,黑當骨。
  諸脈者皆屬於目,諸髓者皆屬於腦,諸筋者皆屬於節,諸血者皆屬於心,諸氣者皆屬於肺,此四支八谿之朝夕也。
  故人臥,血歸於肝,肝受血而能視,足受血而能步,掌受血而能握,指受血而能攝。臥出而風吹之,血凝於膚者為痹,凝於脈者為泣,凝於足者為厥。此三者,血行而不得反其空,故為痹厥也。人有大谷十二分,小谿三百五十四名,少十二俞,此皆衛氣之所留止,邪氣之所客也,針石緣而去之。
  診病之始五決為紀,欲知其始,先建其母,所謂五決者五脈也。
  是以頭痛巔疾,下虛上實過在足少陰,巨陽,甚則入腎。徇蒙招尤,目冥耳聾,下實上虛,過在足少陽、厥陰,甚則入肝。腹滿(月真)脹,支鬲脅,下厥上冒,過在足太陰,陽明。咳嗽上氣,厥在胸中,過在手陽明太陰。心煩頭痛病在鬲中,過在手巨陽,少陰。
  夫脈之小大滑澀浮沉,可以指別;五藏之象,可以類推;五藏相音,可以意識;五色微診,可以目察。能合脈色,可以萬全。赤,脈之至也喘而堅,診曰有積氣在中,時害於食,名曰心痹,得之外疾,思慮而心虛,故邪從之。白,脈之至也,喘而浮,上虛下實,驚,有積氣在胸中,喘而虛,名曰肺痹寒熱,得之醉而使內也。青,脈之至也長而左右彈,有積氣在心下支胠,名曰肝痹,得之寒濕,與疝同法,腰痛足清頭痛。黃,脈之至也大而虛,有積氣在腹中,有厥氣名曰厥疝,女子同法,得之疾使四支,汗出當風。黑,脈之至也,上堅而大,有積氣在小腹與陰,名曰腎痹,得之沐浴清水而臥。
  凡相五色之奇脈,面黃目青,面黃目赤,面黃目白,面黃目黑者,皆不死也。面青目赤,面赤目白,面青目黑,面黑目白,面赤目青,皆死也。


\section{五藏別論篇第十一}

  黃帝問曰:余聞方士,或以腦髓為藏,或以腸胃為藏,或以為府,敢問更相反,皆自謂是,不知其道,願聞其說。
  岐伯對曰:腦髓骨脈膽女子胞,此六者地氣之所生也,皆藏於陰而像於地,故藏而不寫,名曰奇恆之府。夫胃大腸小腸三焦膀胱,此五者,天氣之所生也,其氣象天,故寫而不藏,此受五藏濁氣,名曰傳化之府,此不能久留,輸瀉者也。魄門亦為五藏使,水谷不得久藏。所謂五藏者,藏精氣而不寫也,故滿而不能實。六府者,傳化物而不藏,故實而不能滿也。所以然者,水谷入口,則胃實而腸虛;食下,則腸實而胃虛。故曰:實而不滿,滿而不實也。
  帝曰:氣口何以獨為五藏主?岐伯曰:胃者,水穀之海,六府之大源也。五味入口,藏於胃,以養五藏氣,氣口亦太陰也。是以五藏六府之氣味,皆出於胃,變見於氣口。故五氣入鼻,藏於心肺,心肺有病,而鼻為之不利也。
  凡治病必察其下,適其脈,觀其志意與其病也。拘於鬼神者,不可與言至德。惡於針石者,不可與言至巧。病不許治者,病必不治,治之無功矣。


\section{異法方宜論篇第十二}

  黃帝問曰:醫之治病也,一病而治各不同,皆愈何也?岐伯對曰:地勢使然也。故東方之域,天地之所始生也,魚鹽之地,海濱傍水,其民食魚而嗜咸,皆安其處,美其食,魚者使人熱中,鹽者勝血,故其民皆黑色疏理,其病皆為癰瘍,其治宜砭石,故砭石者,亦從東方來。
  西方者,金玉之域,沙石之處,天地之所收引也,其民陵居而多風,水土剛強,其民不衣而褐薦,其民華食而脂肥,故邪不能傷其形體,其病生於內,其治宜毒藥,故毒藥者,亦從西方來。
  北方者,天地所閉藏之域也,其地高陵居,風寒冰冽,其民樂野處而乳食,藏寒生滿病,其治宜灸焫,故灸焫者,亦從北方來。
  南方者,天地所長養,陽之所盛處也,其地下,水土弱,霧露之所聚也,其民嗜酸而食胕,故其民皆致理而赤色,其病攣痹,其治宜微針,故九針者,亦從南方來。
  中央者,其地平以濕,天地所以生萬物也眾,其民食雜而不勞,故其病多痿厥寒熱,其治宜導引按蹻,故導引按蹻者,亦從中央出也。
  故聖人雜合以治,各得其所宜,故治所以異而病皆愈者,得病之情,知治之大體也。
\section{移精變氣論篇第十三}

  黃帝問曰:余聞古之治病,惟其移精變氣,可祝由而已。今世治病,毒藥治其內,針石治其外,或愈或不愈,何也?
  岐伯對曰:往古人居禽獸之間,動作以避寒,陰居以避暑,內無眷慕之累,外無伸宦之形,此恬憺之世,邪不能深入也。故毒藥不能治其內,針石不能治其外,故可移精祝由而已。當今之世不然,憂患緣其內,苦形傷其外,又失四時之從,逆寒暑之宜,賊風數至,虛邪朝夕,內至五藏骨髓,外傷空竅肌膚,所以小病必甚,大病必死,故祝由不能已也。
  帝曰:善。余欲臨病人,觀死生,決嫌疑,欲知其要,如日月光,可得聞乎?岐伯曰:色脈者,上帝之所貴也,先師之所傳也。上古使僦貸季,理色脈而通神明,合之金木水火土四時八風六合,不離其常,變化相移,以觀其妙,以知其要,欲知其要,則色脈是矣。色以應日,脈以應月,常求其要,則其要也。夫色之變化,以應四時之脈,此上帝之所貴,以合於神明也,所以遠死而近生。生道以長,命曰聖王。中古之治病,至而治之,湯液十日,以去八風五痹之病,十日不巳,治以草蘇草荄之枝,本末為助,標本已得,邪氣乃服。暮世之治病也則不然,治不本四時,不知日月,不審逆從,病形已成,乃欲微針治其外,湯液治其內,粗工凶凶,以為可攻,故病未已,新病復起。
  帝曰:願聞要道。岐伯曰:治之要極,無失色脈,用之不惑,治之大則。逆從到行,標本不得,亡神失國。去故就新,乃得真人。帝曰:余聞其要於夫子矣,夫子言不離色脈,此余之所知也。岐伯曰:治之極於一。帝曰:何謂一?岐伯曰:一者,因得之。帝曰:奈何?岐伯曰:閉戶塞牖,系之病者,數問其情,以從其意,得神者昌,失神者亡。帝曰:善。


\section{湯液醪醴論篇第十四}

  黃帝問曰:為五穀湯液及醪醴,奈何?岐伯對曰:必以稻米,炊之稻薪,稻米者完,稻薪者堅。帝曰:何以然?岐伯曰:此得天地之和,高下之宜,故能至完,伐取得時,故能至堅也。
  帝曰:上古聖人作湯液醪醴,為而不用,何也?岐伯曰:自古聖人之作湯液醪醴者,以為備耳,夫上古作湯液,故為而弗服也。中古之世,道德稍衰,邪氣時至,服之萬全。帝曰:今之世不必已何也。岐伯曰:當今之世,必齊毒藥攻其中,鑱石針艾治其外也。
  帝曰:形弊血盡而功不立者何?岐伯曰:神不使也。帝曰:何謂神不使?岐伯曰:針石道也。精神不進,志意不治,故病不可愈。今精壞神去,榮衛不可復收。何者,嗜欲無窮,而憂患不止,精氣弛壞,營泣衛除,故神去之而病不愈也。
  帝曰:夫病之始生也,極微極精,必先入結於皮膚。今良工皆稱曰:病成名曰逆,則針石不能治,良藥不能及也。今良工皆得其法,守其數,親戚兄弟遠近音聲日聞於耳,五色日見於目,而病不愈者,亦何暇不早乎。岐伯曰:病為本,工為標,標本不得,邪氣不服,此之謂也。
  帝曰:其有不從毫毛而生,五藏陽以竭也,津液充郭,其魄獨居,孤精於內,氣耗於外,形不可與衣相保,此四極急而動中,是氣拒於內,而形施於外,治之奈何?岐伯曰:平治於權衡,去宛陳莝,微動四極,溫衣,繆刺其處,以復其形。開鬼門,潔淨府,精以時服,五陽已布,疏滌五藏,故精自生,形自盛,骨肉相保,巨氣乃平。帝曰:善。


\section{玉版論要篇第十五}

  黃帝問曰:余聞揆度奇恆,所指不同,用之奈何?岐伯對曰:揆度者,度病之淺深也。奇恆者,言奇病也。請言道之至數,五色脈變,揆度奇恆,道在於一。神轉不回,回則不轉,乃失其機,至數之要,迫近於微,著之玉版,命曰合玉機。
  容色見上下左右,各在其要。其色見淺者,湯液主治,十日已。其見深者,必齊主治,二十一日已。其見大深者,醪酒主治,百日已。色夭面脫,不治,百日盡已。脈短氣絕死,病溫虛甚死。色見上下左右,各在其要。上為逆,下為從。女子右為逆,左為從;男子左為逆,右為從。易,重陽死,重陰死。陰陽反他,治在權衡相奪,奇恆事也,揆度事也。
  搏脈痹躄,寒熱之交。脈孤為消氣,虛洩為奪血。孤為逆,虛為從。行奇恆之法,以太陰始。行所不勝曰逆,逆則死;行所勝曰從,從則活。八風四時之勝,終而復始,逆行一過,不復可數,論要畢矣。


\section{診要經終論篇第十六}

  黃帝問曰:診要何如?岐伯對曰:正月二月,天氣始方,地氣始發,人氣在肝。三月四月,天氣正方,地氣定發,人氣在脾。五月六月,天氣盛,地氣高,人氣在頭。七月八月,陰氣始殺,人氣在肺。九月十月,陰氣始冰,地氣始閉,人氣在心。十一月十二月,冰復,地氣合,人氣在腎。
  故春刺散俞,及與分理,血出而止,甚者傳氣,間者環也。夏刺絡俞,見血而止,盡氣閉環,痛病必下。秋刺皮膚,循理,上下同法,神變而止。冬刺俞竅於分理,甚者直下,間者散下。春夏秋冬,各有所刺,法其所在。
  春刺夏分,脈亂氣微,入淫骨髓,病不能愈,令人不嗜食,又且少氣。春刺秋分,筋攣逆氣,環為嗽,病不愈,令人時驚,又且哭。春刺冬分,邪氣著藏,令人脹,病不愈,又且欲言語。
  夏刺春分,病不愈,令人解墮,夏刺秋分,病不愈,令人心中欲無言,惕惕如人將捕之。夏刺冬分,病不愈,令人少氣,時欲怒。
  秋刺春分,病不已,令人惕然,欲有所為,起而忘之。秋刺夏分,病不已,令人益嗜臥,又且善夢。秋刺冬分,病不已,令人灑灑時寒。
  冬刺春分,病不已,令人欲臥不能眠,眠而有見。冬刺夏分,病不愈,氣上,發為諸痹。冬刺秋分,病不已,令人善渴。
  凡刺胸腹者,必避五藏。中心者,環死;中脾者,五日死;中腎者,七日死;中肺者,五日死;中鬲者,皆為傷中,其病雖愈,不過一歲必死。刺避五藏者,知逆從也。所謂從者,鬲與脾腎之處,不知者反之。刺胸腹者,必以布憿著之,乃從單布上刺,刺之不愈,復刺。刺針必肅,刺腫搖針,經刺勿搖,此刺之道也。
  帝曰:願聞十二經脈之終,奈何?岐伯曰:太陽之脈,其終也,戴眼反折,瘛瘲,其色白,絕汗乃出,出則死矣。少陽終者,耳聾,百節皆縱,目寰絕系,絕系一日半死,其死也,色先青白,乃死矣。陽明終者,口目動作,善驚忘言,色黃,其上下經盛,不仁,則終矣。少陰終者,面黑齒長而垢,腹脹閉,上下不通而終矣。太陰終者,腹脹閉不得息,善噫善嘔,嘔則逆,逆則面赤,不逆則上下不通,不通則面黑皮,毛焦而終矣。厥陰終者,中熱嗌干,善溺心煩,甚則舌卷卵上縮而終矣。此十二經之所敗也。


\section{脈要精微論篇第十七}

  黃帝問曰:診法何如?岐伯對曰:診法常以平旦,陰氣未動,陽氣未散,飲食未進,經脈未盛,絡脈調勻,氣血未亂,故乃可診有過之脈。
  切脈動靜而視精明,察五色,觀五藏有餘不足,六府強弱,形之盛衰,以此參伍,決死生之分。
  夫脈者,血之府也,長則氣治,短則氣病,數則煩心,大則病進,上盛則氣高,下盛則氣脹,代則氣衰,細則氣少,澀則心痛,渾渾革至如湧泉,病進而色弊,綿綿其去如弦絕,死。
  夫精明五色者,氣之華也。赤欲如白裹朱,不欲如赭;白欲如鵝羽,不欲如鹽;青欲如蒼璧之澤,不欲如藍;黃欲如羅裹雄黃,不欲如黃土;黑欲如重漆色,不欲如地蒼。五色精微象見矣,其壽不久也。夫精明者,所以視萬物,別白黑,審短長。以長為短,以白為黑,如是則精衰矣。
  五藏者,中之守也,中盛藏滿,氣勝傷恐者,聲如從室中言,是中氣之濕也。言而微,終日乃復言者,此奪氣也。衣被不斂,言語善惡,不避親疏者,此神明之亂也。倉廩不藏者,是門戶不要也。水泉不止者,是膀胱不藏也。得守者生,失守者死。
  夫五藏者,身之強也,頭者精明之府,頭頃視深,精神將奪矣。背者胸中之府,背曲肩隨,府將壞矣。腰者腎之府,轉搖不能,腎將憊矣。膝者筋之府,屈伸不能,行則僂附,筋將憊矣。骨者髓之府,不能久立,行則振掉,骨將憊矣。得強則生,失強則死。
  岐伯曰:反四時者,有餘為精,不足為消。應太過,不足為精;應不足,有餘為消。陰陽不相應,病名曰關格。
  帝曰:脈其四時動奈何,知病之所在奈何,知病之所變奈何,知病乍在內奈何,知病乍在外奈何,請問此五者,可得聞乎。岐伯曰:請言其與天運轉大也。萬物之外,六合之內,天地之變,陰陽之應,彼春之暖,為夏之暑,彼秋之忿,為冬之怒,四變之動,脈與之上下,以春應中規,夏應中矩,秋應中衡,冬應中權。是故冬至四十五日,陽氣微上,陰氣微下;夏至四十五日,陰氣微上,陽氣微下。陰陽有時,與脈為期,期而相失,知脈所分,分之有期,故知死時。微妙在脈,不可不察,察之有紀,從陰陽始,始之有經,從五行生,生之有度,四時為宜,補寫勿失,與天地如一,得一之情,以知死生。是故聲合五音,色合五行,脈合陰陽。
  是知陰盛則夢涉大水恐懼,陽盛則夢大火燔灼,陰陽俱盛則夢相殺毀傷;上盛則夢飛,下盛則夢墮;甚飽則夢予,甚飢則夢取;肝氣盛則夢怒,肺氣盛則夢哭;短蟲多則夢聚眾,長蟲多則夢相擊毀傷。
  是故持脈有道,虛靜為保。春日浮,如魚之遊在波;夏日在膚,泛泛乎萬物有餘;秋日下膚,蟄蟲將去;冬日在骨,蟄蟲周密,君子居室。故曰:知內者按而紀之,知外者終而始之。此六者,持脈之大法。
  心脈搏堅而長,當病舌卷不能言;其耎而散者,當消環自已。肺脈搏堅而長,當病唾血;其耎而散者,當病灌汗,至今不復散發也。肝脈搏堅而長,色不青,當病墜若搏,因血在脅下,令人喘逆;其耎而散色澤者,當病溢飲,溢飲者喝暴多飲,而易入肌皮腸胃之外也。胃脈搏堅而長,其色赤,當病折髀;其耎而散者,當病食痹。脾脈搏堅而長,其色黃,當病少氣;其耎而散色不澤者,當病足胻腫,若水狀也。腎脈搏堅而長,其色黃而赤者,當病折腰;其而散者,當病少血,至今不復也。
  帝曰:診得心脈而急,此為何病,病形何如?岐伯曰:病名心疝,少腹當有形也。帝曰:何以言之。岐伯曰:心為牡藏,小腸為之使,故曰少腹當有形也。帝曰:診得胃脈,病形何如?岐伯曰:胃脈實則脹,虛則洩。
  帝:病成而變何謂?岐伯曰:風成為寒熱,癉成為消中,厥成為巔疾,久風為飧洩,脈風成為癘,病之變化,不可勝數。
  帝曰:諸癰腫筋攣骨痛,此皆安生。岐伯曰:此寒氣之腫,八風之變也。帝曰:治之奈何?岐伯曰:此四時之病,以其勝治之,愈也。
  帝曰:有故病五藏發動,因傷脈色,各何以知其久暴至之病乎。岐伯曰:悉乎哉問也。徵其脈小色不奪者,新病也;徵其脈不奪其色奪者,此久病也;徵其脈與五色俱奪者,此久病也;徵其脈與五色俱不奪者,新病也。肝與腎脈並至,其色蒼赤,當病毀傷,不見血,已見血,濕若中水也。
  尺內兩傍,則季脅也,尺外以候腎,尺裡以候腹。中附上,左外以候肝,內以候鬲;右,外以候胃,內以候脾。上附上,右外以候肺,內以候胸中;左,外以候心,內以候羶中。前以候前,後以候後。上竟上者,胸喉中事也;下竟下者,少腹腰股膝脛足中事也。
  粗大者,陰不足陽有餘,為熱中也。來疾去徐,上實下虛,為厥巔疾;來徐去疾,上虛下實,為惡風也。故中惡風者,陽氣受也。有脈俱沉細數者,少陰厥也;沉細數散者,寒熱也;浮而散者為眴僕。諸浮不躁者皆在陽,則為熱,其有躁者在手。諸細而沉者皆在陰,則為骨痛;其有靜者在足。數動一代者,病在陽之脈也,洩及便膿血。諸過者,切之,澀者陽氣有餘也,滑者陰氣有餘也。陽氣有餘,為身熱無汗,陰氣有餘,為多汗身寒,陰陽有餘,則無汗而寒。推而外之,內而不外,有心腹積也。推而內之,外而不內,身有熱也。推而上之,上而不下,腰足清也。推而下之,下而不上,頭項痛也。按之至骨,脈氣少者,腰脊痛而身有痹也。
\section{平人氣象論篇第十八}

  黃帝問曰:平人何如?岐伯對曰:人一呼脈再動,一吸脈亦再動,呼吸定息脈五動,閏以太息,命曰平人。平人者,不病也。常以不病調病人,醫不病,故為病人平息以調之為法。人一呼脈一動,一吸脈一動,曰少氣。人一呼脈三動,一吸脈三動而躁,尺熱曰病溫,尺不熱脈滑曰病風,脈澀曰痹。人一呼脈四動以上曰死,脈絕不至曰死,乍疏乍數曰死。
  平人之常氣稟於胃,胃者,平人之常氣也,人無胃氣曰逆,逆者死。
  春胃微弦曰平,弦多胃少曰肝病,但弦無胃曰死,胃而有毛曰秋病,毛甚曰今病。藏真散於肝,肝藏筋膜之氣也,夏胃微鉤曰平,鉤多胃少曰心病,但鉤無胃曰死,胃而有石曰冬病,石甚曰今病。藏真通於心,心藏血脈之氣也。長夏胃微軟弱曰平,弱多胃少曰脾病,但代無胃曰死,軟弱有石曰冬病,弱甚曰今病。藏真濡於脾,脾藏肌肉之氣也。秋胃微毛曰平,毛多胃少曰肺病,但毛無胃曰死,毛而有弦曰春病,弦甚曰今病。藏真高於肺,以行榮衛陰陽也。冬胃微石曰平,石多胃少曰腎病,但石無胃曰死,石而有鉤曰夏病,鉤甚曰今病。藏真下於腎,腎藏骨髓之氣也。
  胃之大絡,名曰虛裡,貫鬲絡肺,出於左乳下,其動應衣,脈宗氣也。盛喘數絕者,則病在中;結而橫,有積矣;絕不至曰死。乳之下其動應衣,宗氣洩也。
  欲知寸口太過與不及,寸口之脈中手短者,曰頭痛。寸口脈中手長者,曰足脛痛。寸口脈中手促上擊者,曰肩背痛。寸口脈沉而堅者,曰病在中。寸口脈浮而盛者,曰病在外。寸口脈沉而弱,曰寒熱及疝瘕少腹痛。寸口脈沉而橫,曰脅下有積,腹中有橫積痛。寸口脈沉而喘,曰寒熱。脈盛滑堅者,曰病在外。脈小實而堅者,病在內。脈小弱以澀,謂之久病。脈滑浮而疾者,謂之新病。脈急者,曰疝瘕少腹痛。脈滑曰風。脈澀曰痹。緩而滑曰熱中。盛而緊曰脹。
  脈從陰陽,病易已;脈逆陰陽,病難已。脈得四時之順,曰病無他;脈反四時及不間藏,曰難已。
  臂多青脈,曰脫血。尺脈緩澀,謂之解(亻亦)。安臥脈盛,謂之脫血。尺澀脈滑,謂之多汗。尺寒脈細,謂之後洩。脈尺常熱者,謂之熱中。
  肝見庚辛死,心見壬癸死,脾見甲乙死,肺見丙丁死,腎見戊己死,是謂真藏見,皆死。
  頸脈動喘疾欬,曰水。目裹微腫如臥蠶起之狀,曰水。溺黃赤安臥者,黃疸。已食如飢者,胃疸。面腫曰風。足脛腫曰水。目黃者曰黃疸。婦人手少陰脈動甚者,妊子也。
  脈有逆從,四時未有藏形,春夏而脈瘦,秋冬而脈浮大,命曰逆四時也。風熱而脈靜,洩而脫血脈實,病在中,脈虛,病在外,脈澀堅者,皆難治,命曰反四時也。
  人以水谷為本,故人絕水谷則死,脈無胃氣亦死,所謂無胃氣者,但得真藏脈不得胃氣也。所謂脈不得胃氣者,肝不弦腎不石也。
  太陽脈至,洪大以長;少陽脈至,乍數乍疏,乍短乍長;陽明脈至,浮大而短。
  夫平心脈來,纍纍如連珠,如循琅玕,曰心平,夏以胃氣為本,病心脈來,喘喘連屬,其中微曲,曰心病,死心脈來,前曲後居,如操帶鉤,曰心死。
  平肺脈來,厭厭聶聶,如落榆莢,曰肺平,秋以胃氣為本。病肺脈來,不上不下,如循雞羽,曰肺病。死肺脈來,如物之浮,如風吹毛,曰肺死。
  平肝脈來,軟弱招招,如揭長竿末梢,曰肝平,春以胃氣為本。病肝脈來,盈實而滑,如循長竿,曰肝病。死肝脈來,急益勁,如新張弓弦,曰肝死。
  平脾脈來,和柔相離,如雞踐地,曰脾平,長夏以胃氣為本。病脾脈來,實而盈數,如雞舉足,曰脾病。死脾脈來,銳堅如烏之喙,如鳥之距,如屋之漏,如水之流,曰脾死。
  平腎脈來,喘喘纍纍如鉤,按之而堅,曰腎平,冬以胃氣為本。病腎脈來,如引葛,按之益堅,曰腎病。死腎脈來,發如奪索,辟辟如彈石,曰腎死。

\section{玉機真藏論篇第十九}

  黃帝問曰:春脈如弦,何如而弦?岐伯對曰:春脈者肝也,東方木也,萬物之所以始生也,故其氣來,軟弱輕虛而滑,端直以長,故曰弦,反此者病。帝曰:何如而反。岐伯曰:其氣來實而強,此謂太過,病在外;其氣來不實而微,此謂不及,病在中。帝曰:春脈太過與不及,其病皆何如?岐伯曰:太過則令人善忘,忽忽眩冒而巔疾;其不及,則令人胸痛引背,下則兩胠脅滿。帝曰:善。
  夏脈如鉤,何如而鉤?岐伯曰:夏脈者心也,南方火也,萬物之所以盛長也,故其氣來盛去衰,故曰鉤,反此者病。帝曰:何如而反。岐伯曰:其氣來盛去亦盛,此謂太過,病在外;其氣來不盛去反盛,此謂不及,病在中。帝曰:夏脈太過與不及,其病皆何如?岐伯曰:太過則令人身熱而膚痛,為浸淫;其不及,則令人煩心,上見欬唾,下為氣洩。帝曰:善。
  秋脈如浮,何如而浮?岐伯曰:秋脈者肺也,西方金也,萬物之所以收成也,故其氣來,輕虛以浮,來急去散,故曰浮,反此者病。帝曰:何如而反。岐伯曰:其氣來,毛而中央堅,兩傍虛,此謂太過,病在外;其氣來,毛而微,此謂不及,病在中。帝曰:秋脈太過與不及,其病皆何如?岐伯曰:太過則令人逆氣而背痛,慍慍然;其不及,則令人喘,呼吸少氣而欬,上氣見血,下聞病音。帝曰:善。
  冬脈如營,何如而營?岐伯曰:冬脈者腎也,北方水也,萬物之所以合藏也,故其氣來,沉以搏,故曰營,反此者病。帝曰:何如而反。岐伯曰:其氣來如彈石者,此謂太過,病在外;其去如數者,此謂不及,病在中。帝曰:冬脈太過與不及,其病皆何如?岐伯曰:太過,則令人解(亻亦),脊脈痛而少氣不欲言;其不及,則令人心懸如病飢,眇中清,脊中痛,少腹滿,小便變。帝曰:善。
  帝曰:四時之序,逆從之變異也,然脾脈獨何主。岐伯曰:脾脈者土也,孤藏以灌四傍者也。帝曰:然則脾善惡,可得見之乎。岐伯曰:善者不可得見,惡者可見。帝曰:惡者何如可見。岐伯曰:其來如水之流者,此謂太過,病在外;如鳥之喙者,此謂不及,病在中。帝曰:夫子言脾為孤藏,中央土以灌四傍,其太過與不及,其病皆何如?岐伯曰:太過,則令人四支不舉;其不及,則令人九竅不通,名曰重強。
  帝瞿然而起,再拜而稽首曰:善。吾得脈之大要,天下至數,五色脈變,揆度奇恆,道在於一,神轉不回,回則不轉,乃失其機,至數之要,迫近以微,著之玉版,藏之藏府,每旦讀之,名曰玉機。
  五藏受氣於其所生,傳之於其所勝,氣舍於其所生,死於其所不勝。病之且死,必先傳行至其所不勝,病乃死。此言氣之逆行也,故死。肝受氣於心,傳之於脾,氣舍於腎,至肺而死。心受氣於脾,傳之於肺,氣舍於肝,至腎而死。脾受氣於肺,傳之於腎,氣舍於心,至肝而死。肺受氣於腎,傳之於肝,氣舍於脾,至心而死。腎受氣於肝,傳之於心,氣舍於肺,至脾而死。此皆逆死也。一日一夜五分之,此所以佔死生之早暮也。
  黃帝曰:五藏相通,移皆有次,五藏有病,則各傳其所勝。不治,法三月若六月,若三日若六日,傳五藏而當死,是順傳所勝之次。故曰:別於陽者,知病從來;別於陰者,知死生之期。言知至其所困而死。
  是故風者百病之長也,今風寒客於人,使人毫毛畢直,皮膚閉而為熱,當是之時,可汗而發也;或痹不仁腫痛,當是之時,可湯熨及火灸刺而去之。弗治,病入舍於肺,名曰肺痹,發欬上氣。弗治,肺即傳而行之肝,病名曰肝痹,一名曰厥,脅痛出食,當是之時,可按若刺耳。弗治,肝傳之脾,病名曰脾風,發癉,腹中熱,煩心出黃,當此之時,可按可藥可浴。弗治,脾傳之腎,病名曰疝瘕,少腹冤熱而痛,出白,一名曰蠱,當此之時,可按可藥。弗治,腎傳之心,病筋脈相引而急,病名曰瘛,當此之時,可灸可藥。弗治,滿十日,法當死。腎因傳之心,心即復反傳而行之肺,發寒熱,法當三歲死,此病之次也。
  然其捽髮者,不必治於傳,或其傳化有不以次,不以次入者,憂恐悲喜怒,令不得以其次,故令人有大病矣。因而喜大虛則腎氣乘矣,怒則肝氣乘矣,悲則肺氣乘矣,恐則脾氣乘矣,憂則心氣乘矣,此其道也。故病有五,五五二十五變,及其傳化。傳,乘之名也。
  大骨枯槁,大肉陷下,胸中氣滿,喘息不便,其氣動形,期六月死,真藏脈見,乃予之期日。大骨枯槁,大肉陷下,胸中氣滿,喘息不便,內痛引肩項,期一月死,真藏見,乃予之期日。大骨枯槁,大肉陷下,胸中氣滿,喘息不便,內痛引肩項,身熱脫肉破(月囷),真藏見,十月之內死。大骨枯槁,大肉陷下,肩髓內消,動作益衰,真藏來見,期一歲死,見其真藏,乃予之期日。大骨枯槁,大肉陷下,胸中氣滿,腹內痛,心中不便,肩項身熱,破(月囷)脫肉,目匡陷,真藏見,目不見人,立死,其見人者,至其所不勝之時則死。
  急虛身中卒至,五藏絕閉,脈道不通,氣不往來,譬如墮溺,不可為期。其脈絕不來,若人一息五六至,其形肉不脫,真藏雖不見,猶死也。
  真肝脈至,中外急,如循刀刃責責然,如按琴瑟弦,色青白不澤,毛折,乃死。真心脈至,堅而搏,如循薏苡子纍纍然,色赤黑不澤,毛折,乃死。真肺脈至,大而虛,如以毛羽中人膚,色白赤不澤,毛折,乃死。真腎脈至,搏而絕,如指彈石辟辟然,色黑黃不澤,毛折,乃死。真脾脈至,弱而乍數乍疏,色黃青不澤,毛折,乃死。諸真藏脈見者,皆死,不治也。
  黃帝曰:見真藏曰死,何也。岐伯曰:五藏者,皆稟氣於胃,胃者,五藏之本也,藏氣者,不能自致於手太陰,必因於胃氣,乃至於手太陰也,故五藏各以其時,自為而至於手太陰也。故邪氣勝者,精氣衰也,故病甚者,胃氣不能與之俱至於手太陰,故真藏之氣獨見,獨見者病勝藏也,故曰死。帝曰:善。
  黃帝曰:凡治病,察其形氣色澤,脈之盛衰,病之新故,乃治之無後其時。形氣相得,謂之可治;色澤以浮,謂之易己;脈從四時,謂之可治;脈弱以滑,是有胃氣,命曰易治,取之以時。形氣相失,謂之難治;色夭不澤,謂之難已;脈實以堅,謂之益甚;脈逆四時,為不可治。必察四難,而明告之。
  所謂逆四時者,春得肺脈,夏得腎脈,秋得心脈,冬得脾脈,其至皆懸絕沉澀者,命曰逆。四時未有藏形,於春夏而脈沉澀,秋冬而脈浮大,名曰逆四時也。
  病熱脈靜,洩而脈大,脫血而脈實,病在中脈實堅,病在外,脈不實堅者,皆難治。
  黃帝曰:余聞虛實以決死生,願聞其情。岐伯曰:五實死,五虛死。帝曰:願聞五實五虛。岐伯曰:脈盛,皮熱,腹脹,前後不通,悶瞀,此謂五實。脈細,皮寒,氣少,洩利前後,飲食不入,此謂五虛。帝曰:其時有生者,何也。岐伯曰:漿粥入胃,洩注止,則虛者活;身汗得後利,則實者活。此其候也。


\section{三部九候論篇第二十}

  黃帝問曰:余聞九針於夫子,眾多博大,不可勝數。余願聞要道,以屬子孫,傳之後世,著之骨髓,藏之肝肺,歃血而受,不敢妄洩,令合天道,必有終始,上應天光星辰歷紀,下副四時五行,貴賤更互,冬陰夏陽,以人應之奈何,願聞其方。
  岐伯對曰:妙乎哉問也!此天地之至數。帝曰:願聞天地之至數,合於人形,血氣通,決死生,為之奈何?岐伯曰:天地之至數,始於一,終於九焉。一者天,二者地,三者人,因而三之,三三者九,以應九野。故人有三部,部有三候,以決死生,以處百病,以調虛實,而除邪疾。
  帝曰:何謂三部。岐伯曰:有下部,有中部,有上部,部各有三候,三候者,有天有地有人也,必指而導之,乃以為真。上部天,兩額之動脈;上部地,兩頰之動脈;上部人,耳前之動脈。中部天,手太陰也;中部地,手陽明也;中部人,手少陰也。下部天,足厥陰也;下部地,足少陰也;下部人,足太陰也。故下部之天以侯肝,地以候腎,人以候脾胃之氣。
  帝曰:中部之候奈何?岐伯曰:亦有天,亦有地,亦有人。天以候肺,地以候胸中之氣,人以候心。帝曰:上部以何候之。岐伯曰:亦有天,亦有地,亦有人,天以候頭角之氣,地以候口齒之氣,人以候耳目之氣。三部者,各有天,各有地,各有人。三而成天,三而成地,三而成人,三而三之,合則為九,九分為九野,九野為九藏。故神藏五,形藏四,合為九藏。五藏已敗,其色必夭,夭必死矣。
  帝曰:以候奈何?岐伯曰:必先度其形之肥瘦,以調其氣之虛實,實則寫之,虛則補之。必先去其血脈而後調之,無問其病,以平為期。
  帝曰:決死生奈何?岐伯曰:形盛脈細,少氣不足以息者,危。形瘦脈大,胸中多氣者,死。形氣相得者,生。參伍不調者,病。三部九候皆相失者,死。上下左右之脈相應如參舂者,病甚。上下左右相失不可數者,死。中部之候雖獨調,與眾藏相失者,死。中部之候相減者,死。目內陷者死。
  帝曰:何以知病之所在。岐伯曰:察九候,獨小者病,獨大者病,獨疾者病,獨遲者病,獨熱者病,獨寒者病,獨陷下者病。以左手足上,去踝五寸按之,庶右手足當踝而彈之,其應過五寸以上,蠕蠕然者,不病;其應疾,中手渾渾然者,病;中手徐徐然者,病;其應上不能至五寸,彈之不應者,死。是以脫肉身不去者,死。中部乍疏乍數者,死。其脈代而鉤者,病在絡脈。九候之相應也,上下若一,不得相失。一候後則病,二候後則病甚,三候後則病危。所謂後者,應不俱也。察其府藏,以知死生之期。必先知經脈,然後知病脈,真藏脈見者勝死。足太陽氣絕者,其足不可屈伸,死必戴眼。
  帝曰:冬陰夏陽奈何?岐伯曰:九候之脈,皆沉細懸絕者為陰,主冬,故以夜半死。盛躁喘數者為陽,主夏,故以日中死。是故寒熱病者,以平旦死。熱中及熱病者,以日中死。病風者,以日夕死。病水者,以夜半死。其脈乍疏乍數乍遲乍疾者,日乘四季死。形肉已脫,九候雖調,猶死。七診雖見,九候皆從者不死。所言不死者,風氣之病,及經月之病,似七診之病而非也,故言不死。若有七診之病,其脈候亦敗者死矣,必發噦噫。必審問其所始病,與今之所方病,而後各切循其脈,視其經絡浮沉,以上下逆從循之,其脈疾者不病,其脈遲者病,脈不往來者死,皮膚著者死。
  帝曰:其可治者奈何?岐伯曰:經病者治其經,孫絡病者治其孫絡血,血病身有痛者,治其經絡。其病者在奇邪,奇邪之脈則繆刺之。留瘦不移,節而刺之。上實下虛,切而從之,索其結絡脈,刺出其血,以見通之。瞳子高者,太陽不足,戴眼者,太陽已絕,此決死生之要,不可不察也。手指及手外踝上五指,留針。


\section{經脈別論篇第二十一}

  黃帝問曰:人之居處動靜勇怯,脈亦為之變乎。岐伯對曰:凡人之驚恐恚勞動靜,皆為變也。是以夜行則喘出於腎,淫氣病肺。有所墮恐,喘出於肝,淫氣害脾。有所驚恐,喘出於肺,淫氣傷心。度水跌僕,喘出於腎與骨,當是之時,勇者氣行則已,怯者則著而為病也。故曰:診病之道,觀人勇怯,骨肉皮膚,能知其情,以為診法也。
  故飲食飽甚,汗出於胃。驚而奪精,汗出於心。持重遠行,汗出於腎。疾走恐懼,汗出於肝。搖體勞苦,汗出於脾。故春秋冬夏,四時陰陽,生病起於過用,此為常也。
  食氣入胃,散精於肝,淫氣於筋。食氣入胃,濁氣歸心,淫精於脈。脈氣流經,經氣歸於肺,肺朝百脈,輸精於皮毛。毛脈合精,行氣於府。府精神明,留於四藏,氣歸於權衡。權衡以平,氣口成寸,以決死生。
  飲入於胃,游溢精氣,上輸於脾。脾氣散精,上歸於肺,通調水道,下輸膀胱。水精四布,五經並行,合於四時五藏陰陽,揆度以為常也。
  太陽藏獨至,厥喘虛氣逆,是陰不足陽有餘也,表裡當俱寫,取之下俞,陽明藏獨至,是陽氣重並也,當寫陽補陰,取之下俞。少陽藏獨至,是厥氣也,蹻前卒大,取之下俞,少陽獨至者,一陽之過也。太陰藏搏者,用心省真,五脈氣少,胃氣不平,三陰也,宜治其下俞,補陽寫陰。一陽獨嘯,少陽厥也,陽並於上,四脈爭張,氣歸於腎,宜治其經絡,寫陽補陰。一陰至,厥陰之治也,真陰(疒肙)心,厥氣留薄,發為白汗,調食和藥,治在下俞。
  帝曰:太陽藏何象。岐伯曰:像三陽而浮也。帝曰:少陽藏何象。岐伯曰:像一陽也,一陽藏者,滑而不實也。帝曰:陽明藏何象。岐伯曰:象大浮也,太陰藏搏,言伏鼓也。二陰搏至,腎沉不浮也。


\section{藏氣法時論篇第二十二}

  黃帝問曰:合人形以法四時五行而治,何如而從,何如而逆,得失之意,願聞其事。岐伯對曰:五行者,金木水火土也,更貴更賤,以知死生,以決成敗,而定五藏之氣,間甚之時,死生之期也。
  帝曰:願卒聞之。岐伯曰:肝主春,足厥陰少陽主治,其日甲乙,肝苦急,急食甘以緩之。心主夏,手少陰太陽主治,其日丙丁,心苦緩,急食酸以收之。脾主長夏,足太陰陽明主治,其日戊己,脾苦濕,急食苦以燥之。肺主秋,手太陰陽明主治,其日庚辛,肺苦氣上逆,急食苦以洩之。腎主冬,足少陰太陽主治,其日壬癸,腎苦燥,急食辛以潤之,開腠理,致津液,通氣也。
  病在肝,愈於夏,夏不愈,甚於秋,秋不死,持於冬,起於春,禁當風。肝病者,愈在丙丁,丙丁不愈,加於庚辛,庚辛不死,持於壬癸,起於甲乙。肝病者,平旦慧,下晡甚,夜半靜。肝欲散,急食辛以散之,用辛補之,酸寫之。
  病在心,愈在長夏,長夏不愈,甚於冬,冬不死,持於春,起於夏,禁溫食熱衣。心病者,愈在戊己,戊己不愈,加於壬癸,壬癸不死,持於甲乙,起於丙丁。心病者,日中慧,夜半甚,平旦靜。心欲軟,急食咸以軟之,用咸補之,甘寫之。
  病在脾,愈在秋,秋不愈,甚於春,春不死,持於夏,起於長夏,禁溫食飽食濕地濡衣。脾病者,愈在庚辛,庚辛不愈,加於甲乙,甲乙不死,持於丙丁,起於戊己。脾病者,日昳慧,日出甚,下晡靜。脾欲緩,急食甘以緩之,用苦寫之,甘補之。
  病在肺,愈在冬,冬不愈,甚於夏,夏不死,持於長夏,起於秋,禁寒飲食寒衣。肺病者,愈在壬癸,壬癸不愈,加於丙丁,丙丁不死,持於戊己,起於庚辛。肺病者,下晡慧,日中甚,夜半靜。肺欲收,急食酸以收之,用酸補之,辛寫之。
  病在腎,愈在春,春不愈,甚於長夏,長夏不死,持於秋,起於冬,禁犯焠(火矣) 熱食溫灸衣。腎病者,愈在甲乙,甲乙不愈,甚於戊己,戊己不死,持於庚辛,起於壬癸。腎病者,夜半慧,四季甚,下晡靜。腎欲堅,急食苦以堅之,用苦補之,咸寫之。
  夫邪氣之客於身也,以勝相加,至其所生而愈,至其所不勝而甚,至於所生而持,自得其位而起。必先定五藏之脈,乃可言間甚之時,死生之期也。
  肝病者,兩脅下痛引少腹,令人善怒,虛則目(目巟)(目巟)無所見,耳無所聞,善恐,如人將捕之,取其經,厥陰與少陽,氣逆,則頭痛耳聾不聰頰腫。取血者。
  心病者,胸中痛,脅支滿,脅下痛,膺背肩甲間痛,兩臂內痛;虛則胸腹大,脅下與腰相引而痛,取其經,少陰太陽,舌下血者。其變病,刺隙中血者。
  脾病者,身重善肌肉痿,足不收行,善瘈,腳下痛;虛則腹滿腸鳴,飧洩食不化,取其經,太陰陽明少陰血者。
  肺病者,喘咳逆氣,肩背痛,汗出,尻陰股膝髀腨(骨行)足皆痛;虛則少氣不能報息,耳聾嗌干,取其經,太陰足太陽之外厥陰內血者。
  腎病者,腹大脛腫,喘咳身重,寢汗出,憎風;虛則胸中痛,大腹小腹痛,清厥意不樂,取其經,少陰太陽血者。
  肝色青,宜食甘,粳米牛肉棗葵皆甘。心色赤,宜食酸,小豆犬肉李韭皆酸。肺色白,宜食苦,麥羊肉杏薤皆苦。脾色黃,宜食咸,大豆豕肉栗藿皆咸。腎色黑,宜食辛,黃黍雞肉桃蔥皆辛。辛散,酸收,甘緩,苦堅,咸軟。
  毒藥攻邪,五穀為養,五果為助,五畜為益,五菜為充,氣味合而服之,以補精益氣。此五者,有辛酸甘苦咸,各有所利,或散,或收,或緩,或急,或堅,或軟,四時五藏,病隨五味所宜也。


\section{宣明五氣篇第二十三}

  五味所入:酸入肝,辛入肺,苦入心,咸入腎,甘入脾,是謂五入。
  五氣所病:心為噫,肺為咳,肝為語,脾為吞,腎為欠為嚏,胃為氣逆,為噦為恐,大腸小腸為洩,下焦溢為水,膀胱不利為癃,不約為遺溺,膽為怒,是謂五病。
  五精所並:精氣並於心則喜,並於肺則悲,並於肝則憂,並於脾則畏,並於腎則恐,是謂五並,虛而相併者也。
  五藏所惡:心惡熱,肺惡寒,肝惡風,脾惡濕,腎惡燥,是謂五惡。
  五藏化液:心為汗,肺為涕,肝為淚,脾為涎,腎為唾,是謂五液。
  五味所禁:辛走氣,氣病無多食辛;咸走血,血病無多食咸;苦走骨,骨病無多食苦;甘走肉,肉病無多食甘;酸走筋,筋病無多食酸;是謂五禁,無令多食。
  五病所發:陰病發於骨,陽病發於血,陰病發於肉,陽病發於冬,陰病發於夏,是謂五發。
  五邪所亂:邪入於陽則狂,邪入於陰則痹,搏陽則為巔疾,搏陰則為瘖,陽入之陰則靜,陰出之陽則怒,是謂五亂。
  五邪所見:春得秋脈,夏得冬脈,長夏得春脈,秋得夏脈,冬得長夏脈,名曰陰出之陽,病善怒不治,是謂五邪。皆同命,死不治。
  五藏所藏:心藏神,肺藏魄,肝藏魂,脾藏意,腎藏志,是謂五藏所藏。
  五藏所主:心主脈,肺主皮,肝主筋,脾主肉,腎主骨,是謂五主。
  五勞所傷:久視傷血,久臥傷氣,久坐傷肉,久立傷骨,久行傷筋,是謂五勞所傷。
  五脈應像:肝脈弦,心脈鉤,脾脈代,肺脈毛,腎脈石,是謂五藏之脈。


\section{血氣形志篇第二十四}

  夫人之常數,太陽常多血少氣,少陽常少血多氣,陽明常多氣多血,少陰常少血多氣,厥陰常多血少氣,太陰常多氣少血,此天之常數。足太陽與少陰為表裡,少陽與厥陰為表裡,陽明與太陰為表裡,是為足陰陽也。手太陽與少陰為表裡,少陽與心主為表裡,陽明與太陰為表裡,是為手之陰陽也。今知手足陰陽所苦,凡治病必先去其血,乃去其所苦,伺之所欲,然後寫有餘,補不足。
  欲知背俞,先度其兩乳間,中折之,更以他草度去半已,即以兩隅相拄也,乃舉以度其背,令其一隅居上,齊脊大柱,兩隅在下,當其下隅者,肺之俞也。復下一度,心之俞也。復下一度,左角肝之俞也,右角脾之俞也。復下一度,腎之俞也。是謂五藏之俞,灸刺之度也。
  形樂志苦,病生於脈,治之以灸刺。形樂志樂,病生於肉,治之以針石。形苦志樂,病生於筋,治之以熨引。形苦志苦,病生於咽嗌,治之以百藥。形數驚恐,經絡不通,病生於不仁,治之以按摩醪藥。是謂五形志也。
  刺陽明出血氣,刺太陽,出血惡氣,刺少陽,出氣惡血,刺太陰,出氣惡血,刺少陰,出氣惡血,刺厥陰,出血惡氣也。

\section{寶命全形論篇第二十五}

  黃帝問曰:天覆地載,萬物悉備,莫貴於人,人以天地之氣生,四時之法成,君王眾庶,盡欲全形,形之疾病,莫知其情,留淫日深,著於骨髓,心私慮之,余欲針除其疾病,為之奈何?
  岐伯對曰:夫鹽之味咸者,其氣令器津洩;弦絕者,其音嘶敗;木敷者,其葉發;病深者,其聲噦。人有此三者,是謂壞府,毒藥無治,短針無取,此皆絕皮傷肉,血氣爭黑。
  帝曰:余念其痛,心為之亂惑,反甚其病,不可更代,百姓聞之,以為殘賊,為之奈何?岐伯曰:夫人生於地,懸命於天,天地合氣,命之曰人。人能應四時者,天地為之父母;知萬物者,謂之天子。天有陰陽,人有十二節;天有寒暑,人有虛實。能經天地陰陽之化者,不失四時;知十二節之理者,聖智不能欺也;能存八動之變,五勝更立;能達虛實之數者,獨出獨入,呿吟至微,秋毫在目。
  帝曰:人生有形,不離陰陽,天地合氣,別為九野,分為四時,月有小大,日有短長,萬物並至,不可勝量,虛實呿吟,敢問其方。
  岐伯曰:木得金而伐,火得水而滅,土得木而達,金得火而缺,水得土而絕,萬物盡然,不可勝竭。故針有懸布天下者五,黔首共余食,莫知之也。一曰治神,二曰知養身,三曰知毒藥為真,四曰制砭石小大,五曰知府藏血氣之診。五法俱立,各有所先。今末世之刺也,虛者實之,滿者洩之,此皆眾工所共知也。若夫法天則地,隨應而動,和之者若響,隨之者若影,道無鬼神,獨來獨往。
  帝曰:願聞其道。
  岐伯曰:凡刺之真,必先治神,五藏已定,九候已備,後乃存針,眾脈不見,眾凶弗聞,外內相得,無以形先,可玩往來,乃施於人。人有虛實,五虛勿近,五實勿遠,至其當發,間不容瞚。手動若務,針耀而勻,靜意視義,觀適之變,是謂冥冥,莫知其形,見其烏烏,見其稷稷,從見其飛,不知其誰,伏如橫弩,起如發機。
  帝曰:何如而虛?何如而實?岐伯曰:刺虛者須其實,刺實者須其虛,經氣已至,慎守勿失,深淺在志,遠近若一,如臨深淵,手如握虎,神無營於眾物。


\section{八正神明論篇第二十六}

  黃帝問曰:用針之服,必有法則焉,今何法何則?岐伯對曰:法天則地,合以天光。
  帝曰:願卒聞之。岐伯曰:凡刺之法,必候日月星辰四時八正之氣,氣定乃刺之。是故天溫日明,則人血淖液而衛氣浮,故血易寫,氣易行;天寒日陰,則人血凝泣,而衛氣沉。月始生,則血氣始精,衛氣始行;月郭滿,則血氣實,肌肉堅;月郭空,則肌肉減,經絡虛,衛氣去,形獨居。是以因天時而調血氣也。是以天寒無刺,天溫無疑。月生無寫,月滿無補,月郭空無治,是謂得時而調之。因天之序,盛虛之時,移光定位,正立而待之。故日月生而寫,是謂藏虛;月滿而補,血氣揚溢,絡有留血,命曰重實;月郭空而治,是謂亂經。陰陽相錯,真邪不別,沉以留止,外虛內亂,淫邪乃起。
  帝曰:星辰八正何候?
  岐伯曰:星辰者,所以制日月之行也。八正者,所以候八風之虛邪以時至者也。四時者,所以分春秋冬夏之氣所在,以時調之也,八正之虛邪,而避之勿犯也。以身之虛,而逢天之虛,兩虛相感,其氣至骨,入則傷五藏,工候救之,弗能傷也,故曰天忌不可不知也。
  帝曰:善。其法星辰者,余聞之矣,願聞法往古者。
  岐伯曰:法往古者,先知針經也。驗於來今者,先知日之寒溫、月之虛盛,以候氣之浮沉,而調之於身,觀其立有驗也。觀其冥冥者,言形氣榮衛之不形於外,而工獨知之,以日之寒溫,月之虛盛,四時氣之浮沉,參伍相合而調之,工常先見之,然而不形於外,故曰觀於冥冥焉。通於無窮者,可以傳於後世也,是故工之所以異也,然而不形見於外,故俱不能見也。視之無形,嘗之無味,故謂冥冥,若神彷彿。虛邪者,八正之虛邪氣也。正邪者,身形若用力,汗出,腠理開,逢虛風,其中人也微,故莫知其情,莫見其形。上工救其萌牙,必先見三部九候之氣,盡調不敗而救之,故曰上工。下工救其已成,救其已敗。救其已成者,言不知三部九候之相失,因病而敗之也,知其所在者,知診三部九候之病脈處而治之,故曰守其門戶焉,莫知其情而見邪形也。
  帝曰:余聞補寫,未得其意。
  岐伯曰:寫必用方,方者,以氣方盛也,以月方滿也,以日方溫也,以身方定也,以息方吸而內針,乃復候其方吸而轉針,乃復候其方呼而徐引針,故曰寫必用方,其氣而行焉。補必用員,員者行也,行者移也,刺必中其,復以吸排針也。故員與方,非針也。故養神者,必知形之肥瘦,榮衛血氣之盛衰。血氣者,人之神,不可不謹養。
  帝曰:妙乎哉論也。合人形於陰陽四時,虛實之應,冥冥之期,其非夫子孰能通之。然夫子數言形與神,何謂形,何謂神,願卒聞之。
  岐伯曰:請言形、形乎形、目冥冥,問其所病,索之於經,慧然在前,按之不得,不知其情,故曰形。
  帝曰:何謂神?
  岐伯曰:,神乎神,耳不聞,目明,心開而志先,慧然獨悟,口弗能言,俱視獨見,適若昏,昭然獨明*請言神*,若風吹雲,故曰神。三部九候為之原,九針之論,不必存也。


\section{離合真邪論篇第二十七}

  黃帝問曰:余聞九針九篇,夫子乃因而九之,九九八十一篇,余盡通其意矣。經言氣之盛衰,左右頃移,以上調下,以左調右,有餘不足,補瀉於滎輸,余知之矣。此皆榮衛之頃移,虛實之所生,非邪氣從外入於經也。余願聞邪氣之在經也,其病人何如?取之奈何?
  岐伯對曰:夫聖人之起度數,必應於天地,故天有宿度,地有經水,人有經脈。天地溫和,則經水安靜;天寒地凍,則經水凝泣;天暑地熱,則經水沸溢;卒風暴起,則經水波湧而隴起。夫邪之入於脈也,寒則血凝泣,暑則氣淖澤,虛邪因而入客,亦如經水之得風也,經之動脈,其至也亦時隴起,其行於脈中循循然,其至寸口中手也,時大時小,大則邪至,小則平,其行無常處,在陰與陽,不可為度,從而察之,三部九候,卒然逢之,早遏其路,吸則內針,無令氣忤;靜以久留,無令邪布;吸則轉針,以得氣為故;候呼引針,呼盡乃去;大氣皆出,故命曰寫。
  帝曰:不足者補之,奈何?
  岐伯曰:必先捫而循之,切而散之,推而按之,彈而怒之,抓而下之,通而取之,外引其門,以閉其神。呼盡內針,靜以久留,以氣至為故,如待所貴,不知日暮,其氣以至,適而自護,候吸引針,氣不得出,各在其處,推闔其門,令神氣存,大氣留止,故命曰補。
  帝曰:候氣奈何?
  岐伯曰:夫邪去絡入於經也,舍於血脈之中,其寒溫未相得,如湧波之起也,時來時去,故不常在。故曰方其來也,必按而止之,止而取之,無逢其沖而寫之。真氣者,經氣也,經氣太虛,故曰其來不可逢,此之謂也。故曰候邪不審,大氣已過,寫之則真氣脫,脫則不復,邪氣復至,而病益蓄,故曰其往不可追,此之謂也。不可掛以發者,待邪之至時而髮針寫矣,若先若後者,血氣已盡,其病不可下,故曰知其可取如發機,不知其取如扣椎,故曰知機道者不可掛以發,不知機者扣之不發,此之謂也。
  帝曰:補寫奈何?
  岐伯曰:此攻邪也,疾出以去盛血,而復其真氣,此邪新客,溶溶未有定處也,推之則前,引之則止,逆而刺之,溫血也。刺出其血,其病立已。
  帝曰:善。然真邪以合,波隴不起,候之奈何?
  岐伯曰:審捫循三部九候之盛虛而調之,察其左右上下相失及相減者,審其病藏以期之。不知三部者,陰陽不別,天地不分,地以候地,天以候天,人以候人,調之中府,以定三部,故曰刺不知三部九候病脈之處,雖有大過且至,工不能禁也。誅罰無過,命曰大惑,反亂大經,真不可復,用實為虛,以邪為真,用針無義,反為氣賊,奪人正氣,以從為逆,榮衛散亂,真氣已失,邪獨內著,絕人長命,予人夭殃,不知三部九候,故不能久長。因不知合之四時五行,因加相勝,釋邪攻正,絕人長命。邪之新客來也,未有定處,推之則前,引之則止,逢而寫之,其病立已。


\section{通評虛實論篇第二十八}

  黃帝問曰:何謂虛實?岐伯對曰:邪氣盛則實,精氣奪則虛。
  帝曰:虛實何如?岐伯曰:氣虛者肺虛也,氣逆者足寒也,非其時則生,當其時則死。余藏皆如此。
  帝曰:何謂重實?岐伯曰:所謂重實者,言大熱病,氣熱脈滿,是謂重實。
  帝曰:經絡俱實何如?何以治之?岐伯曰:經絡皆實,是寸脈急而尺緩也,皆當治之,故曰滑則從,澀則逆也。夫虛實者,皆從其物類始,故五藏骨肉滑利,可以長久也。
  帝曰:絡氣不足,經氣有餘,何如?岐伯曰:絡氣不足,經氣有餘者,脈口熱而尺寒也,秋冬為逆,春夏為從,治主病者。
  帝曰:經虛絡滿,何如?岐伯曰:經虛絡滿者,尺熱滿,脈口寒澀也,此春夏死秋冬生也。
  帝曰:治此者奈何?岐伯曰:絡滿經虛,灸陰刺陽;經滿絡虛,刺陰灸陽。 帝曰:何謂重虛?岐伯曰:脈氣上虛尺虛,是謂重虛。帝曰:何以治之?岐伯曰:所謂氣虛者,言無常也。尺虛者,行步恇然。脈虛者,不像陰也。如此者,滑則生,澀則死也。
  帝曰:寒氣暴上,脈滿而實何如?岐伯曰:實而滑則生,實而逆則死。
  帝曰:脈實滿,手足寒,頭熱,何如?岐伯曰:春秋則生,冬夏則死。脈浮而澀,澀而身有熱者死。
  帝曰:其形盡滿何如?岐伯曰:其形盡滿者,脈急大堅,尺澀而不應也,如是者,故從則生,逆則死。帝曰:何謂從則生,逆則死?岐伯曰:所謂從者,手足溫也;所謂逆者,手足寒也。
  帝曰:乳子而病熱,脈懸小者何如?岐伯曰:手足溫則生,寒則死。
  帝曰:乳子中風熱,喘鳴肩息者,脈何如?岐伯曰:喘鳴肩息者,脈實大也,緩則生,急則死。
  帝曰:腸澼便血何如?岐伯曰:身熱則死,寒則生。帝曰:腸澼下白沫何如?岐伯曰:脈沉則生,脈浮則死。帝曰:腸下膿血何如?岐伯曰:脈懸絕則死,滑大則生。帝曰:腸澼之屬,身不熱,脈不懸絕何如?岐伯曰:滑大者曰生,懸澀者曰死,以藏期之。
  帝曰:癲疾何如?岐伯曰:脈搏大滑,久自已;脈小堅急,死不治。帝曰:癲疾之脈,虛實何如?岐伯曰:虛則可治,實則死。
  帝曰:消癉虛實何如?岐伯曰:脈實大,病久可治;脈懸小堅,病久不可治。
  帝曰:形度骨度脈度筋度,何以知其度也?
  帝曰:春亟治經絡;夏亟治經輸;秋亟治六府;冬則閉塞,閉塞者,用藥而少針石也。所謂少針石者,非癰疽之謂也,癰疽不得頃時回。癰不知所,按之不應手,乍來乍已,刺手太陰傍三痏與纓脈各二,掖癰大熱,刺足少陽五;刺而熱不止,刺手心主三,刺手太陰經絡者大骨之會各三。暴癰筋軟,隨分而痛,魄汗不盡,胞氣不足,治在經俞。
  腹暴滿,按之不下,取手太陽經絡者,胃之募也,少陰俞去脊椎三寸傍五,用員利針。霍亂,刺俞傍五,足陽明及上傍三。刺癇驚脈五,針手太陰各五,刺經太陽五,刺手少陰經絡傍者一,足陽明一,上踝五寸刺三針。
  凡治消癉、僕擊、偏枯、痿厥、氣滿發逆,肥貴人,則高梁之疾也。隔塞閉絕,上下不通,則暴憂之疾也。暴厥而聾,偏塞閉不通,內氣暴薄也。不從內,外中風之病,故瘦留著也。蹠跛,寒風濕之病也。
  黃帝曰:黃疸暴痛,癲疾厥狂,久逆之所生也。五藏不平,六府閉塞之所生也。頭痛耳鳴,九竅不利,腸胃之所生也。


\section{太陰陽明論篇第二十九}

  黃帝問曰:太陰陽明為表裡,脾胃脈也,生病而異者何也?岐伯對曰:陰陽異位,更虛更實,更逆更從,或從內,或從外,所從不同,故病異名也。
  帝曰:願聞其異狀也。岐伯曰:陽者,天氣也,主外;陰者,地氣也,主內。故陽道實,陰道虛。故犯賊風虛邪者,陽受之;食飲不節,起居不時者,陰受之。陽受之,則入六府,陰受之,則入五藏。入六府,則身熱不時臥,上為喘呼;入五藏,則(月真)滿閉塞,下為飧洩,久為腸澼。故喉主天氣,咽主地氣。故陽受風氣,陰受濕氣。故陰氣從足上行至頭,而下行循臂至指端;陽氣從手上行至頭,而下行至足。故曰陽病者上行極而下,陰病者下行極而上。故傷於風者,上先受之;傷於濕者,下先受之。
  帝曰:脾病而四支不用何也?岐伯曰:四支皆稟氣於胃,而不得至經,必因於脾,乃得稟也。今脾病不能為胃行其津液,四支不得稟水谷氣,氣日以衰,脈道不利,筋骨肌肉,皆無氣以生,故不用焉。
  帝曰:脾不主時何也?岐伯曰:脾者土也,治中央,常以四時長四藏,各十八日寄治,不得獨主於時也。脾藏者常著胃土之精也,土者生萬物而法天地,故上下至頭足,不得主時也。
  帝曰:脾與胃以膜相連耳,而能為之行其津液何也?岐伯曰:足太陰者三陰也,其脈貫胃屬脾絡嗌,故太陰為之行氣於三陰。陽明者表也,五藏六府之海也,亦為之行氣於三陽。藏府各因其經而受氣於陽明,故為胃行其津液,四支不得稟水谷氣,日以益衰,陰道不利,筋骨肌肉無氣以生,故不用焉。


\section{陽明脈解篇第三十}

  黃帝問曰:足陽明之脈病,惡人與火,聞木音則惕然而驚,鐘鼓不為動,聞木音而驚,何也?願聞其故。岐伯對曰:陽明者胃脈也,胃者,土也,故聞木音而驚者,土惡木也。帝曰:善。其惡火何也?岐伯曰:陽明主肉,其脈血氣盛,邪客之則熱,熱甚則惡火。
  帝曰:其惡人何也?岐伯曰:陽明厥則喘而惋,惋則惡人。帝曰:或喘而死者,或喘而生者,何也?岐伯曰:厥逆連藏則死,連經則生。
  帝曰:善。病甚則棄衣而走,登高而歌,或至不食數日,逾垣上屋,所上之處,皆非其素所能也,病反能者何也?岐伯曰:四支者,諸陽之本也,陽盛則四支實,實則能登高也。
  帝曰:其棄衣而走者,何也?岐伯曰:熱盛於身,故棄衣欲走也。帝曰:其妄言罵詈,不避親疏而歌者,何也?岐伯曰:陽盛則使人妄言罵詈不避親疏,而不欲食,不欲食,故妄走也。


\section{熱論篇第三十一}

  黃帝問曰:今夫熱病者,皆傷寒之類也,或愈或死,其死皆以六七日之間,其愈皆以十日以上者,何也?不知其解,願聞其故。
  岐伯對曰:巨陽者,諸陽之屬也,其脈連於風府,故為諸陽主氣也。人之傷於寒也,則為病熱,熱雖甚不死;其兩感於寒而病者,必不免於死。
  帝曰:願聞其狀。岐伯曰:傷寒一日,巨陽受之,故頭項痛腰脊強。二日陽明受之,陽明主肉,其脈俠鼻絡於目,故身熱目疼而鼻干,不得臥也。三日少陽受之,少陽主膽,其脈循脅絡於耳,故胸脅痛而耳聾。三陽經絡皆受其病,而未入於藏者,故可汗而已。四日太陰受之,太陰脈布胃中絡於嗌,故腹滿而嗌干。五日少陰受之,少陰脈貫腎絡於肺,系舌本,故口燥舌干而喝。六日厥陰受之,厥陰脈循陰器而絡於肝,故煩滿而囊縮。三陰三陽,五藏六府皆受病,榮衛不行,五藏不通則死矣。
  其不兩感於寒者,七日巨陽病衰,頭痛少愈;八日陽明病衰,身熱少愈;九日少陽病衰,耳聾微聞;十日太陰病衰,腹減如故,則思飲食;十一日少陰病衰,渴止不滿,舌干已而嚏;十二日厥陰病衰,囊縱少腹微下,大氣皆去,病日已矣。帝曰:治之奈何?岐伯曰:治之各通其藏脈,病日衰已矣。其未滿三日者,可汗而已;其滿三日者,可洩而已。
  帝曰:熱病已癒,時有所遺者,何也?岐伯曰:諸遺者,熱甚而強食之,故有所遺也。若此者,皆病已衰,而熱有所藏,因其谷氣相薄,兩熱相合,故有所遺也。帝曰:善。治遺奈何?岐伯曰:視其虛實,調其逆從,可使必已矣。帝曰:病熱當何禁之?岐伯曰:病熱少愈,食肉則復,多食則遺,此其禁也。
  帝曰:其病兩感於寒者,其脈應與其病形何如?岐伯曰:兩感於寒者,病一日則巨陽與少陰俱病,則頭痛口乾而煩滿;二日則陽明與太陰俱病,則腹滿身熱,不欲食譫言;三日則少陽與厥陰俱病,則耳聾囊縮而厥,水漿不入,不知人,六日死。帝曰:五藏已傷,六府不通,榮衛不行,如是之後,三日乃死,何也?岐伯曰:陽明者,十二經脈之長也,其血氣盛,故不知人,三日其氣乃盡,故死矣。
  凡病傷寒而成溫者,先夏至日者為病溫,後夏至日者為病暑,暑當與汗皆出,勿止。


\section{刺熱篇第三十二}

  肝熱病者,小便先黃,腹痛多臥身熱,熱爭,則狂言及驚,脅滿痛,手足躁,不得安臥;庚辛甚,甲乙大汗,氣逆則庚辛死。刺足厥陰少陽。其逆則頭痛員員,脈引沖頭也。
  心熱病者,先不樂,數日乃熱,熱爭則卒心痛,煩悶善嘔,頭痛面赤,無汗;壬癸甚,丙丁大汗,氣逆則壬癸死。刺手少陰太陽。
  脾熱病者,先頭重頰痛,煩心顏青,欲嘔身熱,熱爭則腰痛不可用俛仰,腹滿洩,兩頷痛;甲乙甚,戊己大汗,氣逆則甲乙死。刺足太陰陽明。
  肺熱病者,先淅然厥,起毫毛,惡風寒,舌上黃,身熱。熱爭則喘欬,痛走胸膺背,不得大息,頭痛不堪,汗出而寒;丙丁甚,庚辛大汗,氣逆則丙丁死。刺手太陰陽明,出血如大豆,立已。
  腎熱病者,先腰痛(骨行)痠,苦喝數飲,身熱,熱爭則項痛而強,(骨行)寒且痠,足下熱,不欲言,其逆則項痛員員澹澹然;戊己甚,壬癸大汗,氣逆則戊己死。刺足少陰太陽。諸汗者,至其所勝日汗出也。
  肝熱病者,左頰先赤;心熱病者,顏先赤;脾熱病者,鼻先赤;肺熱病者,右頰先赤;腎熱病者,頤先赤。病雖未發,見赤色者刺之,名曰治未病。熱病從部所起者,至期而已;其刺之反者,三週而已;重逆則死。諸當汗者,至其所勝日,汗大出也。
  諸治熱病,以飲之寒水,乃刺之;必寒衣之,居止寒處,身寒而止也。
  熱病先胸脅痛,手足躁,刺足少陽,補足太陰,病甚者為五十九刺。熱病始手臂痛者,刺手陽明太陰而汗出止。熱病始於頭首者,刺項太陽而汗出止。熱病始於足脛者,刺足陽明而汗出止。熱病先身重骨痛,耳聾好瞑,刺足少陰,病甚為五十九刺。熱病先眩冒而熱,胸脅滿,刺足少陰少陽。
  太陽之脈,色榮顴骨,熱病也,榮未交,曰今且得汗,待時而已。與厥陰脈爭見者,死期不過三日。其熱病內連腎,少陽之脈色也。少陽之脈,色榮頰前,熱病也,榮未交,曰今且得汗,待時而已,與少陰脈爭見者,死期不過三日。
  熱病氣穴:三椎下間主胸中熱,四椎下間主鬲中熱,五椎下間主肝熱,六椎下間主脾熱,七椎下間主腎熱,榮在骶也,項上三椎陷者中也。頰下逆顴為大瘕,下牙車為腹滿,顴後為脅痛。頰上者,鬲上也。


\section{評熱病論篇第三十三}

  黃帝問曰:有病溫者,汗出輒復熱,而脈躁疾不為汗衰,狂言不能食,病名為何?岐伯對曰:病名陰陽交,交者死也。帝曰:願聞其說。岐伯曰:人所以汗出者,皆生於谷,谷生於精。今邪氣交爭於骨肉而得汗者,是邪卻而精勝也。精勝,則當能食而不復熱,復熱者邪氣也,汗者精氣也;今汗出而輒復熱者,是邪勝也,不能食者,精無俾也,病而留者,其壽可立而傾也。且夫《熱論》曰:汗出而脈尚躁盛者死。今脈不與汗相應,此不勝其病也,其死明矣。狂言者是失志,失志者死。今見三死,不見一生,雖愈必死也。
  帝曰:有病身熱汗出煩滿,煩滿不為汗解,此為何病?岐伯曰:汗出而身熱者,風也;汗出而煩滿不解者,厥也,病名曰風厥。帝曰:願卒聞之。岐伯曰:巨陽主氣,故先受邪;少陰與其為表裡也,得熱則上從之,從之則厥也。帝曰:治之奈何?岐伯曰:表裡刺之,飲之服湯。
  帝曰:勞風為病何如?岐伯曰:勞風法在肺下,其為病也,使人強上冥視,唾出若涕,惡風而振寒,此為勞風之病。帝曰:治之奈何?岐伯曰:以救俛仰。巨陽引。精者三日,中年者五日,不精者七日,咳出青黃涕,其狀如膿,大如彈丸,從口中若鼻中出,不出則傷肺,傷肺則死也。
  帝曰:有病腎風者,面胕然(疒龍)壅,害於言,可刺不?岐伯曰:虛不當刺,不當刺而刺,後五日其氣必至。帝曰:其至何如?岐伯曰:至必少氣時熱,時熱從胸背上至頭,汗出,手熱,口乾苦渴,小便黃,目下腫,腹中鳴,身重難以行,月事不來,煩而不能食,不能正偃,正偃則欬,病名曰風水,論在《刺法》中。
  帝曰:願聞其說。岐伯曰:邪之所湊,其氣必虛,陰虛者,陽必湊之,故少氣時熱而汗出也。小便黃者,少腹中有熱也。不能正偃者,胃中不和也。正偃則咳甚,上迫肺也。諸有水氣者,微腫先見於目下也。帝曰:何以言?岐伯曰:水者陰也,目下亦陰也,腹者至陰之所居,故水在腹者,必使目下腫也。真氣上逆,故口苦舌干,臥不得正偃,正偃則咳出清水也。諸水病者,故不得臥,臥則驚,驚則咳甚也。腹中鳴者,病本於胃也。薄脾則煩不能食,食不下者,胃脘隔也。身重難以行者,胃脈在足也。月事不來者,胞脈閉也,胞脈者屬心而絡於胞中,今氣上迫肺,心氣不得下通,故月事不來也。帝曰:善。


\section{逆調論篇第三十四}

  黃帝問曰:人身非常溫也,非常熱也,為之熱而煩滿者何也?岐伯對曰:陰氣少而陽氣勝,故熱而煩滿也。
  帝曰:人身非衣寒也,中非有寒氣也,寒從中生者何?岐伯曰:是人多痹氣也,陽氣少,陰氣多,故身寒如從水中出。
  帝曰:人有四支熱,逢風寒如炙如火者,何也?岐伯曰:是人者,陰氣虛,陽氣盛,四支者陽也,兩陽相得,而陰氣虛少,少水不能滅盛火,而陽獨治,獨治者,不能生長也,獨勝而止耳,逢風而如炙如火者,是人當肉爍也。
  帝曰:人有身寒,湯火不能熱,厚衣不能溫,然不凍栗,是為何病?岐伯曰:是人者,素腎氣勝,以水為事;太陽氣衰,腎脂枯不長;一水不能勝兩火,腎者水也,而生於骨,腎不生,則髓不能滿,故寒甚至骨也。所以不能凍栗者,肝一陽也,心二陽也,腎孤藏也,一水不能勝二火,故不能凍栗,病名曰骨痹,是人當攣節也。
  帝曰:人之肉苛者,雖近衣絮,猶尚苛也,是謂何疾?岐伯曰:榮氣虛衛氣實也,榮氣虛則不仁,衛氣虛則不用,榮衛俱虛,則不仁且不用,肉如故也,人身與志不相有,曰死。
  帝曰:人有逆氣不得臥而息有音者;有不得臥而息無音者;有起居如故而息有音者;有得臥,行而喘者;有不得臥,不能行而喘者;有不得臥,臥而喘者;皆何藏使然?願聞其故。岐伯曰:不得臥而息有音者,是陽明之逆也,足三陽者下行,今逆而上行,故息有音也。陽明者,胃脈也,胃者六府之海,其氣亦下行,陽明逆不得從其道,故不得臥也。《下經》曰:胃不和則臥不安。此之謂也。夫起居如故而息有音者,此肺之絡脈逆也。絡脈不得隨經上下,故留經而不行,絡脈之病人也微,故起居如故而息有音也。夫不得臥,臥則喘者,是水氣之客也;夫水者,循津液而流也,腎者,水藏,主津液,主臥與喘也。帝曰:善。


\section{瘧論篇第三十五}

黃帝問曰:夫痎瘧皆生於風,其蓄作有時者何也?岐伯對曰:瘧之始發也,先起於毫毛,伸欠乃作,寒慄鼓頷,腰脊俱痛,寒去則內外皆熱,頭痛如破,渴欲冷飲。
  帝曰:何氣使然?願聞其道。岐伯曰:陰陽上下交爭,虛實更作,陰陽相移也。陽並於陰,則陰實而陽虛,陽明虛,則寒慄鼓頷也;巨陽虛,則腰背頭項痛;三陽俱虛,則陰氣勝,陰氣勝則骨寒而痛;寒生於內,故中外皆寒;陽盛則外熱,陰虛則內熱,外內皆熱則喘而渴,故欲冷飲也。
  此皆得之夏傷於暑,熱氣盛,藏於皮膚之內,腸胃之外,此榮氣之所舍也。此令人汗空疏,腠理開,因得秋氣,汗出遇風,及得之以浴,水氣舍於皮膚之內,與衛氣並居。衛氣者,晝日行於陽,夜行於陰,此氣得陽而外出,得陰而內搏,內外相薄,是以日作。
  帝曰:其間日而作者何也?岐伯曰:其氣之舍深,內薄於陰,陽氣獨發,陰邪內著,陰與陽爭不得出,是以間日而作也。
  帝曰:善。其作日晏與其日早者,何氣使然?岐伯曰:邪氣客於風府,循膂而下,衛氣一日一夜大會於風府,其明日日下一節,故其作也晏,此先客於脊背也。每至於風府則腠理開,腠理開則邪氣入,邪氣入則病作,以此日作稍益晏也。其出於風府,日下一節,二十五日下至骶骨,二十六日入於脊內,注於伏膂之脈;其氣上行,九日出於缺盆之中,其氣日高,故作日益早也。其間日發者,由邪氣內薄於五藏,橫連募原也。其道遠,其氣深,其行遲,不能與衛氣俱行,不得皆出,故間日乃作也。
  帝曰:夫子言衛氣每至於風府,腠理乃發,發則邪氣入,入則病作。今衛氣日下一節,其氣之發也,不當風府,其日作者奈何?岐伯曰:此邪氣客於頭項循膂而下者也,故虛實不同,邪中異所,則不得當其風府也。故邪中於頭項者,氣至頭項而病;中於背者,氣至背而病;中於腰脊者,氣至腰脊而病;中於手足者,氣至手足而病。衛氣之所在,與邪氣相合,則病作。故風無常府,衛氣之所發,必開其腠理,邪氣之所合,則其府也。
  帝曰:善。夫風之與瘧也,相似同類,而風獨常在,瘧得有時而休者何也?岐伯曰:風氣留其處,故常在,瘧氣隨經絡沉以內薄,故衛氣應乃作。
  帝曰:瘧先寒而後熱者,何也?岐伯曰:夏傷於大暑,其汗大出,腠理開發,因遇夏氣淒滄之水寒,藏於腠理皮膚之中,秋傷於風,則病成矣,夫寒者,陰氣也,風者,陽氣也,先傷於寒而後傷於風,故先寒而後熱也,病以時作,名曰寒瘧。
  帝曰:先熱而後寒者,何也?岐伯曰:此先傷於風而後傷於寒,故先熱而後寒也,亦以時作,名曰溫瘧。
  其但熱而不寒者,陰氣先絕,陽氣獨發,則少氣煩冤,手足熱而欲嘔,名曰癉瘧。
  帝曰:夫經言有餘者寫之,不足者補之。今熱為有餘,寒為不足。夫瘧者之寒,湯火不能溫也,及其熱,冰水不能寒也,此皆有餘不足之類。當此之時,良工不能止,必須其自衰,乃刺之,其故何也?願聞其說。
  岐伯曰:經言無刺熇熇之熱,無刺渾渾之脈,無刺漉漉之汗,故為其病逆,未可治也。夫瘧之始發也,陽氣並於陰,當是之時,陽虛而陰盛,外無氣,故先寒慄也。陰氣逆極,則復出之陽,陽與陰復並於外,則陰虛而陽實,故先熱而渴。夫瘧氣者,並於陽則陽勝,並於陰則陰勝,陰勝則寒,陽勝則熱。瘧者,風寒之氣不常也,病極則復,至病之發也,如火之熱,如風雨不可當也。故經言曰:方其盛時必毀,因其衰也,事必大昌,此之謂也。夫瘧之未發也,陰未並陽,陽未並陰,因而調之,真氣得安,邪氣乃亡,故工不能治其已發,為其氣逆也。
  帝曰:善。攻之奈何?早晏何如?岐伯曰:瘧之且發也,陰陽之且移也,必從四末始也。陽已傷,陰從之,故先其時堅束其處,令邪氣不得入,陰氣不得出,審候見之,在孫絡盛堅而血者皆取之,此真往而未得並者也。
  帝曰:瘧不發,其應何如?岐伯曰:瘧氣者,必更盛更虛,當氣之所在也,病在陽,則熱而脈躁;在陰,則寒而脈靜;極則陰陽俱衰,衛氣相離,故病得休;衛氣集,則復病也。
  帝曰:時有間二日或至數日發,或渴或不渴,其故何也?岐伯曰:其間日者,邪氣與衛氣客於六府,而有時相失,不能相得,故休數日乃作也。瘧者,陰陽更勝也,或甚或不甚,故或渴或不渴。
  帝曰:論言夏傷於暑,秋必病瘧。今瘧不必應者,何也?岐伯曰:此應四時者也。其病異形者,反四時也。其以秋病者寒甚,以冬病者寒不甚,以春病者惡風,以夏病者多汗。
  帝曰:夫病溫瘧與寒瘧而皆安舍,舍於何藏?岐伯曰:溫瘧者,得之冬中於風,寒氣藏於骨髓之中,至春則陽氣大發,邪氣不能自出,因遇大暑,腦髓爍,肌肉消,腠理髮洩,或有所用力,邪氣與汗皆出,此病藏於腎,其氣先從內出之於外也。如是者,陰虛而陽盛,陽盛則熱矣,衰則氣復反入,入則陽虛,陽虛則寒矣,故先熱而後寒,名曰溫瘧。
  帝曰:癉瘧何如?岐伯曰:癉瘧者,肺素有熱。氣盛於身,厥逆上衝,中氣實而不外洩,因有所用力,腠理開,風寒舍於皮膚之內、分肉之間而發,發則陽氣盛,陽氣盛而不衰則病矣。其氣不及於陰,故但熱而不寒,氣內藏於心,而外舍於分肉之間,令人消爍脫肉,故命曰癉瘧。帝曰:善。

\section{刺瘧篇第三十六}

足太陽之瘧,令人腰痛頭重,寒從背起,先寒後熱,熇熇暍暍然,熱止汗出,難已,刺隙中出血。
  足少陽之瘧,令人身體解(亻亦),寒不甚,熱不甚,惡見人,見人心惕惕然,熱多汗出甚,刺足少陽。
  足陽明之瘧,令人先寒,灑淅灑淅,寒甚久乃熱,熱去汗出,喜見日月光火氣,乃快然,刺足陽明跗上。
  足太陰之瘧,令人不樂,好太息,不嗜食,多寒熱汗出,病至則善嘔,嘔已乃衰,即取之。
  足少陰之瘧,令人嘔吐甚,多寒熱,熱多寒少,欲閉戶牖而處,其病難已。
  足厥陰之瘧,令人腰痛少腹滿,小便不利,如癃狀,非癃也,數便,意恐懼,氣不足,腹中悒悒,刺足厥陰。
  肺瘧者,令人心寒,寒甚熱,熱間善驚,如有所見者,刺手太陰陽明。
  心瘧者,令人煩心甚,欲得清水,反寒多,不甚熱,刺手少陰。
  肝瘧者,令人色蒼蒼然,太息,其狀若死者,刺足厥陰見血。
  脾瘧者,令人寒,腹中痛,熱則腸中鳴,鳴已汗出,刺足太陰。
  腎瘧者,令人灑灑然,腰脊痛,宛轉,大便難,目眴眴然,手足寒,刺足太陽少陰。
  胃瘧者,令人且病也,善飢而不能食,食而支滿腹大,刺足陽明太陰橫脈出血。
  瘧發身方熱,刺跗上動脈,開其空,出其血,立寒;瘧方欲寒,刺手陽明太陰,足陽明太陰。瘧脈滿大急,刺背俞,用中針,傍伍胠俞各一,適肥瘦出其血也。瘧脈小實急,灸脛少陰,刺指井。瘧脈滿大急,刺背俞,用五胠俞背俞各一,適行至於血也。
  瘧脈緩大虛,便宜用藥,不宜用針。凡治瘧,先發如食頃乃可以治,過之則失時也。諸瘧而脈不見,刺十指間出血,血去必已,先視身之赤如小豆者盡取之。十二瘧者,其發各不同時,察其病形,以知其何脈之病也。先其發時如食頃而刺之,一刺則衰,二刺則知,三刺則已;不已,刺舌下兩脈出血,不已,刺隙中盛經出血,又刺項已下俠脊者必已。舌下兩脈者,廉泉也。
  刺瘧者,必先問其病之所先發者,先刺之。先頭痛及重者,先刺頭上及兩額兩眉間出血。先項背痛者,先刺之。先腰脊痛者,先刺隙中出血。先手臂痛者,先刺手少陰陽明十指間。先足脛痠痛者,先刺足陽明十指間出血。風瘧,瘧發則汗出惡風,刺三陽經背俞之血者。(骨行)痠痛甚,按之不可,名曰胕髓病,以饞針針絕骨出血,立已。身體小痛,刺至陰,諸陰之井無出血,間日一刺。瘧不渴,間日而作,刺足太陽;渴而間日作,刺足少陽;溫瘧汗不出,為五十九刺。

\section{氣厥論篇第三十七}

  黃帝問曰:五藏六府,寒熱相移者何?岐伯曰:腎移寒於肝,癰腫少氣。脾移寒於肝,癰腫筋攣。肝移寒於心,狂隔中。心移寒於肺,肺消,肺消者飲一溲二,死不治。肺移寒於腎,為湧水,湧水者,按腹不堅,水氣客於大腸,疾行則鳴濯濯如囊裹漿,水之病也。
  脾移熱於肝,則為驚衄。肝移熱於心,則死。心移熱於肺,傳為鬲消。肺移熱於腎,傳為柔痓。腎移熱於脾,傳為虛,腸澼,死,不可治。
  胞移熱於膀胱,則癃溺血。膀胱移熱於小腸,鬲腸不便,上為口糜。小腸移熱於大腸,為虙瘕,為沉。大腸移熱於胃,善食而瘦入,謂之食亦。胃移熱於膽,亦曰食亦。膽移熱於腦,則辛頞鼻淵,鼻淵者,濁涕下不止也,傳為衄蔑瞑目,故得之氣厥也。


\section{咳論篇第三十八}

  黃帝問曰:肺之令人咳,何也?岐伯對曰:五藏六府皆令人咳,非獨肺也。帝曰:願聞其狀。岐伯曰:皮毛者,肺之合也,皮毛先受邪氣,邪氣以從其合也。其寒飲食入胃,從肺脈上至於肺,則肺寒,肺寒則外內合邪,因而客之,則為肺咳。五藏各以其時受病,非其時,各傳以與之。人與天地相參,故五藏各以治時,感於寒則受病,微則為咳,甚者為洩為痛。乘秋則肺先受邪,乘春則肝先受之,乘夏則心先受之,乘至陰則脾先受之,乘冬則腎先受之。
  帝曰:何以異之?岐伯曰:肺咳之狀,而喘息有音,甚則唾血。心咳之狀,則心痛,喉中介介如梗狀,甚則咽腫喉痹。肝咳之狀,咳則兩脅下痛,甚則不可以轉,轉則兩胠下滿。脾咳之狀,咳則右脅下下痛,陰陰引肩背,甚則不可以動,動則咳劇。腎咳之狀,咳則腰背相引而痛,甚則咳涎。
  帝曰:六府之咳奈何?安所受病?岐伯曰:五藏之久咳,乃移於六府。脾咳不已,則胃受之,胃咳之狀,咳而嘔,嘔甚則長蟲出。肝咳不已,則膽受之,膽咳之狀,咳嘔膽汁,肺咳不已,則大腸受之,大腸咳狀,咳而遺失。心咳不已,則小腸受之,小腸咳狀,咳而失氣,氣與咳俱失。腎咳不已,則膀胱受之,膀胱咳狀,咳而遺溺。久咳不已,則三焦受之,三焦咳狀,咳而腹滿,不欲食飲,此皆聚於胃,關於肺,使人多涕唾而面浮腫氣逆也。
  帝曰:治之奈何?岐伯曰:治藏者治其俞,治府者治其合,浮腫者治其經。帝曰:善。


\section{舉痛論篇第三十九}

  黃帝問曰:余聞善言天者,必有驗於人;善言古者,必有合於今;善言人者,必有厭於己。如此,則道不惑而要數極,所謂明也。今余問於夫子,令言而可知,視而可見,捫而可得,令驗於己而發蒙解惑,可得而聞乎?岐伯再拜稽首對曰:何道之問也?
  帝曰:願聞人之五藏卒痛,何氣使然?岐伯對曰:經脈流行不止、環周不休,寒氣入經而稽遲,泣而不行,客於脈外則血少,客於脈中則氣不通,故卒然而痛。
  帝曰:其痛或卒然而止者,或痛甚不休者,或痛甚不可按者,或按之而痛止者,或按之無益者,或喘動應手者,或心與背相引而痛者,或脅肋與少腹相引而痛者,或腹痛引陰股者,或痛宿昔而成積者,或卒然痛死不知人,有少間復生者,或痛而嘔者,或腹痛而後洩者,或痛而閉不通者,凡此諸痛,各不同形,別之奈何?
  岐伯曰:寒氣客於脈外則脈寒,脈寒則縮踡,縮踡則脈絀急,絀急則外引小絡,故卒然而痛,得炅則痛立止;因重中於寒,則痛久矣。
  寒氣客於經脈之中,與炅氣相薄則脈滿,滿則痛而不可按也。寒氣稽留,炅氣從上,則脈充大而血氣亂,故痛甚不可按也。
  寒氣客於腸胃之間,膜原之下,血不得散,小絡急引故痛,按之則血氣散,故按之痛止。
  寒氣客於俠脊之脈,則深按之不能及,故按之無益也。
  寒氣客於衝脈,衝脈起於關元,隨腹直上,寒氣客則脈不通,脈不通則氣因之,故揣動應手矣。
  寒氣客於背俞之脈則脈泣,脈泣則血虛,血虛則痛,其俞注於心,故相引而痛,按之則熱氣至,熱氣至則痛止矣。
  寒氣客於厥陰之脈,厥陰之脈者,絡陰器繫於肝,寒氣客於脈中,則血泣脈急,故脅肋與少腹相引痛矣。
  厥氣客於陰股,寒氣上及少腹,血泣在下相引,故腹痛引陰股。
  寒氣客於小腸膜原之間,絡血之中,血泣不得注於大經,血氣稽留不得行,故宿昔而成積矣。
  寒氣客於五藏,厥逆上洩,陰氣竭,陽氣未入,故卒然痛死不知人,氣復反則生矣。
  寒氣客於腸胃,厥逆上出,故痛而嘔也。
  寒氣客於小腸,小腸不得成聚,故後洩腹痛矣。
  熱氣留於小腸,腸中痛,癉熱焦喝,則堅幹不得出,故痛而閉不通矣。
  帝曰:所謂言而可知者也。視而可見奈何?岐伯曰:五藏六府,固盡有部,視其五色,黃赤為熱,白為寒,青黑為痛,此所謂視而可見者也。
  帝曰:捫而可得奈何?岐伯曰:視其主病之脈,堅而血及陷下者,皆可捫而得也。
  帝曰:善。余知百病生於氣也。怒則氣上,喜則氣緩,悲則氣消,恐則氣下,寒則氣收,炅則氣洩,驚則氣亂,勞則氣耗,思則氣結,九氣不同,何病之生?岐伯曰:怒則氣逆,甚則嘔血及飧洩,故氣上矣。喜則氣和志達,榮衛通利,故氣緩矣。悲則心繫急,肺布葉舉,而上焦不通,榮衛不散,熱氣在中,故氣消矣。恐則精卻,卻則上焦閉,閉則氣還,還則下焦脹,故氣不行矣。寒則腠理閉,氣不行,故氣收矣。炅則腠理開,榮衛通,汗大洩,故氣洩。驚則心無所倚,神無所歸,慮無所定,故氣亂矣。勞則喘息汗出,外內皆越,故氣耗矣。思則心有所存,神有所歸,正氣留而不行,故氣結矣。


\section{腹中論篇第四十}

  黃帝問曰:有病心腹滿,旦食則不能暮食,此為何病?岐伯對曰:名為鼓脹。帝曰:治之奈何?岐伯曰:治之以雞矢醴,一劑知,二劑已。帝曰:其時有復發者何也?岐伯曰:此飲食不節,故時有病也。雖然其病且已,時故當病,氣聚於腹也。
  帝曰:有病胸脅支滿者,妨於食,病至則先聞腥臊臭,出清液,先唾血,四支清,目眩,時時前後血,病名為何?何以得之?岐伯曰:病名血枯。此得之年少時,有所大脫血:若醉入房中,氣竭肝傷,故月事衰少不來也。帝曰:治之奈何?復以何術?岐伯曰:以四烏骨一藘茹二物併合之,丸以雀卵,大如小豆,以五丸為後飯,飲以鮑魚汁,利腸中及傷肝也。
  帝曰:病有少腹盛,上下左右皆有根,此為何病?可治不?岐伯曰:病名曰伏梁。帝曰:伏梁何因而得之?岐伯曰:裹大膿血,居腸胃之外,不可治,治之每切,按之致死。帝曰:何以然?岐伯曰:此下則因陰,必下膿血,上則迫胃脘,生鬲,俠胃脘內癰,此久病也,難治。居齊上為逆,居齊下為從,勿動亟奪,論在《刺法》中。
  帝曰:人有身體髀股(骨行)皆腫,環齊而痛,是為何病?岐伯曰:病名伏梁,此風根也。其氣溢於大腸而著於肓,肓之原在齊下,故環齊而痛也,不可動之,動之為水溺澀之病。
  帝曰:夫子數言熱中消中,不可服高梁芳草石藥,石藥發瘨,芳草發狂。夫熱中消中者,皆富貴人也,今禁高梁,是不合其心,禁芳草石藥,是病不愈,願聞其說。岐伯曰:夫芳草之氣美,石藥之氣悍,二者其氣急疾堅勁,故非緩心和人,不可以服此二者。帝曰:不可以服此二者,何以然?岐伯曰:夫熱氣慓悍,藥氣亦然,二者相遇,恐內傷脾,脾者土也而惡木,服此藥者,至甲乙日更論。
  帝曰:善。有病膺腫頸痛胸滿腹脹,此為何病?何以得之?岐伯曰:名厥逆。帝曰:治之奈何?岐伯曰:灸之則瘖,石之則狂,須其氣並,乃可治也。帝曰:何以然?岐伯曰:陽氣重上,有餘於上,灸之則陽氣入陰,入則瘖,石之則陽氣虛,虛則狂;須其氣並而治之,可使全也。
  帝曰:善。何以知懷子之且生也?岐伯曰:身有病而無邪脈也。
  帝曰:病熱而有所痛者何也?岐伯曰:病熱者,陽脈也,以三陽之動也,人迎一盛少陽,二盛太陽,三盛陽明,入陰也。夫陽入於陰,故病在頭與腹,乃(月真)脹而頭痛也。帝曰:善。

\section{刺腰痛篇第四十一}

  足太陽脈令人腰痛,引項脊尻背如重狀;刺其隙中太陽正經出血,春無見血。
  少陽令人腰痛,如以針刺其皮中,循循然不可以俯仰,不可以顧,刺少陽成骨之端出血,成骨在膝外廉之骨獨起者,夏無見血。
  陽明令人腰痛,不可以顧,顧如有見者,善悲,刺陽明於(骨行)前三痏,上下和之出血,秋無見血。
  足少陰令人腰痛,痛引脊內廉,刺少陰於內踝上二痏,春無見血,出血太多,不可復也。
  厥陰之脈,令人腰痛,腰中如張弓弩弦;刺厥陰之脈,在腨踵魚腹之外,循之纍纍然,乃刺之,其病令人善言,默默然不慧,刺之三痏。
  解脈令人腰痛,痛引肩,目然,時遺溲,刺解脈,在膝筋肉分間隙外廉之橫脈出血,血變而止。
  解脈令人腰痛如引帶,常如折腰狀,善恐,刺解脈在隙中結絡如黍米,刺之血射以黑,見赤血而已。
  同陰之脈,令人腰痛,痛如小錘居其中,怫然腫;刺同陰之脈,在外踝上絕骨之端,為三痏。
  陽維之脈,令人腰痛,痛上怫然腫;刺陽維之脈,脈與太陽合腨下間,去地一尺所。
  衡絡之脈,令人腰痛,不可以俛仰,仰則恐僕,得之舉重傷腰,衡絡絕,惡血歸之,刺之在隙陽筋之間,上隙數寸,衡居為二痏出血。
  會陰之脈,令人腰痛,痛上漯漯然汗出,汗干令人欲飲,飲已欲走,刺直陽之脈上三痏,在蹻上隙下五寸橫居,視其盛者出血。
  飛陽之脈,令人腰痛,痛上怫怫然,甚則悲以恐;刺飛陽之脈,在內踝上五寸,少陰之前,與陰維之會。
  昌陽之脈,令人腰痛,痛引膺,目(目巟)(目巟)然,甚則反折,舌卷不能言;刺內筋為二痏,在內踝上大筋前,太陰後,上踝二寸所。
  散脈,令人腰痛而熱,熱甚生煩,腰下如有橫木居其中,甚則遺溲;刺散脈,在膝前骨肉分間,絡外廉束脈,為三痏。
  肉裡之脈,令人腰痛,不可以咳,咳則筋縮急;刺肉裡之脈為二痏,在太陽之外,少陽絕骨之後。
  腰痛俠脊而痛至頭,幾幾然,目(目巟)(目巟)欲僵仆,刺足太陽隙中出血。腰痛上寒,刺足太陽陽明;上熱,刺足厥陰;不可以俛仰,刺足少陽;中熱而喘,刺足少陰,刺隙中出血。
  腰痛上寒,不可顧,刺足陽明;上熱,刺足太陰;中熱而喘,刺足少陰。大便難,刺足少陰。少腹滿,刺足厥陰。如折,不可以俛仰,不可舉,刺足太陽,引脊內廉,刺足少陰。
  腰痛引少腹控(月少),不可以仰。刺腰尻交者,兩髁胂上。以月生死為痏數,髮針立已。左取右,右取左。


\section{風論篇第四十二}

  黃帝問曰:風之傷人也,或為寒熱,或為熱中,或為寒中,或為癘風,或為偏枯,或為風也,其病各異,其名不同,或內至五藏六府,不知其解,願聞其說。
  岐伯對曰:風氣藏於皮膚之間,內不得通,外不得洩;風者,善行而數變,腠理開則灑然寒,閉則熱而悶,其寒也則衰食飲,其熱也則消肌肉,故使人怢慄而不能食,名曰寒熱。
  風氣與陽明入胃,循脈而上至目內眥,其人肥則風氣不得外洩,則為熱中而目黃;人瘦則外洩而寒,則為寒中而泣出。
  風氣與太陽俱入,行諸脈俞,散於分肉之間,與衛氣相干,其道不利,故使肌肉憤(月真)而有瘍,衛氣有所凝而不行,故其肉有不仁也。癘者,有榮氣熱府,其氣不清,故使其鼻柱壞而色敗,皮膚瘍潰,風寒客於脈而不去,名曰癘風,或名曰寒熱。
  以春甲乙傷於風者為肝風,以夏丙丁傷於風者為心風,以季夏戊己傷於邪者為脾風,以秋庚辛中於邪者為肺風,以冬壬癸中於邪者為腎風。
  風中五藏六府之俞,亦為藏府之風,各入其門戶所中,則為偏風。風氣循風府而上,則為腦風;風入系頭,則為目風,眼寒;飲酒中風,則為漏風;入房汗出中風,則為內風;新沐中風,則為首風;久風入中,則為腸風飧洩;外在腠理,則為洩風。故風者百病之長也,至其變化,乃為他病也,無常方,然致有風氣也。
  帝曰:五藏風之形狀不同者何?願聞其診及其病能。
  岐伯曰:肺風之狀,多汗惡風,色皏然白,時咳短氣,晝日則差,暮則甚,診在眉上,其色白。
  心風之狀,多汗惡風,焦絕,善怒嚇,赤色,病甚則言不可快,診在口,其色赤。
  肝風之狀,多汗惡風,善悲,色微蒼,嗌干善怒,時憎女子,診在目下,其色青。
  脾風之狀,多汗惡風,身體怠惰,四支不欲動,色薄微黃,不嗜食,診在鼻上,其色黃。
  腎風之狀,多汗惡風,面(疒龍)然浮腫,脊痛不能正立,其色炲,隱曲不利,診在肌上,其色黑。
  胃風之狀,頸多汗惡風,食飲不下,鬲塞不通,腹善滿,失衣則(月真)脹,食寒則洩,診形瘦而腹大。
  首風之狀,頭面多汗,惡風,當先風一日則病甚,頭痛不可以出內,至其風日,則病少愈。
  漏風之狀,或多汗,常不可單衣,食則汗出,甚則身汗,喘息惡風,衣常濡,口乾善渴,不能勞事。
  洩風之狀,多汗,汗出洩衣上,口中干,上漬其風,不能勞事,身體盡痛則寒。帝曰:善。


\section{痹論篇第四十三}

  黃帝問曰:痹之安生?岐伯對曰:風寒濕三氣雜至,合而為痹也。其風氣勝者為行痹,寒氣勝者為痛痹,濕氣勝者為著痹也。
  帝曰:其有五者何也?岐伯曰:以冬遇此者為骨痹,以春遇此者為筋痹,以夏遇此者為脈痹,以至陰遇此者為肌痹,以秋遇此者為皮痹。
  帝曰:內舍五藏六府,何氣使然?岐伯曰:五藏皆有合,病久而不去者,內舍於其合也。故骨痹不已,復感於邪,內舍於腎;筋痹不已,復感於邪,內舍於肝;脈痹不已,復感於邪,內舍於心;肌痹不已,復感於邪,內舍於脾;皮痹不已,復感於邪,內舍於肺。所謂痹者,各以其時,重感於風寒濕之氣也。
  凡痹之客五藏者,肺痹者,煩滿喘而嘔;心痹者,脈不通,煩則心下鼓,暴上氣而喘,嗌干善噫,厥氣上則恐;肝痹者,夜臥則驚,多飲數小便,上為引如懷;腎痹者,善脹,尻以代踵,脊以代頭;脾痹者,四支懈惰,發咳嘔汁,上為大塞;腸痹者,數飲而出不得,中氣喘爭,時發飧洩;胞痹者,少腹膀胱,按之內痛,若沃以湯,澀於小便,上為清涕。
  陰氣者,靜則神藏,躁則消亡,飲食自倍,腸胃乃傷。淫氣喘息,痹聚在肺;淫氣憂思,痹聚在心;淫氣遺溺,痹聚在腎;淫氣乏竭,痹聚在肝;淫氣肌絕,痹聚在脾。
  諸痹不巳,亦益內也,其風氣勝者,其人易已也。
  帝曰:痹,其時有死者,或疼久者,或易已者,其故何也?岐伯曰:其入藏者死,其留連筋骨間者疼久,其留皮膚間者易已。
  帝曰:其客於六府者何也?岐伯曰:此亦其食飲居處,為其病本也。六府亦各有俞,風寒濕氣中其俞,而食飲應之,循俞而入,各舍其府也。
  帝曰:以針治之奈何?岐伯曰:五藏有俞,六府有合,循脈之分,各有所發,各隨其過,則病瘳也。
  帝曰:榮衛之氣,亦令人痹乎?岐伯曰:榮者,水谷之精氣也,和調於五藏,灑陳於六府,乃能入於脈也。故循脈上下,貫五藏,絡六府也。衛者,水谷之悍氣也,其氣慓疾滑利,不能入於脈也,故循皮膚之中,分肉之間,熏於肓膜,散於胸腹,逆其氣則病,從其氣則愈,不與風寒濕氣合,故不為痹。
  帝曰:善。痹或痛,或不痛,或不仁,或寒,或熱,或燥,或濕,其故何也?岐伯曰:痛者,寒氣多也,有寒故痛也。其不痛不仁者,病久入深,榮衛之行澀,經絡時疏,故不通,皮膚不營,故為不仁。其寒者,陽氣少,陰氣多,與病相益,故寒也。其熱者,陽氣多,陰氣少,病氣勝,陽遭陰,故為痹熱。其多汗而濡者,此其逢濕甚也,陽氣少,陰氣盛,兩氣相感,故汗出而濡也。
  帝曰:夫痹之為病,不痛何也?岐伯曰:痹在於骨則重,在於脈則血凝而不流,在於筋則屈不伸,在於肉則不仁,在於皮則寒,故具此五者則不痛也。凡痹之類,逢寒則蟲,逢熱則縱。帝曰:善。


\section{痿論篇第四十四}

  黃帝問曰:五藏使人痿何也?岐伯對曰:肺主身之皮毛,心主身之血脈,肝主身之筋膜,脾主身之肌肉,腎主身之骨髓。故肺熱葉焦,則皮毛虛弱急薄,著則生痿躄也;心氣熱,則下脈厥而上,上則下脈虛,虛則生脈痿,樞折挈,脛縱而不任地也;肝氣熱,則膽洩口苦筋膜干,筋膜干則筋急而攣,發為筋痿;脾氣熱,則胃干而渴,肌肉不仁,發為肉痿;腎氣熱,則腰脊不舉,骨枯而髓減,發為骨痿。
  帝曰:何以得之?岐伯曰:肺者,藏之長也,為心之蓋也;有所失亡,所求不得,則發肺鳴,鳴則肺熱葉焦,故曰,五藏因肺熱葉焦發為痿躄,此之謂也。悲哀太甚,則胞絡絕,胞絡絕,則陽氣內動,發則心下崩,數溲血也。故《本病》曰:大經空虛,發為肌痹,傳為脈痿。思想無窮,所願不得,意淫於外,入房太甚,宗筋弛縱,發為筋痿,及為白淫,故《下經》曰:筋痿者,生於肝使內也。有漸於濕,以水為事,若有所留,居處相濕,肌肉濡漬,痹而不仁,發為肉痿。故《下經》曰:肉痿者,得之濕地也。有所遠行勞倦,逢大熱而渴,渴則陽氣內伐,內伐則熱舍於腎,腎者水藏也,今水不勝火,則骨枯而髓虛,故足不任身,發為骨痿。故《下經》曰:骨痿者,生於大熱也。
  帝曰:何以別之?岐伯曰:肺熱者色白而毛敗,心熱者色赤而絡脈溢,肝熱者色蒼而爪枯,脾熱者色黃而肉蠕動;腎熱者色黑而齒槁。
  帝曰:如夫子言可矣,論言治痿者獨取陽明,何也?岐伯曰:陽明者,五藏六府之海,主潤宗筋,宗筋主骨而利機關也。衝脈者,經脈之海也,主滲灌谿谷,與陽明合於宗筋,陰陽總宗筋之會,會於氣街,而陽明為之長,皆屬於帶脈,而絡於督脈。故陽明虛則宗筋縱,帶脈不引,故足痿不用也。
  帝曰:治之奈何?岐伯曰:各補其滎而通其俞,調其虛實,和其逆順,筋、脈、骨、肉各以其時受月,則病已矣。帝曰:善。


\section{厥論篇第四十五}

  黃帝問曰:厥之寒熱者何也?岐伯對曰:陽氣衰於下,則為寒厥;陰氣衰於下,則為熱厥。
  帝曰:熱厥之為熱也,必起於足下者何也?岐伯曰:陽氣起於足五指之表,陰脈者集於足下,而聚於足心,故陽氣盛則足下熱也。
  帝曰:寒厥之為寒也,必從五指而上於膝者何也?岐伯曰:陰氣起於五指之裡,集於膝下而聚於膝上,故陰氣盛,則從五指至膝上寒,其寒也,不從外,皆從內也。
  帝曰:寒厥何失而然也?岐伯曰:前陰者,宗筋之所聚,太陰陽明之所合也。春夏則陽氣多而陰氣少,秋冬則陰氣盛而陽氣衰。此人者質壯,以秋冬奪於所用,下氣上爭不能復,精氣溢下,邪氣因從之而上也;氣因於中,陽氣衰,不能滲營其經絡,陽氣日損,陰氣獨在,故手足為之寒也。
  帝曰:熱厥何如而然也?岐伯曰;灑入於胃,則絡脈滿而經脈虛;脾主為胃行其津液者也,陰氣虛則陽氣入,陽氣入則胃不和,胃不和則精氣竭,精氣竭則不營其四支也。此人必數醉若飽以入房,氣聚於脾中不得散,酒氣與谷氣相薄,熱盛於中,故熱偏於身內熱而溺赤也。夫酒氣盛而慓悍,腎氣有衰,陽氣獨盛,故手足為之熱也。
  帝曰:厥或令人腹滿,或令人暴不知人,或至半日遠至一日乃知人者何也?岐伯曰:陰氣盛於上則下虛,下虛則腹脹滿;陽氣盛於上,則下氣重上,而邪氣逆,逆則陽氣亂,陽氣亂則不知人也。
  帝曰:善。願聞六經脈之厥狀病能也。岐伯曰:巨陽之厥,則腫首頭重,足不能行,發為眴僕;陽明之厥,則癲疾欲走呼,腹滿不得臥,面赤而熱,妄見而妄言;少陽之厥,則暴聾頰腫而熱,脅痛,(骨行)不可以運;太陰之厥,則腹滿(月真)脹,後不利,不欲食,食則嘔,不得臥;少陰之厥,則口乾溺赤,腹滿心痛;厥陰之厥,則少腹腫痛,腹脹,涇溲不利,好臥屈膝,陰縮腫,(骨行)內熱。盛則寫之,虛則補之,不盛不虛,以經取之。
  太陰厥逆,(骨行)急攣,心痛引腹,治主病者;少陰厥逆,虛滿嘔變,下洩清,治主病者;厥陰厥逆,攣、腰痛,虛滿前閉,譫言,治主病者;三陰俱逆,不得前後,使人手足寒,三日死。太陽厥逆,僵仆,嘔血善衄,治主病者;少陽厥逆,機關不利,機關不利者,腰不可以行,項不可以顧,發腸癰不可治,驚者死;陽明厥逆,喘咳身熱,善驚,衄,嘔血。
  手太陰厥逆,虛滿而咳,善嘔沫,治主病者;手心主、少陰厥逆,心痛引喉,身熱死,不可治。手太陽厥逆,耳聾泣出,項不可以顧,腰不可以俛仰,治主病者;手陽明、少陽厥逆,發喉痹、嗌腫,治主病者。

\section{病能論篇第四十六}

  黃帝問曰:人病胃脘癰者,診當何如?岐伯對曰:診此者當候胃脈,其脈當沉細,沉細者氣逆,逆者人迎甚盛,甚盛則熱;人迎者胃脈也,逆而盛,則熱聚於胃口而不行,故胃脘為癰也。
  帝曰:善。人有臥而有所不安者何也?岐伯曰:藏有所傷,及精有所之寄則安,故人不能懸其病也。
  帝曰:人之不得偃臥者何也?岐伯曰:肺者藏之蓋也,肺氣盛則脈大,脈大則不得偃臥,論在《奇恆陰陽》中。
  帝曰:有病厥者,診右脈沉而緊,左脈浮而遲,不然病主安在?岐伯曰:冬診之,右脈固當沉緊,此應四時,左脈浮而遲,此逆四時,在左當主病在腎,頗關在肺,當腰痛也。帝曰:何以言之?岐伯曰:少陰脈貫腎絡肺,今得肺脈,腎為之病,故腎為腰痛之病也。
  帝曰:善。有病頸癰者,或石治之,或針灸治之,而皆已,其真安在?岐伯曰:此同名異等者也。夫癰氣之息者,宜以針開除去之;夫氣盛血聚者,宜石而寫之。此所謂同病異治也。
  帝曰:有病怒狂者,此病安生?岐伯曰:生於陽也,帝曰:陽何以使人狂?岐伯曰:陽氣者,因暴折而難決,故善怒也,病名曰陽厥。帝曰:何以知之?岐伯曰:陽明者常動,巨陽少陽不動,不動而動大疾,此其候也。帝曰:治之奈何?岐伯曰:奪其食即已。夫食入於陰,長氣於陽,故奪其食即已。使之服以生鐵洛為飲,夫生鐵洛者,下氣疾也。
  帝曰:善。有病身熱解墯,汗出如浴,惡風少氣,此為何病?岐伯曰:病名曰酒風。帝曰:治之奈何?岐伯曰:以澤瀉,術各十分,麋銜五分,合,以三指撮,為後飯。
  所謂深之細者,其中手如針也,摩之切之,聚者堅也,博者大也。《上經》者,言氣之通天也;《下經》者,言病之變化也;《金匱》者,決死生也;《揆度》者,切度之也;《奇恆》者,言奇病也。所謂奇者,使奇病不得以四時死也;恆者,得以四時死也。所謂揆者,方切求之也,言切求其脈理也;度者,得其病處,以四時度之也。


\section{奇病論篇第四十七}

黃帝問曰:人有重身,九月而瘖,此為何也?岐伯對曰:胞之絡脈絕也。帝曰:何以言之?岐伯曰:胞絡者係於腎,少陰之脈,貫腎系舌本,故不能言。帝曰:治之奈何?岐伯曰:無治也,當十月復。《刺法》曰:無損不足,益有餘,以成其疹,然後調之。所謂無損不足者,身羸瘦,無用鑱石也;無益其有餘者,腹中有形而洩之,洩之則精出而病獨擅中,故曰疹成也。
  帝曰:病脅下滿氣逆,二三歲不已,是為何病?岐伯曰:病名曰息積,此不妨於食,不可灸刺,積為導引服藥,藥不能獨治也。
  帝曰:人有身體髀股(骨行)皆腫,環齊而痛,是為何病?岐伯曰:病名曰伏梁。此風根也,其氣溢於大腸,而著於肓,肓之原在齊下,故環齊而痛也。不可動之,動之為水溺濇之病也。
  帝曰:人有尺脈數甚,筋急而見,此為何病?岐伯曰:此所謂疹筋,是人腹必急,白色黑色見,則病甚。
  帝曰:人有病頭痛以數歲不已,此安得之?名為何病?岐伯曰:當有所犯大寒,內至骨髓,髓者以腦為主,腦逆故令頭痛,齒亦痛,病名曰厥逆。帝曰:善。
  帝曰:有病口甘者,病名為何?何以得之?岐伯曰:此五氣之溢也,名曰脾癉。夫五味入口,藏於胃,脾為之行其精氣,津液在脾,故令人口甘也;此肥美之所發也,此人必數食甘美而多肥也,肥者令人內熱,甘者令人中滿,故其氣上溢,轉為消渴。治之以蘭,除陳氣也。
  帝曰:有病口苦,取陽陵泉,口苦者病名為何?何以得之?岐伯曰:病名曰膽癉。夫肝者中之將也,取決於膽,咽為之使。此人者,數謀慮不決,故膽虛氣上溢,而口為之苦。治之以膽募俞,治在《陰陽十二官相使》中。
  帝曰:有癃者,一日數十溲,此不足也。身熱如炭,頸膺如格,人迎躁盛,喘息氣逆,此有餘也。太陰脈微細如發者,此不足也。其病安在?名為何病?岐伯曰:病在太陰,其盛在胃,頗在肺,病名曰厥,死不治。此所謂得五有餘二不足也。帝曰:何謂五有餘二不足?岐伯曰:所謂五有餘者,五病之氣有餘也;二不足者,亦病氣之不足也。今外得五有餘,內得二不足,此其身不表不裡,亦正死明矣。
  帝曰:人生而有病巔疾者,病名曰何?安所得之?岐伯曰:病名為胎病。此得之在母腹中時,其母有所大驚,氣上而不下,精氣並居,故令子發為巔疾也。
  帝曰:有病(疒龍)然如有水狀,切其脈大緊,身無痛者,形不瘦,不能食,食少,名為何病?岐伯曰:病生在腎,名為腎風。腎風而不能食,善驚,驚已,心氣痿者死。帝曰:善。

\section{大奇論篇第四十八}

  肝滿腎滿肺滿皆實,即為腫。肺之雍,喘而兩胠滿;肝雍,兩胠滿,臥則驚,不得小便;腎雍,腳下至少腹滿,脛有大小,髀(骨行)大跛,易偏枯。
  心脈滿大,癇瘛筋攣;肝脈小急,癇瘛筋攣;肝脈騖,暴有所驚駭,脈不至若瘖,不治自已。
  腎脈小急,肝脈小急,心脈小急,不鼓皆為瘕。
  腎肝並沉為石水,並浮為風水,並虛為死,並小弦欲驚。
  腎脈大急沉,肝脈大急沉,皆為疝。
  心脈搏滑急為心疝,肺脈沉搏為肺疝。
  三陽急為瘕,三陰急為疝,二陰急為癇厥,二陽急為驚。
  脾脈外鼓,沉為腸澼,久自已。肝脈小緩為腸澼,易治。腎脈小搏沉,為腸澼下血,血溫身熱者死。心肝澼亦下血,二藏同病者可治。其脈小沉濇為腸澼,其身熱者死,熱見七日死。
  胃脈沉鼓濇,胃外鼓大,心脈小堅急,皆鬲偏枯。男子發左,女子發右,不瘖舌轉,可治,三十日起,其從者,瘖,三歲起。年不滿二十者,三歲死。
  脈至而搏,血衄身熱者死,脈來懸鉤浮為常脈。
  脈至如喘,名曰暴厥。暴厥者,不知與人言。脈至如數,使人暴驚,三四日自已。
  脈至浮合,浮合如數,一息十至以上,是經氣予不足也,微見九十日死;脈至如火薪然,是心精之予奪也,草干而死;脈至如散葉,是肝氣予虛也,木葉落而死;脈至如省客,省客者,脈塞而鼓,是腎氣予不足也,懸去棗華而死;脈至如丸泥,是胃精予不足也,榆莢落而死;脈至如橫格,是膽氣予不足也,禾熟而死;脈至如弦縷,是胞精予不足也,病善言,下霜而死,不言可治;脈至如交漆,交漆者,左右傍至也,微見三十日死;脈至如湧泉,浮鼓肌中,太陽氣予不足也,少氣味,韭英而死;脈至如頹土之狀,按之不得,是肌氣予不足也,五色先見,黑白壘發死;脈至如懸雍,懸雍者,浮揣切之益大,是十二俞之予不足也,水凝而死;脈至如偃刀,偃刀者,浮之小急,按之堅大急,五藏菀熟,寒熱獨並於腎也,如此其人不得坐,立春而死;脈至如丸滑不直手,不直手者,按之不可得也,是大腸氣予不足也,棗葉生而死;脈至如華者,令人善恐,不欲坐臥,行立常聽,是小腸氣予不足也,季秋而死。


\section{脈解篇第四十九}

  太陽所謂腫腰脽痛者,正月太陽寅,寅太陽也,正月陽氣出在上,而陰氣盛,陽未得自次也,故腫腰脽痛也。病偏虛為跛者,正月陽氣凍解地氣而出也,所謂偏虛者,冬寒頗有不足者,故偏虛為跛也。所謂強上引背者,陽氣大上而爭,故強上也。所謂耳鳴者,陽氣萬物盛上而躍,故耳鳴也。所謂甚則狂巔疾者,陽盡在上,而陰氣從下,下虛上實,故狂巔疾也,所謂浮為聾者,皆在氣也。所謂入中為瘖者,陽盛已衰,故為瘖也。內奪而厥,則為瘖俳,此腎虛也。少陰不至者,厥也。
  少陽謂心脅痛者,言少陽盛也,盛者心之所表也。九月陽氣盡而陰氣盛,故心脅痛也。所謂不可反側者,陰氣藏物也,物藏則不動,故不可反側也。所謂甚則躍者,九月萬物盡衰,草木畢落而墮,則氣去陽而之陰,氣盛而陽之下長,故謂躍。
  陽明所謂灑灑振寒者,陽明者午也,五月盛陽之陰也,陽盛而陰氣加之,故灑灑振寒也。所謂脛腫而股不收者,是五月盛陽之陰也,陽者衰於五月,而一陰氣上,與陽始爭,故脛腫而股不收也。所謂上喘而為水者,陰氣下而復上,上則邪客於藏府間,故為水也。所謂胸痛少氣者,水氣在藏府也,水者,陰氣也,陰氣在中,故胸痛少氣也。所謂甚則厥,惡人與火,聞木音則惕然而驚者,陽氣與陰氣相薄,水火相惡,故惕然而驚也。所謂欲獨閉戶牖而處者,陰陽相薄也,陽盡而陰盛,故欲獨閉戶牖而居。所謂病至則欲乘高而歌,棄衣而走者,陰陽復爭,而外並於陽,故使之棄衣而走也。所謂客孫脈則頭痛鼻鼽腹腫者,陽明並於上,上者則其孫絡太陰也,故頭痛鼻鼽腹腫也。
  太陰所謂病脹者,太陰子也,十一月萬物氣皆藏於中,故曰病脹;所謂上走心為噫者,陰盛而上走於陽明,陽明絡屬心,故曰上走心為噫也;所謂食則嘔者,物盛滿而上溢,故嘔也;所謂得後與氣則快然如衰者,十二月陰氣下衰,而陽氣且出,故曰得後與氣則快然如衰也。
  少陰所謂腰痛者,少陰者,腎也,十月萬物陽氣皆傷,故腰痛也。所謂嘔咳上氣喘者,陰氣在下,陽氣在上,諸陽氣浮,無所依從,故嘔咳上氣喘也。所謂色色不能久立久坐,起則目(目巟)(目巟)無所見者,萬物陰陽不定未有主也,秋氣始至,微霜始下,而方殺萬物,陰陽內奪,故目(目巟)(目巟)無所見也。所謂少氣善怒者,陽氣不治,陽氣不治,則陽氣不得出,肝氣當治而未得,故善怒,善怒者,名曰煎厥。所謂恐如人將捕之者,秋氣萬物未有畢去,陰氣少,陽氣入,陰陽相薄,故恐也。所謂惡聞食臭者,胃無氣,故惡聞食臭也。所謂面黑如地色者,秋氣內奪,故變於色也。所謂咳則有血者,陽脈傷也,陽氣未盛於上而脈滿,滿則咳,故血見於鼻也。
  厥陰所謂頹疝,婦人少腹腫者,厥陰者辰也,三月陽中之陰,邪在中,故曰頹疝少腹腫也。所謂腰脊痛不可以俯仰者,三月一振榮華,萬物一俯而不仰也。所謂頹癃疝膚脹者,曰陰亦盛而脈脹不通,故曰頹癃疝也。所謂甚則嗌乾熱中者,陰陽相薄而熱,故嗌干也。


\section{刺要論篇第五十}

  黃帝問曰:願聞刺要。岐伯對曰:病有浮沉,刺有淺深,各至其理,無過其道,過之則內傷,不及則生外壅,壅則邪從之,淺深不得,反為大賊,內動五藏,後生大病。故曰:病有在毫毛腠理者,有在皮膚者,有在肌肉者,有在脈者,有在筋者,有在骨者,有在髓者。
  是故刺毫毛腠理無傷皮,皮傷則內動肺,肺動則秋病溫瘧,淅淅然寒慄。
  刺皮無傷肉,肉傷則內動脾,脾動則七十二日四季之月,病腹脹煩,不嗜食。
  刺肉無傷脈,脈傷則內動心,心動則夏病心痛。
  刺脈無傷筋,筋傷則內動肝,肝動則春病熱而筋弛。
  刺筋無傷骨,骨傷則內動腎,腎動則冬病脹腰痛。
   刺骨無傷髓,髓傷則銷鑠(骨行)酸,體解(亻亦)然不去矣。


\section{刺齊論篇第五十一}

  黃帝問曰:願聞刺淺深之分。岐伯對曰:刺骨者無傷筋,刺筋者無傷肉,刺肉者無傷脈,刺脈者無傷皮,刺皮者無傷肉,刺肉者無傷筋,刺筋者無傷骨。
  帝曰:余未知其所謂,願聞其解。岐伯曰:刺骨無傷筋者,針至筋而去,不及骨也。刺筋無傷肉者,至肉而去,不及筋也。刺肉無傷脈者,至脈而去,不及肉也。刺脈無傷皮者,至皮而去,不及脈也。
  所謂刺皮無傷肉者,病在皮中,針入皮中,無傷肉也。刺肉無傷筋者,過肉中筋也。刺筋無傷骨者,過筋中骨也。此之謂反也。


\section{刺禁論篇第五十二}
 
  黃帝問曰:願聞禁數。岐伯對曰:藏有要害,不可不察,肝生於左,肺藏於右,心部於表,腎治於裡,脾為之使,胃為之市。鬲肓之上,中有父母,七節之傍,中有小心,從之有福,逆之有咎。
  刺中心,一日死,其動為噫。刺中肝,五日死,其動為語。刺中腎,六日死,其動為嚏。刺中肺,三日死,其動為咳。刺中脾,十日死,其動為吞。刺中膽,一日半死,其動為嘔。
  刺跗上,中大脈,血出不止死。刺面,中溜脈,不幸為盲。刺頭,中腦戶,入腦立死。刺舌下,中脈太過,血出不止為瘖。刺足下布絡中脈,血不出為腫。刺隙中大脈,令人僕脫色。刺氣街中脈,血不出為腫,鼠僕。刺脊間中髓,為傴。刺乳上,中乳房,為腫,根蝕。刺缺盆中內陷,氣洩,令人喘咳逆。刺手魚腹內陷,為腫。
  無刺大醉,令人氣亂。無刺大怒,令人氣逆。無刺大勞人,無刺新飽人,無刺大飢人,無刺大渴人,無刺大驚人。
  刺陰股中大脈,血出不止死。刺客主人內陷中脈,為內漏、為聾。刺膝髕出液,為跛。刺臂太陰脈,出血多立死。刺足少陰脈,重虛出血,為舌難以言。刺膺中陷,中肺,為喘逆仰息。刺肘中內陷,氣歸之,為不屈伸。刺陰股下三寸內陷,令人遺溺。刺掖下脅間內陷,令人咳。刺少腹,中膀胱,溺出,令人少腹滿。刺腨腸內陷為腫。刺匡上陷骨中脈,為漏、為盲。刺關節中液出,不得屈伸。


\section{刺志論篇第五十三}

  黃帝問曰:願聞虛實之要。岐伯對曰:氣實形實,氣虛形虛,此其常也,反此者病。谷盛氣盛,谷虛氣虛,此其常也,反此者病。脈實血實,脈虛血虛,此其常也,反此者病。
  帝曰:如何而反?岐伯曰:氣虛身熱,此謂反也;谷入多而氣少,此謂反也;谷不入而氣多,此謂反也;脈盛血少,此謂反也;脈少血多,此謂反也。
  氣盛身寒,得之傷寒。氣虛身熱,得之傷暑。谷入多而氣少者,得之有所脫血,濕居下也。谷入少而氣多者,邪在胃及與肺也。脈小血多者,飲中熱也。脈大血少者,脈有風氣,水漿不入,此之謂也。
  夫實者,氣入也;虛者,氣出也;氣實者,熱也;氣虛者,寒也。入實者,左手開針空也;入虛者,左手閉針空也。


\section{針解篇第五十四}

  黃帝問曰:願聞九針之解,虛實之道。岐伯對曰:刺虛則實之者,針下熱也,氣實乃熱也。滿而洩之者,針下寒也,氣虛乃寒也。菀陳則除之者,出惡血也。邪勝則虛之者,出針勿按;徐而疾則實者,徐出針而疾按之;疾而徐則虛者,疾出針而徐按之;言實與虛者,寒溫氣多少也。若無若有者,疾不可知也。察後與先者,知病先後也。為虛與實者,工勿失其法。若得若失者,離其法也。虛實之要,九針最妙者,為其各有所宜也。補寫之時者,與氣開闔相合也。九針之名,各不同形者,針窮其所當補寫也。
  刺實須其虛者,留針陰氣隆至,乃去針也;刺虛須其實者,陽氣隆至,針下熱乃去針也。經氣已至,慎守勿失者,勿變更也。深淺在志者,知病之內外也;近遠如一者,深淺其候等也。如臨深淵者,不敢墮也。手如握虎者,欲其壯也。神無營於眾物者,靜志觀病人,無左右視也;義無邪下者,欲端以正也;必正其神者,欲瞻病人目制其神,令氣易行也。所謂三里者,下膝三寸也;所謂跗之者,舉膝分易見也;巨虛者,蹻足(骨行)獨陷者;下廉者,陷下者也。
  帝曰:余聞九針,上應天地四時陰陽,願聞其方,令可傳於後世以為常也。岐伯曰:夫一天、二地、三人、四時、五音、六律、七星、八風、九野,身形亦應之,針各有所宜,故曰九針。人皮應天,人肉應地,人脈應人,人筋應時,人聲應音,人陰陽合氣應律,人齒面目應星,人出入氣應風,人九竅三百六十五絡應野,故一針皮,二針肉,三針脈,四針筋,五針骨,六針調陰陽,七針益精,八針除風,九針通九竅,除三百六十五節氣,此之謂各有所主也。人心意應八風,人氣應天,人發齒耳目五聲應五音六律,人陰陽脈血氣應地,人肝目應之九。九竅三百六十五。人一以觀動靜天二以候五色七星應之,以候發毋澤五音一,以候宮商角徵羽六律有餘,不足應之二地一,以候高下有餘九野一節俞應之,以候閉節,三人變一分人,候齒洩多血少十分角之變,五分以候緩急,六分不足三分寒關節第九,分四時人寒溫燥濕四時,一應之以候相反,一四方各作解。


\section{長刺節論篇第五十五}

  刺家不診,聽病者言,在頭,頭疾痛,為藏針之,刺至骨,病已上,無傷骨肉及皮,皮者道也。
  陰刺,入一傍四處。治寒熱。深專者,刺大藏,迫藏刺背,背俞也。刺之迫藏,藏會,腹中寒熱去而止。與刺之要,髮針而淺出血。
  治腐腫者刺腐上,視癰小大深淺刺,刺大者多血,小者深之,必端內針為故止。
  病在少腹有積,刺皮(骨盾)以下,至少腹而止;刺俠脊兩傍四椎間,刺兩髂季脅肋間,導腹中氣熱下已。
  病在少腹,腹痛不得大小便,病名曰疝,得之寒;刺少腹兩股間,刺腰髁骨間,刺而多之,盡炅病已。
  病在筋,筋攣節痛,不可以行,名曰筋痹。刺筋上為故,刺分肉間,不可中骨也;病起筋炅,病已止。
  病在肌膚,肌膚盡痛,名曰肌痹,傷於寒濕。刺大分、小分,多髮針而深之,以熱為故;無傷筋骨,傷筋骨,癰發若變;諸分盡熱,病已止。
  病在骨,骨重不可舉,骨髓痠痛,寒氣至,名曰骨痹。深者刺,無傷脈肉為故,其道大分小分,骨熱病已止。
  病在諸陽脈,且寒且熱,諸分且寒且熱,名曰狂。刺之虛脈,視分盡熱,病已止。
  病初發,歲一發,不治月一發,不治,月四五發,名曰癲病。刺諸分諸脈,其無寒者以針調之,病已止。
  病風且寒且熱,炅汗出,一日數過,先刺諸分理絡脈;汗出且寒且熱,三日一刺,百日而已。
  病大風,骨節重,鬚眉墮,名曰大風,刺肌肉為故,汗出百日,刺骨髓,汗出百日,凡二百日,鬚眉生而止針。


\section{皮部論篇第五十六}

  黃帝問曰:余聞皮有分部,脈有經紀,筋有結絡,骨有度量。其所生病各異,別其分部,左右上下,陰陽所在,病之始終,願聞其道。
  岐伯對曰:欲知皮部以經脈為紀者,諸經皆然。陽明之陽,名曰害蜚,上下同法。視其部中有浮絡者,皆陽明之絡也。其色多青則痛,多黑則痹,黃赤則熱,多白則寒,五色皆見,則寒熱也。絡盛則入客於經,陽主外,陰主內。
  少陽之陽,名曰樞持,上下同法。視其部中有浮絡者,皆少陽之絡也,絡盛則入客於經,故在陽者主內,在陰者主出,以滲於內,諸經皆然。
  太陽之陽,名曰關樞,上下同法。視其部中有浮絡者,皆太陽之絡也。絡盛則入客於經。
  少陰之陰,名曰樞儒,上下同法。視其部中有浮絡者,皆少陰之絡也。絡盛則入客於經,其入經也,從陽部注於經;其出者,從陰內注於骨。
  心主之陰,名曰害肩,上下同法。視其部中有浮絡者,皆心主之絡也。絡盛則入客於經。
  太陰之陰,名曰關蟄,上下同法。視其部中有浮絡者,皆太陰之絡也。絡盛則入客於經。凡十二經絡脈者,皮之部也。
  是故百病之始生也,必先於皮毛,邪中之則腠理開,開則入客於絡脈,留而不去,傳入於經,留而不去,傳入於府,廩於腸胃。邪之始入於皮毛也,晰然起毫毛,開腠理;其入於絡也,則絡脈盛色變;其入客於經也,則感虛乃陷下。其留於筋骨之間,寒多則筋攣骨痛,熱多則筋弛骨消,肉爍(月囷)破,毛直而敗。
  帝曰:夫子言皮之十二部,其生病皆何如?岐伯曰:皮者脈之部也,邪客於皮則腠理開,開則邪入客於絡脈,絡脈滿則注於經脈,經脈滿則入舍於府藏也,故皮者有分部,不與而生大病也。


\section{經絡論篇第五十七}

  黃帝問曰:夫絡脈之見也,其五色各異,青黃赤白黑不同,其故何也?岐伯對曰:經有常色而絡無常變也。
  帝曰:經之常色何如?岐伯曰:心赤,肺白、肝青、脾黃、腎黑,皆亦應其經脈之色也。
  帝曰:絡之陰陽,亦應其經乎?岐伯曰:陰絡之色應其經,陽絡之色變無常,隨四時而行也。寒多則凝泣,凝泣則青黑;熱多則淖澤,淖澤則黃赤;此皆常色,謂之無病,五色具見者,謂之寒熱。帝曰:善。


\section{氣穴論篇第五十八}

  黃帝問曰:余聞氣穴三百六十五,以應一歲,未知其所,願卒聞之。岐伯稽首再拜對曰:窘乎哉問也!其非聖帝,孰能窮其道焉!因請溢意盡言其處。帝捧手逡巡而卻曰:夫子之開余道也,目未見其處,耳未聞其數,而目以明,耳以聰矣。岐伯曰:此所謂聖人易語,良馬易御也。帝曰:余非聖人之易語也,世言真數開人意,今余所訪問者真數,發蒙解惑,未足以論也。然余願聞夫子溢志盡言其處,令解其意,請藏之金匱,不敢復出。
  岐伯再拜而起曰:臣請言之,背與心相控而痛,所治天突與十椎及上紀,上紀者,胃脘也,下紀者,關元也。背胸邪系陰陽左右,如此其病前後痛濇,胸脅痛而不得息,不得臥,上氣短氣偏痛,脈滿起,斜出尻脈,絡胸脅支心貫鬲,上肩加天突,斜下肩交十椎下。
  藏俞五十穴,府俞七十二穴,熱俞五十九穴,水俞五十七穴,頭上五行行五,五五二十五穴,中兩傍各五,凡十穴,大椎上兩傍各一,凡二穴,目瞳子浮白二穴,兩髀厭分中二穴,犢鼻二穴,耳中多所聞二穴,眉本二穴,完骨二穴,頂中央一穴,枕骨二穴,上關二穴,大迎二穴,下關二穴,天柱二穴,巨虛上下廉四穴,曲牙二穴,天突一穴,天府二穴,天牖二穴,扶突二穴,天窗二穴,肩解二穴,關元一穴,委陽二穴,肩貞二穴,瘖門一穴,齊一穴,胸俞十二穴,背俞二穴,膺俞十二穴,分肉二穴,踝上橫二穴,陰陽蹻四穴,水俞在諸分,熱俞在氣穴,寒熱俞在兩骸厭中二穴,大禁二十五,在天府下五寸,凡三百六十五穴,針之所由行也。
  帝曰:余已知氣穴之處,游針之居,願聞孫絡谿谷,亦有所應乎?岐伯曰:孫絡三百六十五穴會,亦以應一歲,以溢奇邪,以通榮衛,榮衛稽留,衛散榮溢,氣竭血著,外為發熱,內為少氣,疾寫無怠,以通榮衛,見而寫之,無問所會。
  帝曰:善。願聞谿谷之會也。岐伯曰:肉之大會為谷,肉之小會為谿,肉分之間,谿谷之會,以行榮衛,以會大氣。邪溢氣壅,脈熱肉敗榮衛不行,必將為膿,內銷骨髓,外破大膕,留於節湊,必將為敗。積寒留舍,榮衛不居,卷肉縮筋,肋肘不得伸,內為骨痹,外為不仁,命曰不足,大寒留於谿谷也。谿谷三百六十五穴會,亦應一歲,其小痹淫溢,循脈往來,微針所及,與法相同。
  帝乃辟左右而起,再拜曰:今日發蒙解惑,藏之金匱,不敢復出,乃藏之金蘭之室,署曰氣穴所在。岐伯曰:孫絡之脈別經者,其血盛而當寫者,亦三百六十五脈,並注於絡,傳注十二絡脈,非獨十四絡脈也,內解寫於中者十脈。

\section{氣府論篇第五十九}

  足太陽脈氣所發者七十八穴:兩眉頭各一,入發至項三寸半,傍五,相去三寸,其浮氣在皮中者凡五行,行五,五五二十五,項中大筋兩傍各一,風府兩傍各一,俠背以下至尻尾二十一節,十五間各一,五藏之俞各五,六府之俞各六,委中以下至足小指傍各六俞。
  足少陽脈氣所發者六十二穴:兩角上各二,直目上髮際內各五,耳前角上各一,耳前角下各一,銳發下各一,客主人各一,耳後陷中各一,下關各一,耳下牙車之後各一,缺盆各一,掖下三寸,脅下至胠,八間各一,髀樞中傍各一,膝以下至足小指次指各六俞。
  足陽明脈氣所發者六十八穴:額顱髮際傍各三,面鼽骨空各一,大迎之骨空各一,人迎各一,缺盆外骨空各一,膺中骨間各一,俠鳩尾之外,當乳下三寸,俠胃脘各五,俠齊廣三寸各三,下齊二寸俠之各三。氣街動脈各一,伏菟上各一,三里以下至足中指各八俞,分之所在穴空。
  手太陽脈氣所發者三十六穴:目內眥各一,目外眥各一,鼽骨下各一,耳郭上各一,耳中各一,巨骨穴各一,曲掖上骨穴各一,柱骨上陷者各一,上天窗四寸各一,肩解各一,肩解下三寸各一,肘以下至手小指本各六俞。
  手陽明脈氣所發者二十二穴:鼻空外廉、項上各二,大迎骨空各一,柱骨之會各一,髃骨之會各一,肘以下至手大指次指本各六俞。
  手少陽脈氣所發者三十二穴:鼽骨下各一,眉後各一,角上各一,下完骨後各一,項中足太陽之前各一,俠扶突各一,肩貞各一,肩貞下三寸分間各一,肘以下至手小指次指本各六俞。
  督脈氣所發者二十八穴:項中央二,髮際後中八,面中三,大椎以下至尻尾及傍十五穴,至骶下凡二十一節,脊椎法也。
  任脈之氣所發者二十八穴:喉中央二,膺中骨陷中各一,鳩尾下三寸,胃脘五寸,胃脘以下至橫骨六寸半一,腹脈法也。下陰別一,目下各一,下唇一,齗交一。
  衝脈氣所發者二十二穴:俠鳩尾外各半寸至齊寸一,俠齊下傍各五分至橫骨寸一,腹脈法也。
  足少陰舌下,厥陰毛中急脈各一,手少陰各一,陰陽蹻各一,手足諸魚際脈氣所發者,凡三百六十五穴也。


\section{骨空論篇第六十}

  黃帝問曰:余聞風者百病之始也,以針治之奈何?岐伯對曰:風從外入,令人振寒,汗出頭痛,身重惡寒,治在風府,調其陰陽,不足則補,有餘則寫。
  大風頸項痛,刺風府,風府在上椎。大風汗出,灸譩譆,譩譆在背下俠脊傍三寸所,厭之,令病者呼譩譆,譩譆應手。
  從風憎風,刺眉頭。失枕,在肩上橫骨間。折,使榆臂,齊肘正,灸脊中。
  絡季脅引少腹而痛脹,刺譩譆。
  腰痛不可以轉搖,急引陰卵,刺八髎與痛上,八髎在腰尻分間。
  鼠瘻,寒熱,還刺寒府,寒府在附膝外解營。取膝上外者使之拜,取足心者使之跪。
  任脈者,起於中極之下,以上毛際,循腹裡上關元,至咽喉,上頤循面入目。衝脈者,起於氣街,並少陰之經,俠齊上行,至胸中而散。任脈為病,男子內結七疝,女子帶下瘕聚。衝脈為病,逆氣裡急。
  督脈為病,脊強反折。督脈者,起於少腹以下骨中央,女子入系廷孔,其孔,溺孔之端也。其絡循陰器合篡間,繞篡後,別繞臀,至少陰與巨陽中絡者合,少陰上股內後廉,貫脊屬腎,與太陽起於目內眥,上額交巔,上入絡腦,還出別下項,循肩髆,內俠脊抵腰中,入循膂絡腎。其男子循莖下至篡,與女子等。其少腹直上者,貫齊中央,上貫心入喉,上頤環唇,上繫兩目之下中央。此生病,從少腹上衝心而痛,不得前後,為沖疝;其女子不孕,癃痔遺溺嗌干。督脈生病治督脈,治在骨上,甚者在齊下營。
  其上氣有音者,治其喉中央,在缺盆中者,其病上衝喉者治其漸,漸者,上俠頤也。
  蹇,膝伸不屈,治其楗。坐而膝痛,治其機。立而暑解,治其骸關。膝痛,痛及拇指治其膕。坐而膝痛如物隱者,治其關。膝痛不可屈伸,治其背內。連(骨行)若折,治陽明中俞髎。若別,治巨陽少陰滎。淫濼脛痠,不能久立,治少陽之維,在外上五寸。
  輔骨上,橫骨下為楗,俠髖為機,膝解為骸關,俠膝之骨為連骸,骸下為輔,輔上為膕,膕上為關,頭橫骨為枕。
  水俞五十七穴者,尻上五行,行五;伏菟上兩行,行五,左右各一行,行五;踝上各一行,行六穴,髓空在腦後三分,在顱際銳骨之下,一在齗基下,一在項後中復骨下,一在脊骨上空在風府上。脊骨下空,在尻骨下空。數髓空在面俠鼻,或骨空在口下當兩肩。兩髆骨空,在髆中之陽。臂骨空在臂陽,去踝四寸兩骨空之間。股骨上空在股陽,出上膝四寸。(骨行)骨空在輔骨之上端,股際骨空在毛中動下。尻骨空在髀骨之後,相去四寸。扁骨有滲理湊,無髓孔,易髓無孔。
  灸寒熱之法,先灸項大椎,以年為壯數,次灸橛骨,以年為壯數,視背俞陷者灸之,舉臂肩上陷者灸之,兩季脅之間灸之,外踝上絕骨之端灸之,足小指次指間灸之,腨下陷脈灸之,外踝後灸之,缺盆骨上切之堅痛如筋者灸之,膺中陷骨間灸之,掌束骨下灸之,齊下關元三寸灸之,毛際動脈灸之,膝下三寸分間灸之,足陽明跗上動脈灸之,巔上一灸之。犬所齧之處灸之三壯,即以犬傷病法灸之。凡當灸二十九處,傷食灸之,不已者,必視其經之過於陽者,數刺其俞而藥之。


\section{水熱穴論篇第六十一}

  黃帝問曰:少陰何以主腎?腎何以主水?岐伯對曰:腎者,至陰也,至陰者,盛水也。肺者,太陰也,少陰者,冬脈也,故其本在腎,其末在肺,皆積水也。
  帝曰:腎何以能聚水而生病?岐伯曰:腎者,胃之關也,關門不利,故聚水而從其類也。上下溢於皮膚,故為胕腫,胕腫者,聚水而生病也。
  帝曰:諸水皆生於腎乎?岐伯曰:腎者,牝藏也,地氣上者屬於腎,而生水液也,故曰至陰。勇而勞甚則腎汗出,腎汗出逢於風,內不得入於藏府,外不得越於皮膚,客於玄府,行於皮裡,傳為胕腫,本之於腎,名曰風水。所謂玄府者,汗空也。
  帝曰:水俞五十七處者,是何主也?岐伯曰:腎俞五十七穴,積陰之所聚也,水所從出入也。尻上五行行五者,此腎俞,故水病下為胕腫大腹,上為喘呼,不得臥者,標本俱病,故肺為喘呼,腎為水腫,肺為逆不得臥,分為相輸俱受者,水氣之所留也。伏菟上各二行行五者,此腎之街也,三陰之所交結於腳也。踝上各一行行六者,此腎脈之下行也,名曰太沖。凡五十七穴者,皆藏之陰絡,水之所客也。
  帝曰:春取絡脈分肉,何也?岐伯曰:春者木始治,肝氣始生,肝氣急,其風疾,經脈常深,其氣少,不能深入,故取絡脈分肉間。
  帝曰:夏取盛經分腠,何也?岐伯曰:夏者火始治,心氣始長,脈瘦氣弱,陽氣留溢,熱熏分腠,內至於經,故取盛經分腠,絕膚而病去者,邪居淺也。所謂盛經者,陽脈也。
  帝曰:秋取經俞,何也?岐伯曰:秋者金始治,肺將收殺,金將勝火,陽氣在合,陰氣初勝,濕氣及體,陰氣未盛,未能深入,故取俞以寫陰邪,取合以虛陽邪,陽氣始衰,故取於合。
  帝曰:冬取井榮,何也?岐伯曰:冬者水始治,腎方閉,陽氣衰少,陰氣堅盛,巨陽伏沉,陽脈乃去,故取井以下陰逆,取榮以陽氣。故曰:冬取井榮,春不鼽衄,此之謂也。
  帝曰:夫子言治熱病五十九俞,余論其意,未能領別其處,願聞其處,因聞其意。岐伯曰:頭上五行行五者,以越諸陽之熱逆也;大杼、膺俞、缺盆、背俞,此八者,以寫胸中之熱也;氣街、三里、巨虛上下廉,此八者,以寫胃中之熱也;雲門、髃骨、委中、髓空,此八者,以寫四支之熱也;五藏俞傍五,此十者,以寫五藏之熱也。凡此五十九穴者,皆熱之左右也。
  帝曰:人傷於寒而傳為熱,何也?岐伯曰:夫寒盛,則生熱也。

\section{調經論篇第六十二}

  黃帝問曰:余聞刺法言,有餘寫之,不足補之,何謂有餘?何謂不足?岐伯對曰:有餘有五,不足亦有五,帝欲何問?帝曰:願盡聞之。岐伯曰:神有餘有不足,氣有餘有不足,血有餘有不足,形有餘有不足,志有餘有不足,凡此十者,其氣不等也。
  帝曰:人有精氣津液,四支、九竅、五藏十六部、三百六十五節,乃生百病,百病之生,皆有虛實。今夫子乃言有餘有五,不足亦有五,何以生之乎?岐伯曰:皆生於五藏也。夫心藏神,肺藏氣,肝藏血,脾藏肉,腎藏志,而此成形。志意通,內連骨髓,而成身形五藏。五藏之道,皆出於經隧,以行血氣,血氣不和,百病乃變化而生,是故守經隧焉。
  帝曰:神有餘不足何如?岐伯曰:神有餘則笑不休,神不足則悲。血氣未並,五藏安定,邪客於形,灑淅起於毫毛,未入於經絡也,故命曰神之微。帝曰:補寫奈何?岐伯曰:神有餘,則寫其小絡之血,出血勿之深斥,無中其大經,神氣乃平。神不足者,視其虛絡,按而致之,刺而利之,無出其血,無洩其氣,以通其經,神氣乃平。帝曰:刺微奈何?岐伯曰:按摩勿釋,著針勿斥,移氣於不足,神氣乃得復。
  帝曰:善。有餘不足奈何?岐伯曰:氣有餘則喘咳上氣,不足則息利少氣。血氣未並,五藏安定,皮膚微病,命曰白氣微洩。帝曰:補寫奈何?岐伯曰:氣有餘,則寫其經隧,無傷其經,無出其血,無洩其氣。不足,則補其經隧,無出其氣。帝曰:刺微奈何?岐伯曰:按摩勿釋,出針視之,曰我將深之,適人必革,精氣自伏,邪氣散亂,無所休息,氣洩腠理,真氣乃相得。
  帝曰:善。血有餘不足奈何?岐伯曰:血有餘則怒,不足則恐。血氣未並,五藏安定,孫絡水溢,則經有留血。帝曰:補寫奈何?岐伯曰:血有餘,則寫其盛經出其血。不足,則視其虛經內針其脈中,久留而視;脈大,疾出其針,無令血洩。帝曰:刺留血,奈何?岐伯曰:視其血絡,刺出其血,無令惡血得入於經,以成其疾。
  帝曰:善。形有餘不足奈何?岐伯曰:形有餘則腹脹、涇溲不利,不足則四支不用。血氣未並,五藏安定,肌肉蠕動,命曰微風。帝曰:補寫奈何?岐伯曰:形有餘則寫其陽經,不足則補其陽絡。帝曰:刺微奈何?岐伯曰:取分肉間,無中其經,無傷其絡,衛氣得復,邪氣乃索。
  帝曰:善。志有餘不足奈何?岐伯曰:志有餘則腹脹飧洩,不足則厥。血氣未並,五藏安定,骨節有動。帝曰:補寫奈何?岐伯曰:志有餘則寫然筋血者,不足則補其復溜。帝曰:刺未並奈何?岐伯曰:即取之,無中其經,邪所乃能立虛。
  帝曰:善。余已聞虛之形,不知其何以生!岐伯曰:氣血以並,陰陽相頃,氣亂於衛,血逆於經,血氣離居,一實一虛。血並於陰,氣並於陽,故為驚狂;血並於陽,氣並於陰,乃為炅中;血並於上,氣並於下,心煩惋善怒;血並於下,氣並於上,亂而喜忘。帝曰:血並於陰,氣並於陽,如是血氣離居,何者為實?何者為虛?岐伯曰:血氣者,喜溫而惡寒,寒則泣不能流,溫則消而去之,是故氣之所並為血虛,血之所並為氣虛。
  帝曰:人之所有者,血與氣耳。今夫子乃言血並為虛,氣並為虛,是無實乎?岐伯曰:有者為實,無者為虛,故氣並則無血,血並則無氣,今血與氣相失,故為虛焉。絡之與孫脈俱輸於經,血與氣並,則為實焉。血之與氣並走於上,則為大厥,厥則暴死,氣復反則生,不反則死。
  帝曰:實者何道從來?虛者何道從去?虛實之要,願聞其故。岐伯曰:夫陰與陽,皆有俞會,陽注於陰,陰滿之外,陰陽勻平,以充其形,九候若一,命曰平人。夫邪之生也,或生於陰,或生於陽。其生於陽者,得之風雨寒暑;其生於陰者,得之飲食居處,陰陽喜怒。
  帝曰:風雨之傷人奈何?岐伯曰:風雨之傷人也,先客於皮膚,傳入於孫脈,孫脈滿則傳入於絡脈,絡脈滿則輸於大經脈,血氣與邪並客於分腠之間,其脈堅大,故曰實。實者外堅充滿,不可按之,按之則痛。帝曰:寒濕之傷人奈何?岐伯曰:寒濕之中人也,皮膚不收,肌肉堅緊,榮血泣,衛氣去,故曰虛。虛者聶辟,氣不足,按之則氣足以溫之,故快然而不痛。
  帝曰:善。陰之生實奈何?岐伯曰:喜怒不節,則陰氣上逆,上逆則下虛,下虛則陽氣走之,故曰實矣。帝曰:陰之生虛奈何?岐伯曰:喜則氣下,悲則氣消,消則脈虛空,因寒飲食,寒氣熏滿,則血泣氣去,故曰虛矣。
  帝曰:經言陽虛則外寒,陰虛則內熱,陽盛則外熱,陰盛則內寒,余已聞之矣,不知其所由然也。岐伯曰:陽受氣於上焦,以溫皮膚分肉之間。令寒氣在外,則上焦不通,上焦不通,則寒氣獨留於外,故寒慄。帝曰:陰虛生內熱奈何?岐伯曰:有所勞倦,形氣衰少,谷氣不盛,上焦不行,下脘不通,胃氣熱,熱氣熏胸中,故內熱。帝曰:陽盛生外熱奈何?岐伯曰:上焦不通利,則皮膚緻密,腠理閉塞,玄府不通,衛氣不得洩越,故外熱。帝曰:陰盛生內寒奈何?岐伯曰:厥氣上逆,寒氣積於胸中而不寫,不寫則溫氣去,寒獨留,則血凝泣,凝則脈不通,其脈盛大以濇,故中寒。
  帝曰:陰與陽並,血氣以並,病形以成,刺之奈何?岐伯曰:刺此者,取之經隧,取血於營,取氣於衛,用形哉,因四時多少高下。帝曰:血氣以並,病形以成,陰陽相頃,補寫奈何?岐伯曰:寫實者氣盛乃內針,針與氣俱內,以開其門,如利其戶;針與氣俱出,精氣不傷,邪氣乃下,外門不閉,以出其疾;搖大其道,如利其路,是謂大寫,必切而出,大氣乃屈。帝曰:補虛奈何?岐伯曰:持針勿置,以定其意,候呼內針,氣出針入,針空四塞,精無從去,方實而疾出針,氣入針出,熱不得還,閉塞其門,邪氣布散,精氣乃得存,動氣候時,近氣不失,遠氣乃來,是謂追之。
  帝曰:夫子言虛實者有十,生於五藏,五藏五脈耳。夫十二經脈皆生其病,今夫子獨言五藏,夫十二經脈者,皆絡三百六十五節,節有病必被經脈,經脈之病,皆有虛實,何以合之?岐伯曰:五藏者,故得六府與為表裡,經絡支節,各生虛實,其病所居,隨而調之。病在脈,調之血;病在血,調之絡;病在氣,調之衛;病在肉,調之分肉;病在筋,調之筋;病在骨,調之骨;燔針劫刺其下及與急者;病在骨,焠針藥熨;病不知所痛,兩蹻為上;身形有痛,九候莫病,則繆刺之;痛在於左而右脈病者,巨刺之。必謹察其九候,針道備矣。


\section{繆刺論篇第六十三}

  黃帝問曰:余聞繆刺,未得其意,何謂繆刺?岐伯對曰:夫邪之客於形也,必先舍於皮毛,留而不去,入舍於孫脈,留而不去,入舍於絡脈,留而不去,入舍於經脈,內連五藏,散於腸胃,陰陽俱感,五藏乃傷,此邪之從皮毛而入,極於五藏之次也,如此則治其經焉。今邪客於皮毛,入舍於孫絡,留而不去,閉塞不通,不得入於經,流溢於大絡,而生奇病也。夫邪客大絡者,左注右,右注左,上下左右,與經相干,而佈於四末,其氣無常處,不入於經俞,命曰繆刺。
  帝曰:願聞繆刺,以左取右以右取左,奈何?其與巨刺何以別之?岐伯曰:邪客於經,左盛則右病,右盛則左病,亦有移易者,左痛未已而右脈先病,如此者,必巨刺之,必中其經,非絡脈也。故絡病者,其痛與經脈繆處,故命曰繆刺。
  帝曰:願聞繆刺奈何?取之何如?岐伯曰:邪客於足少陰之絡,令人卒心痛,暴脹,胸脅支滿,無積者,刺然骨之前出血,如食頃而已。不已,左取右,右取左。病新發者,取五日,已。
  邪客於手少陽之絡,令人喉痹舌卷,口乾心煩,臂外廉痛,手不及頭,刺手中指次指爪甲上,去端如韭葉各一痏,壯者立已,老者有頃已,左取右,右取左,此新病數日已。
  邪客於足厥陰之絡,令人卒疝暴痛,刺足大指爪甲上,與肉交者各一痏,男子立已,女子有頃已,左取右,右取左。
  邪客於足太陽之絡,令人頭項肩痛,刺足小指爪甲上,與肉交者各一痏,立已,不已,刺外踝下三痏,左取右,右取左,如食頃已。
  邪客於手陽明之絡,令人氣滿胸中,喘息而支胠,胸中熱,刺手大指、次指爪甲上,去端如韭葉各一痏,左取右,右取左,如食頃已。
  邪客於臂掌之間,不可得屈,刺其踝後,先以指按之痛,乃刺之,以月死生為數,月生一日一痏,二日二痏,十五日十五痏,十六日十四痏。
  邪客於足陽蹻之脈,令人目痛從內眥始,刺外踝之下半寸所各二痏,左刺右,右刺左,如行十里頃而已。
  人有所墮墜,惡血留內,腹中滿脹,不得前後,先飲利藥,此上傷厥陰之脈,下傷少陰之絡,刺足內踝之下,然骨之前,血脈出血,刺足跗上動脈,不已,刺三毛上各一痏,見血立已,左刺右,右刺左。善悲驚不樂,刺如右方。
  邪客於手陽明之絡,令人耳聾,時不聞音,刺手大指次指爪甲上,去端如韭葉各一痏,立聞,不已,刺中指爪甲上與肉交者,立聞,其不時聞者,不可刺也。耳中生風者,亦刺之如此數,左刺右,右刺左。
  凡痹往來行無常處者,在分肉間痛而刺之,以月死生為數,用針者隨氣盛衰,以為痏數,針過其日數則脫氣,不及日數則氣不寫,左刺右,右刺左,病已,止,不已,復刺之如法,月生一日一痏,二日二痏,漸多之;十五日十五痏,十六日十四,漸少之。
  邪客於足陽明之經,令人鼽衄上齒寒,足中指次指爪甲上,與肉交者各一痏,左刺右,右刺左。
  邪客於足少陽之絡,令人脅痛不得息,咳而汗出,刺足小指次指爪甲上,與肉交者各一痏,不得息立已,汗出立止,咳者溫衣飲食,一日已。左刺右,右刺左,病立已,不已,復刺如法。
  邪客於足少陰之絡,令人嗌痛,不可內食,無故善怒,氣上走賁上,刺足下中央之脈各三痏,凡六刺,立已,左刺右,右刺左。嗌中腫,不能內唾,時不能出唾者,刺然骨之前,出血立已,左刺右,右刺左。
  邪客於足太陰之絡,令人腰痛,引少腹控(月少),不可以仰息,刺腰尻之解,兩胂之上,是腰俞,以月死生為痏數,髮針立已,左刺右,右刺左。
  邪客於足太陽之絡,令人拘攣背急,引脅而痛,刺之從項始,數脊椎俠脊,疾按之應手如痛,刺之傍三痏,立已。
  邪客於足少陽之絡,令人留於樞中痛,髀不可舉,刺樞中以毫針,寒則久留針,以月死生為數,立已。
  治諸經刺之,所過者不病,則繆刺之。
  耳聾,刺手陽明,不已,刺其通脈出耳前者。
  齒齲,刺手陽明,不已,刺其脈入齒中,立已。
  邪客於五藏之間,其病也,脈引而痛,時來時止,視其病,繆刺之於手足爪甲上,視其脈,出其血,間日一刺,一刺不已,五刺已。
  繆傳引上齒,齒唇寒痛,視其手背脈血者去之,足陽明中指爪甲上一痏,手大指次指爪甲上各一痏,立已,左取右,右取左。
  邪客於手足少陰太陰足陽明之絡,此五絡,皆會於耳中,上絡左角,五絡俱竭,令人身脈皆動,而形無知也,其狀若屍,或曰屍厥,刺其足大指內側爪甲上,去端如韭葉,後刺足心,後刺足中指爪甲上各一痏,後刺手大指內側,去端如韭葉,後刺手心主,少陰銳骨之端各一痏,立已。不已,以竹管吹其兩耳,鬄其左角之發方一寸,燔治,飲以美酒一杯,不能飲者灌之,立已。
  凡刺之數,先視其經脈,切而從之,審其虛而調之,不調者經刺之,有痛而經不病者繆刺之,因視其皮部有血絡者盡取之,此繆刺之數也。


\section{四時刺逆從論篇第六十四}

  厥陰有餘,病陰痹;不足病生熱痹;滑則病狐疝風;濇則病少腹積氣。
  少陰有餘,病皮痹隱軫;不足病肺痹;滑則病肺風疝;濇則病積溲血。
  太陰有餘,病肉痹寒中;不足病脾痹;滑則病脾風疝;濇則病積心腹時滿。
  陽明有餘,病脈痹,身時熱;不足病心痹;滑則病心風疝;濇則病積時善驚 。
  太陽有餘,病骨痹身重;不足病腎痹;滑則病腎風疝;濇則病積時善巔疾。
  少陽有餘,病筋痹脅滿;不足病肝痹;滑則病肝風疝;濇則病積時筋急目痛。
  是故春氣在經脈,夏氣在孫絡,長夏氣在肌肉,秋氣在皮膚,冬氣在骨髓中。帝曰:余願聞其故。岐伯曰:春者,天氣始開,地氣始洩,凍解冰釋,水行經通,故人氣在脈。夏者,經滿氣溢,入孫絡受血,皮膚充實。長夏者,經絡皆盛,內溢肌中。秋者,天氣始收,腠理閉塞,皮膚引急。冬者蓋藏,血氣在中,內著骨髓,通於五藏。是故邪氣者,常隨四時之氣血而入客也,至其變化不可為度,然必從其經氣,辟除其邪,除其邪,則亂氣不生。
  帝曰:逆四時而生亂氣奈何?岐伯曰:春刺絡脈,血氣外溢,令人少氣;春刺肌肉,血氣環逆,令人上氣;春刺筋骨,血氣內著,令人腹脹。夏刺經脈,血氣乃竭,令人解(亻亦);夏刺肌肉,血氣內卻,令人善恐;夏刺筋骨,血氣上逆,令人善怒。秋刺經脈,血氣上逆,令人善忘;秋刺絡脈,氣不外行,令人臥不欲動;秋刺筋骨,血氣內散,令人寒慄。冬刺經脈,血氣皆脫,令人目不明;冬刺絡脈,內氣外洩,留為大痹;冬刺肌肉,陽氣竭絕,令人善忘。凡此四時刺者,大逆之病,不可不從也,反之,則生亂氣相淫病焉。故刺不知四時之經,病之所生,以從為逆,正氣內亂,與精相薄。必審九候,正氣不亂,精氣不轉。
  帝曰:善。刺五藏,中心一日死,其動為噫;中肝五日死,其動為語;中肺三日死,其動為咳;中腎六日死,其動為嚏欠;中脾十日死,其動為吞。刺傷人五藏必死,其動則依其藏之所變候知其死也。


\section{標本病傳論篇第六十五}

  黃帝問曰:病有標本,刺有逆從,奈何?岐伯對曰:凡刺之方,必別陰陽,前後相應,逆從得施,標本相移。故曰:有其在標而求之於標,有其在本而求之於本,有其在本而求之於標,有其在標而求之於本,故治有取標而得者,有取本而得者,有逆取而得者,有從取而得者。故知逆與從,正行無問,知標本者,萬舉萬當,不知標本,是謂妄行。
  夫陰陽逆從,標本之為道也,小而大,言一而知百病之害。少而多,淺而博,可以言一而知百也。以淺而知深,察近而知遠,言標與本,易而勿及。治反為逆,治得為從。先病而後逆者治其本,先逆而後病者治其本,先寒而後生病者治其本,先病而後生寒者治其本,先熱而後生病者治其本,先熱而後生中滿者治其標,先病而後洩者治其本,先洩而後生他病者治其本,必且調之,乃治其他病,先病而後生中滿者治其標,先中滿而後煩心者治其本。人有客氣,有同氣。小大不利治其標,小大利治其本。病發而有餘,本而標之,先治其本,後治其標;病發而不足,標而本之,先治其標,後治其本。謹察間甚,以意調之,間者並行,甚者獨行。先小大不利而後生病者治其本。
  夫病傳者,心病先心痛,一日而咳,三日脅支痛,五日閉塞不通,身痛體重;三日不已,死。冬夜半,夏日中。
  肺病喘咳,三日而脅支滿痛,一日身重體痛,五日而脹,十日不已,死。冬日入,夏日出。
  肝病頭目眩脅支滿,三日體重身痛,五日而脹,三日腰脊少腹痛脛,三日不已,死。冬日入,夏早食。
  脾病身痛體重,一日而脹,二日少腹腰脊痛脛酸,三日背(月呂)筋痛,小便閉,十日不已,死。冬人定,夏晏食。
  腎病少腹腰脊痛,(骨行)酸,三日背(月呂)筋痛,小便閉;三日腹脹;三日兩脅支痛,三日不已,死。冬大晨,夏晏晡。
  胃病脹滿,五日少腹腰脊痛,(骨行)酸;三日背(月呂)筋痛,小便閉;五日身體重;六日不已,死。冬夜半後,夏日昳。
  膀胱病小便閉,五日少腹脹,腰脊痛,(骨行)酸;一日腹脹;一日身體痛;二日不已,死。冬雞鳴,夏下晡。
  諸病以次是相傳,如是者,皆有死期,不可刺。間一藏止,及至三四藏者,乃可刺也。


\section{天元紀大論篇第六十六}

  黃帝問曰:天有五行,御五位,以生寒暑燥濕風;人有五藏,化五氣,以生喜怒思憂恐。論言五運相襲而皆治之,終期之日,週而復始,余已知之矣,願聞其與三陰三陽之候,奈何合之?
  鬼臾區稽首再拜對曰:昭乎哉問也。夫五運陰陽者,天地之道也,萬物之綱紀,變化之父母,生殺之本始,神明之府也,可不通乎!故物生謂之化,物極謂之變,陰陽不測謂之神,神用無方謂之聖。夫變化之為用也,在天為玄,在人為道,在地為化,化生五味,道生智,玄生神。神在天為風,在地為木;在天為熱,在地為火;在天為濕,在地為土;在天為燥,在地為金;在天為寒,在地為水;故在天為氣,在地成形,形氣相感而化生萬物矣。然天地者,萬物之上下也;左右者,陰陽之道路也;水火者,陰陽之徵兆也;金木者,生成之終始也。氣有多少,形有盛衰,上下相召,而損益彰矣。
  帝曰:願聞五運之主時也何如?鬼臾區曰:五氣運行,各終期日,非獨主時也。帝曰:請聞其所謂也。鬼臾區曰:臣積考《太始天元冊》文曰:太虛寥廓,肇基化元,萬物資始,五運終天,布氣真靈,總統坤元,九星懸朗,七曜周旋,曰陰曰陽,曰柔曰剛,幽顯既位,寒暑弛張,生生化化,品物咸章。臣斯十世,此之謂也。
  帝曰:善。何謂氣有多少,形有盛衰?鬼臾區曰:陰陽之氣各有多少,故曰三陰三陽也。形有盛衰,謂五行之治,各有太過不及也。故其始也,有餘而往,不足隨之,不足而往,有餘從之,知迎知隨,氣可與期。應天為天符,承歲為歲直,三合為治。
  帝曰:上下相召奈何?鬼臾區曰:寒暑燥濕風火,天之陰陽也,三陰三陽上奉之。木火土金水火,地之陰陽也,生長化收藏下應之。天以陽生陰長,地以陽殺陰藏。天有陰陽,地亦有陰陽。木火土金水火,地之陰陽也,生長化收藏。故陽中有陰,陰中有陽。所以欲知天地之陰陽者,應天之氣,動而不息,故五歲而右遷,應地之氣,靜而守位,故六期而環會,動靜相召,上下相臨,陰陽相錯,而變由生也。
  帝曰:上下週紀,其有數乎?鬼臾區曰:天以六為節,地以五為制,周天氣者,六期為一備;終地紀者,五歲為一週。君火以明,相火以位,五六相合而七百二十氣為一紀,凡三十歲;千四百四十氣,凡六十歲而為一週,不及太過,斯皆見矣。
  帝曰:夫子之言,上終天氣,下畢地紀,可謂悉矣。余願聞而藏之,上以治民,下以治身,使百姓昭著,上下和親,德澤下流,子孫無憂,傳之後世,無有終時,可得聞乎?鬼臾區曰:至數之機,迫迮以微,其來可見,其往可追,敬之者昌,慢之者亡。無道行私,必得天殃,謹奉天道,請言真要。
  帝曰:善言始者,必會於終,善言近者,必知其遠,是則至數極而道不惑,所謂明矣,願夫子推而次之,令有條理,簡而不匱,久而不絕,易用難忘,為之綱紀,至數之要,願盡聞之。鬼臾區曰:昭乎哉問!明乎哉道!如鼓之應桴,響之應聲也。臣聞之:甲己之歲,土運統之;乙庚之歲,金運統之;丙辛之歲,水運統之;丁壬之歲,木運統之;戊癸之歲,火運統之。
  帝曰:其於三陰三陽,合之奈何?鬼臾區曰:子午之歲,上見少陰;丑未之歲,上見太陰;寅申之歲,上見少陽;卯酉之歲,上見陽明;辰戌之歲,上見太陽;巳亥之歲,上見厥陰。少陰,所謂標也,厥陰,所謂終也。厥陰之上,風氣主之;少陰之上,熱氣主之;太陰之上,濕氣主之;少陽之上,相火主之;陽明之上,燥氣主之;太陽之上,寒氣主之。所謂本也,是謂六元。帝曰:光乎哉道!明乎哉論!請著之玉版,藏之金匱,署曰《天元紀》。


\section{五運行大論篇第六十七}

  黃帝坐明堂,始正天綱,臨觀八極,考建五常,請天師而問之曰:論言天地之動靜,神明為之紀;陰陽之升降,寒暑彰其兆。余聞五運之數於夫子,夫子之所言,正五氣之各主歲爾,首甲定運,余因論之。鬼臾區曰:土主甲己,金主乙庚,水主丙辛,木主丁壬,火主戊癸。子午之上,少陰主之;丑未之上,太陰主之;寅申之上,少陽主之;卯酉之上,陽明主之;辰戌之上,太陽主之;巳亥之上,厥陰主之。不合陰陽,其故何也?
  岐伯曰:是明道也,此天地之陰陽也。夫數之可數者,人中之陰陽也,然所合,數之可得者也。夫陰陽者,數之可十,推之可百,數之可千,推之可萬。天地陰陽者,不以數推,以象之謂也。
  帝曰:願聞其所始也。岐伯曰:昭乎哉問也!臣覽《太始天元冊》文,丹天之氣,經於牛女戊分;黅天之氣,經於心尾已分;蒼天之氣,經於危室柳鬼;素天之氣,經於亢氐昴畢;玄天之氣,經於張翼婁胃。所謂戊己分者,奎璧角軫,則天地之門戶也。夫候之所始,道之所生,不可不通也。
  帝曰:善。論言天地者,萬物之上下,左右者,陰陽之道路,未知其所謂也。岐伯曰:所謂上下者,歲上下見陰陽之所在也。左右者,諸上見厥陰,左少陰,右太陽;見少陰,左太陰,右厥陰;見太陰,左少陽,右少陰;見少陽,左陽明,右太陰;見陽明,左太陽,右少陽;見太陽,左厥陰,右陽明。所謂面北而命其位,言其見也。
  帝曰:何謂下?岐伯曰:厥陰在上,則少陽在下,左陽明右太陰。少陰在上則陽明在下,左太陽右少陽。太陰在上則太陽在下,左厥陰右陽明。少陽在上則厥陰在下,左少陰右太陽。陽明在上則少陰在下,左太陰右厥陰。太陽在上則太陰在下,左少陽右少陰。所謂面南而命其位,言其見也。上下相遘,寒暑相臨,氣相得則和,不相得則疾。帝曰:氣相得而病者,何也?岐伯曰:以下臨上,不當位也。帝曰:動靜何如?岐伯曰:上者右行,下者左行,左右周天,余而復會也。帝曰:余聞鬼臾區曰,應地者靜。今夫子乃言下者左行,不知其所謂也,願聞何以生之乎?岐伯曰:天地動靜,五行遷復,雖鬼臾區其上候而巳,猶不能遍明。夫變化之用,天垂象,地成形,七曜緯虛,五行麗地。地者,所以載生成之形類也。虛者,所以列應天之精氣也。形精之動,猶根本之與枝葉也,仰觀其象,雖遠可知也。
  帝曰:地之為下,否乎?岐伯曰:地為人之下,太虛之中者也。帝曰:馮乎?岐伯曰:大氣舉之也。燥以干之,暑以蒸之,風以動之,濕以潤之,寒以堅之,火以溫之。故風寒在下,燥熱在上,濕氣在中,火遊行其間,寒暑六入,故令虛而生化也。故燥勝則地幹,暑勝則地熱,風勝則地動,濕勝則地泥,寒勝則地裂,火勝則地固矣。
  帝曰:天地之氣,何以候之?岐伯曰:天地之氣,勝復之作,不形於診也。《脈法》曰:天地之變,無以脈診,此之謂也。
  帝曰:間氣何如?岐伯曰:隨氣所在,期於左右。帝曰:期之奈何?岐伯曰:從其氣則和,違其氣則病,不當其位者病,迭移其位者病,失守其位者危,尺寸反者死,陰陽交者死。先立其年,以知其氣,左右應見,然後乃可以言死生之逆順也。
  帝曰:寒暑燥濕風火,在人合之奈何?其於萬物何以生化?岐伯曰:東方生風,風生木,木生酸,酸生肝,肝生筋,筋生心。其在天為玄,在人為道,在地為化。化生五味,道生智,玄生神,化生氣。神在天為風,在地為木,在體為筋,在氣為柔,在藏為肝。其性為暄,其德為和,其用為動,其色為蒼,其化為榮,其蟲毛,其政為散,其令宣發,其變摧拉,其眚為隕,其味為酸,其志為怒。怒傷肝,悲勝怒;風傷肝,燥勝風;酸傷筋,辛勝酸。
  南方生熱,熱生火,火生苦,苦生心,心生血,血生脾。其在天為熱,在地為火,在體為脈,在氣為息,在藏為心。其性為暑,其德為顯,其用為躁,其色為赤,其化為茂,其蟲羽,其政為明,其令郁蒸,其變炎爍,其眚燔焫,其味為苦,其志為喜。喜傷心,恐勝喜;熱傷氣,寒勝熱;苦傷氣,咸勝苦。
  中央生濕,濕生土,土生甘,甘生脾,脾生肉,肉生肺。其在天為濕,在地為土,在體為肉,在氣為充,在藏為脾。其性靜兼,其德為濡,其用為化,其色為黃,其化為盈,其蟲裸,其政為謐,其令雲雨,其變動注,其眚淫潰,其味為甘,其志為思。思傷脾,怒勝思;濕傷肉,風勝濕;甘傷脾,酸勝甘。
  西方生燥,燥生金,金生辛,辛生肺,肺生皮毛,皮毛生腎。其在天為燥,在地為金,在體為皮毛,在氣為成,在藏為肺。其性為涼,其德為清,其用為固,其色為白,其化為斂,其蟲介,其政為勁,其令霧露,其變肅殺,其眚蒼落,其味為辛,其志為憂。憂傷肺,喜勝憂;,熱傷皮毛,寒勝熱;辛傷皮毛,苦勝辛。
  北方生寒,寒生水,水生咸,咸生腎,腎生骨髓,髓生肝。其在天為寒,在地為水,在體為骨,在氣為堅,在藏為腎。其性為凜,其德為寒,其用為藏,其色為黑,其化為肅,其蟲鱗,其政為靜,其令霰雪,其變凝冽,其眚冰雹,其味為咸,其志為恐。恐傷腎,思勝恐;寒傷血,燥勝寒;咸傷血,甘勝咸。正氣更立,各有所先,非其位則邪,當其位則正。
  帝曰:病生之變何如?岐伯曰:氣相得則微,不相得則甚。帝曰:主歲何如?岐伯曰:氣有餘,則制己所勝而侮所不勝;其不及,則己所不勝侮而乘之,己所勝輕而侮之。侮反受邪。侮而受邪,寡於畏也。帝曰:善。


\section{六微旨大論篇第六十八}

  黃帝問曰:嗚呼!遠哉,天之道也,如迎浮雲,若視深淵,視深淵尚可測,迎浮雲莫知其極。夫子數言謹奉天道,余聞而藏之,心私異之,不知其所謂也。願夫子溢志盡言其事,令終不滅,久而不絕,天之道可得聞乎?岐伯稽首再拜對曰:明乎哉問,天之道也!此因天之序,盛衰之時也。
  帝曰:願聞天道六六之節盛衰何也?岐伯曰:上下有位,左右有紀。故少陽之右,陽明治之;陽明之右,太陽治之;太陽之右,厥陰治之;厥陰之右,少陰治之;少陰之右,太陰治之;太陰之右,少陽治之。此所謂氣之標,蓋南面而待之也。故曰:因天之序,盛衰之時,移光定位,正立而待之,此之謂也。
  少陽之上,火氣治之,中見厥陰;陽明之上,燥氣治之,中見太陰;太陽之上,寒氣治之,中見少陰;厥陰之上,風氣治之,中見少陽;少陰之上,熱氣治之,中見太陽;太陰之上,濕氣治之,中見陽明。所謂本也,本之下,中之見也,見之下,氣之標也。本標不同,氣應異象。
  帝曰:其有至而至,有至而不至,有至而太過,何也?岐伯曰:至而至者和;至而不至,來氣不及也;未至而至,來氣有餘也。帝曰:至而不至,未至而至如何?岐伯曰:應則順,否則逆,逆則變生,變則病。帝曰:善。請言其應。岐伯曰:物,生其應也。氣,脈其應也。
  帝曰:善。願聞地理之應六節氣位何如?岐伯曰:顯明之右,君火之位也;君火之右,退行一步,相火治之;復行一步,土氣治之;復行一步,金氣治之;復行一步,水氣治之;復行一步,木氣治之;復行一步,君火治之。
  相火之下,水氣承之;水位之下,土氣承之;土位之下,風氣承之;風位之下,金氣承之;金位之下,火氣承之;君火之下,陰精承之。帝曰:何也?岐伯曰:亢則害,承乃制,制則生化,外列盛衰,害則敗亂,生化大病。
  帝曰:盛衰何如?岐伯曰:非其位則邪,當其位則正,邪則變甚,正則微。帝曰:何謂當位?岐伯曰:木運臨卯,火運臨午,土運臨四季,金運臨酉,水運臨子,所謂歲會,氣之平也。帝曰:非位何如?岐伯曰:歲不與會也。
  帝曰:土運之歲,上見太陰;火運之歲,上見少陽少陰;金運之歲,上見陽明;木運之歲,上見厥陰;水運之歲,上見太陽,奈何?岐伯曰:天之與會也。故《天元冊》曰天符。
  天符歲會何如?岐伯曰:太一天符之會也。
  帝曰:其貴賤何如?岐伯曰:天符為執法,歲位為行令,太一天符為貴人。帝曰:邪之中也奈何?岐伯曰:中執法者,其病速而危;中行令者,其病徐而持;中貴人者,其病暴而死。帝曰:位之易也何如?岐伯曰:君位臣則順,臣位君則逆,逆則其病近,其害速;順則其病遠,其害微。所謂二火也。
  帝曰:善。願聞其步何如?岐伯曰:所謂步者,六十度而有奇,故二十四步積盈百刻而成日也。
  帝曰:六氣應五行之變何如?岐伯曰:位有終始,氣有初中,上下不同,求之亦異也。帝曰:求之奈何?岐伯曰:天氣始於甲,地氣始於子,子甲相合,命曰歲立,謹候其時,氣可與期。
  帝曰:願聞其歲,六氣始終,早晏何如?岐伯曰:明乎哉問也!甲子之歲,初之氣,天數始於水下一刻,終於八十七刻半;二之氣始於八十七刻六分,終於七十五刻;三之氣,始於七十六刻,終於六十二刻半;四之氣,始於六十二刻六分,終於五十刻;五之氣,始於五十一刻,終於三十七刻半;六之氣,始於三十七刻六分,終於二十五刻。所謂初六,天之數也。
  乙丑歲,初之氣,天數始於二十六刻,終於一十二刻半;二之氣,始於一十二刻六分,終於水下百刻;三之氣,始於一刻,終於八十七刻半;四之氣,始於八十七刻六分,終於七十五刻;五之氣,始於七十六刻,終於六十二刻半;六之氣,始於六十二刻六分,終於五十刻。所謂六二,天之數也。
  丙寅歲,初之氣,天數始於五十一刻,終於三十七刻半;二之氣,始於三十七刻六分,終於二十五刻;三之氣,始於二十六刻,終於一十二刻半;四之氣,始於一十二刻六分,終於水下百刻;五之氣,始於一刻,終於八十七刻半;六之氣,始於八十七刻六分,終於七十五刻。所謂六三,天之數也。
  丁卯歲,初之氣,天數始於七十六刻,終於六十二刻半;二之氣,始於六十二刻六分,終於五十刻;三之氣,始於五十一刻,終於三十七刻半;四之氣,始於三十七刻六分,終於二十五刻;五之氣,始於二十六刻,終於一十二刻半;六之氣,始於一十二刻六分,終於水下百刻。所謂六四,天之數也。次戊辰歲,初之氣復始於一刻,常如是無已,週而復始。
  帝曰:願聞其歲候何如?岐伯曰:悉乎哉問也!日行一週,天氣始於一刻,日行再周,天氣始於二十六刻,日行三週,天氣始於五十一刻,日行四周,天氣始於七十六刻,日行五週,天氣復始於一刻,所謂一紀也。是故寅午戌歲氣會同,卯未亥歲氣會同,辰申子歲氣會同,巳酉丑歲氣會同,終而復始。
  帝曰:願聞其用也。岐伯曰:言天者求之本,言地者求之位,言人者求之氣交。帝曰:何謂氣交?岐伯曰:上下之位,氣交之中,人之居也。故曰:天樞之上,天氣主之;天樞之下,地氣主之;氣交之分,人氣從之,萬物由之,此之謂也。
  帝曰:何謂初中?岐伯曰:初凡三十度而有奇,中氣同法。帝曰:初中何也?岐伯曰:所以分天地也。帝曰:願卒聞之。岐伯曰:初者地氣也,中者天氣也。
  帝曰:其升降何如?岐伯曰:氣之升降,天地之更用也。帝曰:願聞其用何如?岐伯曰:升已而降,降者謂天;降已而升,升者謂地。天氣下降,氣流於地;地氣上升,氣騰於天。故高下相召,升降相因,而變作矣。
  帝曰:善。寒濕相遘,燥熱相臨,風火相值,其有聞乎?岐伯曰:氣有勝復,勝復之作,有德有化,有用有變,變則邪氣居之。帝曰:何謂邪乎?岐伯曰:夫物之生從於化,物之極由乎變,變化之相薄,成敗之所由也。故氣有往復,用有遲速,四者之有,而化而變,風之來也。帝曰:遲速往復,風所由生,而化而變,故因盛衰之變耳。成敗倚伏游乎中,何也?岐伯曰:成敗倚伏生乎動,動而不已,則變作矣。
  帝曰:有期乎?岐伯曰:不生不化,靜之期也。帝曰:不生化乎?岐伯曰:出入廢則神機化滅,升降息則氣立孤危。故非出入,則無以生長壯老已;非升降,則無以生長化收藏。是以升降出入,無器不有。故器者生化之宇,器散則分之,生化息矣。故無不出入,無不升降,化有小大,期有近遠,四者之有而貴常守,反常則災害至矣。故曰無形無患,此之謂也。帝曰:善。有不生不化乎?岐伯曰:悉乎哉問也!與道合同,惟真人也。帝曰:善。

\section{氣交變大論篇第六十九}

  黃帝問曰:五運更治,上應天期,陰陽往復,寒暑迎隨,真邪相薄,內外分離,六經波蕩,五氣頃移,太過不及,專勝兼併,願言其始,而有常名,可得聞乎?岐伯稽首再拜對曰:昭乎哉問也!是明道也。此上帝所貴,先師傳之,臣雖不敏,往聞其旨。帝曰:余聞得其人不教,是謂失道,傳非其人,慢洩天寶。余誠菲德,未足以受至道,然而眾子哀其不終,願夫子保於無窮,流於無極,余司其事,則而行之奈何?岐伯曰:請遂言之也。《上經》曰:夫道者上知天文,下知地理,中知人事,可以長久,此之謂也。帝曰:何謂也?岐伯曰:本氣位也,位天者,天文也;位地者,地理也;通於人氣之變化者,人事也。故太過者先天,不及者後天,所謂治化而人應之也。
  帝曰:五運之化,太過何如?岐伯曰:歲木太過,風氣流行,脾土受邪。民病飧洩,食減,體重,煩冤,腸鳴腹支滿,上應歲星。甚則忽忽善怒,眩冒巔疾。化氣不政,生氣獨治,雲物飛動,草木不寧,甚而搖落,反脅痛而吐甚,沖陽絕者死不治,上應太白星。
  歲火太過,炎暑流行,金肺受邪。民病瘧,少氣咳喘,血溢血洩注下,嗌燥耳聾,中熱肩背熱,上應熒惑星。甚則胸中痛,脅支滿脅痛,膺背肩胛間痛,兩臂內痛,身熱骨痛而為浸淫。收氣不行,長氣獨明,雨水霜寒,上應辰星。上臨少陰少陽,火燔焫,冰泉涸,物焦槁,病反譫妄狂越,咳喘息鳴,下甚血溢洩不已,太淵絕者死不治,上應熒惑星。
  歲土太過,雨濕流行,腎水受邪。民病腹痛,清厥意不樂,體重煩冤,上應鎮星。甚則肌肉萎,足痿不收,行善瘈,腳下痛,飲發中滿食減,四支不舉。變生得位,藏氣伏,化氣獨治之,泉湧河衍,涸澤生魚,風雨大至,土崩潰,鱗見於陸,病腹滿溏洩腸鳴,反下甚而太谿絕者,死不治,上應歲星。
  歲金太過,燥氣流行,肝木受邪。民病兩脅下少腹痛,目赤痛眥瘍,耳無所聞。肅殺而甚,則體重煩冤,胸痛引背,兩脅滿且痛引少腹,上應太白星。甚則喘咳逆氣,肩背痛,尻陰股膝髀腨(骨行)足皆病,上應熒惑星。收氣峻,生氣下,草木斂,蒼干凋隕,病反暴痛,脅不可反側,咳逆甚而血溢,太沖絕者,死不治,上應太白星。
  歲水太過,寒氣流行,邪害心火。民病身熱煩心,躁悸,陰厥上下中寒,譫妄心痛,寒氣早至,上應辰星。甚則腹大脛腫,喘咳,寢汗出憎風,大雨至,埃霧朦郁,上應鎮星。上臨太陽,雨冰雪,霜不時降,濕氣變物,病反腹滿腸鳴溏洩,食不化,渴而妄冒,神門絕者,死不治,上應熒惑辰星。
  帝曰:善。其不及何如?岐伯曰:悉乎哉問也!歲木不及,燥乃大行,生氣失應,草木晚榮,肅殺而甚,則剛木辟著,悉萎蒼干,上應太白星,民病中清,胠脅痛,少腹痛,腸鳴溏洩,涼雨時至,上應太白星,其谷蒼。上臨陽明,生氣失政,草木再榮,化氣乃急,上應太白鎮星,其主蒼早。復則炎暑流火,濕性燥,柔脆草木焦槁,下體再生,華實齊化,病寒熱瘡瘍疿胗癰痤,上應熒惑太白,其谷白堅。白露早降,收殺氣行,寒雨害物,蟲食甘黃,脾土受邪,赤氣後化,心氣晚治,上勝肺金,白氣乃屈,其谷不成,咳而鼽,上應熒惑太白星。
  歲火不及,寒乃大行,長政不用,物榮而下,凝慘而甚,則陽氣不化,乃折榮美,上應辰星,民病胸中痛,脅支滿,兩脅痛,膺背肩胛間及兩臂內痛,郁冒朦昧,心痛暴瘖,胸腹大,脅下與腰背相引而痛,甚則屈不能伸,髖髀如別,上應熒惑辰星,其谷丹。復則埃郁,大雨且至,黑氣乃辱,病溏腹滿,食飲不下,寒中腸鳴,洩注腹痛,暴攣痿痹,足不任身,上應鎮星辰星,玄谷不成。
  歲土不及,風乃大行,化氣不令,草木茂榮,飄揚而甚,秀而不實,上應歲星,民病飧洩霍亂,體重腹痛,筋骨繇復,肌肉瞤酸,善怒,藏氣舉事,蟄蟲早附,咸病寒中,上應歲星鎮星,其谷黅。復則收政嚴峻,名木蒼凋,胸脅暴痛,下引少腹,善太息,蟲食甘黃,氣客於脾,黅谷乃減,民食少失味,蒼谷乃損,上應太白歲星。上臨厥陰,流水不冰,蟄蟲來見,藏氣不用,白乃不復,上應歲星,民乃康。
  歲金不及,炎火乃行,生氣乃用,長氣專勝,庶物以茂,燥爍以行,上應熒惑星,民病肩背瞀重,鼽嚏血便注下,收氣乃後,上應太白星,其谷堅芒。復則寒雨暴至,乃零冰雹霜雪殺物,陰厥且格,陽反上行,頭腦戶痛,延及囟頂發熱,上應辰星,丹谷不成,民病口瘡,甚則心痛。
  歲水不及,濕乃大行,長氣反用,其化乃速,暑雨數至,上應鎮星,民病腹滿身重,濡洩寒瘍流水,腰股痛發,膕腨股膝不便,煩冤,足痿,清厥,腳下痛,甚則跗腫,藏氣不政,腎氣不衡,上應辰星,其谷秬。上臨太陰,則大寒數舉,蟄蟲早藏,地積堅冰,陽光不治,民病寒疾於下,甚則腹滿浮腫,上應鎮星,其主黅谷。復則大風暴發,草偃木零,生長不鮮,面色時變,筋骨並辟,肉瞤瘛,目視(目巟)(目巟),物疏璺,肌肉胗發,氣並鬲中,痛於心腹,黃氣乃損,其谷不登,上應歲星。
  帝曰:善。願聞其時也。岐伯曰:悉乎哉問也!木不及,春有鳴條律暢之化,則秋有霧露清涼之政。春有慘淒殘賊之勝,則夏有炎暑燔爍之復。其眚東,其藏肝,其病內舍胠脅,外在關節。
  火不及,夏有炳明光顯之化,則冬有嚴肅霜寒之政。夏有慘淒凝冽之勝,則不時有埃昏大雨之復。其眚南,其藏心,其病內舍膺脅,外在經絡。
  土不及,四維有埃雲潤澤之化,則春有鳴條鼓拆之政。四維發振拉飄騰之變,則秋有肅殺霖霪之復。其眚四維,其藏脾,其病內舍心腹,外在肌肉四支。
  金不及,夏有光顯郁蒸之令,則冬有嚴凝整肅之應。夏有炎爍燔燎之變,則秋有冰雹霜雪之復。其眚西,其藏肺,其病內舍膺脅肩背,外在皮毛。
  水不及,四維有湍潤埃雲之化,則不時有和風生發之應。四維發埃驟注之變,則不時有飄蕩振拉之復。其眚北,其藏腎,其病內舍腰脊骨髓,外在谿谷腨膝。夫五運之政,猶權衡也,高者抑之,下者舉之,化者應之,變者復之,此生長化成收藏之理,氣之常也,失常則天地四塞矣。故曰:天地之動靜,神明為之紀,陰陽之往復,寒暑彰其兆,此之謂也。
  帝曰:夫子之言五氣之變,四時之應,可謂悉矣。夫氣之動亂,觸遇而作,發無常會,卒然災合,何以期之?岐伯曰:夫氣之動變,固不常在,而德化政令災變,不同其候也。帝曰:何謂也?岐伯曰:東方生風,風生木,其德敷和,其化生榮,其政舒啟,其令風,其變振發,其災散落。南方生熱,熱生火,其德彰顯,其化蕃茂,其政明曜,其令熱,其變銷爍,其災燔焫。中央生濕,濕生土,其德溽蒸,其化豐備,其政安靜,其令濕,其變驟注,其災霖潰。西方生燥,燥生金,其德清潔,其化緊斂,其政勁切,其令燥,其變肅殺,其災蒼隕。北方生寒,寒生水,其德淒滄,其化清謐,其政凝肅,其令寒,其變凓冽,其災冰雪霜雹。是以察其動也,有德有化,有政有令,有變有災,而物由之,而人應之也。
  帝曰:夫子之言歲候,不及其太過,而上應五星。今夫德化政令,災眚變易,非常而有也,卒然而動,其亦為之變乎?岐伯曰:承天而行之,故無妄動,無不應也。卒然而動者,氣之交變也,其不應焉。故曰:應常不應卒,此之謂也。帝曰:其應奈何?岐伯曰:各從其氣化也。
  帝曰:其行之徐疾逆順何如?岐伯曰:以道留久,逆守而小,是謂省下;以道而去,去而速來,曲而過之,是謂省遺過也;久留而環,或離或附,是謂議災與其德也;應近則小,應遠則大。芒而大倍常之一,其化甚;大常之二,其眚即也;小常之一,其化減;小常之二,是謂臨視,省下之過與其德也。德者福之,過者伐之。是以象之見也,高而遠則小,下而近則大,故大則喜怒邇,小則禍福遠。歲運太過,則運星北越,運氣相得,則各行以道。故歲運太過,畏星失色而兼其母,不及則色兼其所不勝。肖者瞿瞿,莫知其妙,閔閔之當,孰者為良,妄行無徵,是畏候王。
  帝曰:其災應何如?岐伯曰:亦各從其化也。故時至有盛衰,凌犯有逆順,留守有多少,形見有善惡,宿屬有勝負,徵應有吉凶矣。
  帝曰:其善惡,何謂也?岐伯曰:有喜有怒,有憂有喪,有澤有燥,此象之常也,必謹察之。帝曰:六者高下異乎?岐伯曰:象見高下,其應一也,故人亦應之。
  帝曰:善。其德化政令之動靜損益皆何如?岐伯曰:夫德化政令災變,不能相加也。勝復盛衰,不能相多也。往來小大,不能相過也。用之升降,不能相無也。各從其動而復之耳。
  帝曰:其病生何如?岐伯曰:德化者氣之祥,政令者氣之章,變易者復之紀,災眚者傷之始,氣相勝者和,不相勝者病,重感於邪則甚也。
  帝曰:善。所謂精光之論,大聖之業,宣明大道,通於無窮,究於無極也。余聞之,善言天者,必應於人,善言古者,必驗於今,善言氣者,必彰於物,善言應者,同天地之化,善言化言變者,通神明之理,非夫子孰能言至道歟!乃擇良兆而藏之靈室,每旦讀之,命曰《氣交變》,非齋戒不敢發,慎傳也。


\section{五常政大論篇第七十}

  黃帝問曰:太虛寥廓,五運迴薄,衰盛不同,損益相從,願聞平氣何如而名?何如而紀也?岐伯對曰:昭乎哉問也!木曰敷和,火曰升明,土曰備化,金曰審平,水曰靜順。
  帝曰:其不及奈何?岐伯曰:木曰委和,火曰伏明,土曰卑監,金曰從革,水曰涸流。帝曰:太過何謂?岐伯曰:木曰發生,火曰赫曦,土曰敦阜,金曰堅成,水曰流衍。
  帝曰:三氣之紀,願聞其候。岐伯曰:悉乎哉問也!敷和之紀,木德周行,陽舒陰布,五化宣平,其氣端,其性隨,其用曲直,其化生榮,其類草木,其政發散,其候溫和,其令風,其藏肝,肝其畏清,其主目,其谷麻,其果李,其實核,其應春,其蟲毛,其畜犬,其色蒼,其養筋,其病裡急支滿,其味酸,其音角,其物中堅,其數八。
  升明之紀,正陽而治,德施周普,五化均衡,其氣高,其性速,其用燔灼,其化蕃茂,其類火,其政明曜,其候炎暑,其令熱,其藏心,心其畏寒,其主舌,其穀麥,其果杏,其實絡,其應夏,其蟲羽,其畜馬,其色赤,其養血,其病瞤瘛,其味苦,其音徵,其物脈,其數七。
  備化之紀,氣協天休,德流四政,五化齊修,其氣平,其性順,其用高下,其化豐滿,其類土,其政安靜,其候溽蒸,其令濕,其藏脾,脾其畏風,其主口,其谷稷,其果棗,其實肉,其應長夏,其蟲裸,其畜牛,其色黃,其養肉,其病否,其味甘,其音宮,其物膚,其數五。
  審平之紀,收而不爭,殺而無犯,五化宣明,其氣潔,其性剛,其用散落,其化堅斂,其類金,其政勁肅,其候清切,其令燥,其藏肺,肺其畏熱,其主鼻,其穀稻,其果桃,其實殼,其應秋,其蟲介,其畜雞,其色白,其養皮毛,其病咳,其味辛,其音商,其物外堅,其數九。
  靜順之紀,藏而勿害,治而善下,五化咸整,其氣明,其性下,其用沃衍,其化凝堅,其類水,其政流演,其候凝肅,其令寒,其藏腎,腎其畏濕,其主二陰,其谷豆,其果栗,其實濡,其應冬,其蟲鱗,其畜彘,其色黑,其養骨髓,其病厥,其味咸,其音羽,其物濡,其數六。
  故生而勿殺,長而勿罰,化而勿制,收而勿害,藏而勿抑,是謂平氣。
  委和之紀,是謂勝生。生氣不政,化氣乃揚,長氣自平,收令乃早。涼雨時降,風雲並興,草木晚榮,蒼干凋落,物秀而實,膚肉內充。其氣斂,其用聚,其動緛戾拘緩,其發驚駭,其藏肝,其果棗李,其實核殼,其谷稷稻,其味酸辛,其色白蒼,其畜犬雞,其蟲毛介,其主霧露淒滄,其聲角商。其病搖動注恐,從金化也,少角與判商同,上角與正角同,上商與正商同;其病支廢腫瘡瘍,其甘蟲,邪傷肝也,上宮與正宮同。蕭飋肅殺,則炎赫沸騰,眚於三,所謂復也。其主飛蠹蛆雉,乃為雷霆。
  伏明之紀,是謂勝長。長氣不宣,藏氣反布,收氣自政,化令乃衡,寒清數舉,暑令乃薄。承化物生,生而不長,成實而稚,遇化已老,陽氣屈伏,蟄蟲早藏。其氣鬱,其用暴,其動彰伏變易,其發痛,其藏心,其果栗桃,其實絡濡,其谷豆稻,其味苦咸,其色玄丹,其畜馬彘,其蟲羽鱗,其主冰雪霜寒,其聲徵羽。其病昏惑悲忘,從水化也,少徵與少羽同,上商與正商同,邪傷心也。凝慘凜冽,則暴雨霖霪,眚於九,其主驟注雷霆震驚,沉(蕓去草頭令)淫雨。
  卑監之紀,是謂減化。化氣不令,生政獨彰,長氣整,雨乃愆,收氣平,風寒並興,草木榮美,秀而不實,成而秕也。其氣散,其用靜定,其動瘍湧分潰癰腫。其發濡滯,其藏脾,其果李栗,其實濡核,其谷豆麻,其味酸甘,其色蒼黃,其畜牛犬,其蟲裸毛,其主飄怒振發,其聲宮角,其病留滿否塞,從木化也,少宮與少角同,上宮與正宮同,上角與正角同,其病飧洩,邪傷脾也。振拉飄揚,則蒼干散落,其眚四維,其主敗折虎狼,清氣乃用,生政乃辱。
  從革之紀,是謂折收。收氣乃後,生氣乃揚,長化合德,火政乃宣,庶類以蕃。其氣揚,其用躁切,其動鏗禁瞀厥,其發咳喘,其藏肺,其果李杏,其實殼絡,其谷麻麥,其味苦辛,其色白丹,其畜雞羊,其蟲介羽,其主明曜炎爍,其聲商徵,其病嚏咳鼽衄,從火化也,少商與少徵同,上商與正商同,上角與正角同,邪傷肺也。炎光赫烈,則冰雪霜雹,眚於七,其主鱗伏彘鼠,歲氣早至,乃生大寒。
  涸流之紀,是謂反陽,藏令不舉,化氣乃昌,長氣宣佈,蟄蟲不藏,土潤水泉減,草木條茂,榮秀滿盛。其氣滯,其用滲洩,其動堅止,其發燥槁,其藏腎,其果棗杏,其實濡肉,其谷黍稷,其味甘咸,其色黅玄,甚畜彘牛,其蟲鱗裸,其主埃郁昏翳,其聲羽宮,其病痿厥堅下,從土化也,少羽與少宮同,上宮與正宮同,其病癃閟,邪傷腎也,埃昏驟雨,則振拉摧拔,眚於一,其主毛顯狐貉,變化不藏。
  故乘危而行,不速而至,暴虐無德,災反及之,微者復微,甚者復甚,氣之常也。
  發生之紀,是謂啟陳,土疏洩,蒼氣達,陽和布化,陰氣乃隨,生氣淳化,萬物以榮。其化生,其氣美,其政散,其令條舒,其動掉眩巔疾,其德鳴靡啟坼,其變振拉摧拔,其谷麻稻,其畜雞犬,其果李桃,其色青黃白,其味酸甘辛,其象春,其經足厥陰少陽,其藏肝脾,其蟲毛介,其物中堅外堅,其病怒,太角與上商同,上徵則其氣逆,其病吐利。不務其德,則收氣復,秋氣勁切,甚則肅殺,清氣大至,草木凋零,邪乃傷肝。
  赫曦之紀,是謂蕃茂,陰氣內化,陽氣外榮,炎暑施化,物得以昌。其化長,其氣高,其政動,其令鳴顯,其動炎灼妄擾,其德暄暑郁蒸,其變炎烈沸騰,其穀麥豆,其畜羊彘,其果杏栗,其色赤白玄,其味苦辛咸,其象夏,其經手少陰太陽,手厥陰少陽,其藏心肺,其蟲羽鱗,其物脈濡,其病笑瘧瘡瘍血流狂妄目赤,上羽與正徵同,其收齊,其病痓,上徵而收氣後也。暴烈其政,藏氣乃復,時見凝慘,甚則雨水霜雹切寒,邪傷心也。
  敦阜之紀,是謂廣化,厚德清靜,順長以盈,至陰內實,物化充成,煙埃朦郁,見於厚土,大雨時行,濕氣乃用,燥政乃辟,其化員,其氣豐,其政靜,其令周備,其動濡積並稸,其德柔潤重淖,其變震驚飄驟崩潰,其谷稷麻,其畜牛犬,其果棗李,其色黅玄蒼,其味甘咸酸,其象長夏,其經足太陰陽明,其藏脾腎,其蟲裸毛,其物肌核,其病腹滿,四支不舉,大風迅至,邪傷脾也。
  堅成之紀,謂收引,天氣潔,地氣明,陽氣隨,陰治化,燥行其政,物以司成,收氣繁布,化洽不終。其化成,其氣削,其政肅,其令銳切,其動暴折瘍疰,其德霧露蕭飋,其變肅殺凋零,其穀稻黍,其畜雞馬,其果桃杏,其色白青丹,其味辛酸苦,其象秋,其經手太陰陽明,其藏肺肝,其蟲介羽,其物殼絡,其病喘喝,胸憑仰息。上徵與正商同,其生齊,其病咳,政暴變,則名木不營,柔脆焦首,長氣斯救,大火流,炎爍且至,蔓將槁,邪傷肺也。
  流衍之紀,是謂封藏,寒司物化,天地嚴凝,藏政以布,長令不揚。其化凜,其氣堅,其政謐,其令流注,其動漂洩沃湧,其德凝慘寒雰,其變冰雪霜雹,其谷豆稷,其畜彘牛,其果栗棗,其色黑丹黅,其味咸苦甘,其象冬,其經足少陰太陽,其藏腎心,其蟲鱗裸,其物濡滿,其病脹,上羽而長氣不化也。政過則化氣大舉,而埃昏氣交,大雨時降,邪傷腎也。故曰:不恆其德,則所勝來復,政恆其理,則所勝同化,此之謂也。
  帝曰:天不足西北,左寒而右涼;地不滿東南,右熱而左溫,其故何也?岐伯曰:陰陽之氣,高下之理,太少之異也。東南方,陽也,陽者其精降於下,故右熱而左溫。西北方,陰也,陰者其精奉於上,故左寒而右涼。是以地有高下,氣有溫涼,高者氣寒,下者氣熱。故適寒涼者脹之,之溫熱者瘡,下之則脹已,汗之則瘡已,此湊理開閉之常,太少之異耳。
  帝曰:其於壽夭何如?岐伯曰:陰精所奉其人壽,陽精所降其人夭。帝曰:善。其病也,治之奈何?岐伯曰:西北之氣散而寒之,東南之氣收而溫之,所謂同病異治也。故曰:氣寒氣涼,治以寒涼,行水漬之。氣溫氣熱,治以溫熱,強其內守。必同其氣,可使平也,假者反之。
  帝曰:善。一州之氣生化壽夭不同,其故何也?岐伯曰:高下之理,地勢使然也。崇高則陰氣治之,污下則陽氣治之,陽勝者先天,陰勝者後天,此地理之常,生化之道也。帝曰:其有壽夭乎?岐伯曰:高者其氣壽,下者其氣夭,地之小大異也,小者小異,大者大異。故治病者,必明天道地理,陰陽更勝,氣之先後,人之壽夭,生化之期,乃可以知人之形氣矣。
  帝曰:善。其歲有不病,而藏氣不應不用者,何也?岐伯曰:天氣制之,氣有所從也。帝曰:願卒聞之。岐伯曰:少陽司天,火氣下臨,肺氣上從,白起金用,草木眚,火見燔焫,革金且耗,大暑以行,咳嚏鼽衄鼻窒,曰瘍,寒熱胕腫。風行於地,塵沙飛揚,心痛胃脘痛,厥逆鬲不通,其主暴速。
  陽明司天,燥氣下臨,肝氣上從,蒼起木用而立,土乃眚,淒滄數至,木伐草萎,脅痛目赤,掉振鼓慄,筋痿不能久立。暴熱至,土乃暑,陽氣鬱發,小便變,寒熱如瘧,甚則心痛,火行於槁,流水不冰,蟄蟲乃見。
  太陽司天,寒氣下臨,心氣上從,而火且明,丹起金乃眚,寒清時舉,勝則水冰,火氣高明,心熱煩,嗌干善渴,鼽嚏,喜悲數欠,熱氣妄行,寒乃復,霜不時降,善忘,甚則心痛。土乃潤,水豐衍,寒客至,沉陰化,濕氣變物,水飲內稸,中滿不食,皮(疒帬)肉苛,筋脈不利,甚則胕腫,身後癰。
  厥陰司天,風氣下臨,脾氣上從,而土且隆,黃起,水乃眚,土用革,體重肌肉萎,食減口爽,風行太虛,雲物搖動,目轉耳鳴。火縱其暴,地乃暑,大熱消爍,赤沃下,蟄蟲數見,流水不冰,其發機速。
  少陰司天,熱氣下臨,肺氣上從,白起金用,草木眚,喘嘔寒熱,嚏鼽衄鼻窒,大暑流行,甚則瘡瘍燔灼,金爍石流。地乃燥清,淒滄數至,脅痛善太息,肅殺行,草木變。
  太陰司天,濕氣下臨,腎氣上從,黑起水變,埃冒雲雨,胸中不利,陰痿,氣大衰,而不起不用。當其時,反腰脽痛,動轉不便也,厥逆。地乃藏陰,大寒且至,蟄蟲早附,心下否痛,地裂冰堅,少腹痛,時害於食,乘金則止水增,味乃咸,行水減也。
  帝曰:歲有胎孕不育,治之不全,何氣使然?岐伯曰:六氣五類,有相勝制也,同者盛之,異者衰之,此天地之道,生化之常也。故厥陰司天,毛蟲靜,羽蟲育,介蟲不成;在泉,毛蟲育,裸蟲耗,羽蟲不育。少陰司天,羽蟲靜,介蟲育,毛蟲不成;在泉,羽蟲育,介蟲耗不育。太陰司天,裸蟲靜,鱗蟲育,羽蟲不成;在泉,裸蟲育,鱗蟲不成。少陽司天,羽蟲靜,毛蟲育,裸蟲不成;在泉,羽蟲育,介蟲耗,毛蟲不育。陽明司天,介蟲靜,羽蟲育,介蟲不成;在泉,介蟲育,毛蟲耗,羽蟲不成。太陽司天,鱗蟲靜,裸蟲育;在泉,鱗蟲耗,裸蟲不育。諸乘所不成之運,則甚也。故氣主有所製,歲立有所生,地氣制己勝,天氣制勝己,天制色,地制形,五類衰盛,各隨其氣之所宜也。故有胎孕不育,治之不全,此氣之常也,所謂中根也。根於外者亦五,故生化之別,有五氣五味五色五類五宜也。帝曰:何謂也?岐伯曰:根於中者,命曰神機,神去則機息。根於外者,命曰氣立,氣止則化絕。故各有制,各有勝,各有生,各有成。故曰:不知年之所加,氣之同異,不足以言生化,此之謂也。
  帝曰:氣始而生化,氣散而有形,氣布而蕃育,氣終而像變,其致一也。然而五味所資,生化有薄,成熟有多少,終始不同,其故何也?岐伯曰:地氣制之也,非天不生,地不長也。帝曰:願聞其道。岐伯曰:寒熱燥濕,不同其化也。故少陽在泉,寒毒不生,其味辛,其治苦酸,其谷蒼丹。陽明在泉,濕毒不生,其味酸,其氣濕,其治辛苦甘,其谷丹素。太陽在泉,熱毒不生,其味苦,其治淡咸,其谷黅秬。厥陰在泉,清毒不生,其味甘,其治酸苦,其谷蒼赤,其氣專,其味正。少陰在泉,寒毒不生,其味辛,其治辛苦甘,其谷白丹。太陰在泉,燥毒不生,其味咸,其氣熱,其治甘咸,其谷黅秬。化淳則咸守,氣專則辛化而俱治。
  故曰:補上下者從之,治上下者逆之,以所在寒熱盛衰而調之。故曰:上取下取,內取外取,以求其過。能毒者以厚藥,不勝毒者以薄藥,此之謂也。氣反者,病在上,取之下;病在下,取之上;病在中,傍取之。治熱以寒,溫而行之;治寒以熱,涼而行之;治溫以清,冷而行之;治清以溫,熱而行之。故消之削之,吐之下之,補之寫之,久新同法。
  帝曰:病在中而不實不堅,且聚且散,奈何?岐伯曰:悉乎哉問也!無積者求其藏,虛則補之,藥以袪之,食以隨之,行水漬之,和其中外,可使畢已。
  帝曰:有毒無毒,服有約乎?岐伯曰:病有久新,方有大小,有毒無毒,固宜常制矣。大毒治病,十去其六;常毒治病,十去其;,小毒治病,十去其八;無毒治病,十去其九。谷肉果菜,食養盡之,無使過之,傷其正也。不盡,行復如法,必先歲氣,無伐天和,無盛盛,無虛虛,而遺人天殃,無致邪,無失正,絕人長命。帝曰:其久病者,有氣從不康,病去而瘠,奈何?岐伯曰:昭乎哉聖人之問也!化不可代,時不可違。夫經絡以通,血氣以從,復其不足,與眾齊同,養之和之,靜以待時,謹守其氣,無使頃移,其形乃彰,生氣以長,命曰聖王。故《大要》曰:無代化,無違時,必養必和,待其來復,此之謂也。帝曰:善。


\section{六元正紀大論篇第七十一}

  黃帝問曰:六化六變,勝復淫治,甘苦辛咸酸淡先後,余知之矣。夫五運之化,或從天氣,或逆天氣,或從天氣而逆地氣,或從地氣而逆天氣,或相得,或不相得,余未能明其事。欲通天之紀,從地之理,和其運,調其化,使上下合德,無相奪倫,天地升降,不失其宜,五運宣行,勿乖其政,調之正味,從逆奈何?岐伯稽首再拜對曰:昭乎哉問也。此天地之綱紀,變化之淵源,非聖帝孰能窮其至理歟!臣雖不敏,請陳其道,令終不滅,久而不易。
  帝曰:願夫子推而次之,從其類序,分其部主,別其宗司,昭其氣數,明其正化,可得聞乎?岐伯曰:先立其年以明其氣,金木水火土運行之數,寒暑燥濕風火臨御之化,則天道可見,民氣可調,陰陽卷舒,近而無惑,數之可數者,請遂言之。
  帝曰:太陽之政奈何?岐伯曰:辰戌之紀也。太陽太角太陰壬辰壬戌,其運風,其化鳴紊啟拆,其變振拉摧拔,其病眩掉目瞑。
  太角少徵太宮少商太羽太陽太徵太陰戊辰戊戌同正徵,其運熱,其化暄暑鬱燠,其變炎烈沸騰,其病熱郁。
  太徵少宮太商少羽少角太陽太宮太陰甲辰歲會甲戌歲會,其運陰埃,其化柔潤重澤,其變震驚飄驟,其病濕下重。
  太宮少商太羽太角少徵太陽太商太陰庚辰庚戌,其運涼,其化霧露蕭,其變肅殺凋零,其病燥背瞀胸滿。
  太商少羽少角太徵少宮太陽太羽太陰丙辰天符丙戌天符,其運寒,其化凝慘凓冽,其變冰雪霜雹,其病大寒留於谿谷。
  太羽太角少徵太宮少商
  凡此太陽司天之政,氣化運行先天,天氣肅,地氣靜,寒凝太虛,陽氣不令,水土合德,上應辰星鎮星。其谷玄黅,其政肅,其令徐。寒政大舉,澤無陽焰,則火發待時。少陽中治,時雨乃涯,止極雨散,還於太陰,雲朝北極,濕化乃布,澤流萬物,寒敷於上,雷動於下,寒濕之氣,持於氣交。民病寒濕,發肌肉萎,足痿不收,濡寫血熱。初之氣,地氣遷,氣乃大溫,草乃早榮,民乃厲,溫病乃作,身熱頭痛嘔吐,肌腠瘡瘍。二之氣,大涼反至,民乃慘,草乃遇寒,火氣遂抑,民病氣鬱中滿,寒乃始。三之氣,天政布,寒氣行,雨乃降,民病寒,反熱中,癰疽注下,心熱瞀悶,不治者死。四之氣,風濕交爭,風化為雨,乃長乃化乃成,民病大熱少氣,肌肉萎,足痿,注下赤白。五之氣,陽復化,草乃長,乃化乃成,民乃舒。終之氣,地氣正,濕令行,陰凝太虛,埃昬郊野,民乃慘淒,寒風以至,反者孕乃死。故歲宜苦以燥之溫之,必折其郁氣,先資其化源,抑其運氣,扶其不勝,無使暴過而生其疾,食歲谷以全其真,避虛邪以安其正。適氣同異,多少制之,同寒濕者燥熱化,異寒濕者燥濕化,故同者多之,異者少之,用寒遠寒,用涼遠涼,用溫遠溫,用熱遠熱,食宜同法。有假者反常,反是者病,所謂時也。
  帝曰:善。陽明之政奈何?岐伯曰:卯酉之紀也。陽明少角少陰,清熱勝復同,同正商。丁卯歲會丁酉,其運風清熱。少角太徵少宮太商少羽陽明少徵少陰,寒雨勝復同,同正商。癸卯癸酉,其運熱寒雨。少徵太宮少商太羽太角陽明少宮少陰,風涼勝復同。己卯己酉,其運雨風涼。少宮太商少羽少角太徵陽明少商少陰,熱寒勝復同,同正商。乙卯天符,乙酉歲會,太一天符,其運涼熱寒。少商太羽太角少徵太宮陽明少羽少陰,雨風勝復同,辛卯少宮同。辛酉辛卯其運寒雨風。少羽少角太徵太宮太商
  凡此陽明司天之政,氣化運行後天,天氣急,地氣明,陽專其令,炎暑大行,物燥以堅,淳風乃治,風燥橫運,流於氣交,多陽少陰,雲趨雨府,濕化乃敷。燥極而澤,其谷白丹,間谷命太者,其耗白甲品羽,金火合德,上應太白熒惑。其政切,其令暴,蟄蟲乃見,流水不冰,民病咳嗌塞,寒熱發,暴振凓癃閟,清先而勁,毛蟲乃死,熱後而暴,介蟲乃殃,其發躁,勝復之作,擾而大亂,清熱之氣,持於氣交。初之氣,地氣遷,陰始凝,氣始肅,水乃冰,寒雨化。其病中熱脹面目浮腫,善眠,鼽衄,嚏欠,嘔,小便黃赤,甚則淋。二之氣,陽乃布,民乃舒,物乃生榮。厲大至,民善暴死。三之氣,天政布,涼乃行,燥熱交合,燥極而澤,民病寒熱。四之氣,寒雨降,病暴僕,振慄譫妄,少氣,嗌干引飲,及為心痛癰腫瘡瘍瘧寒之疾,骨痿血便。五之氣,春令反行,草乃生榮,民氣和。終之氣,陽氣布,候反溫,蟄蟲來見,流水不冰,民乃康平,其病溫。故食歲谷以安其氣,食間谷以去其邪,歲宜以咸以苦以辛,汗之、清之、散之,安其運氣,無使受邪,折其郁氣,資其化源。以寒熱輕重少多其制,同熱者多天化,同清者多地化,用涼遠涼,用熱遠熱,用寒遠寒,用溫遠溫,食宜同法。有假者反之,此其道也。反是者,亂天地之經,擾陰陽之紀也。
  帝曰:善。少陽之政奈何?岐伯曰:寅申之紀也。少陽太角厥陰壬寅壬申,其運風鼓,其化鳴紊啟坼,其變振拉摧拔,其病掉眩,支脅,驚駭。太角少徵太宮少商太羽少陽太徵厥陰戊寅天符戊申天符,其運暑,其化暄囂鬱燠,其變炎烈沸騰,其病上熱郁,血溢血洩心痛。太徵少宮太商少羽少角少陽太商厥陰甲寅甲申,其運陰雨,其化柔潤重澤,其變震驚飄驟,其病體重,胕腫痞飲。太宮少商太羽太角少徵少陽太商厥陰庚寅庚申同正商,其運涼,其化霧露清切,其變肅殺凋零,其病肩背胸中。太商少羽少角太徵少宮少陽太羽厥陰丙寅丙申,其運寒肅,其化凝慘凓冽,其變冰雪霜雹,其病寒浮腫。太羽太角少徵太宮少商.
  凡此少陽司天之政,氣化運行先天,天氣正,地氣擾,風乃暴舉,木偃沙飛,炎火乃流,陰行陽化,雨乃時應,火木同德,上應熒惑歲星。其谷丹蒼,其政嚴,其令擾。故風熱參布,雲物沸騰,太陰橫流,寒乃時至,涼雨並起。民病寒中,外發瘡瘍,內為洩滿。故聖人遇之,和而不爭。往復之作,民病寒熱瘧洩,聾瞑嘔吐,上怫腫色變。初之氣,地氣遷,風勝乃搖,寒乃去,候乃大溫,草木早榮。寒來不殺,溫病乃起,其病氣怫於上,血溢目赤,咳逆頭痛,血崩脅滿,膚腠中瘡。二之氣,火反郁,白埃四起,雲趨雨府,風不勝濕,雨乃零,民乃康。其病熱鬱於上,咳逆嘔吐,瘡發於中,胸嗌不利,頭痛身熱,昬憒膿瘡。三之氣,天政布,炎暑至,少陽臨上,雨乃涯。民病熱中,聾瞑血溢,膿瘡咳嘔,鼽衄渴嚏欠,喉痹目赤,善暴死。四之氣,涼乃至,炎暑間化,白露降,民氣和平,其病滿身重。五之氣,陽乃去,寒乃來,雨乃降,氣門乃閉,剛木早凋,民避寒邪,君子周密。終之氣,地氣正,風乃至,萬物反生,霿霧以行。其病關閉不禁,心痛,陽氣不藏而咳。抑其運氣,贊所不勝,必折其郁氣,先取化源,暴過不生,苛疾不起。故歲宜咸辛宜酸,滲之洩之,漬之發之,觀氣寒溫以調其過,同風熱者多寒化,異風熱者少寒化,用熱遠熱,用溫遠溫,用寒遠寒,用涼遠涼,食宜同法,此其道也。有假者反之,反是者,病之階也。
  帝曰:善。太陰之政奈何?岐伯曰:丑未之紀也。太陰少角太陽,清熱勝復同,同正宮,丁丑丁未,其運風清熱。少角太徵少宮太商少羽太陰少徵太陽,寒雨勝復同,癸丑癸未,其運熱寒雨。少徵太宮少商太羽太角太陰少宮太陽,風清勝復同,同正宮,己丑太一天符,己未太一天符,其運雨風清。少宮太商少羽少角太徵太陰少商太陽,熱寒勝復同,乙丑乙未,其運涼熱寒。少商太羽太角少徵太宮太陰少羽太陽,雨風勝復同,同正宮。辛丑辛未,其運寒雨風。少羽少角太徵少宮太商凡此太陰司天之政,氣化運行後天,陰專其政,陽氣退辟,大風時起,天氣下降,地氣上騰,原野昏霿,白埃四起,雲奔南極,寒雨數至,物成於差夏。民病寒濕,腹滿,身(月真)憤,胕腫,痞逆寒厥拘急。濕寒合德,黃黑埃昏,流行氣交,上應鎮星辰星。其政肅,其令寂,其谷黅玄。故陰凝於上,寒積於下,寒水勝火,則為冰雹,陽光不治,殺氣乃行。故有餘宜高,不及宜下,有餘宜晚,不及宜早,土之利,氣之化也,民氣亦從之,間谷命其太也。初之氣,地氣遷,寒乃去,春氣正,風乃來,生布萬物以榮,民氣條舒,風濕相薄,雨乃後。民病血溢,筋絡拘強,關節不利,身重筋痿。二之氣,大火正,物承化,民乃和,其病溫厲大行,遠近咸若,濕蒸相薄,雨乃時降。三之氣,天政布,濕氣降,地氣騰,雨乃時降,寒乃隨之。感於寒濕,則民病身重胕腫,胸腹滿。四之氣,畏火臨,溽蒸化,地氣騰,天氣否隔,寒風曉暮,蒸熱相薄,草木凝煙,濕化不流,則白露陰布,以成秋令。民病腠理熱,血暴溢瘧,心腹滿熱,臚脹,甚則胕腫。五之氣,慘令已行,寒露下,霜乃早降,草木黃落,寒氣及體,君子周密,民病皮腠。終之氣,寒大舉,濕大化,霜乃積,陰乃凝,水堅冰,陽光不治。感於寒則病人關節禁固,腰脽痛,寒濕推於氣交而為疾也。必折其郁氣,而取化源,益其歲氣,無使邪勝,食歲谷以全其真,食間谷以保其精。故歲宜以苦燥之溫之,甚者發之洩之。不發不洩,則濕氣外溢,肉潰皮拆而水血交流。必贊其陽火,令御甚寒,從氣異同,少多其判也,同寒者以熱化,同濕者以燥化,異者少之,同者多之,用涼遠涼,用寒遠寒,用溫遠溫,用熱遠熱,食宜同法。假者反之,此其道也,反是者病也。
  帝曰:善,少陰之政奈何?岐伯曰:子午之紀也。少陰太角陽明壬子壬午,其運風鼓,其化鳴紊啟折,其變振拉摧拔,其病支滿。太角少徵太宮少商太羽少陰太徵陽明戊子天符戊午太一天符,其運炎暑,其化暄曜鬱燠,其變炎烈沸騰,其病上熱血溢。太徵少宮太商少羽少角少陰太宮陽明甲子甲午,其運陰雨,其化柔潤時雨,其變震驚飄驟,其病中滿身重。太宮少商太羽太角少徵少陰太商陽明庚子庚午同正商。其運涼勁,其化霧露蕭飋,其變肅殺凋零,其病下清。太商少羽少角太徵少宮少陰太羽陽明丙子歲會丙午,其運寒,其化凝慘凓冽,其變冰雪霜雹,其病寒下。太羽太角少徵太宮少商。
  凡此少陰司天之政,氣化運行先天,地氣肅,天氣明,寒交暑,熱加燥,雲馳雨府,濕化乃行,時雨乃降,金火合德,上應熒惑太白。其政明,其令切,其谷丹白。水火寒熱持於氣交而為病始也。熱病生於上,清病生於下,寒熱凌犯而爭於中,民病咳喘,血溢血洩,鼽嚏,目赤,眥瘍,寒厥入胃,心痛,腰痛,腹大,嗌干腫上。初之氣,地氣遷,燥將去,寒乃始,蟄復藏,水乃冰,霜復降,風乃至,陽氣鬱,民反周密,關節禁固,腰脽痛,炎暑將起,中外瘡瘍。二之氣,陽氣布,風乃行,春氣以正,萬物應榮,寒氣時至,民乃和,其病淋,目瞑目赤,氣鬱於上而熱。三之氣,天政布,大火行,庶類蕃鮮,寒氣時至。民病氣厥心痛,寒熱更作,咳喘目赤。四之氣,溽暑至,大雨時行,寒熱互至。民病寒熱,嗌干,黃癉,鼽衄,飲發。五之氣,畏火臨,暑反至,陽乃化,萬物乃生乃長榮,民乃康,其病溫。終之氣,燥令行,余火內格,腫於上,咳喘,甚則血溢。寒氣數舉,則霿霧翳,病生皮腠,內舍於脅,下連少腹而作寒中,地將易也。必抑其運氣,資其歲勝,折其郁發,先取化源,無使暴過而生其病也。食歲谷以全真氣,食間谷以辟虛邪。歲宜咸以之,而調其上,甚則以苦發之,以酸收之,而安其下,甚則以苦洩之。適氣同異而多少之,同天氣者以寒清化,同地氣者以溫熱化,用熱遠熱,用涼遠涼,用溫遠溫,用寒遠寒,食宜同法。有假則反,此其道也,反是者病作矣。
  帝曰:善。厥陰之政奈何?岐伯曰:巳亥之紀也。厥陰少角少陽,清熱勝復同,同正角。丁巳天符,丁亥天符,其運風清熱。少角太徵少宮太商少羽厥陰少徵少陽,寒雨勝復同,癸巳癸亥,其運熱寒雨。少徵太宮少商太羽太角厥陰少宮少陽,風清勝復同,同正角。己巳己亥,其運雨風清。少宮太商少羽少角太徵厥陰少商少陽,寒熱勝復同,同正角。乙巳乙亥,其運涼熱寒。少商太羽太角少徵太宮厥陰少羽少陽,雨風勝復同,辛巳辛亥,其運寒雨風。少羽少角太徵少宮太商
  凡此厥陰司天之政,氣化運行後天,諸同正歲,氣化運行同天,天氣擾,地氣正,風生高遠,炎熱從之,雲趨雨府,濕化乃行,風火同德,上應歲星熒惑。其政撓,其令速,其谷蒼丹,間谷言太者,其耗文角品羽。風燥火熱,勝復更作,蟄蟲來見,流水不冰,熱病行於下,風病行於上,風燥勝復形於中。初之氣,寒始肅,殺氣方至,民病寒於右之下。二之氣,寒不去,華雪水冰,殺氣施化,霜乃降,名草上焦,寒雨數至,陽復化,民病熱於中。三之氣,天政布,風乃時舉,民病泣出耳鳴掉眩。四之氣,溽暑濕熱相薄,爭於左之上,民病黃疸而為胕腫。五之氣,燥濕更勝,沉陰乃布,寒氣及體,風雨乃行。終之氣,畏火司令,陽乃大化,蟄蟲出見,流水不冰,地氣大發,草乃生,人乃舒,其病溫厲,必折其郁氣,資其化源,贊其運氣,無使邪勝,歲宜以辛調上,以咸調下,畏火之氣,無妄犯之,用溫遠溫,用熱遠熱,用涼遠涼,用寒遠寒,食宜同法。有假反常,此之道也,反是者病。
  帝曰:善。夫子言可謂悉矣,然何以明其應乎?岐伯曰:昭乎哉問也!夫六氣者,行有次,止有位,故常以正月朔日平旦視之,睹其位而知其所在矣。運有餘,其至先,運不及,其至後,此天之道,氣之常也。運非有餘非不足,是謂正歲,其至當其時也。帝曰:勝復之氣,其常在也,災眚時至,候也奈何?岐伯曰:非氣化者,是謂災也。
  帝曰:天地之數,終始奈何?岐伯曰:悉乎哉問也!是明道也。數之始,起於上而終於下,歲半之前,天氣主之,歲半之後,地氣主之,上下交互,氣交主之,歲紀畢矣。故曰位明,氣月可知乎,所謂氣也。帝曰:余司其事,則而行之,不合其數,何也?岐伯曰:氣用有多少,化洽有盛衰,衰盛多少,同其化也。帝曰:願聞同化何如?岐伯曰:風溫春化同,熱曛昏火夏化同,勝與復同,燥清煙露秋化同,雲雨昏暝埃長夏化同,寒氣霜雪冰冬化同,此天地五運六氣之化,更用盛衰之常也。
  帝曰:五運行同天化者,命曰天符,余知之矣。願聞同地化者何謂也?岐伯曰:太過而同天化者三,不及而同天化者亦三,太過而同地化者三,不及而同地化者亦三,此凡二十四歲也。帝曰:願聞其所謂也。岐伯曰:甲辰甲戌太宮下加太陰,壬寅壬申太角下加厥陰,庚子庚午太商下加陽明,如是者三。癸巳癸亥少徵下加少陽,辛丑辛未少羽下加太陽,癸卯癸酉少徵下加少陰,如是者三。戊子戊午太徵上臨少陰,戊寅戊申太徵上臨少陽,丙辰丙戌太羽上臨太陽,如是者三。丁巳丁亥少角上臨厥陰,乙卯乙酉少商上臨陽明,己丑己未少宮上臨太陰,如是者三。除此二十四歲,則不加不臨也。帝曰:加者何謂?岐伯曰:太過而加同天符,不及而加同歲會也。帝曰:臨者何謂?岐伯曰:太過不及,皆曰天符,而變行有多少,病形有微甚,生死有早晏耳。
  帝曰:夫子言用寒遠寒,用熱遠熱,余未知其然也,願聞何謂遠?岐伯曰:熱無犯熱,寒無犯寒,從者和,逆者病,不可不敬畏而遠之,所謂時與六位也。帝曰:溫涼何如?岐伯曰:司氣以熱,用熱無犯,司氣以寒,用寒無犯,司氣以涼,用涼無犯,司氣以溫,用溫無犯,間氣同其主無犯,異其主則小犯之,是謂四畏,必謹察之。帝曰:善。其犯者何如?岐伯曰:天氣反時,則可依時,及勝其主則可犯,以平為期,而不可過,是謂邪氣反勝者。故曰:無失天信,無逆氣宜,無翼其勝,無贊其復,是謂至治。
  帝曰:善。五運氣行主歲之紀,其有常數乎?岐伯曰:臣請次之。甲子甲午歲,上少陰火,中太宮土運,下陽明金,熱化二,雨化五,燥化四,所謂正化日也。其化上咸寒,中苦熱,下酸熱,所謂藥食宜也。
  乙丑乙未歲,上太陰土,中少商金運,下太陽水,熱化寒化勝復同,所謂邪氣化日也。災七宮。濕化五,清化四,寒化六,所謂正化日也。其化上苦熱,中酸和,下甘熱,所謂藥食宜也。
  丙寅丙申歲,上少陽相火,中太羽水運,下厥陰木。火化二,寒化六,風化三,所謂正化日也。其化上咸寒,中咸溫下辛溫,所謂藥食宜也。
  丁卯丁酉歲,上陽明金,中少角木運,下少陰火,清化熱化勝復同,所謂邪氣化日也。災三宮。燥化九,風化三,熱化七,所謂正化日也。其化上苦小溫,中辛和,下咸寒,所謂藥食宜也。
  戊辰戊戌歲,上太陽水,中太徵火運,下太陰土。寒化六,熱化七,濕化五,所謂正化日也。其化上苦溫,中甘和,下甘溫,所謂藥食宜也。
  己巳己亥歲,上厥陰木,中少宮土運,下少陽相火,風化清化勝復同,所謂邪氣化日也。災五宮。風化三,濕化五,火化七,所謂正化日也。其化上辛涼,中甘和,下咸寒,所謂藥食宜也。
  庚午庚子歲,上少陰火,中太商金運,下陽明金,熱化七,清化九,燥化九,所謂正化日也。其化上咸寒,中辛溫,下酸溫,所謂藥食宜也。
  辛未辛丑歲,上太陰土,中少羽水運,下太陽水,雨化風化勝復同,所謂邪氣化日也。災一宮。雨化五,寒化一,所謂正化日也。其化上苦熱,中苦和,下苦熱,所謂藥食宜也。
  壬申壬寅歲,上少陽相火,中太角木運,下厥陰木,火化二,風化八,所謂正化日也。其化土咸寒,中酸和,下辛涼,所謂藥食宜也。
  癸酉癸卯歲,上陽明金,中少徵火運,下少陰火,寒化雨化勝復同,所謂邪氣化日也。災九宮。燥化九,熱化二,所謂正化日也。其化上苦小溫,中咸溫,下咸寒,所謂藥食宜也。
  甲戌甲辰歲,上太陽水,中太宮土運,下太陰土,寒化六,濕化五,正化日也。其化上苦熱,中苦溫,下苦溫,藥食宜也。
  乙亥乙巳歲,上厥陰木,中少商金運,下少陽相火,熱化寒化勝復同,邪氣化日也。災七宮。風化八,清化四,火化二,正化度也。其化上辛涼,中酸和,下咸寒,藥食宜也。
  丙子丙午歲,上少陰火,中太羽水運,下陽明金,熱化二,寒化六,清化四,正化度也。其化上咸寒,中咸熱,下酸溫,藥食宜也。
  丁丑丁未歲,上太陰土,中少角木運,下太陽水,清化熱化勝復同,邪氣化度也。災三宮。雨化五,風化三,寒化一,正化度也。其化上苦溫,中辛溫,下甘熱,藥食宜也。
  戊寅戊申歲,上少陽相火,中太徵火運,下厥陰木,火化七,風化三,正化度也。其化上咸寒,中甘和,下辛涼,藥食宜也。
  己卯己酉歲,上陽明金,中少宮土運,下少陰火,風化清化勝復同,邪氣化度也。災五宮。清化九,雨化五,熱化七,正化度也,其化上苦小溫,中甘和,下咸寒,藥食宜也。
  庚辰庚戌歲,上太陽水,中太商金運,下太陰土。寒化一,清化九,雨化五,正化度也。其化上苦熱,中辛溫,下甘熱,藥食宜也。
  辛巳辛亥歲,上厥陰木,中少羽水運,下少陽相火,雨化風化勝復同,邪氣化度也。災一宮。風化三,寒化一,火化七,正化度也。其化上辛涼,中苦和,下咸寒,藥食宜也。
  壬午壬子歲,上少陰火,中太角木運,下陽明金。熱化二,風化八,清化四,正化度也。其化上咸寒,中酸涼,下酸溫,藥食宜也。
  癸未癸丑歲,上太陰土,中少徵火運,下太陽水,寒化雨化勝復同,邪氣化度也。災九宮。雨化五,火化二,寒化一,正化度也。其化上苦溫,中咸溫,下甘熱,藥食宜也。
  甲申甲寅歲,上少陽相火,中太宮土運,下厥陰木。火化二,雨化五,風化八,正化度也。其化上咸寒,中咸和,下辛涼,藥食宜也。
  乙酉乙卯歲,上陽明金,中少商金運,下少陰火,熱化寒化勝復同,邪氣化度也。災七宮。燥化四,清化四,熱化二,正化度也。其化上苦小溫,中苦和,下咸寒,藥食宜也。
  丙戌丙辰歲,上太陽水,中太羽水運,下太陰土。寒化六,雨化五,正化度也。其化上苦熱,中咸溫,下甘熱,藥食宜也。
  丁亥丁巳歲,上厥陰木,中少角木運,下少陽相火,清化熱化勝復同,邪氣化度也。災三宮。風化三,火化七,正化度也。其化上辛涼,中辛和,下咸寒,藥食宜也。
  戊子戊午歲,上少陰火,中太徵火運,下陽明金。熱化七,清化九,正化度也。其化上咸寒,中甘寒,下酸溫,藥食宜也。
  己丑己未歲,上太陰土,中少宮土運,下太陽水,風化清化勝復同,邪氣化度也。災五宮。雨化五,寒化一,正化度也。其化上苦熱,中甘和,下甘熱,藥食宜也。
  庚寅庚申歲,上少陽相火,中太商金運,下厥陰木。火化七,清化九,風化三,正化度也。其化上咸寒,中辛溫,下辛涼,藥食宜也。
  辛卯辛酉歲,上陽明金,中少羽水運,下少陰火,雨化風化勝復同,邪氣化度也。災一宮。清化九,寒化一,熱化七,正化度也。其化上苦小溫,中苦和,下咸寒,藥食宜也。
  壬辰壬戌歲,上太陽水,中太角木運,下太陰土。寒化六,風化八,雨化五,正化度也。其化上苦溫,中酸和,下甘溫,藥食宜也。
  癸巳癸亥歲,上厥陰木,中少徵火運,下少陽相火,寒化雨化勝復同,邪氣化度也。災九宮。風化八,火化二,正化度也。其化上辛涼,中咸和,下咸寒,藥食宜也。
  凡此定期之紀,勝復正化,皆有常數,不可不察。故知其要者一言而終,不知其要,流散無窮,此之謂也。
  帝曰:善。五運之氣,亦復歲乎?岐伯曰:郁極乃發,待時而作者也。帝曰:請問其所謂也?岐伯曰:五常之氣,太過不及,其發異也。帝曰:願卒聞之。岐伯曰:太過者暴,不及者徐,暴者為病甚,徐者為病持。帝曰:太過不及,其數何如?岐伯曰:太過者其數成,不及者其數生,土常以生也。
  帝曰:其發也何如?岐伯曰:土郁之發,岩谷震驚,雷殷氣交,埃昏黃黑,化為白氣,飄驟高深,擊石飛空,洪水乃從,川流漫衍,田牧土駒。化氣乃敷,善為時雨,始生始長,始化始成。故民病心腹脹,腸鳴而為數後,甚則心痛脅(月真),嘔吐霍亂,飲發注下,胕腫身重。雲奔雨府,霞擁朝陽,山澤埃昏。其乃發也,以其四氣。雲橫天山,浮游生滅,怫之先兆。
  金郁之發,天潔地明,風清氣切,大涼乃舉,草樹浮煙,燥氣以行,霿霧數起,殺氣來至,草木蒼干,金乃有聲。故民病咳逆,心脅滿,引少腹善暴痛,不可反側,嗌乾麵塵色惡。山澤焦枯,土凝霜鹵,怫乃發也,其氣五。夜零白露,林莽聲淒,怫之兆也。
  水郁之發,陽氣乃辟,陰氣暴舉,大寒乃至,川澤嚴凝,寒雰結為霜雪,甚則黃黑昏翳,流行氣交,乃為霜殺,水乃見祥。故民病寒客心痛,腰脽痛,大關節不利,屈伸不便,善厥逆,痞堅腹滿。陽光不治,空積沉陰,白埃昏暝,而乃發也,其氣二火前後。太虛深玄,氣猶麻散,微見而隱,色黑微黃,怫之先兆也。
  木郁之發,太虛埃昏,雲物以擾,大風乃至,屋發折木,木有變。故民病胃脘當心而痛,上支兩脅,鬲咽不通,食飲不下,甚則耳鳴眩轉,目不識人,善暴僵仆。太虛蒼埃,天山一色,或氣濁色,黃黑郁若,橫雲不起,雨而乃發也,其氣無常。長川草偃,柔葉呈陰,松吟高山,虎嘯岩岫,怫之先兆也。
  火郁之發,太虛腫翳,大明不彰,炎火行,大暑至,山澤燔燎,材木流津,廣廈騰煙,土浮霜鹵,止水乃減,蔓草焦黃,風行惑言,濕化乃後。故民病少氣,瘡瘍癰腫,脅腹胸背,面首四支(月真)憤,臚脹,瘍痱,嘔逆,瘛瘲骨痛,節乃有動,注下溫瘧,腹中暴痛,血溢流注,精液乃少,目赤心熱,甚則瞀悶懊憹,善暴死。刻終大溫,汗濡玄府,其乃發也,其氣四。動復則靜,陽極反陰,濕令乃化乃成。華髮水凝,山川冰雪,焰陽午澤,怫之先兆也。有怫之應而後報也,皆觀其極而乃發也,木發無時,水隨火也。謹候其時,病可與期,失時反歲,五氣不行,生化收藏,政無恆也。
  帝曰:水發而雹雪,土發而飄驟,木發而毀折,金髮而清明,火發而曛昧,何氣使然?岐伯曰:氣有多少,發有微甚,微者當其氣,甚者兼其下,徵其下氣而見可知也。
  帝曰:善。五氣之發,不當位者何也?岐伯曰:命其差。帝曰:差有數乎?岐伯曰:後皆三十度而有奇也。
  帝曰:氣至而先後者何?岐伯曰:運太過則其至先。運不及則其至後,此候之常也。帝曰:當時而至者何也?岐伯曰:非太過,非不及,則至當時,非是者眚也。
  帝曰:善。氣有非時而化者何也?岐伯曰:太過者當其時,不及者歸其己勝也。
  帝曰:四時之氣,至有早晏高下左右,其候何如?岐伯曰:行有逆順,至有遲速,故太過者化先天,不及者化後天。
  帝曰:願聞其行何謂也?岐伯曰:春氣西行,夏氣北行,秋氣東行,冬氣南行。故春氣始於下,秋氣始於上,夏氣始於中,冬氣始於標,春氣始於左,秋氣始於右,冬氣始於後,夏氣始於前,此四時正化之常。故至高之地,冬氣常在,至下之地,春氣常在。必謹察之。帝曰:善。
  黃帝問曰:五運六氣之應見,六化之正,六變之紀,何如?岐伯對曰:夫六氣正紀,有化有變,有勝有復,有用有病,不同其候,帝欲何乎?帝曰:願盡聞之。岐伯曰:請遂言之。夫氣之所至也,厥陰所至為和平,少陰所至為暄,太陰所至為埃溽,少陽所至為炎暑,陽明所至為清勁,太陽所至為寒雰,時化之常也。
  厥陰所至為風府,為璺啟;少陰所至為火府,為舒榮;太陰所至為雨府,為員盈;少陽所至為熱府,為行出;陽明所至為司殺府,為庚蒼;太陽所至為寒府,為歸藏;司化之常也。
  厥陰所至為生,為風搖;少陰所至為榮,為形見;太陰所至為化,為雲雨;少陽所至為長,為蕃鮮;陽明所至為收,為霧露;太陽所至為藏,為周密;氣化之常也。
  厥陰所至為風生,終為肅;少陰所至為熱生,中為寒;太陰所至為濕生,終為注雨;少陽所至為火生,終為蒸溽;陽明所至為燥生,終為涼;太陽所至為寒生,中為溫;德化之常也。
  厥陰所至為毛化,少陰所至為羽化,太陰所至為裸化,少陽所至羽化,陽明所至為介化,太陽所至為鱗化,德化之常也。
  厥陰所至為生化,少陰所至為榮化,太陰所至為濡化,少陽所至為茂化,陽明所至為堅化,太陽所至為藏化,布政之常也。
  厥陰所至為飄怒大涼,少陰所至為大暄寒,太陰所至為雷霆驟雨烈風,少陽所至為飄風燔燎霜凝,陽明所至為散落溫,太陽所至為寒雪冰雹白埃,氣變之常也。
  厥陰所至為撓動,為迎隨;少陰所至為高明,焰為曛;太陰所至為沉陰,為白埃,為晦暝;少陽所至為光顯,為彤雲,為曛;陽明所至為煙埃,為霜,為勁切,為淒鳴;太陽所至為剛固,為堅芒,為立;令行之常也。
  厥陰所至為裡急;少陰所至為瘍胗身熱;太陰所至為積飲否隔;少陽所至為嚏嘔,為瘡瘍;陽明所至為浮虛;太陽所至為屈伸不利;病之常也。
  厥陰所至為支痛;少陰所至為驚惑,惡寒,戰慄,譫妄;太陰所至為稸滿,少陽所至為驚躁,瞀昧,暴病;陽明所至為鼽,尻陰膝髀腨(骨行)足病;太陽所至為腰痛;病之常也。
  厥陰所至為緛戾;少陰所至為悲妄衄衊;太陰所至為中滿霍亂吐下;少陽所至為喉痹,耳鳴嘔湧;陽明所至皴揭;太陽所至為寢汗,痙;病之常也。
  厥陰所至為脅痛嘔洩,少陰所至為語笑,太陰所至為重胕腫,少陽所至為暴注、瞤瘛、暴死,陽明所至為鼽嚏,太陽所至為流洩禁止,病之常也。
  凡此十二變者,報德以德,報化以化,報政以政,報令以令,氣高則高,氣下則下,氣後則後,氣前則前,氣中則中,氣外則外,位之常也。故風勝則動,熱勝則腫,燥勝則干,寒勝則浮,濕勝則濡洩,甚則水閉胕腫,隨氣所在,以言其變耳。
  帝曰:願聞其用也。岐伯曰:夫六氣之用,各歸不勝而為化。故太陰雨化,施於太陽;太陽寒化,施於少陰;少陰熱化,施於陽明;陽明燥化,施於厥陰;厥陰風化,施於太陰。各命其所在以徵之也。帝曰:自得其位何如?岐伯曰:自得其位,常化也。帝曰:願聞所在也。岐伯曰:命其位而方月可知也。
  帝曰:六位之氣盈虛何如?岐伯曰:太少異也,太者之至徐而常,少者暴而亡。帝曰:天地之氣盈虛何如?岐伯曰:天氣不足,地氣隨之,地氣不足,天氣從之,運居其中而常先也。惡所不勝,歸所同和,隨運歸從而生其病也。故上勝則天氣降而下,下勝則地氣遷而上,多少而差其分,微者小差,甚者大差,甚則位易氣交易,則大變生而病作矣。《大要》曰:甚紀五分,微紀七分,其差可見,此之謂也。
  帝曰:善。論言熱無犯熱,寒無犯寒。余欲不遠熱,不遠熱奈何?岐伯曰:悉乎哉問也!發表不遠熱,攻裡不遠寒。帝曰:不發不攻而犯寒犯熱,何如?岐伯曰:寒熱內賊,其病益甚。帝曰:願聞無病者何如?岐伯曰:無者生之,有者甚之。帝曰:生者何如?岐伯曰:不遠熱則熱至,不遠寒則寒至。寒至則堅否腹滿,痛急下利之病生矣。熱至則身熱,吐下霍亂,癰疽瘡瘍,瞀郁注下,瞤瘛腫脹,嘔,鼽衄頭痛,骨節變,肉痛,血溢血洩,淋閟之病生矣。帝曰:治之奈何?岐伯曰:時必順之,犯者治以勝也。
  黃帝問曰:婦人重身,毒之何如?岐伯曰:有故無殞,亦無殞也。帝曰:願聞其故何謂也?岐伯曰:大積大聚,其可犯也,衰其大半而止,過者死。
  帝曰:善。郁之甚者治之奈何?岐伯曰:木郁達之,火郁發之,土郁奪之,金郁洩之,水郁折之,然調其氣,過者折之,以其畏也,所謂寫之。帝曰:假者何如?岐伯曰:有假其氣,則無禁也。所謂主氣不足,客氣勝也。帝曰:至哉聖人之道!天地大化運行之節,臨御之紀,陰陽之政,寒暑之令,非夫子孰能通之!請藏之靈蘭之室,署曰《六元正紀》,非齋戒不敢示,慎傳也。


刺法論七十二(亡)

本病論七十三(亡)

\section{至真要大論篇第七十四}

  黃帝問曰:五氣交合,盈虛更作,余知之矣。六氣分治,司天地者,其至何如?岐伯再拜對曰:明乎哉問也!天地之大紀,人神之通應也。帝曰:願聞上合昭昭,下合冥冥奈何?岐伯曰:此道之所主,工之所疑也。
  帝曰:願聞其道也。岐伯曰:厥陰司天,其化以風;少陰司天,其化以熱;太陰司天,其化以濕;少陽司天,其化以火;陽明司天,其化以燥;陽司天,其化以寒。以所臨藏位,命其病者也。
  帝曰:地化奈何?岐伯曰:司天同候,間氣皆然。帝曰:間氣何謂?岐伯曰:司左右者,是謂間氣也。帝曰:何以異之?岐伯曰:主歲者紀歲,間氣者紀步也。帝曰:善。歲主奈何?岐伯曰:厥陰司天為風化,在泉為酸化,司氣為蒼化,間氣為動化。少陰司天為熱化,在泉為苦化,不司氣化,居氣為灼化。太陰司天為濕化,在泉為甘化,司氣為黅化,間氣為柔化。少陽司天為火化,在泉苦化,司氣為丹化,間氣為明化。陽明司天為燥化,在泉為辛化,司氣為素化,間氣為清化。太陽司天為寒化,在泉為咸化,司氣為玄化,間氣為藏化。故治病者,必明六化分治,五味五色所生,五藏所宜,乃可以言盈虛病生之緒也。
  帝曰:厥陰在泉而酸化先,余知之矣。風化之行也,何如?岐伯曰:風行於地,所謂本也,餘氣同法。本乎天者,天之氣也,本乎地者,地之氣也,天地合氣,六節分而萬物化生矣。故曰:謹候氣宜,無失病機,此之謂也。
  帝曰:其主病何如?岐伯曰:司歲備物,則無遺主矣。帝曰:先歲物何也?岐伯曰:天地之專精也。帝曰:司氣者何如?岐伯曰:司氣者主歲同,然有餘不足也。帝曰:非司歲物何謂也?岐伯曰:散也,故質同而異等也,氣味有薄厚,性用有躁靜,治保有多少,力化有淺深,此之謂也。
  帝曰:歲主藏害何謂?岐伯曰:以所不勝命之,則其要也。帝曰:治之奈何?岐伯曰:上淫於下,所勝平之,外淫於內,所勝治之。帝曰:善。平氣何如?岐伯曰:謹察陰陽所在而調之,以平為期,正者正治,反者反治。
  帝曰:夫子言察陰陽所在而調之,論言人迎與寸口相應,若引繩小大齊等,命曰平,陰之所在寸口何如?岐伯曰:視歲南北,可知之矣。帝曰:願卒聞之。岐伯曰:北政之歲,少陰在泉,則寸口不應;厥陰在泉,則右不應;太陰在泉,則左不應。南政之歲,少陰司天,則寸口不應;厥陰司天,則右不應;太陰司天,則左不應。諸不應者,反其診則見矣。帝曰:尺候何如?岐伯曰:北政之歲,三陰在下,則寸不應;三陰在上,則尺不應。南政之歲,三陰在天,則寸不應;三陰在泉,則尺不應,左右同。故曰:知其要者,一言而終,不知其要,流散無窮,此之謂也。
  帝曰:善。天地之氣,內淫而病何如?岐伯曰:歲厥陰在泉,風淫所勝,則地氣不明,平野昧,草乃早秀。民病灑灑振寒,善伸數欠,心痛支滿,兩脅裡急,飲食不下,鬲咽不通,食則嘔,腹脹善噫,得後與氣,則快然如衰,身體皆重。
  歲少陰在泉,熱淫所勝,則焰浮川澤,陰處反明。民病腹中常鳴,氣上衝胸,喘不能久立,寒熱皮膚痛,目瞑齒痛(出頁)腫,惡寒發熱如瘧,少腹中痛,腹大,蟄蟲不藏。
  歲太陰在泉,草乃早榮,濕淫所勝,則埃昏岩谷,黃反見黑,至陰之交。民病飲積,心痛,耳聾,渾渾焞焞,嗌腫喉痹,陰病血見,少腹痛腫,不得小便,病沖頭痛,目似脫,項似拔,腰似折,髀不可以回,膕如結,腨如別。
  歲少陽在泉,火淫所勝,則焰明郊野,寒熱更至。民病注洩赤白,少腹痛溺赤,甚則血便,少陰同候。
  歲陽明在泉,燥淫所勝,則霿霧清瞑。民病喜嘔,嘔有苦,善太息,心脅痛不能反側,甚則嗌乾麵塵,身無膏澤,足外反熱。
  歲太陽在泉,寒淫所勝,則凝肅慘慄。民病少腹控睾,引腰脊,上衝心痛,血見,嗌痛頷腫。
  帝曰:善。治之奈何?岐伯曰:諸氣在泉,風淫於內,治以辛涼,佐以苦,以甘緩之,以辛散之。熱淫於內,治以咸寒,佐以甘苦,以酸收之,以苦發之。濕淫於內,治以苦熱,佐以酸淡,以苦燥之,以淡洩之。火淫於內,治以咸冷,佐以苦辛,以酸收之,以苦發之。燥淫於內,治以苦溫,佐以甘辛,以苦下之。寒淫於內,治以甘熱,佐以苦辛,以咸寫之,以辛潤之,以苦堅之。
  帝曰:善。天氣之變何如?岐伯曰:厥陰司天,風淫所勝,則太虛埃昏,雲物以擾,寒生春氣,流水不冰,民病胃脘當心而痛,上支兩脅,鬲咽不通,飲食不下,舌本強,食則嘔,冷洩腹脹,溏洩,瘕水閉,蟄蟲不去,病本於脾。沖陽絕,死不治。
  少陰司天,熱淫所勝,怫熱至,火行其政,民病胸中煩熱,嗌干,右胠滿,皮膚痛,寒熱咳喘,大雨且至,唾血血洩,鼽衄嚏嘔,溺色變,甚則瘡瘍胕腫,肩背臂臑及缺盆中痛,心痛肺(月真),腹大滿,膨膨而喘咳,病本於肺。尺澤絕,死不治。
  太陰司天,濕淫所勝,則沉陰且布,雨變枯槁,胕腫骨痛,陰痹,陰痹者,按之不得,腰脊頭項痛,時眩,大便難,陰氣不用,飢不欲食,咳唾則有血,心如懸,病本於腎。太谿絕,死不治。
  少陽司天,火淫所勝,則溫氣流行,金政不平,民病頭痛,發熱惡寒而瘧,熱上皮膚痛,色變黃赤,傳而為水,身面胕腫,腹滿仰息,洩注赤白,瘡瘍咳唾血,煩心,胸中熱,甚則鼽衄,病本於肺。天府絕,死不治。
  陽明司天,燥淫所勝,則木乃晚榮,草乃晚生,筋骨內變,民病左胠脅痛,寒清於中,感而瘧,大涼革候,咳,腹中鳴,注洩鶩溏,名木斂,生菀於下,草焦上首,心脅暴痛,不可反側,嗌乾麵塵,腰痛,丈夫頹疝,婦人少腹痛,目昧眥,瘍瘡痤癰,蟄蟲來見,病本於肝。太沖絕,死不治。
  太陽司天,寒淫所勝,則寒氣反至,水且冰,血變於中,發為癰瘍,民病厥心痛,嘔血血洩鼽衄,善悲,時眩僕,運火炎烈,雨暴乃雹,胸腹滿,手熱肘攣,掖腫,心澹澹大動,胸脅胃脘不安,面赤目黃,善噫嗌干,甚則色炲,渴而欲飲,病本於心。神門絕,死不治。所謂動氣知其藏也。
  帝曰:善。治之奈何?岐伯曰:司天之氣,風淫所勝,平以辛涼,佐以苦甘,以甘緩之,以酸寫之。熱淫所勝,平以咸寒,佐以苦甘,以酸收之。濕淫所勝,平以苦熱,佐以酸辛,以苦燥之,以淡洩之。濕上甚而熱,治以苦溫,佐以甘辛,以汗為故而止。火淫所勝,平以酸冷,佐以苦甘,以酸收之,以苦發之,以酸復之,熱淫同。燥淫所勝,平以苦濕,佐以酸辛,以苦下之。寒淫所勝,平以辛熱,佐以甘苦,以咸寫之。
  帝曰:善。邪氣反勝,治之奈何?岐伯曰:風司於地,清反勝之,治以酸溫,佐以苦甘,以辛平之。熱司於地,寒反勝之,治以甘熱,佐以苦辛,以咸平之。濕司於地,熱反勝之,治以苦冷,佐以咸甘,以苦平之。火司於地,寒反勝之,治以甘熱,佐以苦辛,以咸平之。燥司於地,熱反勝之,治以平寒,佐以苦甘,以酸平之,以和為利。寒司於地,熱反勝之,治以咸冷,佐以甘辛,以苦平之。
  帝曰:其司天邪勝何如?岐伯曰:風化於天,清反勝之,治以酸溫,佐以甘苦。熱化於天,寒反勝之,治以甘溫,佐以苦酸辛。濕化於天,熱反勝之,治以苦寒,佐以苦酸。火化於天,寒反勝之,治以甘熱,佐以苦辛。燥火於天,熱反勝之,治以辛寒,佐以苦甘。寒化於天,熱反勝之,治以咸冷,佐以苦辛。
  帝曰:六氣相勝奈何?岐伯曰:厥陰之勝,耳鳴頭眩,憒憒欲吐,胃鬲如寒,大風數舉,裸蟲不滋,胠脅氣並,化而為熱,小便黃赤,胃脘當心而痛,上支兩脅,腸鳴飧洩,少腹痛,注下赤白,甚則嘔吐,鬲咽不通。
  少陰之勝,心下熱,善飢,齊下反動,氣游三焦,炎暑至,木乃津,草乃萎,嘔逆躁煩,腹滿痛,溏洩,傳為赤沃。
  太陰之勝,火氣內郁,瘡瘍於中,流散於外,病在胠脅,甚則心痛,熱格,頭痛喉痹項強,獨勝則濕氣內郁,寒迫下焦,痛留頂,互引眉間,胃滿,雨數至,燥化乃見,少腹滿,腰脽重強,內不便,善注洩,足下溫,頭重,足脛胕腫,飲發於中,胕腫於上。
  少陽之勝,熱客於胃,煩心心痛,目赤欲嘔,嘔酸善飢,耳痛溺赤,善驚譫妄,暴熱消爍,草萎水涸,介蟲乃屈,少腹痛,下沃赤白。
  陽明之勝,清發於中,左胠脅痛,溏洩,內為嗌塞,外發頹疝,大涼肅殺,華英改容,毛蟲乃殃,胸中不便,嗌塞而咳。
  太陽之勝,凝凓且至,非時水冰,羽乃後化,痔瘧發,寒厥入胃,則內生心痛,陰中乃瘍,隱曲不利,互引陰股,筋肉拘苛,血脈凝泣,絡滿色變,或為血洩,皮膚否腫,腹滿食減,熱反上行,頭項囟頂腦戶中痛,目如脫,寒入下焦,傳為濡寫。
  帝曰:治之奈何?岐伯曰:厥陰之勝,治以甘清,佐以苦辛,以酸寫之。少陰之勝,治以辛寒,佐以苦咸,以甘寫之。太陰之勝,治以咸熱,佐以辛甘,以苦寫之。少陽之勝,治以辛寒,佐以甘咸,以甘寫之。陽明之勝,治以酸溫,佐以辛甘,以苦洩之。太陽之勝,治以甘熱,佐以辛酸,以咸寫之。
  帝曰:六氣之復何如?岐伯曰:悉乎哉問也!厥陰之復,少腹堅滿,裡急暴痛,偃木飛沙,裸蟲不榮,厥心痛,汗發嘔吐,飲食不入,入而復出,筋骨掉眩,清厥,甚則入脾,食痹而吐。沖陽絕,死不治。
  少陰之復,燠熱內作,煩躁鼽嚏,少腹絞痛,火見燔焫,嗌燥,分注時止,氣動於左,上行於右,咳,皮膚痛,暴瘖心痛,郁冒不知人,乃灑淅惡寒,振慄譫妄,寒已而熱,渴而欲飲,少氣骨痿,隔腸不便,外為浮腫,噦噫,赤氣後化,流水不冰,熱氣大行,介蟲不復,病疿胗瘡瘍,癰疽痤痔,甚則入肺,咳而鼻淵。天府絕,死不治。
  太陰之復,濕變乃舉,體重中滿,食飲不化,陰氣上厥,胸中不便,飲發於中,咳喘有聲,大雨時行,鱗見於陸,頭頂痛重,而掉瘛尤甚,嘔而密默,唾吐清液,甚則入腎竅,寫無度。太谿絕,死不治。
  少陽之復,大熱將至,枯燥燔爇,介蟲乃耗,驚瘛咳衄,心熱煩躁,便數憎風,厥氣上行,面如浮埃,目乃瞤瘛,火氣內發,上為口麋嘔逆,血溢血洩,發而為瘧,惡寒鼓慄,寒極反熱,嗌絡焦槁,渴引水漿,色變黃赤,少氣脈萎,化而為水,傳為胕腫,甚則入肺,咳而血洩。尺澤絕,死不治。
  陽明之復,清氣大舉,森木蒼干,毛蟲乃厲,病生胠脅,氣歸於左,善太息,甚則心痛否滿,腹脹而洩,嘔苦咳噦,煩心,病在鬲中,頭痛,甚則入肝,驚駭筋攣。太沖絕,死不治。
  太陽之復,厥氣上行,水凝雨冰,羽蟲乃死。心胃生寒,胸膈不利,心痛否滿,頭痛善悲,時眩僕,食減,腰脽反痛,屈伸不便,地裂冰堅,陽光不治,少腹控睾,引腰脊,上衝心,唾出清水,及為噦噫,甚則入心,善忘善悲。神門絕,死不治。
  帝曰:善,治之奈何?岐伯曰:厥陰之復,治以酸寒,佐以甘辛,以酸寫之,以甘緩之。少陰之復,治以咸寒,佐以苦辛,以甘寫之,以酸收之,辛苦發之,以咸軟之。太陰之復,治以苦熱,佐以酸辛,以苦寫之,燥之,洩之。少陽之復,治以咸冷,佐以苦辛,以咸軟之,以酸收之,辛苦發之,發不遠熱,無犯溫涼,少陰同法。陽明之復,治以辛溫,佐以苦甘,以苦洩之,以苦下之,以酸補之。太陽之復,治以咸熱,佐以甘辛,以苦堅之。治諸勝復,寒者熱之,熱者寒之,溫者清之,清者溫之,散者收之,抑者散之,燥者潤之,急者緩之,堅者耎之,脆者堅之,衰者補之,強者寫之,各安其氣,必清必靜,則病氣衰去,歸其所宗,此治之大體也。
  帝曰:善。氣之上下,何謂也?岐伯曰:身半以上,其氣三矣,天之分也,天氣主之。身半以下,其氣三矣,地之分也,地氣主之。以名命氣,以氣命處,而言其病。半,所謂天樞也。故上勝而下俱病者,以地名之,下勝而上俱病者,以天名之。所謂勝至,報氣屈伏而未發也,復至則不以天地異名,皆如復氣為法也。
  帝曰:勝復之動,時有常乎?氣有必乎?岐伯曰:時有常位,而氣無必也。帝曰:願聞其道也。岐伯曰:初氣終三氣,天氣主之,勝之常也。四氣盡終氣,地氣主之,復之常也。有勝則復,無勝則否。帝曰:善。復已而勝何如?岐伯曰:勝至則復,無常數也,衰乃止耳。復已而勝,不復則害,此傷生也。帝曰:復而反病何也?岐伯曰:居非其位,不相得也,大復其勝則主勝之,故反病也,所謂火燥熱也。帝曰:治之何如?岐伯曰:夫氣之勝也,微者隨之,甚者制之。氣之復也,和者平之,暴者奪之,皆隨勝氣,安其屈伏,無問其數,以平為期,此其道也。
  帝曰:善。客主之勝復奈何?岐伯曰:客主之氣,勝而無復也。帝曰:其逆從何如?岐伯曰:主勝逆,客勝從,天之道也。
  帝曰:其生病何如?岐伯曰:厥陰司天,客勝則耳鳴掉眩,甚則咳;主勝則胸脅痛,舌難以言。少陰司天,客勝則鼽嚏頸項強,肩背瞀熱,頭痛少氣,發熱耳聾目暝,甚則胕腫血溢,瘡瘍咳喘;主勝則心熱煩躁,甚則脅痛支滿。太陰司天,客勝則首面胕腫,呼吸氣喘;主勝則胸腹滿,食已而瞀。少陽司天,客勝則丹胗外發,及為丹熛瘡瘍,嘔逆喉痹,頭痛嗌腫,耳聾血溢,內為瘛瘲;主勝則胸滿咳仰息,甚而有血,手熱。陽明司天,清復內余,則咳衄嗌塞,心鬲中熱,咳不止而白血出者死。太陽司天,客勝則胸中不利,出清涕,感寒則咳;主勝則喉嗌中鳴。
  厥陰在泉,客勝則大關節不利,內為痙強拘瘛,外為不便;主勝則筋骨繇並,腰腹時痛。少陰在泉,客勝則腰痛,尻股膝髀腨(骨行)足病,瞀熱以酸,胕腫不能久立,溲便變;主勝則厥氣上行,心痛發熱,鬲中,眾痹皆作,發於胠脅,魄汗不藏,四逆而起。太陰在泉,客勝則足痿下重,便溲不時,濕客下焦,發而濡寫,及為腫,隱曲之疾;主勝則寒氣逆滿,食飲不下,甚則為疝。少陽在泉,客勝則腰腹痛而反惡寒,甚則下白溺白;主勝則熱反上行而客於心,心痛發熱,格中而嘔。少陰同候。陽明在泉,客勝則清氣動下,少腹堅滿而數便寫;主勝則腰重腹痛,少腹生寒,下為鶩溏,則寒厥於腸,上衝胸中,甚則喘,不能久立。太陽在泉,寒復內余,則腰尻痛,屈伸不利,股脛足膝中痛。
  帝曰:善,治之奈何?岐伯曰:高者抑之,下者舉之,有餘折之,不足補之,佐以所利,和以所宜,必安其主客,適其寒溫,同者逆之,異者從之。
  帝曰:治寒以熱,治熱以寒,氣相得者逆之,不相得者從之,余己知之矣。其於正味何如?岐伯曰:木位之主,其寫以酸,其補以辛。火位之主,其寫以甘,其補以咸。土位之主,其寫以苦,其補以甘。金位之主,其寫以辛,其補以酸。水位之主,其寫以咸,其補以苦。厥陰之客,以辛補之,以酸寫之,以甘緩之。少陰之客,以咸補之,以甘寫之,以咸收之。太陰之客,以甘補之,以苦寫之,以甘緩之。少陽之客,以咸補之,以甘寫之,以咸軟之。陽明之客,以酸補之。以辛寫之,以苦洩之。太陽之客,以苦補之,以咸寫之,以苦堅之,以辛潤之。開發腠理,致津液通氣也。
  帝曰:善。願聞陰陽之三也何謂?岐伯曰:氣有多少,異用也。帝曰:陽明何謂也?岐伯曰:兩陽合明也。帝曰:厥陰何也?岐伯曰:兩陰交盡也。
  帝曰:氣有多少,病有盛衰,治有緩急,方有大小,願聞約奈何?岐伯曰:氣有高下,病有遠近,證有中外,治有輕重,適其至所為故也。《大要》曰:君一臣二,奇之制也;君二臣四,偶之制也;君二臣三,奇之制也;君三臣六,偶之制也。故曰:近者奇之,遠者偶之,汗者不以奇,下者不以偶,補上治上制以緩,補下治下制以急,急則氣味厚,緩則氣味薄,適其至所,此之謂也。病所遠而中道氣味之者,食而過之,無越其制度也。是故平氣之道,近而奇偶,制小其服也。遠而奇偶,制大其服也。大則數少,小則數多。多則九之,少則二之。奇之不去則偶之,是謂重方。偶之不去,則反佐以取之,所謂寒熱溫涼,反從其病也。
  帝曰:善。病生於本,余知之矣。生於標者,治之奈何?岐伯曰:病反其本,得標之病,治反其本,得標之方。
  帝曰:善。六氣之勝,何以候之?岐伯曰:乘其至也。清氣大來,燥之勝也,風木受邪,肝病生焉。熱氣大來,火之勝也,金燥受邪,肺病生焉。寒氣大來,水之勝也,火熱受邪,心病生焉。濕氣大來,土之勝也,寒水受邪,腎病生焉。風氣大來,木之勝也,土濕受邪,脾病生焉。所謂感邪而生病也。乘年之虛,則邪甚也。失時之和,亦邪甚也。遇月之空,亦邪甚也。重感於邪,則病危矣。有勝之氣,其必來復也。
  帝曰:其脈至何如?岐伯曰:厥陰之至,其脈弦,少陰之至,其脈鉤,太陰之至,其脈沉,少陽之至,大而浮,陽明之至,短而濇,太陽之至,大而長。至而和則平,至而甚則病,至而反者病,至而不至者病,未至而至者病,陰陽易者危。
  帝曰:六氣標本,所從不同,奈何?岐伯曰:氣有從本者,有從標本者,有不從標本者也。帝曰:願卒聞之。岐伯曰:少陽太陰從本,少陰太陽從本從標,陽明厥陰,不從標本,從乎中也。故從本者,化生於本,從標本者,有標本之化,從中者,以中氣為化也。帝曰:脈從而病反者,其診何如?岐伯曰:脈至而從,按之不鼓,諸陽皆然。帝曰:諸陰之反,其脈何如?岐伯曰:脈至而從,按之鼓甚而盛也。
  是故百病之起,有生於本者,有生於標者,有生於中氣者,有取本而得者,有取標而得者,有取中氣而得者,有取標本而得者,有逆取而得者,有從取而得者。逆,正順也。若順,逆也。故曰:知標與本,用之不殆,明知逆順,正行無問。此之謂也。不知是者,不足以言診,足以亂經。故《大要》曰:粗工嘻嘻,以為可知,言熱未已,寒病復始,同氣異形,迷診亂經,此之謂也,夫標本之道,要而博,小而大,可以言一而知百病之害,言標與本,易而勿損,察本與標,氣可令調,明知勝復,為萬民式,天之道畢矣。
  帝曰:勝復之變,早晏何如?岐伯曰:夫所勝者,勝至已病,病已慍慍,而復已萌也。夫所復者,勝盡而起,得位而甚,勝有微甚,復有少多,勝和而和,勝虛而虛,天之常也。帝曰:勝復之作,動不當位,或後時而至,其故何也?岐伯曰:夫氣之生,與其化衰盛異也。寒暑溫涼盛衰之用,其在四維。故陽之動,始於溫,盛於暑;陰之動,始於清,盛於寒。春夏秋冬,各差其分。故《大要》曰:彼春之暖,為夏之暑,彼秋之忿,為冬之怒,謹按四維,斥候皆歸,其終可見,其始可知。此之謂也。帝曰:差有數乎?岐伯曰:又凡三十度也。帝曰:其脈應皆何如?岐伯曰:差同正法,待時而去也。《脈要》曰:春不沉,夏不弦,冬不濇,秋不數,是謂四塞。沉甚曰病,弦甚曰病,澀甚曰病,數其曰病,參見曰病,復見曰病,未去而去曰病,去而不去曰病,反者死。故曰:氣之相守司也,如權衡之不得相失也。夫陰陽之氣,清靜則生化治,動則苛疾起,此之謂也。
  帝曰:幽明何如?岐伯曰:兩陰交盡故曰幽,兩陽合明故曰明,幽明之配,寒暑之異也。帝曰:分至何如?岐伯曰:氣至之謂至,氣分之謂分,至則氣同,分則氣異,所謂天地之正紀也。
  帝曰:夫子言春秋氣始於前,冬夏氣始於後,余已知之矣。然六氣往復,主歲不常也,其補寫奈何?岐伯曰:上下所主,隨其攸利,正其味,則其要也,左右同法。《大要》曰:少陽之主,先甘後咸;陽明之主,先辛後酸;太陽之主,先咸後苦;厥陰之主,先酸後辛;少陰之主,先甘後咸;太陰之主,先苦後甘。佐以所利,資以所生,是謂得氣。
  帝曰:善。夫百病之生也,皆生於風寒暑濕燥火,以之化之變也。經言盛者寫之,虛者補之,余錫以方士,而方士用之,尚未能十全,余欲令要道必行,桴鼓相應,猶拔刺雪汙,工巧神聖,可得聞乎?岐伯曰:審察病機,無失氣宜,此之謂也。帝曰:願聞病機何如?岐伯曰:諸風掉眩,皆屬於肝。諸寒收引,皆屬於腎。諸氣膹郁,皆屬於肺。諸濕腫滿,皆屬於脾。諸熱瞀瘈,皆屬於火。諸痛癢瘡,皆屬於心。諸厥固洩,皆屬於下。諸痿喘嘔,皆屬於上。諸禁鼓慄,如喪神守,皆屬於火。諸痙項強,皆屬於濕。諸逆衝上,皆屬於火。諸脹腹大,皆屬於熱。諸躁狂越,皆屬於火。諸暴強直,皆屬於風。諸病有聲,鼓之如鼓,皆屬於熱。諸病胕腫,痛酸驚駭,皆屬於火。諸轉反戾,水液渾濁,皆屬於熱。諸病水液,澄澈清冷,皆屬於寒。諸嘔吐酸,暴注下迫,皆屬於熱。故《大要》曰:謹守病機,各司其屬,有者求之,無者求之,盛者責之,虛者責之,必先五勝,疏其血氣,令其調達,而致和平,此之謂也。
  帝曰:善,五味陰陽之用何如?岐伯曰:辛甘發散為陽,酸苦湧洩為陰,鹹味湧洩為陰,淡味滲洩為陽。六者或收或散,或緩或急,或燥或潤,或軟或堅,以所利而行之,調其氣,使其平也。帝曰:非調氣而得者,治之奈何?有毒無毒,何先何後?願聞其道。岐伯曰:有毒無毒,所治為主,適大小為制也。帝曰:請言其制。岐伯曰:君一臣二,制之小也;君一臣三佐五,制之中也;君一臣三佐九,制之大也。寒者熱之,熱者寒之,微者逆之,甚者從之,堅者削之,客者除之,勞者溫之,結者散之,留者政之,燥者濡之,急者緩之,散者收之,損者溫之,逸者行之,驚者平之,上之下之,摩之浴之,薄之劫之,開之發之,適事為故。
  帝曰:何謂逆從?岐伯曰:逆者正治,從者反治,從少從多,觀其事也。帝曰:反治何謂?岐伯曰:熱因寒用,寒因熱用,塞因塞用,通因通用,必伏其所主,而先其所因,其始則同,其終則異,可使破積,可使潰堅,可使氣和,可使必已。帝曰:善。氣調而得者何如?岐伯曰:逆之從之,逆而從之,從而逆之,疏氣令調,則其道也。
  帝曰:善。病之中外何如?岐伯曰:從內之外者調其內;從外之內者治其外;從內之外而盛於外者,先調其內而後治其外;從外之內而盛於內者,先治其外,而後調其內;中外不相及,則治主病。
  帝曰:善。火熱復,惡寒發熱,有如瘧狀,或一日發,或間數日發,其故何也?岐伯曰:勝復之氣,會遇之時,有多少也。陰氣多而陽氣少,則其發日遠;陽氣多而陰氣少,則其發日近。此勝復相薄,盛衰之節,瘧亦同法。
  帝曰:論言治寒以熱,治熱以寒,而方士不能廢繩墨而更其道也。有病熱者,寒之而熱,有病寒者,熱之而寒,二者皆在,新病復起,奈何治?岐伯曰:諸寒之而熱者取之陰,熱之而寒者取之陽,所謂求其屬也。帝曰:善。服寒而反熱,服熱而反寒,其故何也?岐伯曰:治其王氣,是以反也。帝曰:不治王而然者何也?岐伯曰:悉乎哉問也!不治五味屬也。夫五味入胃,各歸所喜,攻酸先入肝,苦先入心,甘先入脾,辛先入肺,咸先入腎,久而增氣,物化之常也。氣增而久,夭之由也。
  帝曰:善。方制君臣何謂也?岐伯曰:主病之謂君,佐君之謂臣,應臣之謂使,非上下三品之謂也。帝曰:三品何謂/岐伯曰:所以明善惡之殊貫也。
  帝曰:善。病之中外何如?岐伯曰:調氣之方,必別陰陽,定其中外,各守其鄉。內者內治,外者外治,微者調之,其次平之,盛者奪之,汗之下之,寒熱溫涼,衰之以屬,隨其攸利,謹道如法,萬舉萬全,氣血正平,長有天命。帝曰:善。

\section{著至教論篇第七十五}

  黃帝坐明堂,召雷公而問之曰:子知醫之道乎?雷公對曰:誦而頗能解,解而未能別,別而未能明,明而未能彰,足以治群僚,不足至侯王。願得受樹天之度,四時陰陽合之,別星辰與日月光,以彰經術,後世益明,上通神農,著至教,疑於二皇。帝曰:善!無失之,此皆陰陽表裡上下雌雄相輸應也,而道上知天文,下知地理,中知人事,可以長久,以教眾庶,亦不疑殆,醫道論篇,可傳後世,可以為寶。
  雷公曰:請受道,諷誦用解。帝曰:子不聞《陰陽傳》乎!曰:不知。曰:夫三陽天為業,上下無常,合而病至,偏害陰陽。雷公曰:三陽莫當,請聞其解。帝曰:三陽獨至者,是三陽並至,並至如風雨,上為巔疾,下為漏病,外無期,內無正,不中經紀,診無上下,以書別。雷公曰:臣治疏愈,說意而已。帝曰:三陽者,至陽也,積並則為驚,病起疾風,至如礔礪,九竅皆塞,陽氣滂溢,干嗌喉塞,並於陰,則上下無常,薄為腸澼,此謂三陽直心,坐不得起,臥者便身全。三陽之病,且以知天下,何以別陰陽,應四時,合之五行。
  雷公曰:陽言不別,陰言不理,請起受解,以為至道。帝曰:子若受傳,不知合至道以惑師教,語子至道之要。病傷五藏,筋骨以消,子言不明不別,是世主學盡矣。腎且絕,惋惋日暮,從容不出,人事不殷。


\section{示從容論篇第七十六}

  黃帝燕坐,召雷公而問之曰:汝受術誦書者,若能覽觀雜學,及於比類,通合道理,為余言子所長,五藏六府,膽胃大小腸,脾胞膀胱,腦髓涕唾,哭泣悲哀,水所從行,此皆人之所生,治之過失,子務明之,可以十全,即不能知,為世所怨。雷公曰:臣請誦《脈經上下篇》,甚眾多矣,別異比類,猶未能以十全,又安足以明之。
  帝曰:子別試通五藏之過,六府之所不和,針石所敗,毒藥所宜,湯液滋味,具言其狀,悉言以對,請問不知。雷公曰:肝虛腎虛脾虛,皆令人體重煩冤,當投毒藥刺灸砭石湯液,或已,或不已,願聞其解。帝曰:公何年之長而問之少,余真問以自謬也。吾問子窈冥,子言《上下篇》以對,何也?夫脾虛浮似肺,腎小浮似脾,肝急沉散似腎,此皆工之所時亂也,然從容得之。若夫三藏土木水參居,此童子之所知,問之何也?
  雷公曰:於此有人,頭痛,筋攣骨重,怯然少氣,噦噫腹滿,時驚,不嗜臥,此何藏之發也?脈浮而弦,切之石堅,不知其解,復問所以三藏者,以知其比類也。帝曰:夫從容之謂也。夫年長則求之於府,年少則求之於經,年壯則求之於藏。今子所言皆失,八風菀熟,五藏消爍,傳邪相受。夫浮而弦者,是腎不足也。沉而石者,是腎氣內著也。怯然少氣者,是水道不行,形氣消索也。咳嗽煩冤者,是腎氣之逆也。一人之氣,病在一藏也。若言三藏俱行,不在法也。
  雷公曰:於此有人,四支解墯,咳喘血洩,而愚診之,以為傷肺,切脈浮大而緊,愚不敢治,粗工下砭石,病癒多出血,血止身輕,此何物也?帝曰:子所能治,知亦眾多,與此病失矣。譬以鴻飛,亦沖於天。夫聖人之治病,循法守度,援物比類,化之冥冥,循上及下,何必守經。今夫脈浮大虛者,是脾氣之外絕,去胃外歸陽明也。夫二火不勝三水,是以脈亂而無常也。四支解墯,此脾精之不行也。咳喘者,是水氣並陽明也。血洩者,脈急血無所行也。若夫以為傷肺者,由失以狂也。不引比類,是知不明也。夫傷肺者,脾氣不守,胃氣不清,經氣不為使,真藏壞決,經脈傍絕,五藏漏洩,不衄則嘔,此二者不相類也。譬如天之無形,地之無理,白與黑相去遠矣。是失,吾過矣。以子知之,故不告子,明引比類從容,是以名曰診輕,是謂至道也。


\section{疏五過論篇第七十七}

  黃帝曰:嗚呼遠哉!閔閔乎若視深淵,若迎浮雲,視深淵尚可測,迎浮雲莫知其際。聖人之術,為萬民式,論裁志意,必有法則,循經守數,接循醫事,為萬民副,故事有五過四德,汝知之乎?雷公避席再拜曰:臣年幼小,蒙愚以惑,不聞五過與四德,比類形名,虛引其經,心無所對。
  帝曰:凡未診病者,必問嘗貴後賤,雖不中邪,病從內生,名曰脫營。嘗富後貧,名曰失精,五氣留連,病有所並。醫工診之,不在藏府,不變軀形,診之而疑,不知病名。身體日減,氣虛無精,病深無氣,灑灑然時驚,病深者,以其外耗於衛,內奪於榮。良工所失,不知病情,此亦治之一過也。
  凡欲診病者,必問飲食居處,暴樂暴苦,始樂後苦,皆傷精氣,精氣竭絕,形體毀沮。暴怒傷陰,暴喜傷陽,厥氣上行,滿脈去形。愚醫治之,不知補寫,不知病情,精華日脫,邪氣乃並,此治之二過也。
  善為脈者,必以比類奇恆,從容知之,為工而不知道,此診之不足貴,此治之三過也。
  診有三常,必問貴賤,封君敗傷,及欲侯王。故貴脫勢,雖不中邪,精神內傷,身必敗亡。始富後貧,雖不傷邪,皮焦筋屈,痿躄為攣。醫不能嚴,不能動神,外為柔弱,亂至失常,病不能移,則醫事不行,此治之四過也。
  凡診者必知終始,有知余緒,切脈問名,當合男女。離絕菀結,憂恐喜怒,五藏空虛,血氣離守,工不能知,何術之語。嘗富大傷,斬筋絕脈,身體復行,令澤不息。故傷敗結積,留薄歸陽,膿積寒炅。粗工治之,亟刺陰陽,身體解散,四支轉筋,死日有期,醫不能明,不問所發,唯言死日,亦為粗工,此治之五過也。
  凡此五者,皆受術不通,人事不明也。故曰:聖人之治病也,必知天地陰陽,四時經紀,五藏六府,雌雄表裡,刺灸砭石,毒藥所主,從容人事,以明經道,貴賤貧富,各異品理,問年少長,勇怯之理,審於分部,知病本始,八正九候,診必副矣。
  治病之道,氣內為寶,循求其理,求之不得,過在表裡。守數據治,無失俞理,能行此術,終身不殆。不知俞理,五藏菀熟,癰發六府,診病不審,是謂失常。謹守此治,與經相明,《上經》《下經》,揆度陰陽,奇恆五中,決以明堂,審於終始,可以橫行。


\section{徵四失論篇第七十八}

  黃帝在明堂,雷公侍坐,黃帝曰:夫子所通書受事眾多矣,試言得失之意,所以得之,所以失之。雷公對曰:循經受業,皆言十全,其時有過失者,請聞其事解也。
  帝曰:子年少,智未及邪,將言以雜合耶?夫經脈十二,絡脈三百六十五,此皆人之所明知,工之所循用也。所以不十全者,精神不專,志意不理,外內相失,故時疑殆。診不知陰陽逆從之理,此治之一失矣。
  受師不卒,妄作雜術,謬言為道,更名自功,妄用砭石,後遺身咎,此治之二失也。
  不適貧富貴賤之居,坐之薄厚,形之寒溫,不適飲食之宜,不別人之勇怯,不知比類,足以自亂,不足以自明,此治之三失也。
  診病不問其始,憂患飲食之失節,起居之過度,或傷於毒,不先言此,卒持寸口,何病能中,妄言作名,為所窮,此治之四失也。
  是以世人之語者,馳千里之外,不明尺寸之論,診無人事。治數之道,從容之葆,坐持寸口,診不中五脈,百病所起,始以自怨,遺師其咎。是故治不能循理,棄術於市,妄治時愈,愚心自得。嗚呼!窈窈冥冥,熟知其道?道之大者,擬於天地,配於四海,汝不知道之諭,受以明為晦。

\section{陰陽類論篇第七十九}

  孟春始至,黃帝燕坐,臨觀八極,正八風之氣,而問雷公曰:陰陽之類,經脈之道,五中所主,何藏最貴?雷公對曰:春甲乙青,中主肝,治七十二日,是脈之主時,臣以其藏最貴。帝曰:卻念上下經,陰陽從容,子所言最貴,其下也。
  雷公致齋七日,旦復侍坐。帝曰:三陽為經,二陽為維,一陽為游部,此知五藏終始。三陽為表,二陰為裡,一陰至絕,作朔晦,卻具合以正其理。雷公曰:受業未能明。
  帝曰:所謂三陽者,太陽為經,三陽脈,至手太陰,弦浮而不沉,決以度,察以心,合之陰陽之論。所謂二陽者,陽明也,至手太陰,弦而沉急不鼓,炅至以病皆死。一陽者,少陽也,至手太陰,上連人迎,弦急懸不絕,此少陽之病也,專陰則死。
  三陰者,六經之所主也,交於太陰,伏鼓不浮,上空志心。二陰至肺,其氣歸膀胱,外連脾胃。一陰獨至,經絕,氣浮不鼓,鉤而滑。此六脈者,乍陰乍陽,交屬相併,繆通五藏,合於陰陽,先至為主,後至為客。
  雷公曰:臣悉盡意,受傳經脈,頌得從容之道,以合《從容》,不知陰陽,不知雌雄。帝曰:三陽為父,二陽為衛,一陽為紀。三陰為母,二陰為雌,一陰為獨使。
  二陽一陰,陽明主病,不勝一陰,軟而動,九竅皆沉。三陽一陰,太陽脈勝,一陰不能止,內亂五藏,外為驚駭。二陰二陽,病在肺,少陰脈沉,勝肺傷脾,外傷四支。二陰二陽皆交至,病在腎,罵詈妄行,巔疾為狂。二陰一陽,病出於腎,陰氣客遊於心脘,下空竅堤,閉塞不通,四支別離。一陰一陽代絕,此陰氣至心,上下無常,出入不知,喉咽乾燥,病在土脾。二陽三陰,至陰皆在,陰不過陽,陽氣不能止陰,陰陽並絕,浮為血瘕,沉為膿胕。陰陽皆壯,下至陰陽。上合昭昭,下合冥冥,診決生死之期,遂合歲首。
  雷公曰:請問短期。黃帝不應。雷公復問。黃帝曰:在經論中。雷公曰:請聞短期。黃帝曰:冬三月之病,病合於陽者,至春正月脈有死徵,皆歸出春。冬三月之病,在理已盡,草與柳葉皆殺,春陰陽皆絕,期在孟春。春三月之病,曰陽殺,陰陽皆絕,期在草干。夏三月之病,至陰不過十日,陰陽交,期在溓水。秋三月之病,三陽俱起,不治自已。陰陽交合者,立不能坐,坐不能起。三陽獨至,期在石水。二陰獨至,期在盛水。

\section{方盛衰論篇第八十}

  雷公請問氣之多少,何者為逆,何者為從。黃帝答曰:陽從左,陰從右,老從上,少從下。是以春夏歸陽為生,歸秋冬為死,反之則歸秋冬為生,是以氣多少,逆皆為厥。
  問曰:有餘者厥耶?答曰:一上不下,寒厥到膝,少者秋冬死,老者秋冬生。氣上不下,頭痛巔疾,求陽不得,求陰不審,五部隔無徵,若居曠野,若伏空室,綿綿乎屬不滿日。
  是以少氣之厥,令人妄夢,其極至迷。三陽絕,三陰微,是為少氣。
  是以肺氣虛,則使人夢見白物,見人斬血藉藉,得其時,則夢見兵戰。腎氣虛,則使人夢見舟船溺人,得其時,則夢伏水中,若有畏恐。肝氣虛,則夢見菌香生草,得其時,則夢伏樹下不敢起。心氣虛,則夢救火陽物,得其時,則夢燔灼。脾氣虛,則夢飲食不足,得其時,則夢築垣蓋屋。此皆五藏氣虛,陽氣有餘,陰氣不足。合之五診,調之陰陽,以在經脈。
  診有十度度人:脈度,藏度,肉度,筋度,俞度。陰陽氣盡,人病自具。脈動無常,散陰頗陽,脈脫不具,診無常行,診必上下,度民君卿。受師不卒,使術不明,不察逆從,是為妄行,持雌失雄,棄陰附陽,不知併合,診故不明,傳之後世,反論自章。
  至陰虛,天氣絕;至陽盛,地氣不足。陰陽並交,至人之所行。陰陽交並者,陽氣先至,陰氣後至。是以聖人持診之道,先後陰陽而持之,《奇恆之勢》乃六十首,診合微之事,追陰陽之變,章五中之情,其中之論,取虛實之要,定五度之事,知此乃足以診。是以切陰不得陽,診消亡,得陽不得陰,守學不湛,知左不知右,知右不知左,知上不知下,知先不知後,故治不久。知醜知善,知病知不病,知高知下,知坐知起,知行知止,用之有紀,診道乃具,萬世不殆。起所有餘,知所不足。度事上下,脈事因格。是以形弱氣虛,死;形氣有餘,脈氣不足,死。脈氣有餘,形氣不足,生。
  是以診有大方,坐起有常,出入有行,以轉神明,必清必淨,上觀下觀,司八正邪,別五中部,按脈動靜,循尺滑濇,寒溫之意,視其大小,合之病能,逆從以得,復知病名,診可十全,不失人情。故診之,或視息視意,故不失條理,道甚明察,故能長久。不知此道,失經絕理,亡言妄期,此謂失道。

\section{解精微論篇第八十一}

  黃帝在明堂,雷公請曰:臣授業,傳之行教以經論,從容形法,陰陽刺灸,湯藥所滋,行治有賢不肖,未必能十全。若先言悲哀喜怒,燥濕寒暑,陰陽婦女,請問其所以然者,卑賤富貴,人之形體,所從群下,通使臨事以適道術,謹聞命矣。請問有毚愚僕漏之問,不在經者,欲聞其狀。帝曰:大矣。
  公請問:哭泣而淚不出者,若出而少涕,其故何也?帝曰:在經有也。復問:不知水所從生,涕所出也。帝曰:若問此者,無益於治也,工之所知,道之所生也。
  夫心者,五藏之專精也,目者,其竅也,華色者,其榮也,是以人有德也,則氣和於目,有亡,憂知於色。是以悲哀則泣下,泣下水所由生。水宗者,積水也,積水者,至陰也,至陰者,腎之精也。宗精之水所以不出者,是精持之也。輔者裹之,故水不行也。夫水之精為志,火之精為神,水火相感,神志俱悲,是以目之水生也。故諺言曰:心悲名曰志悲,志與心精共湊於目也。是以俱悲則神氣傳於心,精上不傳於志,而志獨悲,故泣出也。泣涕者,腦也,腦者,陰也,髓者,骨之充也,故腦滲為涕。志者骨之主也,是以水流而涕從之者,其行類也。夫涕之與泣者,譬如人之兄弟,急則俱死,生則俱生,其志以神悲,是以涕泣俱出而橫行也。夫人涕泣俱出而相從者,所屬之類也。
  雷公曰:大矣。請問人哭泣而淚不出者,若出而少,涕不從之何也?帝曰:夫泣不出者,哭不悲也。不泣者,神不慈也。神不慈則志不悲,陰陽相持,泣安能獨來。夫志悲者惋,惋則沖陰,沖陰則志去目,志去則神不守精,精神去目,涕泣出也。
  且子獨不誦不念夫經言乎,厥則目無所見。夫人厥則陽氣並於上,陰氣並於下。陽並於上,則火獨光也;陰並於下則足寒,足寒則脹也。夫一水不勝五火,故目盲。是以沖風,泣下而不止。夫風之中目也,陽氣內守於精,是火氣燔目,故見風則泣下也。有以比之,夫火疾風生乃能雨,此之類也。




\section{刺法論篇第七十二(遺篇)}

黃帝問曰:升降不前,氣交有變,即成暴郁,余已知之。何如預救生靈,可得卻乎?岐伯稽首再拜對曰:昭乎哉問!臣聞夫子言,既明天元,須窮刺法,可以折郁扶運,補弱全真,寫盛蠲余,令除斯苦。

帝曰:願卒聞之。岐伯曰:升之不前,即有期凶也。木欲升而天柱窒抑之,木欲發郁,亦須待時,當刺足厥陰之井。火欲升而天蓬窒抑之,火欲發郁,亦須待時,君火相火同刺包絡之熒。土欲升而天沖窒抑之,土欲發郁,亦須待時,當刺足太陰之俞。金欲升而天英窒抑之,金欲發郁,亦須待時,當刺手太陰之經。水欲升而天芮窒抑之,水欲發郁,亦須待時,當刺足少陰之合。

帝曰:升之不前,可以預備,願聞其降,可能先防。岐伯曰:既明其升。必達其降也,升降之道,皆可先治也。木欲降而地晶窒抑之,降而不入,抑之郁發,散而可得位,降而郁發,暴如天間之待時也。降而不下,郁可速矣,降可折其所勝也,當刺手太陰之所出,刺手陽明之所入。火欲降,而地玄窒抑之,降而不入,抑之郁發,散而可矣。當折其所勝,可散其郁,當刺足少陰之所出,刺足太陽之所入。土欲降而地蒼窒抑之,降而不下,抑之郁發,散而可入,當折其勝,可散其郁,當刺足厥陰之所出,刺足少陽之所入,金欲降而地彤窒抑,降而不下,抑之郁發,散而可入,當折其勝,可散其郁,當刺心包絡所出,制手少陽所入也。水欲降而地阜窒抑之,降而不下,抑之郁發,散而可入,當折其土,可散其郁,當刺足太陰之所出,刺足陽明之所入。

帝曰:五運之至有前後,與升降往來,有所承抑之,可得聞乎刺法?岐伯曰:當取其化源也。是故太過取之,不及資之,太過取之,次抑其郁,取其運之化源,令折郁氣;不及扶資,以扶運氣,以避虛邪也。資取之法,令出《密語》。黃帝問曰:升降之刺,以知其要。願聞司天未得遷正,使司化之失其常政,即萬化之或其皆妄,然與民為病,可得先除,欲濟群生,願聞其說。岐伯稽首再拜曰:悉乎哉問!言其至理,聖念慈憫,欲濟群生,臣乃盡陳斯道,可申洞微。太陽復布,即厥陰不遷正,不遷正,氣塞於止,當寫足厥陰之所流。厥陰復布,少陰不遷正,不遷正,即氣塞於上,當刺心包絡脈之所流。少陰復布,太陰不遷正,不遷正,即氣留於上,當刺足太陰之所流。太陰復布,少陽不遷正,不遷正,則氣塞未通,當刺手少陽之所流。少陽復布,則陽明不遷正,不遷正,則氣未通上,當刺手太陰之所流。陽明復布,太陽遷正,不遷正,則復塞其氣,當刺足少陰之所流。

帝曰:遷正不前,以通其要。願聞不退,欲折其餘,無令過失,可得明乎?岐伯曰:氣過有餘,復作布正,是名不退位也。使地氣不得後化,新司天未可遷正,故復布化令如故也。巳亥之歲,天數有餘,故厥陰不退位也,風行於上,木化布天,當刺足厥陰之所入。子午之歲,天數有餘,故少陰不退位也,熱行於上,火余化布天,當刺手厥陰之所入。丑未之歲,天數有餘,故太陰不退位也,濕行於上,雨化布天,當刺足太陰之所入。寅申之歲,天數有餘,故少陽不退位也,熱行於上,火化布天,當刺手少陽所入。卯酉之歲,天數有餘,故陽明不退位也,金行於上,燥化布天,當刺手太陰之所入。辰戌之歲,天數有餘,故太陽不退位也,寒行於上,凜水化布天,當刺足少陰之所入。故天地氣逆,化成民病,以法刺之,預可平痾。

黃帝問曰:剛柔二干,失守其位,使天運之氣皆虛乎?與民為病,可得平乎?岐伯曰:深乎哉問!明其奧旨,天地迭移,三年化疫,是謂根之可見,必有逃門。

假令甲子剛柔失守,剛未正,柔孤而有虧,時序不令,即音律非從,如此三年,變大疫也。詳其微甚。察其淺深,欲至而可刺,刺之當先補腎俞,次三日,可刺足太陰之所注。又有下位已卯不至,而甲子孤立者,次三年作土癘,其法補寫,一如甲子同法也。其刺以畢,又不須夜行及遠行,令七日潔,清靜齋戒,所有自來。腎有久痛者,可以寅時面向南,淨神不亂思,閉氣不息七遍,以引頸嚥氣順之,如咽甚硬物,如此七遍後,餌舌下津令無數。

假令丙寅剛柔失守,上剛干失守,下柔不可獨主之,中水運非太過,不可執法而定之。布天有餘,而失守上正,天地不合,即律呂音異,如此即天運失序,後三年變疫。詳其微甚,差有大小,徐至即後三年,至甚即首三年,當先補心俞,次五日,可刺腎之所入。又有下位地甲子辛已柔不附剛,亦名失守,即地運皆虛,後三年變水癘,即刺法皆如此矣。其刺如華,慎其大喜欲情於中,如不忌,即其氣復散也,令靜七日,心欲實,令少思。

假令庚辰剛柔失守,上位失守,下位無合,乙庚金運,故非相招,布天未退,中運勝來,上下相錯,謂之失守,姑洗林鐘,商音不應也。如此則天運化易,三年變大疫。詳天數,差的微甚,微即微,三年至,甚即甚,三年至,當先補肝俞,次三日,可刺肺之所行。刺畢,可靜神七日,慎勿大怒,怒必真氣卻散之。又或在下地甲子乙未失守者,即乙柔干,即上庚獨治之,亦名失守者,即天運孤主之,三年變癘,名曰金癘,其至待時也。詳其地數之等差,亦推其微甚,可知遲速耳。諸位乙庚失守,刺法同。肝欲平,即勿怒。

假令壬午剛柔失守,上壬未近正,下丁獨然,即雖陽年,虧及不同,上下失守,相招其有期,差之微甚,各有其數也,律呂二角,失而不和,同音有日,微甚如見,三年大疫。當刺脾之俞,次三日,可刺肝之所出也。刺畢,靜神七日,勿大醉歌樂,其氣復散,又勿飽食,勿食生物,欲令脾實,氣無滯飽,無久坐,食無太酸,無食一切生物,宜甘宜淡。又或地下甲子丁酉失守其位,未得中司,即氣不當位,下不與壬奉合者,亦名失守,非名合德,故柔不附剛,即地運不合,三年變癘,其刺法亦如木疫之法。

假令戊申剛柔失守,戊癸雖火運,陽年不太過也,上失其剛,柔地獨主,其氣不正,故有邪干,迭移其位,差有淺深,欲至將合,音律先同,如此天運失時,三年之中,火疫至矣,當刺肺之俞。刺畢,靜神七日,勿大悲傷也,悲傷即肺動,而其氣復散也,人欲實肺者,要在息氣也。又或地下甲子癸亥失守者,即柔失守位也,即上失其剛也。即亦名戊癸不相合德者也,即運與地虛,後三年變癘,即名火癘。

是故立地五年,以明失守,以窮法刺,於是疫之與癘,即是上下剛柔之名也,窮歸一體也。即刺疫法,只有五法,即總其諸位失守,故只歸五行而統之也。

黃帝曰:余聞五疫之至,皆相梁易,無問大小,病狀相似,不施救療,如何可得不相移易者?岐伯曰:不相染者,正氣存內,邪氣可幹,避其毒氣,天牝從來,復得其往,氣出於腦,即不邪干。氣出於腦,即室先想心如日,欲將入於疫室,先想青氣自肝而出,左行於東,化作林木;次想白氣自肺而出,右行於西,化作戈甲;次想赤氣自心而出,南行於上,化作焰明;次想黑氣自腎而出,北行於下,化作水;次想黃氣自脾而出,存於中央,化作土。五氣護身之畢,以想頭上如北斗之煌煌,然後可入於疫室。又一法,於春分之日,日未出而吐之。又一法,於雨水日後,三浴以藥洩汗。又一法,小金丹方:辰砂二兩,水磨雄黃一兩,葉子雌黃一兩,紫金半兩,同入合中,外固,了地一尺築地實,不用爐,不須藥製,用火二十斤鍛了也;七日終,候冷七日取,次日出合子埋藥地中,七日取出,順日研之三日,煉白沙蜜為丸,如梧桐子大,每日望東吸日華氣一口,冰水一下丸,和氣咽之,服十粒,無疫干也。

黃帝問曰:人虛即神遊失守位,使鬼神外干,是致夭亡,何以全真?願聞刺法。岐伯稽首再拜曰:昭乎哉問!謂神移失守,雖在其體,然不致死,或有邪干,故令夭壽。只如厥陰失守,天以虛,人氣肝虛,感天重虛。即魂遊於上,邪干,厥大氣,身溫猶可刺之,制其足少陽之所過,次刺肝之俞。人病心虛,又遇群相二火司天失守,感而三虛,遇火不及,黑屍鬼犯之,令人暴亡,可刺手少陽之所過,復刺心俞。人脾病,又遇太陰司天失守,感而三虛,又遇土不及,青屍鬼邪,犯之於人,令人暴亡,可刺足陽明之所過,復刺脾之俞。人肺病,遇陽明司天失守,感而三虛,又遇金不及,有赤屍鬼犯人,令人暴亡,可刺手陽明之所過,復刺肺俞。人腎病,又遇太陽司天失守,感而三虛,又遇水運不及之年,有黃屍鬼,干犯人正氣,吸人神魂,致暴亡,可刺足太陽之所過,復刺腎俞。

黃帝問曰:十二藏之相使,神失位,使神彩之不圓,恐邪干犯,治之可刺?願聞其要。岐伯稽首再拜曰:悉乎哉問!至理道真宗,此非聖帝,焉窮斯源,是謂氣神合道,契符上天。心者,君主之官,神明出焉,可刺手少陰之源。肺者,相傅之官,治節出焉,可刺手太陰之源。肝者,將軍之官,謀虛出焉,可刺足厥陰之源。膽者,中正不官,決斷出焉,可刺足少陽之源。羶中者,臣使之官,喜樂出焉,可刺心包絡所流。脾為諫議之官,知周出焉,可刺脾之源。胃為倉廩之官,五味出焉,可刺胃之源。大腸者,傳道之官,變化出焉,可刺大腸之源。小腸者,受盛之官,化物出焉,可刺小腸之源。腎者,作強之官,伎巧出焉,刺其腎之源。三焦者,決瀆之官,水道出焉,刺三焦之源。膀胱者,州都之官,津液藏焉,氣化則能出矣,刺膀胱之源。凡此十二官者,不得相失也。是故刺法有全神養真之旨,亦法有修真之道,非治疾也。故要修養和神也,道貴常存,補神固根,精氣不散,神守不分,然即神守而雖不去,亦能全真,人神不守,非達至真,至真之要,在乎天玄,神守天息,復入本元,命曰歸宗。

\section{本病論篇第七十三(遺篇)}

黃帝問曰:天元九窒,余已知之,願聞氣交,何名失守?岐伯曰:謂其上下升降,遷正退位,各有經論,上下各有不前,故名失守也。是故氣交失易位,氣交乃變,變易非常,即四失序,萬化不安,變民病也。

帝曰:升降不前,願聞其故,氣交有變,何以明知?岐伯曰:昭乎哉問,明乎道矣?氣交有變,是謂天地機,但欲降而不得降者,地窒刑之。又有五運太過,而先天而至者,即交不前,但欲升而不得其升,中運抑之,但欲降而不得其降,中運抑之。於是有升之不前,降之不下者,有降之不下,升而至天者,有升降俱不前,作如此之分別,即氣交之變。變之有異,常各各不同,災有微甚者也。

帝曰:願聞氣交遇會勝抑之由,變成民病,輕重何如?岐伯曰:勝相會,抑伏使然。是故辰戌之歲,木氣升之,主逢天柱,勝而不前;又遇庚戌,金運先天,中運勝之忽然不前,木運升天,金乃抑之,升而不前,即清生風少,肅殺於春,露霜復降,草木乃萎。民病溫疫早發,咽嗌乃干,四肢滿,肢節皆痛;久而化郁,即大風摧拉,折隕鳴紊。民病卒中偏痹,手足不仁。

是故巳亥之歲,君火升天,主窒天蓬,勝之不前;又厥陰未遷正,則少陰未得升天,水運以至其中者,君火欲升,而中水運抑之,升之不前,即清寒復作,冷生旦暮。民病伏陽,而內生煩熱,心神驚悸,寒熱間作;日久成郁,即暴熱乃至,赤風瞳翳,化疫,溫癘暖作,赤氣彰而化火疫,皆煩而燥渴,渴甚,治之以洩之可止。

是故子午之歲,太陰升天,主窒天沖,勝之不前;又或遇壬子,木運先天而至者,中木運抑之也,升天不前,即風埃四起,時舉埃昏,雨濕不化。民病風厥涎潮,偏痹不隨,脹滿;久而伏郁,即黃埃化疫也。民病夭亡,臉肢府黃疸滿閉。濕令弗布,雨化乃微。

是故丑未之年,少陽升天,主窒天蓬,勝之不前;又或遇太陰未遷正者,即少陰未升天也,水運以至者,升天不前,即寒冰反布,凜冽如冬,水復涸,冰再結,暄暖乍作,冷夏布之,寒暄不時。民病伏陽在內,煩熱生中,心神驚駭,寒熱間爭;以久成郁,即暴熱乃生,赤風氣腫翳,化成疫癘,乃化作伏熱內煩,痹而生厥,甚則血溢。

是故寅申之年,陽明升天,主窒天英,勝之不前;又或遇戊申戊寅,火運先天而至;金欲升天,火運抑之,升之不前。即時雨不降,西風數舉,鹹鹵燥生。民病上熱喘嗽,血溢;久而化郁,即白埃翳霧,清生殺氣,民病脅滿,悲傷,寒鼽嚏,嗌干,手坼皮膚燥。

是故卯酉之年,太陽升天,主窒天芮,勝之不前;又遇陽明未遷正者,即太陽未升天也,土運以至,水欲升天,土運抑之,升之不前,即濕而熱蒸,寒生兩間。民病注下,食不及化;久而成郁,冷來客熱,冰雹卒至。民病厥逆而噦,熱生於內,氣痹於外,足脛痠疼,反生心悸,懊熱,暴煩而復厥。

黃帝曰:升之不前,余已盡知其旨,願聞降之不下,可得明乎?岐伯曰:悉乎哉問也!是之謂天地微旨,可以盡陳斯道。所謂升已必降也,至天三年,次歲必降,降而入地,始為左間也。如此升降往來,命之六紀也。

是故丑未之歲,厥陰降地,主窒地晶,勝而不前;又或遇少陰未退位,即厥陰未降下,金運以至中,金運承之,降之未下,抑之變郁,木欲降下,金運承之,降而不下,蒼埃遠見,白氣承之,風舉埃昏,清燥行殺,霜露復下,肅殺布令。久而不降,抑之化郁,即作風燥相伏,暄而反清,草木萌動,殺霜乃下,蟄蟲未見,懼清傷藏。

是故寅申之歲,少陰降地,主窒地玄,勝之不入;又或遇丙申丙寅,水運太過,先天而至,君火欲降,水運承之,降而不下,即彤雲才見,黑氣反生,暄暖如舒,寒常布雪,凜冽復作,天雲慘淒。久而不降,伏之化郁,寒勝復熱,赤風化疫,民病面赤、心煩、頭痛、目眩也,赤氣彰而溫病欲作也。

是故卯酉之歲,太陰降地,主窒地蒼,勝之不入;又或少陽未退位者,即太陰未得降也;或木運以至,木運承之,降而不下,即黃雲見而青霞彰,郁蒸作而大風,霧翳埃勝,折隕乃作。久而不降也,伏之化郁,天埃黃氣,地布濕蒸。民病四肢不舉、昏眩、肢節痛、腹滿填臆。

是故辰戌之歲,少陽降地,主窒地玄,勝之不入;又或遇水運太過,先天而至也,水運承之,降而不下,即彤雲才見,黑氣反生,暄暖欲生,冷氣卒至,甚則冰雹也。久而不降,伏之化郁,冰氣復熱,赤風化疫,民病面赤、心煩、頭痛、目眩也,赤氣彰而熱病欲作也。

是故巳亥之歲,陽明降地,主窒地彤,用而不入;又或遇太陽未退位,即陽明未得降;即火運以至之,火運承之不下,即天清而肅,赤氣乃彰,暄熱反作。民皆錯倦,夜臥不安,咽乾引飲,懊熱內煩,天清朝暮,暄還復作;久而不降,伏之化郁,天清薄寒,遠生白氣。民病掉眩,手足直而不仁,兩脅作痛,滿目 然。

是故子午之年,太陽降地,主窒地阜勝之,降而不入;又或遇土運太過,先天而至,土運承之,降而不入,即天彰黑氣,暝暗淒慘,才施黃埃而布濕,寒化令氣,蒸濕復令。久而不降,伏之化郁,民病大厥,四肢重怠,陰痿少力,天布沉陰,蒸濕間作。

帝曰:升降不前,晰知其宗,願聞遷正,可得明乎?岐伯曰:正司中位,是謂遷正位,司天不得其遷正者,即前司天,以過交司之日,即遇司天太過有餘日也,即仍舊治天數,新司天未得遷正也。

厥陰不遷正,即風暄不時,花卉萎瘁。民病淋溲,目系轉,轉筋,喜怒,小便赤。風欲令而寒由不去,溫暄不正,春正失時。

少陰不遷正,即冷氣不退,春冷後寒,暄暖不時。民病寒熱,四肢煩痛,腰脊強直。木氣雖有餘,而位不過於君火也。

太陰不遷正,即雲雨失令,萬物枯焦,當生不發。民病手足肢節腫滿,大腹水腫,填臆不食,飧洩脅滿,四肢不舉。雨化欲令,熱猶治之,溫煦於氣,亢而不澤。

少陽不遷正,即炎灼弗令,苗莠不榮,酷暑於秋,肅殺晚至,霜露不時。民病痎瘧,骨熱,心悸,驚駭;甚時血溢。

陽明不遷正,則暑化於前,肅殺於後,草木反榮。民病寒熱,鼽嚏,皮毛折,爪甲枯焦;甚則喘嗽息高,悲傷不樂。熱化乃布,燥化未令,即清勁未行,肺金復病。

陽明不遷正,即冬清反寒,易令於春,殺霜在前,寒冰於後,陽光復治,凜冽不作,民病溫癘至,喉閉嗌干,煩躁而渴,喘息而有音也。寒化待燥,猶治天氣,過失序,與民作災。

帝曰:遷正早晚,以命其旨,願聞退位,可得明哉?岐伯曰:所謂不退者,即天數未終,即天數有餘,名曰復布政,故名曰再治天也。即天令如故,而不退位也。

厥陰不退位,即大風早舉,時雨不降,濕令不化,民病溫疫,疵廢,風生,皆肢節痛,頭目痛,伏熱內煩,咽喉乾引飲。

少陰不退位,即溫生春冬,蟄蟲早至,草木發生,民病膈熱,咽干,血溢,驚駭,小便赤澀,丹瘤,瘡瘍留毒。

太陰不退位,而取寒暑不時,埃昏布作,濕令不去,民病四肢少力,食飲不下,洩注淋滿,足脛寒,陰痿,閉塞,失溺,小便數。

少陽不退位,即熱生於春,暑乃後化,冬溫不凍,流水不冰,蟄蟲出見,民病少氣,寒熱更作,便血,上熱,小腹堅滿,小便赤沃,甚則血溢。

陽明不退位,即春生清冷,草木晚榮,寒熱間作。民病嘔吐,暴注,食飲不下,大便乾燥,四肢不舉,目瞑掉眩。

太陽不退位,即春寒夏作,冷雹乃降,沉陰昏翳,二之氣寒猶不去。民病痹厥,陰痿,失溺,腰膝皆痛,溫癘晚發。

帝曰:天歲早晚,余已知之,願聞地數,可得聞乎?岐伯曰:地下遷正、升天及退位不前之法,即地土產化,萬物失時之化也。

帝曰:余聞天地二甲子,十干十二支,上下經緯天地,數有迭移,失守其位,可得昭乎?岐伯曰:失之迭位者,謂雖得歲正,未得正位之司,即四時不節,即生大疫。注《玄珠密語》云:陽年三十年,除六年天刑,計有太過二十四年,除此六年,皆作太過之用。令不然之旨,今言迭支迭位,皆可作其不及也。

假令甲子陽年,土運太窒,如癸亥天數有餘者,年雖交得甲子,厥陰猶尚治天,地已遷正,陽明在泉,去歲少陽以作右間,即厥陰之地陽明,故不相和奉者也。癸巳相會,土運太過,虛反受木勝,故非太過也,何以言土運太過,況黃鍾不應太窒,木即勝而金還復,金既復而少陰如至,即木勝如火而金復微,如此則甲已失守,後三年化成土疫,晚至丁卯,早至丙寅,土疫至也,大小善惡,推其天地,詳乎太乙。又只如甲子年,如甲至子而合,應交司而治天,即下己卯未遷正,而戊寅少陽未退位者,亦甲已下有合也,即土運非太過,而木乃乘虛而勝土也,金次又行復勝之,即反邪化也。陰陽天地殊異爾,故其大小善惡,一如天地之法旨也。

假令丙寅陽年太過,如乙丑天數有餘者,雖交得丙寅,太陰尚治天也。地已遷正,厥陰司地,去歲太陽以作右間,即天太陰而地厥陰,故地不奉天化也。乙辛相會,水運太虛,反受土勝,故非太過,即太簇之管,太羽不應,土勝而雨化,木復即風,此者丙辛失守其會,後三年化成水疫,晚至己巳,早至戊辰,甚即速,微即徐,水疫至也,大小善惡,推其天地數乃太乙游宮。又只如丙寅年,丙至寅且合,應交司而治天,即辛巳未得遷正,而庚辰太陽未退位者,亦丙辛不合德也,即水運亦小虛而小勝,或有復,後三年化癘,名曰水癘,其狀如水疫。治法如前。假令庚辰陽年太過,如己卯天數有餘者,雖交得庚辰年也,陽明猶尚治天,地已遷正,太陰司地,去歲少陰以作右間,即天陽明而地太陰也,故地不奉天也。乙巳相會,金運太虛,反受火勝,故非太過也,即姑洗之管,太商不應,火勝熱化,水復寒刑,此乙庚失守,其後三年化成金疫也,速至壬午,徐至癸未,金疫至也,大小善惡,推本年天數及太乙也。又只如庚辰,如庚至辰,且應交司而治天,即下乙未得遷正者,即地甲午少陰未退位者,且乙良不合德也,即下乙未柔干失剛,亦金運小虛也,有小勝或無復,且三年化癘,名曰金癘,其狀如金疫也。治法如前。

假令壬午陽年太過,如辛巳天數有餘者,雖交得壬午年也,厥陰猶尚治天,地已遷正,陽明在泉,去歲丙申少陽以作右間,即天厥陰而地陽明,故地不奉天者也。丁辛相合會,木運太虛,反受金勝,故非太過也,即蕤賓之管,太角不應,金行燥勝,火化熱復,甚即速,微即徐。疫至大小善惡,推疫至之年天數及太乙。又只如壬至午,且應交司而治之,即下丁酉未得遷正者,即地下丙申少陽未得退位者,見丁壬不合德也,即丁柔干失賜,亦木運小虛也,有小勝小復。後三年化癘,名曰木癘,其狀如風疫也。治法如前。

假令戊申陽年太過,如丁未天數太過者,雖交得戊申年也。太陰猶尚司天,地已遷正,厥陰在泉,去歲壬戌太陽以退位作右間,即天丁未,地癸亥,故地不奉天化也。丁癸相會,火運太虛,反受水勝,故非太過也,即夷則之管,上太徵不應,此戊癸失守其會,後三年化疫也,速至庚戌,大小善惡,推疫至之年天數及太乙。又只如戊申,如戊至申,且應交司治天,即下癸亥未得遷正者,即地下壬戌太陽未退者,見戊癸亥未合德也,即下癸柔干失剛,見火運小虛,有小勝或無復也,後三年化癘,名曰火癘也。治法如前;治之法,可寒之洩之。

黃帝曰:人氣不足,天氣如虛,人神失守,神光不聚,邪鬼干人,致有夭亡,可得聞乎?岐伯曰:人之五藏,一藏不足,又會天虛,感邪之至也。人憂愁思慮即傷心,又或遇少陰司天,天數不及,太陰作接間至,即謂天虛也,此即人氣天氣同虛也。又遇驚而奪精,汗出於心,因而三虛,神明失守。心為群主之官,神明出焉,神失守位,即神遊上丹田,在帝太一帝群泥丸宮一下。神既失守,神光不聚,卻遇火不及之歲,有黑屍鬼見之,令人暴亡。

人飲食、勞倦即傷脾,又或遇太陰司天,天數不及,即少陽作接間至,即謂之虛也,此即人氣虛而天氣虛也。又遇飲食飽甚,汗出於胃,醉飽行房,汗出於脾,因而三虛,脾神失守,脾為諫議之官,智周出焉。神既失守,神光失位而不聚也,卻遇土不及之年,或已年或甲年失守,或太陰天虛,青屍鬼見之,令人卒亡。

人久坐濕地,強力入水即傷腎,腎為作強之官,伎巧出焉。因而三虛,腎神失守,神志失位,神光不聚,卻遇水不及之年,或辛不會符,或丙年失守,或太陽司天虛,有黃屍鬼至,見之令人暴亡。

人或恚怒,氣逆上而不下,即傷肝也。又遇厥陰司天,天數不及,即少陰作接間至,是謂天虛也,此謂天虛人虛也。又遇疾走恐懼,汗出於肝。肝為將軍之官,謀慮出焉。神位失守,神光不聚,又遇木不及年,或丁年不符,或壬年失守,或厥陰司天虛也,有白屍鬼見之,令人暴亡也。

已上五失守者,天虛而人虛也,神遊失守其位,即有五屍鬼干人,令人暴亡也,謂之曰屍厥。人犯五神易位,即神光不圓也。非但屍鬼,即一切邪犯者,皆是神失守位故也。此謂得守者生,失守者死。得神者昌,失神者亡。


%% \part{黃帝內經·靈樞經}
%% \section{九針十二原第一}

黃帝問於岐伯曰:餘子萬民,養百姓,而收租稅。余哀其不給,而屬有疾病。余欲勿使被毒藥,無用砭石,欲以微針通其經脈,調其血氣,營其逆順出入之會。令可傳於後世,必明為之法。令終而不滅,久而不絕,易用難忘,為之經紀。異其章,別其表裡,為之終始。令各有形,先立針經,願聞其情。
岐伯答曰:臣請推而次之,令有綱紀,始於一,終於九焉。請言其道。小針之要,易陳而難入,粗守形,上守神,神乎神,客在門,未睹其疾,惡知其原。刺之微,在速遲,粗守關,上守機,機之動,不離其空,空中之機,清靜而微,其來不可逢,其往不可追。知機之道者,不可掛以發,不知機道,叩之不發。知其往來,要與之期,粗之暗乎,妙哉,工獨有之。往者為逆,來者為順,明知逆順,正行無問。逆而奪之,惡得無虛,追而濟之,惡得無實,迎之隨之,以意和之,針道畢矣。
凡用針者,虛則實之,滿則洩之,宛陳則除之,邪勝則虛之。大要曰:徐而疾則實,疾而徐則虛。言實與虛,若有若無,察後與先,若存若亡,為虛與實,若得若失。虛實之要,九針最妙,補瀉之時,以針為之。瀉曰:必持內之,放而出之,排陽得針,邪氣得洩。按而引針,是謂內溫,血不得散,氣不得出也。補曰隨之,隨之意若妄之,若行若按,如蚊虻止,如留如還,去如弦絕,令左屬右,其氣故止,外門已閉,中氣乃實,必無留血,急取誅之。持針之道,堅者為寶,正指直刺,無針左右,神在秋毫,屬意病者,審視血脈者,刺之無殆。方刺之時,必在懸陽,及與兩衛,神屬勿去,知病存亡。血脈者,在腧橫居,視之獨澄,切之獨堅。
九針之名,各不同形:一曰鑱針,長一寸六分;二曰員針,長一寸六分;三曰鍉(di2)針,長三寸半;四曰鋒針,長一寸六分;五曰鈹針,長四寸,廣二分半;六曰員利針,長一寸六分;七曰毫針,長三寸六分;八曰長針,長七寸;九曰大針,長四寸。鑱針者,頭大末銳,去瀉陽氣。員針者,針如卵形,揩摩分間,不得傷肌肉,以瀉分氣。鍉針者,鋒如黍粟之銳,主按脈勿陷,以致其氣。鋒針者,刃三隅,以發錮疾。鈹針者,末如劍鋒,以取大膿。員利針者,大如氂,且員且銳,中身微大,以取暴氣。毫針者,尖如蚊虻喙,靜以徐往,微以久留之而養,以取痛痹。長針者,鋒利身薄,可以取遠痹。大針者,尖如梃,其鋒微員,以瀉機關之水也。九針畢矣。
夫氣之在脈也,邪氣在上,濁氣在中,清氣在下。故針陷脈則邪氣出,針中脈則濁氣出,針太深則邪氣反沉,病益。故曰:皮肉筋脈各有所處,病各有所宜,各不同形,各以任其所宜。無實無虛,損不足而益有餘,是謂甚病,病益甚。取五脈者死,取三脈者恇;奪陰者死,奪陽者狂,針害畢矣。刺之而氣不至,無問其數;刺之而氣至,乃去之,勿復針。針各有所宜,各不同形,各任其所為。刺之要,氣至而有效,效之信,若風之吹雲,明乎若見蒼天,刺之道畢矣。
黃帝曰:願聞五藏六府所出之處。岐伯曰:五藏五腧,五五二十五腧;六府六腧,六六三十六腧。經脈十二,絡脈十五,凡二十七氣,以上下,所出為井,所溜為滎,所注為腧,所行為經,所入為合,二十七氣所行,皆在五腧也。節之交,三百六十五會,知其要者,一言而終,不知其要,流散無窮。所言節者,神氣之所遊行出入也,非皮肉筋骨也。觀其色,察其目,知其散復;一其形,聽其動靜,知其邪正。右主推之,左持而御之,氣至而去之。凡將用針,必先診脈,視氣之劇易,乃可以治也。五藏之氣已絕於內,而用針者反實其外,是謂重竭,重竭必死,其死也靜,治之者,輒反其氣,取腋與膺;五藏之氣已絕於外,而用針者反實其內,是謂逆厥,逆厥則必死,其死也躁,治之者,反取四末。刺之害中而不去,則精洩;害中而去,則致氣。精洩則病益甚而恇,致氣則生為癰瘍。
五藏有六府,六府有十二原,十二原出於四關,四關主治五藏。五藏有疾,當取之十二原。十二原者,五藏之所以稟三百六十五節氣味也。五藏有疾也,應出十二原,而原各有所出,明知其原,睹其應,而知五藏之害矣。陽中之少陰,肺也,其原出於太淵,太淵二。陽中之太陽,心也,其原出於大陵,大陵二。陰中之少陽,肝也,其原出於太沖,太沖二。陰中之至陰,脾也,其原出於太白,太白二。陰中之太陰,腎也,其原出於太溪,太溪二。膏之原,出於鳩尾,鳩尾一。肓之原,出於脖胦,脖胦一。凡此十二原者,主治五藏六府之有疾者也。脹取三陽,飧洩取三陰。
今夫五藏之有疾也,譬猶刺也,猶污也,猶結也,猶閉也。刺雖久,猶可拔也;污雖久,猶可雪也;結雖久,猶可解也;閉雖久,猶可決也。或言久疾之不可取者,非其說也。夫善用針者,取其疾也,猶拔刺也,猶雪污也,猶解結也,猶決閉也。疾雖久,猶可畢也。言不可治者,未得其術也。刺諸熱者,如以手探湯;刺寒清者,如人不欲行。陰有陽疾者,取之下陵三里,正往無殆,氣下乃止,不下復始也。疾高而內者,取之陰之陵泉;疾高而外者,取之陽之陵泉也。



\section{本輸第二}

黃帝問於岐伯曰:凡刺之道,必通十二經絡之所終始,絡脈之所別處,五輸之所留,六府之所與合,四時之所出入,五藏之所溜處,闊數之度,淺深之狀,高下所至。願聞其解。

岐伯曰:請言其次也。肺出於少商,少商者,手大指端內側也,為井木;溜於魚際,魚際者,手魚也,為滎;注於太淵,太淵,魚後一寸陷者中也,為腧;行於經渠,經渠,寸口中也,動而不居,為經;入於尺澤,尺澤,肘中之動脈也,為合,手太陰經也。心出於中沖,中沖,手中指之端也,為井木;溜於勞宮,勞宮,掌中中指本節之內間也,為滎;注於大陵,大陵,掌後兩骨之間方下者也,為腧;行於間使,間使之道,兩筋之間,三寸之中也,有過則至,無過則止,為經;入於曲澤,曲澤,肘內廉下陷者之中也,屈而得之,為合,手少陰也。肝出於大敦,大敦者,足大指之端及三毛之中也,為井木;溜於行間,行間,足大指間也,為滎;注於太沖,太沖,行間上二寸陷者之中也,為腧;行於中封,中封,內踝之前一寸半,陷者之中,使逆則宛,使和則通,搖足而得之,為經;入於曲泉,輔骨之下,大筋之上也,屈膝而得之,為合,足厥陰也。脾出於隱白,隱白者,足大指之端內側也,為井木;溜於大都,大都,本節之後,下陷者之中也,為滎;注於太白,太白,腕骨之下也,為腧;行於商丘,商丘,內踝之下,陷者之中也,為經;入於陰之陵泉,陰之陵泉,輔骨之下,陷者之中也,伸而得之,為合,足太陰也。腎出於湧泉,湧泉者,足心也,為井木;溜於然谷,然谷,然骨之下者也,為滎;注於太溪,太溪內踝之後,跟骨之上陷中者也,為腧;行於復留,復留,上內踝二寸,動而不休,為經;入於陰谷,陰谷,輔骨之後,大筋之下,小筋之上也,按之應手,屈膝而得之,為合,足少陰經也。
膀胱出於至陰,至陰者,足小指之端也,為井金;溜於通谷,通谷,本節之前外側也,為滎;注於束骨,束骨,本節之後,陷者中也,為腧;過於京骨,京骨,足外側大骨之下,為原;行於崑崙,崑崙,在外踝之後,跟骨之上,為經;入於委中,委中,膕中央,為合,委而取之,足太陽也。膽出於竅陰,竅陰者,足小指次指之端也,為井金;溜於俠溪,俠溪,足小指次指之間也,為滎;注於臨泣,臨泣,上行一寸半陷者中也,為腧;過於丘墟,丘墟,外踝之前下,陷者中也,為原;行於陽輔,陽輔,外踝之上,輔骨之前,及絕骨之端也,為經;入於陽之陵泉,陽之陵泉,在膝外陷者中也,為合,伸而得之,足少陽也。胃出於厲兌,厲兌者,足大指內次指之端也,為井金;溜於內庭,內庭,次指外間也,為滎;注於陷谷,陷谷者,上中指內間上行二寸陷者中也,為腧;過於沖陽,沖陽,足跗上五寸陷者中也,為原,搖足而得之;行於解溪,解溪,上衝陽一寸半陷者中也,為經;入於下陵,下陵,膝下三寸,胻骨外三里也,為合;復下三里三寸為巨虛上廉,復下上廉三寸為巨虛下廉也,大腸屬上,小腸屬下,足陽明胃脈也,大腸小腸,皆屬於胃,是足陽明也。三焦者,上合手少陽,出於關沖,關沖者,手小指次指之端也,為井金;溜於液門,液門,小指次指之間也,為滎;注於中渚,中渚,本節之後陷者中也,為腧;過於陽池,陽池,在腕上陷者之中也,為原;行於支溝,支溝,上腕三寸,兩骨之間陷者中也,為經;入於天井,天井,在肘外大骨之上陷者中也,為合,屈肘乃得之;三焦下腧,在於足大指之前,少陽之後,出於膕中外廉,名曰委陽,是太陽絡也。手少陽經也。三焦者,足少陽太陰之所將,太陽之別也,上踝五寸,別入貫腨腸,出於委陽,並太陽之正,入絡膀胱,約下焦,實則閉癃,虛則遺溺,遺溺則補之,閉癃則瀉之。手太陽小腸者,上合手太陽,出於少澤,少澤,小指之端也,為井金;溜於前谷,前谷,在手外廉本節前陷者中也,為滎;注於後溪,後溪者,在手外側本節之後也,為腧;過於腕骨,腕骨,在手外側腕骨之前,為原;行於陽谷,陽谷,在銳骨之下陷者中也,為經;入於小海,小海,在肘內大骨之外,去端半寸陷者中也,伸臂而得之,為合,手太陽經也。大腸上合手陽明,出於商陽,商陽,大指次指之端也,為井金;溜於本節之前二間,為滎;注於本節之後三間,為腧;過於合谷,合谷,在大指岐骨之間,為原;行於陽溪,陽溪,在兩筋間陷者中也,為經;入於曲池,在肘外輔骨陷者中,屈臂而得之,為合,手陽明也。是謂五藏六府之腧,五五二十五腧,六六三十六腧也。六府皆出足之三陽,上合於手者也。

缺盆之中,任脈也,名曰天突,一。次任脈側之動脈,足陽明也,名曰人迎,二。次脈手陽明也,名曰扶突,三。次脈手太陽也,名曰天窗,四。次脈足少陽也,名曰天容,五。次脈手少陽也,名曰天牖,六。次脈足太陽也,名曰天柱,七。次脈頸中央之脈,督脈也,名曰風府。腋內動脈,手太陰也,名曰天府。腋下三寸,手心主也,名曰天池。刺上關者,呿不能欠;刺下關者,欠不能呿。刺犢鼻者,屈不能伸;刺兩關者,伸不能屈。

足陽明挾喉之動脈也,其腧在膺中。手陽明次在其腧外,不至曲頰一寸。手太陽當曲頰。足少陽在耳下曲頰之後。手少陽出耳後,上加完骨之上。足太陽挾項大筋之中髮際。陰尺動脈在五里,五腧之禁也。

肺合大腸,大腸者,傳道之府。心合小腸,小腸者,受盛之府。肝合膽,膽者,中精之府。脾合胃,胃者,五穀之府。腎合膀胱,膀胱者,津液之府也。少陽屬腎,腎上連肺,故將兩藏。三焦者,中瀆之府也,水道出焉,屬膀胱,是孤之府也,是六府之所與合者。

春取絡脈諸滎大經分肉之間,甚者深取之,間者淺取之。夏取諸腧孫絡肌肉皮膚之上。秋取諸合,余如春法。冬取諸井諸腧之分,欲深而留之。此四時之序,氣之所處,病之所舍,藏之所宜。轉筋者,立而取之,可令遂已。痿厥者,張而刺之,可令立快也。



\section{小針解第三}

所謂易陳者,易言也。難入者,難著於人也。粗守形者,守刺法也。上守神者,守人之血氣有餘不足,可補瀉也。神客者,正邪其會也。神者,正氣也。客者,邪氣也。在門者,邪循正氣之所出入也。未睹其疾者,先知邪正何經之疾也。惡知其原者,先知何經之病所取之處也。刺之微在數遲者,徐疾之意也。粗守關者,守四支而不知血氣正邪之往來也。上守機者,知守氣也。機之動不離其空中者,知氣之虛實,用針之徐疾也。空中之機清淨以微者,針以得氣,密意守氣勿失也。其來不可逢者,氣盛不可補也。其往不可追者,氣虛不可瀉也。不可掛以發者,言氣易失也。扣之不發者,言不知補瀉之意也,血氣已盡而氣不下也。知其往來者,知氣之逆順盛虛也。要與之期者,知氣之可取之時也。粗之暗者,冥冥不知氣之微密也。妙哉!工獨有之者,盡知針意也。往者為逆者,言氣之虛而小,小者逆也。來者為順者,言形氣之平,平者順也。明知逆順,正行無問者,言知所取之處也。迎而奪之者,瀉也。追而濟之者,補也。

所謂虛則實之者,氣口虛而當補之也。滿則洩之者,氣口盛而當瀉之也。宛陳則除之者,去血脈也。邪勝則虛之者,言諸經有盛者,皆瀉其邪也。徐而疾則實者,言徐內而疾出也。疾而徐則虛者,言疾內而徐出也。言實與虛若有若無者,言實者有氣,虛者無氣也。察後與先若亡若存者,言氣之虛實,補瀉之先後也,察其氣之已下與常存也。為虛與實若得若失者,言補者佖然若有得也,瀉則怳然若有失也。

夫氣之在脈也,邪氣在上者,言邪氣之中人也高,故邪氣在上也。濁氣在中者,言水谷皆入於胃,其精氣上注於肺,濁溜於腸胃,言寒溫不適,飲食不節,而病生於腸胃,故命曰濁氣在中也。清氣在下者,言清濕地氣之中人也,必從足始,故曰清氣在下也。針陷脈則邪氣出者,取之上。針中脈則濁氣出者,取之陽明合也。針太深則邪氣反沉者,言淺浮之病,不欲深刺也,深則邪氣從之入,故曰反沉也。皮肉筋脈各有所處者,言經絡各有所主也。取五脈者死,言病在中,氣不足,但用針盡大瀉其諸陰之脈也。取三陽之脈者,唯言盡瀉三陽之氣,令病人恇然不復也。奪陰者死,言取尺之五里五往者也。奪陽者狂,正言也。

睹其色、察其目、知其散復、一其形、聽其動靜者,言上工知相五色於目,有知調尺寸大小緩急滑澀,以言所病也。知其邪正者,知論虛邪與正邪之風也。右主推之、左持而御之者,言持針而出入也。氣至而去之者,言補瀉氣調而去之也。調氣在於終始一者,持心也。節之交三百六十五會者,絡脈之滲灌諸節者也。所謂五藏之氣已絕於內者,脈口氣內絕不至,反取其外之病處與陽經之合,有留針以致陽氣,陽氣至則內重竭,重竭則死矣,其死也無氣以動,故靜。所謂五藏之氣已絕於外者,脈口氣外絕不至,反取其四末之輸,有留針以致其陰氣,陰氣至則陽氣反入,入則逆,逆則死矣,其死也陰氣有餘,故躁。所以察其目者,五藏使五色循明,循明則聲章,聲章者,則言聲與平生異也。




\section{邪氣藏府病形第四}

黃帝問於岐伯曰:邪氣之中人也奈何?岐伯答曰:邪氣之中人高也。黃帝曰:高下有度乎?岐伯曰:身半已上者,邪中之也;身半已下者,濕中之也。故曰,邪之中人也,無有常,中於陰則溜於府,中於陽則溜於經。黃帝曰:陰之與陽也,異名同類,上下相會,經絡之相貫,如環無端。邪之中人,或中於陰,或中於陽,上下左右,無有恆常,其故何也?岐伯曰:諸陽之會,皆在於面。中人也方乘虛時,及新用力,若飲食汗出腠理開,而中於邪。中於面則下陽明,中於項則下太陽,中於頰則下少陽,其中於膺背兩脅亦中其經。黃帝曰:其中於陰奈何?岐伯答曰:中於陰者,當從臂胻始。夫臂與胻,其陰皮薄,其肉淖澤,故俱受於風,獨傷其陰。黃帝曰:此故傷其藏乎?岐伯答曰:身之中於風也,不必動藏。故邪入於陰經,則其藏氣實,邪氣入而不能客,故還之於府。故中陽則溜於經,中陰則溜於府。黃帝曰:邪之中人藏奈何?岐伯曰:愁憂恐懼則傷心。形寒寒飲則傷肺,以其兩寒相感,中外皆傷,故氣逆而上行。有所墮墜,惡血留內,若有所大怒,氣上而不下,積於脅下,則傷肝。有所擊僕,若醉入房,汗出當風,則傷脾。有所用力舉重,若入房過度,汗出浴水,則傷腎。黃帝曰:五藏之中風奈何?岐伯曰:陰陽俱感,邪乃得往。黃帝曰:善哉。

帝問於岐伯曰:首面與身形也,屬骨連筋,同血合於氣耳。天寒則裂地凌冰,其卒寒或手足懈惰,然而其面不衣何也?岐伯答曰:十二經脈,三百六十五絡,其血氣皆上於面而走空竅,其精陽氣上走於目而為睛,其彆氣走於耳而為聽,其宗氣上出於鼻而為臭,其濁氣出於胃,走唇舌而為味。其氣之津液皆上熏於面,而皮又厚,其肉堅,故天氣甚寒不能勝之也。

黃帝曰:邪之中人,其病形何如?岐伯曰:虛邪之中身也,灑淅動形。正邪之中人也微,先見於色,不如於身,若有若無,若亡若存,有形無形,莫知其情。黃帝曰:善哉。

黃帝問於岐伯曰:余聞之,見其色,知其病,命曰明;按其脈,知其病,命曰神;問其病,知其處,命曰工。余願聞見而知之,按而得之,問而極之,為之奈何?岐伯答曰:夫色脈與尺之相應也,如桴鼓影響之相應也,不得相失也,此亦本末根葉之出候也,故根死則葉枯矣。色脈形肉不得相失也,故知一則為工,知二則為神,知三則神且明矣。黃帝曰:願卒聞之。岐伯答曰:色青者,其脈弦也;赤者,其脈鉤也;黃者,其脈代也;白者,其脈毛;黑者,其脈石。見其色而不得其脈,反得其相勝之脈,則死矣;得其相生之脈,則病已矣。黃帝問於岐伯曰:五藏之所生,變化之病形,何如?岐伯答曰:先定其五色五脈之應,其病乃可別也。黃帝曰:色脈已定,別之奈何?岐伯曰:調其脈之緩、急、小、大、滑、澀,而病變定矣。黃帝曰:調之奈何?岐伯答曰:脈急者,尺之皮膚亦急;脈緩者,尺之皮膚亦緩;脈小者,尺之皮膚亦減而少氣;脈大者,尺之皮膚亦賁而起;脈滑者,尺之皮膚亦滑;脈澀者,尺之皮膚亦澀。凡此變者,有微有甚。故善調尺者,不待於寸,善調脈者,不待於色。能參合而行之者,可以為上工,上工十全九;行二者,為中工,中工十全七;行一者,為下工,下工十全六。

黃帝曰:請問脈之緩、急、小、大、滑、澀之病形何如?岐伯曰:臣請言五藏之病變也。心脈急甚者為瘛瘲;微急為心痛引背,食不下。緩甚為狂笑;微緩為伏梁,在心下,上下行,時唾血。大甚為喉吤,微大為心痹引背,善淚出。小甚為善噦,微小為消癉。滑甚為善渴;微滑為心疝引臍,小腹鳴。澀甚為瘖;微澀為血溢,維厥,耳鳴,顛疾。

肺脈急甚為癲疾;微急為肺寒熱,怠惰,咳唾血,引腰背胸,若鼻息肉不通。緩甚為多汗;微緩為痿瘻,偏風,頭以下汗出不可止。大甚為脛腫;微大為肺痹引胸背,起惡日光。小甚為洩,微小為消癉。滑甚為息賁上氣,微滑為上下出血。澀甚為嘔血;微澀為鼠瘻,在頸支腋之間,下不勝其上,其應善(疒峻-山)矣。

肝脈急甚者為惡言;微急為肥氣在脅下,若覆杯。緩甚為善嘔,微緩為水瘕痹也。大甚為內癰,善嘔衄,微大為肝痹陰縮,咳引小腹。小甚為多飲,微小為消癉。滑甚為憒疝,微滑為遺溺。澀甚為溢飲,微澀為瘈攣筋痹。

脾脈急甚為瘈瘲;微急為膈中,食飲入而還出,後沃沫。緩甚為痿厥;微緩為風痿,四肢不用,心慧然若無病。大甚為擊僕;微大為疝氣,腹裡大膿血,在腸胃之外。小甚為寒熱,微小為消癉。滑甚為(疒貴)癃,微滑為蟲毒蛕蠍蛸腹熱。澀甚為腸(疒貴);微澀為內(疒貴),多下膿血。

腎脈急甚為骨癲疾;微急為沉厥奔豚,足不收,不得前後。緩甚為折脊;微緩為洞,洞者,食不化,下嗌還出。大甚為陰痿;微大為石水,起臍以下至小腹腄腄然,上至胃脘,死不治。小甚為洞洩,微小為消癉。滑甚為癃(疒貴);微滑為骨痿,坐不能起,起則目無所見。澀甚為大癰,微澀為不月沉痔。

黃帝曰:病之六變者,刺之奈何?岐伯答曰:諸急者多寒;緩者多熱;大者多氣少血;小者血氣皆少;滑者陽氣盛,微有熱;澀者多血少氣,微有寒。是故刺急者,深內而久留之。刺緩者,淺內而疾髮針,以去其熱。刺大者,微瀉其氣,無出其血。刺滑者,疾髮針而淺內之,以瀉其陽氣而去其熱。刺澀者,必中其脈,隨其逆順而久留之,必先按而循之,已髮針,疾按其痏,無令其血出,以和其脈。諸小者,陰陽形氣俱不足,勿取以針,而調以甘藥也。

黃帝曰:余聞五藏六府之氣,滎輸所入為合,令何道從入,入安連過,願聞其故。岐伯答曰:此陽脈之別入於內,屬於府者也。黃帝曰:滎輸與合,各有名乎?岐伯答曰:滎輸治外經,合治內府。黃帝曰:治內府奈何?岐伯曰:取之於合。黃帝曰:合各有名乎?岐伯答曰:胃合於三里,大腸合入於巨虛上廉,小腸合入於巨虛下廉,三焦合入於委陽,膀胱合入於委中央,膽合入於陽陵泉。黃帝曰:取之奈何?岐伯答曰:取之三里者,低跗;取之巨虛者,舉足;取之委陽者,屈伸而索之;委中者,屈而取之;陽陵泉者,正豎膝予之齊下至委陽之陽取之。取諸外經者,揄申而從之。

黃帝曰:願聞六府之病。岐伯答曰:面熱者足陽明病,魚絡血者手陽明病,兩跗之上脈豎陷者足陽明病,此胃脈也。大腸病者,腸中切痛而鳴濯濯,冬日重感於寒即洩,當臍而痛,不能久立,與胃同候,取巨虛上廉。胃病者,腹(月真)脹,胃脘當心而痛,上支兩脅,膈咽不通,飲食不下,取之三里也。小腸病者,小腹痛,腰脊控睾而痛,時窘之後,當耳前熱。若寒甚,若獨肩上熱甚,及手小指次指之間熱,若脈陷者,此其候也,手太陽病也,取之巨虛下廉。三焦病者,腹氣滿,小腹尤堅,不得小便,窘急,溢則水,留即為脹,候在足太陽之外大絡,大絡在太陽少陽之間,亦見於脈,取委陽。膀胱病者,小腹偏腫而痛,以手按之,即欲小便而不得,肩上熱若脈陷,及足小指外廉及脛踝後皆熱若脈陷,取委中央。膽病者,善太息,口苦,嘔宿汁,心下澹澹,恐人將捕之,嗌中吤吤然,數唾,在足少陽之本末,亦視其脈之陷下者灸之,其寒熱者取陽陵泉。黃帝曰:刺之有道乎?岐伯答曰:刺此者,必中氣穴,無中肉節,中氣穴則針染於巷,中肉節即皮膚痛,補瀉反則病益篤。中筋則筋緩,邪氣不出,與其真相搏,亂而不去,反還內著,用針不審,以順為逆也。




\section{根結第五}

岐伯曰:天地相感,寒暖相移,陰陽之道,孰少孰多?陰道偶,陽道奇,發於春夏,陰氣少,陽氣多,陰陽不調,何補何瀉?發於秋冬,陽氣少,陰氣多,陰氣盛而陽氣衰,故莖葉枯槁,濕雨下歸,陰陽相移,何瀉何補?奇邪離經,不可勝數,不知根結,五藏六府,折關敗樞,開合而走,陰陽大失,不可復取。九針之玄,要在終始,故能知終始,一言而畢,不知終始,針道咸絕。

太陽根於至陰,結於命門,命門者目也。陽明根於厲兌,結於顙大,顙大者鉗耳也。少陽根於竅陰,結於窗籠,窗籠者耳中也。太陽為開,陽明為合,少陽為樞。故開折則肉節瀆而暴病起矣,故暴病者取之太陽,視有餘不足,瀆者皮肉宛膲而弱也。合折則氣無所止息而痿疾起矣,故痿疾者取之陽明,視有餘不足,無所止息者,真氣稽留,邪氣居之也。樞折即骨繇而不安於地,故骨繇者取之少陽,視有餘不足,骨繇者節緩而不收也,所謂骨繇者搖故也,當窮其本也。太陰根於隱白,結於太倉。少陰根於湧泉,結於廉泉。厥陰根於大敦,結於玉英,絡於羶中。太陰為開,厥陰為合,少陰為樞。故開折則倉廩無所輸膈洞,膈洞者取之太陰,視有餘不足,故開折者氣不足而生病也。合折即氣絕而喜悲,悲者取之厥陰,視有餘不足。樞折則脈有所結而不通,不通者取之少陰,視有餘不足,有結者皆取之不足。

足太陽根於至陰,溜於京骨,注於崑崙,入於天柱、飛揚也。足少陽根於竅陰,溜於丘墟,注於陽輔,入於天容、光明也。足陽明根於厲兌,溜於沖陽,注於下陵,入於人迎、豐隆也。手太陽根於少澤,溜於陽谷,注於少海,入於天窗、支正也。手少陽根於關沖,溜於陽池,注於支溝,入於天牖、外關也。手陽明根於商陽,溜於合谷,注於陽溪,入於扶突、偏歷也。此所謂十二經者,盛絡皆當取之。

一曰一夜五十營,以營五藏之精,不應數者,名曰狂生。所謂五十營者,五藏皆受氣。持其脈口,數其至也,五十動而不一代者,五藏皆受氣;四十動一代者,一藏無氣;三十動一代者,二藏無氣;二十動一代者,三藏無氣;十動一代者,四藏無氣;不滿十動一代者,五藏無氣。予之短期,要在終始。所謂五十動而不一代者,以為常也,以知五藏之期。予之短期者,乍數乍疏也。

黃帝曰:逆順五體者,言人骨節之小大,肉之堅脆,皮之厚薄,血之清濁,氣之滑澀,脈之長短,血之多少,經絡之數,余已知之矣,此皆布衣匹夫之士也。夫王公大人,血食之君,身體柔脆,肌肉軟弱,血氣慓悍滑利,其刺之徐疾淺深多少,可得同之乎?岐伯答曰:膏梁菽藿之味,何可同也。氣滑即出疾,其氣澀則出遲,氣悍則針小而入淺,氣澀則針大而入深,深則欲留,淺則欲疾。以此觀之,刺布衣者深以留之,刺大人者微以徐之,此皆因氣慓悍滑利也。

黃帝曰:形氣之逆順奈何?岐伯曰:形氣不足,病氣有餘,是邪勝也,急瀉之。形氣有餘,病氣不足,急補之。形氣不足,病氣不足,此陰陽氣俱不足也,不可刺之,刺之則重不足,重不足則陰陽俱竭,血氣皆盡,五藏空虛,筋骨髓枯,老者絕滅,壯者不復矣。形氣有餘,病氣有餘,此謂陰陽俱有餘也,急瀉其邪,調其虛實。故曰有餘者瀉之,不足者補之,此之謂也。故曰刺不知逆順,真邪相搏。滿而補之,則陰陽四溢,腸胃充郭,肝肺內(月真),陰陽相錯。虛而瀉之,則經脈空虛,血氣竭枯,腸胃(亻聶)辟,皮膚薄著,毛腠夭膲,予之死期。故曰用針之要,在於知調陰與陽,調陰與陽,精氣乃光,合形與氣,使神內藏。故曰上工平氣,中工亂脈,下工絕氣危生。故曰下工不可不慎也。必審五藏變化之病,五脈之應,經絡之實虛,皮之柔粗,而後取之也。

\section{壽夭剛柔第六}

黃帝問於少師曰:余聞人之生也,有剛有柔,有弱有強,有短有長,有陰有陽,願聞其方。少師答曰:陰中有陰,陽中有陽,審知陰陽,刺之有方,得病所始,刺之有理,謹度病端,與時相應,內合於五藏六府,外合於筋骨皮膚。是故內有陰陽,外亦有陰陽。在內者,五藏為陰,六府為陽;在外者,筋骨為陰,皮膚為陽。故曰病在陰之陰者,刺陰之滎輸;病在陽之陽者,刺陽之合;病在陽之陰者,刺陰之經;病在陰之陽者,刺絡脈。故曰病在陽者命曰風,病在陰者命曰痹,陰陽俱病命曰風痹。病有形而不痛者,陽之類也;無形而痛者,陰之類也。無形而痛者,其陽完而陰傷之也,急治其陰,無攻其陽;有形而不痛者,其陰完而陽傷之也。急治其陽,無攻其陰。陰陽俱動,乍有形,乍無形,加以煩心,命曰陰勝其陽,此謂不表不裡,其形不久。

黃帝問於伯高曰:余聞形氣病之先後,外內之應奈何?伯高答曰:風寒傷形,憂恐忿怒傷氣。氣傷藏,乃病藏;寒傷形,乃應形;風傷筋脈,筋脈乃應。此形氣外內之相應也。黃帝曰:刺之奈何?伯高答曰:病九日者,三刺而已。病一月者,十刺而已。多少遠近,以此衰之。久痹不去身者,視其血絡,盡出其血。黃帝曰:外內之病,難易之治奈何?伯高答曰:形先病而未入藏者,刺之半其日;藏先病而形乃應者,刺之倍其日。此外內難易之應也。

黃帝問於伯高曰:余聞形有緩急,氣有盛衰,骨有大小,肉有堅脆,皮有厚薄,其以立壽夭奈何?伯高曰:形與氣相任則壽,不相任則夭。皮與肉相果則壽,不相果則夭。血氣經絡勝形則壽,不勝形則夭。黃帝曰:何謂形之緩急?伯高答曰:形充而皮膚緩者則壽,形充而皮膚急者則夭。形充而脈堅大者順也,形充而脈小以弱者氣衰,衰則危矣。若形充而顴不起者骨小,骨小而夭矣。形充而大肉(月囷)堅而有分者肉堅,肉堅則壽矣;形充而大肉無分理不堅者肉脆,肉脆則夭矣。此天之生命,所以立形定氣而視壽夭者。必明乎此立形定氣,而後以臨病人,決死生。黃帝曰:余聞壽夭,無以度之。伯高答曰:牆基卑,高不及其地者,不滿三十而死;其有因加疾者,不及二十而死也。黃帝曰:形氣之相勝,以立壽夭奈何?伯高答曰:平人而氣勝形者壽;病而形肉脫,氣勝形者死,形勝氣者危矣。
黃帝曰:余聞刺有三變,何謂三變?伯高答曰:有刺營者,有刺衛者,有刺寒痹之留經者。黃帝曰:刺三變者奈何?伯高答曰:刺營者出血,刺衛者出氣,刺寒痹者內熱。黃帝曰:營衛寒痹之為病奈何?伯高答曰:營之生病也,寒熱少氣,血上下行。衛之生病也,氣痛時來時去,怫愾賁響,風寒客於腸胃之中。寒痹之為病也,留而不去,時痛而皮不仁。黃帝曰:刺寒痹內熱奈何?伯高答曰:刺布衣者,以火焠之。刺大人者,以藥熨之。黃帝曰:藥熨奈何?伯高答曰:用淳酒二十斤,蜀椒一升,乾薑一斤,桂心一斤,凡四種,皆(口父)咀,漬酒中。用綿絮一斤,細白布四丈,並內酒中。置酒馬矢熅中,蓋封涂,勿使洩。五日五夜,出布綿絮,曝干之,干復漬,以盡其汁。每漬必晬其日,乃出干。干,並用滓與綿絮,復布為復巾,長六七尺,為六七巾。則用之生桑炭炙巾,以熨寒痹所刺之處,令熱入至於病所,寒復炙巾以熨之,三十遍而止。汗出以巾拭身,亦三十遍而止。起步內中,無見風。每刺必熨,如此病已矣,此所謂內熱也。




\section{官針第七}

凡刺之要,官針最妙。九針之宜,各有所為,長短大小,各有所施也,不得其用,病弗能移。疾淺針深,內傷良肉,皮膚為癰;病深針淺,病氣不瀉,支為大膿。病小針大,氣瀉太甚,疾必為害;病大針小,氣不洩瀉,亦復為敗。失針之宜,大者瀉,小者不移,已言其過,請言其所施。

病在皮膚無常處者,取以鑱針於病所,膚白勿取。病在分肉間,取以員針於病所。病在經絡痼痹者,取以鋒針。病在脈,氣少當補之者,取以針於井滎分輸。病為大膿者,取以鈹針。病痹氣暴發者,取以員利針。病痹氣痛而不去者,取以毫針。病在中者,取以長針。病水腫不能通關節者,取以大針。病在五藏固居者,取以鋒針,瀉於井滎分輸,取以四時。

凡刺有九,以應九變。一曰輸刺;輸刺者,刺諸經滎輸藏腧也。二曰遠道刺;遠道刺者,病在上,取之下,刺府腧也。三曰經刺;經刺者,刺大經之結絡經分也。四日絡刺;絡刺者,刺小絡之血脈也。五日分刺;分刺者,刺分肉之間也。六曰大瀉刺;大瀉刺者,刺大膿以鈹針也。七曰毛刺;毛刺者,刺浮痹皮膚也。八曰巨刺;巨刺者,左取右,右取左。九曰焠刺;焠刺者,刺燔針則取痹也。

凡刺有十二節,以應十二經。一曰偶刺;偶刺者,以手直心若背,直痛所,一刺前,一刺後,以治心痹,刺此者傍針之也。二曰報刺;報刺者,刺痛無常處也,上下行者,直內無拔針,以左手隨病所按之,乃出針復刺之也。三曰恢刺;恢刺者,直刺傍之,舉之前後,恢筋急,以治筋痹也。四曰齊刺;齊刺者,直入一,傍入二,以治寒氣小深者。或曰三刺;三刺者,治痹氣小深者也。五曰揚刺;揚刺者,正內一,傍內四,而浮之,以治寒氣之博大者也。六曰直針刺;直針刺者,引皮乃刺之,以治寒氣之淺者也。七曰輸刺;輸刺者,直入直出,稀髮針而深之,以治氣盛而熱者也。八曰短刺;短刺者,刺骨痹,稍搖而深之,致針骨所,以上下摩骨也。九曰浮刺;浮刺者,傍入而浮之,以治肌急而寒者也。十日陰刺;陰刺者,左右率刺之,以治寒厥,中寒厥,足踝後少陰也。十一曰傍針刺;傍針刺者,直刺傍刺各一,以治留痹久居者也。十二曰贊刺;贊刺者,直入直出,數髮針而淺之出血,是謂治癰腫也。

脈之所居深不見者刺之,微內針而久留之,以致其空脈氣也。脈淺者勿刺,按絕其脈乃刺之,無令精出,獨出其邪氣耳。所謂三刺則谷氣出者,先淺刺絕皮,以出陽邪;再刺則陰邪出者,少益深,絕皮致肌肉,未入分肉間也;已入分肉之間,則谷氣出。故刺法曰始刺淺之,以遂邪氣而來血氣;後刺深之,以致陰氣之邪;最後刺極深之,以下谷氣。此之謂也。故用針者,不知年之所加,氣之盛衰,虛實之所起,不可以為工也。

凡刺有五,以應五藏。一曰半刺;半刺者,淺內而疾髮針,無針傷肉,如拔毛狀,以取皮氣,此肺之應也。二曰豹文刺;豹文刺者,左右前後針之,中脈為故,以取經絡之血者,此心之應也。三曰關刺;關刺者,直刺左右,盡筋上,以取筋痹,慎無出血,此肝之應也,或曰淵刺,一曰豈刺。四曰合谷刺;合谷刺者,左右雞足,針於分肉之間,以取肌痹,此脾之應也。五曰輸刺;輸刺者,直入直出,深內之至骨,以取骨痺,此腎之應也。



\section{本神第八}

黃帝問於岐伯曰:凡刺之法,先必本於神。血、脈、營、氣、精神,此五藏之所藏也,至於淫泆離藏則精失、魂魄飛揚、志意恍亂、智慮去身者,何因而然乎?天之罪與?人之過乎?何謂德、氣、生、精、神、魂、魄、心、意、志、思、智、慮?請問其故。

岐伯答曰:天之在我者德也,地之在我者氣也,德流氣薄而生者也。故生之來謂之精,兩精相搏謂之神,隨神往來者謂之魂,並精而出入者謂之魄,所以任物者謂之心,心有所憶謂之意,意之所存謂之志,因志而存變謂之思,因思而遠慕謂之慮,因慮而處物謂之智。

故智者之養生也,必順四時而適寒暑,和喜怒而安居處,節陰陽而調剛柔,如是則僻邪不至,長生久視。是故怵惕思慮者則傷神,神傷則恐懼流淫而不止。因哀悲動中者,竭絕而失生。喜樂者,神憚散而不藏。愁憂者,氣閉塞而不行。盛怒者,迷惑而不治。恐懼者,神蕩憚而不收。

心怵惕思慮則傷神,神傷則恐懼自失,破(月囷)脫肉,毛悴色夭,死於冬。脾憂愁而不解則傷意,意傷則悗亂,四支不舉,毛悴色夭,死於春。肝悲哀動中則傷魂,魂傷則狂忘不精,不精則不正當人,陰縮而攣筋,兩脅骨不舉,毛悴色夭,死於秋。肺喜樂無極則傷魄,魄傷則狂,狂者意不存人,皮革焦,毛悴色夭,死於夏。腎盛怒而不止則傷志,志傷則喜忘其前言,腰脊不可以俯仰屈伸,毛卒色夭,死於季夏;恐懼而不解則傷精,精傷則骨痠痿厥,精時自下。是故五藏,主藏精者也,不可傷,傷則失守而陰虛,陰虛則無氣,無氣則死矣。是故用針者,察觀病人之態,以知精神魂魄之存亡得失之意,五者以傷,針不可以治之也。

肝藏血,血舍魂,肝氣虛則恐,實則怒。脾藏營,營舍意,脾氣虛則四支不用,五藏不安,實則腹脹經溲不利。心藏脈,脈舍神,心氣虛則悲,實則笑不休。肺藏氣,氣舍魄,肺氣虛則鼻塞不利少氣,實則喘喝胸盈仰息。腎藏精,精舍志,腎氣虛則厥,實則脹,五藏不安。必審五藏之病形,以知其氣之虛實,謹而調之也。




\section{終始第九}

凡刺之道,畢於終始,明知終始,五藏為紀,陰陽定矣。陰者主藏,陽者主府,陽受氣於四末,陰受氣於五藏。故瀉者迎之,補者隨之,知迎知隨,氣可令和。和氣之方,必通陰陽,五藏為陰,六府為陽,傳之後世,以血為盟,敬之者昌,慢之者亡,無道行私,必得夭殃。謹奉天道,請言終始,終始者,經脈為紀,持其脈口人迎,以知陰陽有餘不足,平與不平,天道畢矣。所謂平人者不病,不病者,脈口人迎應四時也,上下相應而俱往來也,六經之脈不結動也,本末寒溫之相守司也,形肉血氣必相稱也,是謂平人。少氣者,脈口人迎俱少而不稱尺寸也。如是者,則陰陽俱不足,補陽則陰竭,瀉陰則陽脫。如是者,可將以甘藥,不可飲以至劑。如此者弗灸,不已者因而瀉之,則五藏氣壞矣。

人迎一盛,病在足少陽,一盛而躁,病在手少陽。人迎二盛,病在足太陽,二盛而躁,病在手太陽。人迎三盛,病在足陽明,三盛而躁,病在手陽明。人迎四盛,且大且數,名曰溢陽,溢陽為外格。脈口一盛,病在足厥陰,厥陰一盛而躁,在手心主。脈口二盛,病在足少陰,二盛而躁,在手少陰。脈口三盛,病在足太陰,三盛而躁,在手太陰。脈口四盛,且大且數者,名曰溢陰,溢陰為內關,內關不通死不治。人迎與太陰脈口俱盛四倍以上,命曰關格,關格者與之短期。

人迎一盛,瀉足少陽而補足厥陰,二瀉一補,日一取之,必切而驗之,疏取之上,氣和乃止。人迎二盛,瀉足太陽,補足少陰,二瀉一補,二日一取之,必切而驗之,疏取之上,氣和乃止。人迎三盛,瀉足陽明而補足太陰,二瀉一補,日二取之,必切而驗之,疏取之上,氣和乃止。脈口一盛,瀉足厥陰而補足少陽,二補一瀉,日一取之,必切而驗之,疏而取之上,氣和乃止。脈口二盛,瀉足少陰而補足太陽,二補一瀉,二日一取之,必切而驗之,疏取之上,氣和乃止。脈口三盛,瀉足太陰而補足陽明,二補一瀉,日二取之,必切而驗之,疏而取之上,氣和乃止。所以日二取之者,陽明主胃,大富於谷氣,故可日二取之也。人迎與脈口俱盛三倍以上,命曰陰陽俱溢,如是者不開,則血脈閉塞,氣無所行,流淫於中,五藏內傷。如此者,因而灸之,則變易而為他病矣。

凡刺之道,氣調而止,補陰瀉陽,音氣益彰,耳目聰明,反此者血氣不行。所謂氣至而有效者,瀉則益虛,虛者脈大如其故而不堅也,堅如其故者,適雖言故,病未去也。補則益實,實者脈大如其故而益堅也,夫如其故而不堅者,適雖言快,病未去也。故補則實,瀉則虛,痛雖不隨針,病必衰去。必先通十二經脈之所生病,而後可得傳於終始矣。故陰陽不相移,虛實不相頃,取之其經。

凡刺之屬,三刺至谷氣,邪僻妄合,陰陽易居,逆順相反,沉浮異處,四時不得,稽留淫泆,須針而去。故一刺則陽邪出,再刺則陰邪出,三刺則谷氣至,谷氣至而止。所謂谷氣至者,已補而實,已瀉而虛,故以知谷氣至也。邪氣獨去者,陰與陽未能調,而病知愈也。故曰補則實,瀉則虛,痛雖不隨針,病必衰去矣。

陰盛而陽虛,先補其陽,後瀉其陰而和之。陰虛而陽盛,先補其陰,後瀉其陽而和之。三脈動於足大指之間,必審其實虛。虛而瀉之,是謂重虛,重虛病益甚。凡刺此者,以指按之,脈動而實且疾者疾瀉之,虛而徐者則補之,反此者病益甚。其動也,陽明在上,厥陰在中,少陰在下。膺腧中膺,背腧中背。肩膊虛者,取之上。重舌,刺舌柱以鈹針也。手屈而不伸者,其病在筋,伸而不屈者,其病在骨,在骨守骨,在筋守筋。補須一方實,深取之,稀按其痏,以極出其邪氣;一方虛,淺刺之,以養其脈,疾按其痏,無使邪氣得入。邪氣來也緊而疾,谷氣來也徐而和。脈實者,深刺之,以洩其氣;脈虛者,淺刺之,使精氣無得出,以養其脈,獨出其邪氣。

刺諸痛者,其脈皆實。故曰:從腰以上者,手太陰陽明皆主之;從腰以下者,足太陰陽明皆主之。病在上者下取之,病在下者高取之,病在頭者取之足,病在足者取之膕。病生於頭者頭重,生於手者臂重,生於足者足重,治病者先刺其病所從生者也。

春氣在毛,夏氣在皮膚,秋氣在分肉,冬氣在筋骨,刺此病者,各以其時為齊。故刺肥人者,以秋冬之齊;刺瘦人者,以春夏之齊。

病痛者陰也,痛而以手按之不得者陰也,深刺之。病在上者陽也,病在下者陰也。癢者陽也,淺刺之。病先起陰者,先治其陰而後治其陽;病先起陽者,先治其陽而後治其陰。

刺熱厥者,留針反為寒;刺寒厥者,留針反為熱。刺熱厥者,二陰一陽;刺寒厥者,二陽一陰。所謂二陰者,二刺陰也;一陽者,一刺陽也。久病者邪氣入深,刺此病者,深內而久留之,間日而復刺之,必先調其左右,去其血脈,刺道畢矣。

凡刺之法,必察其形氣,形肉未脫,少氣而脈又躁,躁厥者,必為繆刺之,散氣可收,聚氣可布。深居靜處,佔神往來,閉戶塞牖,魂魄不散,專意一神,精氣之分,毋聞人聲,以收其精,必一其神,令志在針,淺而留之,微而浮之,以移其神,氣至乃休。男內女外,堅拒勿出,謹守勿內,是謂得氣。
凡刺之禁:新內勿刺,新刺勿內。已醉勿刺,已刺勿醉。新怒勿刺,已刺勿怒。新勞勿刺,已刺勿勞。已飽勿刺,已刺勿飽。已飢勿刺,已刺勿飢。已渴勿刺,已刺勿渴。大驚大恐,必定其氣,乃刺之。乘車來者,臥而休之,如食頃乃刺之。出行來者,坐而休之,如行十里頃乃刺之。凡此十二禁者,其脈亂氣散,逆其營衛,經氣不次,因而刺之,則陽病入於陰,陰病出為陽,則邪氣復生,粗工勿察,是謂伐身,形體淫泆,乃消腦髓,津液不化,脫其五味,是謂失氣也。

太陽之脈,其終也,戴眼反折瘛瘲,其色白,絕皮乃絕汗,絕汗則終矣。少陽終者,耳聾,百節盡縱,目系絕,目系絕一日半則死矣,其死也,色青白乃死。陽明終者,口目動作,喜驚妄言,色黃,其上下之經盛而不行則終矣。少陰終者,面黑齒長而垢,腹脹閉塞,上下不通而終矣。厥陰終者,中熱嗌干,喜溺心煩,甚則舌卷卵上縮而終矣。太陰終者,腹脹閉不得息,氣噫善嘔,嘔則逆,逆則面赤,不逆則上下不通,上下不通則面黑皮毛燋而終矣。

\section{經脈第十}

雷公問於黃帝曰:禁脈之言,凡刺之理,經脈為始,營其所行,制其度量,內次五藏,外別六府,願盡聞其道。黃帝曰:人始生,先成精,精成而腦髓生,骨為干,脈為營,筋為剛,肉為牆,皮膚堅而毛髮長,谷入於胃,脈道以通,血氣乃行。雷公曰:願卒聞經脈之始生。黃帝曰:經脈者,所以能決死生,處百病,調虛實,不可不通。

肺手太陰之脈,起於中焦,下絡大腸,還循胃口,上膈屬肺,從肺系橫出腋下,下循臑內,行少陰心主之前,下肘中,循臂內上骨下廉,入寸口,上魚,循魚際,出大指之端;其支者,從腕後直出次指內廉,出其端。是動則病肺脹滿膨膨而喘咳,缺盆中痛,甚則交兩手而瞀,此為臂厥。是主肺所生病者,咳,上氣喘渴,煩心胸滿,臑臂內前廉痛厥,掌中熱。氣盛有餘,則肩背痛風寒,汗出中風,小便數而欠。氣虛則肩背痛寒,少氣不足以息,溺色變。為此諸病,盛則瀉之,虛則補之,熱則疾之,寒則留之,陷下則灸之,不盛不虛,以經取之。盛者寸口大三倍於人迎,虛者則寸口反小於人迎也。
大腸手陽明之脈,起於大指次指之端,循指上廉,出合谷兩骨之間,上入兩筋之中,循臂上廉,入肘外廉,上臑外前廉,上肩,出髃骨之前廉,上出於柱骨之會上,下入缺盆絡肺,下膈屬大腸;其支者,從缺盆上頸貫頰,入下齒中,還出挾口,交人中,左之右,右之左,上挾鼻孔。是動則病齒痛頸腫。是主津液所生病者,目黃口乾,鼽衄,喉痹,肩前臑痛,大指次指痛不用。氣有餘則當脈所過者熱腫,虛則寒慄不復。為此諸病,盛則瀉之,虛則補之,熱則疾之,寒則留之,陷下則灸之,不盛不虛,以經取之。盛者人迎大三倍於寸口,虛者人迎反小於寸口也。

胃足陽明之脈,起於鼻之交頞中,旁納太陽之脈,下循鼻外,入上齒中,還出挾口環唇,下交承漿,卻循頤後下廉,出大迎,循頰車,上耳前,過客主人,循髮際,至額顱;其支者,從大迎前下人迎,循喉嚨,入缺盆,下膈屬胃絡脾;其直者,從缺盆下乳內廉,下挾臍,入氣街中;其支者,起於胃口,下循腹裡,下至氣街中而合,以下髀關,抵伏兔,下膝臏中,下循脛外廉,下足跗,入中指內間;其支者,下廉三寸而別,下入中指外間;其支者,別跗上,入大指間,出其端。是動則病灑灑振寒,善呻數欠顏黑,病至則惡人與火,聞木聲則惕然而驚,心欲動,獨閉戶塞牖而處,甚則欲上高而歌,棄衣而走,賁響腹脹,是為骭厥。是主血所生病者,狂瘧溫淫汗出,鼽衄,口喎唇胗,頸腫喉痹,大腹水腫,膝臏腫痛,循膺、乳、氣街、股、伏兔、骭外廉、足跗上皆痛,中指不用。氣盛則身以前皆熱,其有餘於胃,則消谷善飢,溺色黃。氣不足則身以前皆寒慄,胃中寒則脹滿。為此諸病,盛則瀉之,虛則補之,熱則疾之,寒則留之,陷下則灸之,不盛不虛,以經取之。盛者人迎大三倍於寸口,虛者人迎反小於寸口也。
脾足太陰之脈,起於大指之端,循指內側白肉際,過核骨後,上內踝前廉,上踹內,循脛骨後,交出厥陰之前,上膝股內前廉,入腹屬脾絡胃,上膈,挾咽,連舌本,散舌下;其支者,復從胃,別上膈,注心中。是動則病舌本強,食則嘔,胃脘痛,腹脹善噫,得後與氣則快然如衰,身體皆重。是主脾所生病者,舌本痛,體不能動搖,食不下,煩心,心下急痛,溏、瘕、洩,水閉、黃疸,不能臥,強立股膝內腫厥,足大指不用。為此諸病,盛則瀉之,虛則補之,熱則疾之,寒則留之,陷下則灸之,不盛不虛,以經取之。盛者,寸口大三倍於人迎,虛者,寸口反小於人迎也。
心手少陰之脈,起於心中,出屬心繫,下膈絡小腸;其支者,從心繫上挾咽,系目系;其直者,復從心繫卻上肺,下出腋下,下循臑內後廉,行太陰心主之後,下肘內,循臂內後廉,抵掌後銳骨之端,入掌內後廉,循小指之內出其端。是動則病嗌干心痛,渴而欲飲,是為臂厥。是主心所生病者,目黃脅痛,臑臂內後廉痛厥,掌中熱痛。為此諸病,盛則瀉之,虛則補之,熱則疾之,寒則留之,陷下則灸之,不盛不虛,以經取之。盛者寸口大再倍於人迎,虛者寸口反小於人迎也。

小腸手太陽之脈,起於小指之端,循手外側上腕,出踝中,直上循臂骨下廉,出肘內側兩筋之間,上循臑外後廉,出肩解,繞肩胛,交肩上,入缺盆絡心,循嚥下膈,抵胃屬小腸;其支者,從缺盆循頸上頰,至目銳眥,卻入耳中;其支者,別頰上(出頁)抵鼻,至目內眥,斜絡於顴。是動則病嗌痛頷腫,不可以顧,肩似拔,臑似折。是主液所生病者,耳聾目黃頰腫,頸頷肩臑肘臂外後廉痛。為此諸病,盛則瀉之,虛則補之,熱則疾之,寒則留之,陷下則灸之,不盛不虛,以經取之。盛者人迎大再倍於寸口,虛者人迎反小於寸口也。

膀胱足太陽之脈,起於目內眥,上額交巔;其支者,從巔至耳上角;其直者,從巔入絡腦,還出別下項,循肩髆內,挾脊抵腰中,入循膂,絡腎屬膀胱;其支者,從腰中下挾脊貫臀,入膕中;其支者,從髆內左右,別下貫胛,挾脊內,過髀樞,循髀外從後廉下合膕中,以下貫踹內,出外踝之後,循京骨,至小指外側。是動則病沖頭痛,目似脫,項如拔,脊痛,腰似折,髀不可以曲,膕如結,踹如裂,是為踝厥。是主筋所生病者,痔瘧狂顛疾,頭囟項痛,目黃淚出鼽衄,項背腰尻膕踹腳皆痛,小指不用。為此諸病,盛則瀉之,虛則補之,熱則疾之,寒則留之,陷下則灸之,不盛不虛,以經取之。盛者人迎大再倍於寸口,虛者人迎反小於寸口也。

腎足少陰之脈,起於小指之下,邪走足心,出於然谷之下,循內踝之後,別入跟中,以上踹內,出膕內廉,上股內後廉,貫脊屬腎絡膀胱;其直者,從腎上貫肝膈,入肺中,循喉嚨,挾舌本;其支者,從肺出絡心,注胸中。是動則病飢不欲食,面如漆柴,咳唾則有血,喝喝而喘,坐而欲起,目(目巟)(目巟)如無所見,心如懸若飢狀,氣不足則善恐,心惕惕如人將捕之,是為骨厥。是主腎所生病者,口熱舌干,咽腫上氣,嗌干及痛,煩心心痛,黃疸腸澼,脊股內後廉痛,痿厥嗜臥,足下熱而痛。為此諸病,盛則瀉之,虛則補之,熱則疾之,寒則留之,陷下則灸之,不盛不虛,以經取之。灸則強食生肉,緩帶披髮,大杖重履而步。盛者寸口大再倍於人迎,虛者寸口反小於人迎也。
心主手厥陰心包絡之脈,起於胸中,出屬心包絡,下膈,歷絡三膲;其支者,循胸出脅,下腋三寸,上抵腋,下循臑內,行太陰少陰之間,入肘中,下臂行兩筋之間,入掌中,循中指出其端;其支者,別掌中,循小指次指出其端。是動則病手心熱,臂肘攣急,腋腫,甚則胸脅支滿,心中憺憺大動,面赤目黃,喜笑不休。是主脈所生病者,煩心心痛,掌中熱。為此諸病,盛則瀉之,虛則補之,熱則疾之,寒則留之,陷下則灸之,不盛不虛,以經取之。盛者寸口大一倍於人迎,虛者寸口反小於人迎也。

三焦手少陽之脈,起於小指次指之端,上出兩指之間,循手錶腕,出臂外兩骨之間,上貫肘,循臑外上肩,而交出足少陽之後,入缺盆,布羶中,散落心包,下膈,循屬三焦;其支者,從羶中上出缺盆,上項,系耳後,直上出耳上角,以屈下頰至(出頁);其支者,從耳後入耳中,出走耳前,過客主人前,交頰,至目銳眥。是動則病耳聾渾渾焞焞,嗌腫喉痹。是主氣所生病者,汗出,目銳眥痛,頰痛,耳後肩臑肘臂外皆痛,小指次指不用。為此諸病,盛則瀉之,虛則補之,熱則疾之,寒則留之,陷下則灸之,不盛不虛,以經取之。盛者人迎大一倍於寸口,虛者人迎反小於寸口也。

膽足少陽之脈,起於目銳眥,上抵頭角,下耳後,循頸行手少陽之前,至肩上,卻交出手少陽之後,入缺盆;其支者,從耳後入耳中,出走耳前,至目銳眥後;其支者,別銳眥,下大迎,合於手少陽,抵於(出頁),下加頰車,下頸合缺盆以下胸中,貫膈絡肝屬膽,循脅裡,出氣街,繞毛際,橫入髀厭中;其直者,從缺盆下腋,循胸過季脅,下合髀厭中,以下循髀陽,出膝外廉,下外輔骨之前,直下抵絕骨之端,下出外踝之前,循足跗上,入小指次指之間;其支者,別跗上,入大指之間,循大指岐骨內出其端,還貫爪甲,出三毛。是動則病口苦,善太息,心脅痛不能轉側,甚則面微有塵,體無膏澤,足外反熱,是為陽厥。是主骨所生病者,頭痛頷痛,目銳眥痛,缺盆中腫痛,腋下腫,馬刀俠癭,汗出振寒,瘧,胸脅肋髀膝外至脛絕骨外髁前及諸節皆痛,小指次指不用。為此諸病,盛則瀉之,虛則補之,熱則疾之,寒則留之,陷下則灸之,不盛不虛,以經取之。盛者人迎大一倍於寸口,虛者人迎反小於寸口也。

肝足厥陰之脈,起於大指叢毛之際,上循足跗上廉,去內踝一寸,上踝八寸,交出太陰之後,上膕內廉,循股陰入毛中,過陰器,抵小腹,挾胃屬肝絡膽,上貫膈,布脅肋,循喉嚨之後,上入頏顙,連目系,上出額,與督脈會於巔;其支者,從目系下頰裡,環唇內;其支者,復從肝別貫膈,上注肺。是動則病腰痛不可以俯仰,丈夫(疒貴)疝,婦人少腹腫,甚則嗌干,面塵脫色。是主肝所生病者,胸滿嘔逆飧洩,狐疝遺溺閉癃。為此諸病,盛則瀉之,虛則補之,熱則疾之,寒則留之,陷下則灸之,不盛不虛,以經取之。盛者寸口大一倍於人迎,虛者寸口反小於人迎也。

手太陰氣絕則皮毛焦,太陰者行氣溫於皮毛者也,故氣不榮則皮毛焦,皮毛焦則津液去皮節,津液去皮節者則爪枯毛折,毛折者則毛先死,丙篤丁死,火勝金也。手少陰氣絕則脈不通,脈不通則血不流,血不流則髦色不澤,故其面黑如漆柴者,血先死,壬篤癸死,水勝火也。足太陰氣絕者則脈不榮肌肉,唇舌者肌肉之本也,脈不榮則肌肉軟,肌肉軟則舌萎人中滿,人中滿則唇反,唇反者肉先死,甲篤乙死,木勝土也。足少陰氣絕則骨枯,少陰者冬脈也,伏行而濡骨髓者也,故骨不濡則肉不能著也,骨肉不相親則肉軟卻,肉軟卻故齒長而垢發無澤,發無澤者骨先死,戊篤己死,土勝水也。足厥陰氣絕則筋絕,厥陰者肝脈也,肝者筋之合也,筋者聚於陰氣,而脈絡於舌本也,故脈弗榮則筋急,筋急則引舌與卵,故唇青舌卷卵縮則筋先死,庚篤辛死,金勝木也。五陰氣俱絕則目系轉,轉則目運,目運者為志先死,志先死則遠一日半死矣。六陽氣絕,則陰與陽相離,離則腠理髮洩,絕汗乃出,故旦佔夕死,夕佔旦死。

經脈十二者,伏行分肉之間,深而不見;其常見者,足太陰過於外踝之上,無所隱故也。諸脈之浮而常見者,皆絡脈也。六經絡手陽明少陽之大絡,起於五指間,上合肘中。飲酒者,衛氣先行皮膚,先充絡脈,絡脈先盛,故衛氣已平,營氣乃滿,而經脈大盛。脈之卒然動者,皆邪氣居之,留於本末;不動則熱,不堅則陷且空,不與眾同,是以知其何脈之動也。雷公曰:何以知經脈之與絡脈異也?黃帝曰:經脈者常不可見也,其虛實也以氣口知之,脈之見者皆絡脈也。雷公曰:細子無以明其然也。黃帝曰:諸絡脈皆不能經大節之間,必行絕道而出,入復合於皮中,其會皆見於外。故諸刺絡脈者,必刺其結上,甚血者雖無結,急取之以瀉其邪而出其血,留之發為痹也。凡診絡脈,脈色青則寒且痛,赤則有熱。胃中寒,手魚之絡多青矣;胃中有熱,魚際絡赤;其暴黑者,留久痹也;其有赤有黑有青者,寒熱氣也;其青短者,少氣也。凡刺寒熱者皆多血絡,必間日而一取之,血盡乃止,乃調其虛實;其小而短者少氣,甚者瀉之則悶,悶甚則僕不得言,悶則急坐之也。

手太陰之別,名曰列缺,起於腕上分間,並太陰之經直入掌中,散入於魚際。其病實則手銳掌熱,虛則欠(去欠),小便遺數,取之去腕半寸,別走陽明也。手少陰之別,名曰通裡,去腕一寸半,別而上行,循經入於心中,系舌本,屬目系。其實則支膈,虛則不能言,取之掌後一寸,別走太陽也。手心主之別,名曰內關,去腕二寸,出於兩筋之間,循經以上,繫於心包絡。心繫實則心痛,虛則為頭強,取之兩筋間也。手太陽之別,名曰支正,上腕五寸,內注少陰;其別者,上走肘,絡肩髃。實則節弛肘廢,虛則生肬,小者如指痂疥,取之所別也。手陽明之別,名曰偏歷,去腕三寸,別入太陰;其別者,上循臂,乘肩髃,上曲頰偏齒;其別者,入耳合於宗脈。實則齲聾,虛則齒寒痹隔,取之所別也。手少陽之別,名曰外關,去腕二寸,外遶臂,注胸中,合心主。病實則肘攣,虛則不收,取之所別也。足太陽之別,名曰飛揚,去踝七寸,別走少陰。實則鼽窒頭背痛,虛則鼽衄,取之所別也。足少陽之別,名曰光明,去踝五寸,別走厥陰,下絡足跗。實則厥,虛則痿躄,坐不能起,取之所別也。足陽明之別,名曰豐隆,去踝八寸,別走太陰;其別者,循脛骨外廉,上絡頭項,合諸經之氣,下絡喉嗌。其病氣逆則喉痹瘁瘖,實則狂巔,虛則足不收脛枯,取之所別也。足太陰之別,名曰公孫,去本節之後一寸,別走陽明;其別者,入絡腸胃。厥氣上逆則霍亂,實則腸中切痛,虛則鼓脹,取之所別也。足少陰之別,名曰大鍾,當踝後繞跟,別走太陽;其別者,並經上走於心包,下外貫腰脊。其病氣逆則煩悶,實則閉癃,虛則腰痛,取之所別者也。足厥陰之別,名曰蠡溝,去內踝五寸,別走少陽;其別者,徑脛上睾,結於莖。其病氣逆則睾腫卒疝,實則挺長,虛則暴癢,取之所別也。任脈之別,名曰尾翳,下鳩尾,散於腹。實則腹皮痛,虛則癢搔,取之所別也。督脈之別,名曰長強,挾膂上項,散頭上,下當肩胛左右,別走太陽,入貫膂。實則脊強,虛則頭重,高搖之,挾脊之有過者,取之所別也。脾之大絡,名曰大包,出淵腋下三寸,布胸脅。實則身盡痛,虛則百節盡皆縱,此脈若羅絡之血者,皆取之脾之大絡脈也。凡此十五絡者,實則必見,虛則必下,視之不見,求之上下,人經不同,絡脈異所別也。

\section{經別第十一}

黃帝問於岐伯曰:余聞人之合於天道也,內有五藏,以應五節五色五時五味五位也;外有六府,以應六律,六律建陰陽諸經而合之十二月、十二辰、十二節、十二經水、十二時、十二經脈者,此五藏六府之所以應天道。夫十二經脈者,人之所以生,病之所以成,人之所以治,病之所以起,學之所始,工之所止也,粗之所易,上之所難也。請問其離合出入奈何?岐伯稽首再拜曰:明乎哉問也!此粗之所過,上之所息也,請卒言之。

足太陽之正,別入於膕中,其一道下尻五寸,別入於肛,屬於膀胱,散之腎,循膂當心入散;直者,從膂上出於項,復屬於太陽,此為一經也。足少陰之正,至膕中,別走太陽而合,上至腎,當十四顀,出屬帶脈;直者,系舌本,復出於項,合於太陽,此為一合。成以諸陰之別,皆為正也。

足少陽之正,繞髀入毛際,合於厥陰;別者,入季脅之間,循胸裡屬膽,散之上肝貫心,以上挾咽,出頤頷中,散於面,系目系,合少陽於外眥也。足厥陰之正,別跗上,上至毛際,合於少陽,與別俱行,此為二合也。

足陽明之正,上至髀,入於腹裡,屬胃,散之脾,上通於心,上循咽出於口,上頞(出頁),還系目系,合於陽明也。足太陰之正,上至髀,合於陽明,與別俱行,上結於咽,貫舌中,此為三合也。

手太陽之正,指地,別於肩解,入腋走心,系小腸也。手少陰之正,別入於淵腋兩筋之間,屬於心,上走喉嚨,出於面,合目內眥,此為四合也。

手少陽之正,指天,別於巔,入缺盆,下走三焦,散於胸中也。手心主之正,別下淵腋三寸,入胸中,別屬三焦,出循喉嚨,出耳後,合少陽完骨之下,此為五合也。

手陽明之正,從手循膺乳,別於肩髃,入柱骨,下走大腸,屬於肺,上循喉嚨,出缺盆,合於陽明也。手太陰之正,別入淵腋少陰之前,入走肺,散之太陽,上缺盆,循喉嚨,復合陽明,此六合也。




\section{經水第十二}

黃帝問於岐伯曰:經脈十二者,外合於十二經水,而內屬於五藏六府。夫十二經水者,其有大小、深淺、廣狹、遠近各不同,五藏六府之高下、小大、受谷之多少亦不等,相應奈何?夫經水者,受水而行之;五藏者,合神氣魂魄而藏之;六府者,受谷而行之,受氣而揚之;經脈者,受血而營之。合而以治奈何?刺之深淺,灸之壯數,可得聞乎?
岐伯答曰:善哉問也!天至高,不可度,地至廣,不可量,此之謂也。且夫人生於天地之間,六合之內,此天之高、地之廣也,非人力之所度量而至也。若夫八尺之士,皮肉在此,外可度量切循而得之,其死可解剖而視之,其藏之堅脆,府之大小,谷之多少,脈之長短,血之清濁,氣之多少,十二經之多血少氣,與其少血多氣,與其皆多血氣,與其皆少血氣,皆有大數。其治以針艾,各調其經氣,固其常有合乎?

黃帝曰:余聞之,快於耳,不解於心,願卒聞之。岐伯答曰:此人之所以參天地而應陰陽也,不可不察。足太陽外合於清水,內屬於膀胱,而通水道焉。足少陽外合於渭水,內屬於膽。足陽明外合於海水,內屬於胃。足太陰外合於湖水,內屬於脾。足少陰外合於汝水,內屬於腎。足厥陰外合於澠水,內屬於肝。手太陽外合於淮水,內屬於小腸,而水道出焉。手少陽外合於漯水,內屬於三焦。手陽明外合於江水,內屬於大腸。手太陰外合於河水,內屬於肺。手少陰外合於濟水,內屬於心。手心主外合於漳水,內屬於心包。凡此五藏六府十二經水者,外有源泉而內有所稟,此皆內外相貫,如環無端,人經亦然。故天為陽,地為陰,腰以上為天,腰以下為地。故海以北者為陰,湖以北者為陰中之陰,漳以南者為陽,河以北至漳者為陽中之陰,漯以南至江者為陽中之太陽,此一隅之陰陽也,所以人與天地相參也。
黃帝曰:夫經水之應經脈也,其遠近淺深,水血之多少各不同,合而以刺之奈何?岐伯答曰:足陽明,五藏六府之海也,其脈大血多,氣盛熱壯,刺此者不深弗散,不留不瀉也。足陽明刺深六分,留十呼,足太陽深五分,留七呼。足少陽深四分,留五呼。足太陰深三分,留四呼。足少陰深二分,留三呼。足厥陰深一分,留二呼。手之陰陽,其受氣之道近,其氣之來疾,其刺深者皆無過二分,其留皆無過一呼。其少長大小肥瘦,以心撩之,命曰法天之常。灸之亦然。灸而過此者得惡火,則骨枯脈澀;刺而過此者,則脫氣。
黃帝曰:夫經脈之大小,血之多少,膚之厚薄,肉之堅脆,及膕之大小,其可為量度乎?岐伯答曰:其可為量度者,取其中度也,不甚脫肉而血氣不衰也。若失度之人,痟瘦而形肉脫者,惡可以量度刺乎。審切循捫按,視其寒溫盛衰而調之,是謂因適而為之真也。

\section{經筋第十三}

足太陽之筋,起於足小指,上結於踝,邪上結於膝,其下循足外踝,結於踵,上循跟,結於膕;其別者,結於踹外,上膕中內廉,與膕中並上結於臀,上挾脊上項;其支者,別入結於舌本;其直者,結於枕骨,上頭下顏,結於鼻;其支者,為目上網,下結於頄;其支者,從腋後外廉,結於肩髃;其支者,入腋下,上出缺盆,上結於完骨;其支者,出缺盆,邪上出於頄。其病小指支,跟腫痛,膕攣,脊反折,項筋急,肩不舉,腋支,缺盆中紐痛,不可左右搖。治在燔針劫刺,以知為數,以痛為輸,名曰仲春痹也。
足少陽之筋,起於小指次指,上結外踝,上循脛外廉,結於膝外廉;其支者,別起外輔骨,上走髀,前者結於伏兔之上,後者結於尻;其直者,上乘(月少)季協,上走腋前廉,繫於膺乳,結於缺盆;直者,上出腋,貫缺盆,出太陽之前,循耳後,上額角,交巔上,下走頷,上結於頄;支者,結於目眥為外維。其病小指次指支轉筋,引膝外轉筋,膝不可屈伸,膕筋急,前引髀,後引尻,即上乘(月少)季脅痛,上引缺盆膺乳頸,維筋急,從左之右,右目不開,上過右角,並蹻脈而行,左絡於右,故傷左角,右足不用,命曰維筋相交。治在燔針劫刺,以知為數,以痛為輸,名曰孟春痹也。
足陽明之筋,起於中三指,結於跗上,邪外上加於輔骨,上結於膝外廉,直上結於髀樞,上循脅,屬脊;其直者,上循骭,結於膝;其支者,結於外輔骨,合少陽;其直者,上循伏兔,上結於髀,聚於陰器,上腹而布,至缺盆而結,上頸,上挾口,合於頄,下結於鼻,上合於太陽,太陽為目上網,陽明為目下網;其支者,從頰結於耳前。其病足中指支,脛轉筋,腳跳堅,伏兔轉筋,髀前腫,(疒貴)疝,腹筋急,引缺盆及頰,卒口僻,急者目不合,熱則筋縱,目不開。頰筋有寒,則急引頰移口;有熱則筋弛縱緩,不勝收故僻。治之以馬膏,膏其急者,以白酒和桂,以涂其緩者,以桑鉤鉤之,即以生桑灰置之坎中,高下以坐等,以膏熨急頰,且飲美酒,噉美炙肉,不飲酒者,自強也,為之三拊而已。治在燔針劫刺,以知為數,以痛為輸,名曰季春痹也。
足太陰之筋,起於大指之端內側,上結於內踝;其直者,絡於膝內輔骨,上循陰股,洛於髀,聚於陰器,上腹,結於齊,循腹裡,結於肋,散於胸中;其內者,著於脊。其病足大指支,內踝痛,轉筋痛,膝內輔骨痛,陰股引髀而痛,陰器紐痛,下引臍兩脅痛,引膺中脊內痛。治在燔針劫刺,以知為數,以痛為輸,命曰孟秋痹也。
足少陰之筋,起於小指之下,並足太陰之筋邪走內踝之下,結於踵,與太陽之筋合而上結於內輔之下,並太陰之筋而上循陰股,結於陰器,循脊內挾膂,上至項,結於枕骨,與足太陽之筋合。其病足下轉筋,及所過而結者皆痛及轉筋。病在此者主癇瘛及痙,在外者不能俯,在內者不能仰,故陽病者腰反折不能俯,陰病者不能仰。治在燔針劫刺,以知為數,以痛為輸,在內者熨引飲藥。此筋折紐,紐發數甚者,死不治,名曰仲秋痹也。
足厥陰之筋,起於大指之上,上結於內踝之前,上循脛,上結內輔之下,上循陰股,結於陰器,絡諸筋。其病足大指支,內踝之前痛,內輔痛,陰股痛轉筋,陰器不用,傷於內則不起,傷於寒則陰縮入,傷於熱則縱挺不收。治在行水清陰氣。其病轉筋者,治在燔針劫刺,以知為數,以痛為輸,命曰季秋痹也。
手太陽之筋,起於小指之上,結於腕,上循臂內廉,結於肘內銳骨之後,彈之應小指之上,入結於腋下;其支者,後走腋後廉,上繞肩胛,循頸出走太陽之前,結於耳後完骨;其支者,入耳中;直者,出耳上,下結於頷,上屬目外眥。其病小指支,肘內銳骨後廉痛,循臂陰入腋下,腋下痛,腋後廉痛,繞肩胛引頸而痛,應耳中鳴痛,引頷目瞑,良久乃得視,頸筋急則為筋瘻頸腫。寒熱在頸者,治在燔針劫刺之,以知為數,以痛為輸,其為腫者,復而銳之。名曰仲夏痹也。
手少陽之筋,起於小指次指之端,結於腕,上循臂,結於肘,上繞臑外廉,上肩走頸,合手太陽;其支者,當曲頰入系舌本;其支者,上曲牙,循耳前,屬目外眥,上乘頷,結於角。其病當所過者即支轉筋,舌卷。治在燔針劫刺,以知為數,以痛為輸,名曰季夏痹也。
手陽明之筋,起於大指次指之端,結於腕,上循臂,上結於肘外,上臑,結於髃;其支者,繞肩胛,挾脊;直者,從肩髃上頸;其支者,上頰,結於(九頁);直者,上出手太陽之前,上左角,絡頭,下右頷。其病當所過者支痛及轉筋,肩不舉頸,不可左右視。治在燔針劫刺,以知為數,以痛為輸,名曰孟夏痹也。
手太陰之筋,起於大指之上,循指上行,結於魚後,行寸口外側,上循臂,結肘中,上臑內廉,入腋下,出缺盆,結肩前髃,上結缺盆,下結胸裡,散貫賁,合賁下,抵季脅。其病當所過者支轉筋痛,甚成息賁,脅急吐血。治在燔針劫刺,以知為數,以痛為輸,名曰仲冬痹也。
手心主之筋,起於中指,與太陰之筋並行,結於肘內廉,上臂陰,結腋下,下散前後挾脅;其支者,入腋,散胸中,結於臂。其病當所過者支轉筋,前及胸痛息賁。治在燔針劫刺,以知為數,以痛為輸,名曰孟冬痹也。
手少陰之筋,起於小指之內,側結於銳骨,上結肘內廉,上入腋,交太陰,挾乳裡,結於胸中,循臂,下繫於臍。其病內急,心承伏梁,下為肘網。其病當所過者支轉筋,筋痛。治在燔針劫刺,以知為數,以痛為輸。其成伏粱唾血膿者,死不治。經筋之病,寒則反折筋急,熱則筋弛縱不收,陰痿不用。陽急則反折,陰急則俯不伸。焠刺者,刺寒急也,熱則筋縱不收,無用燔針。名曰季冬痹也。
足之陽明,手之太陽,筋急則口目為噼,眥急不能卒視,治皆如右方也




\section{骨度第十四}

黃帝問於伯高曰:脈度言經脈之長短,何以立之?伯高曰:先度其骨節之大小廣狹長短,而脈度定矣。黃帝曰:願聞眾人之度,人長七尺五寸者,其骨節之大小長短各幾何?伯高曰:頭之大骨圍二尺六寸,胸圍四尺五寸,腰圍四尺二寸。

發所覆者,顱至項尺二寸,發以下至頤長一尺,君子終折。結喉以下至缺盆中長四寸。缺盆以下至(骨曷)(骨亏)長九寸。過則肺大,不滿則肺小。(骨曷)(骨亏])以下至天樞長八寸,過則胃大,不及則胃小。天樞以下至橫骨長六寸半,過則迴腸廣長,不滿則狹短。橫骨長六寸半,橫骨上廉以下至內輔之上廉長一尺八寸,內輔之上廉以下至下廉長三寸半,內輔下廉下至內踝長一尺三寸,內踝以下至地長三寸,膝膕以下至跗屬長一尺六寸,跗屬以下至地長三寸,故骨圍大則太過,小則不及。

角以下至柱骨長一尺,行腋中不見者長四寸,腋以下至季脅長一尺二寸,季脅以下至髀樞長六寸,髀樞以下至膝中長一尺九寸,膝以下至外踝長一尺六寸,外踝以下至京骨長三寸,京骨以下至地長一寸。

耳後當完骨者廣九寸,耳前當耳門者廣一尺三寸,兩顴之間相去七寸,兩乳之間廣九寸半,兩髀之間廣六寸半。足長一尺二寸,廣四寸半。

肩至肘長一尺七寸,肘至腕長一尺二寸半,腕至中指本節長四寸,本節至其末長四寸半。項發以下至背骨長二寸半,膂骨以下至尾骶二十一節長三尺,上節長一寸四分,分之一奇分在下,故上七節至於膂骨九寸八分分之七。

此眾人骨之度也,所以立經脈之長短也。是故視其經脈之在於身也,其見浮而堅,其見明而大者,多血;細而沉者,多氣也。



\section{五十營第十五}

黃帝曰:余願聞五十營,奈何?岐伯答曰:天週二十八宿,宿三十六分,人氣行一週,千八分。日行二十八宿,人經脈上下、左右、前後二十八脈,週身十六丈二尺,以應二十八宿,漏水下百刻,以分晝夜。
故人一呼,脈再動,氣行三寸,一吸,脈亦再動,氣行三寸,呼吸定息,氣行六寸。十息氣行六尺,日行二分。二百七十息,氣行十六丈二尺,氣行交通於中,一週於身,下水二刻,日行二十五分。五百四十息,氣行再周於身,下水四刻,日行四十分。二千七百息,氣行十週於身,下水二十刻,日行五宿二十分。一萬三千五百息,氣行五十營於身,水下百刻,日行二十八宿,漏水皆盡,脈終矣。
所謂交通者,並行一數也,故五十營備,得盡天地之壽矣,凡行八百一十丈也。




\section{營氣第十六}

黃帝曰:營氣之道,內谷為寶。谷入於胃,乃傳之肺,流溢於中,布散於外,精專者行於經隧,常營無已,終而復始,是謂天地之紀。故氣從太陰出,注手陽明,上行注足陽明,下行至跗上,注大指間,與太陰合,上行抵髀。從脾注心中,循手少陰出腋下臂,注小指,合手太陽,上行乘腋出(出頁)內,注目內眥,上巔下項,合足太陽,循脊下尻,下行注小指之端,循足心注足少陰,上行注腎,從腎注心,外散於胸中。循心主脈出腋下臂,出兩筋之間,入掌中,出中指之端,還注小指次指之端,合手少陽,上行注羶中,散於三焦,從三焦注膽,出脅注足少陽,下行至跗上,復從跗注大指間,合足厥陰,上行至肝,從肝上注肺,上循喉嚨,入頏顙之竅,究於畜門。其支別者,上額循巔下項中,循脊入骶,是督脈也,絡陰器,上過毛中,入臍中,上循腹裡,入缺盆,下注肺中,復出太陰。此營氣之所行也,逆順之常也。




\section{脈度第十七}

黃帝曰:願聞脈度。岐伯答曰:手之六陽,從手至頭,長五尺,五六三丈。手之六陰,從手至胸中,三尺五寸,三六一丈八尺,五六三尺,合二丈一尺。足之六陽,從足上至頭八尺,六八四丈八尺。足之六陰,從足至胸中,六尺五寸,六六三丈六尺,五六三尺,合三丈九尺。蹻脈從足至目,七尺五寸,二七一丈四尺,二五一尺,合一丈五尺。督脈任脈各四尺五寸,二四八尺,二五一尺,合九尺。凡都合一十六丈二尺,此氣之大經隧也。經脈為裡,支而橫者為絡,絡之別者為孫,盛而血者疾誅之,盛者瀉之,虛者飲藥以補之。

五藏常內閱於上七竅也,故肺氣通於鼻,肺和則鼻能知臭香矣;心氣通於舌,心和則舌能知五味矣;肝氣通於目,肝和則目能辨五色矣;脾氣通於口,脾和則口能知五穀矣;腎氣通於耳,腎和則耳能聞五音矣。五藏不和則七竅不通,六府不和則留為癰。故邪在府則陽脈不和,陽脈不和則氣留之,氣留之則陽氣盛矣。陽氣太盛則陰脈不利,陰脈不利則血留之,血留之則陰氣盛矣。陰氣太盛,則陽氣不能榮也,故曰關。陽氣太盛,則陰氣弗能榮也,故曰格。陰陽俱盛,不得相榮,故曰關格。關格者,不得盡期而死也。

黃帝曰:蹻脈安起安止?何氣榮水?岐伯答曰:蹻脈者,少陰之別,起於然骨之後,上內踝之上,直上循陰股入陰,上循胸裡入缺盆,上出人迎之前,入頄,屬目內眥,合於太陽、陽蹻而上行,氣並相還則為濡目,氣不榮則目不合。

黃帝曰:氣獨行五藏,不榮六府,何也?岐伯答曰:氣之不得無行也,如水之流,如日月之行不休,故陰脈榮其藏,陽脈榮其府,如環之無端,莫知其紀,終而復始。其流溢之氣,內溉藏府,外濡腠理。

黃帝曰:蹻脈有陰陽,何脈當其數?岐伯答曰:男子數其陽,女子數其陰,當數者為經,其不當數者為絡也。




\section{營衛生會第十八}

黃帝問於岐伯曰:人焉受氣?陰陽焉會?何氣為營?何氣為衛?營安從生?衛於焉會?老壯不同氣,陰陽異位,願聞其會。岐伯答曰:人受氣於谷,谷入於胃,以傳與肺,五藏六府,皆以受氣,其清者為營,濁者為衛,營在脈中,衛在脈外,營周不休,五十而復大會。陰陽相貫,如環無端。衛氣行於陰二十五度,行於陽二十五度,分為晝夜,故氣至陽而起,至陰而止。故曰:日中而陽隴為重陽,夜半而陰隴為重陰。故太陰主內,太陽主外,各行二十五度,分為晝夜。夜半為陰隴,夜半後而為陰衰,平旦陰盡而陽受氣矣。日中而陽隴,日西而陽衰,日入陽盡而陰受氣矣。夜半而大會,萬民皆臥,命曰合陰,平旦陰盡而陽受氣,如是無已,與天地同紀。

黃帝曰:老人之不夜瞑者,何氣使然?少壯之人不晝瞑者,何氣使然?岐伯答曰:壯者之氣血盛,其肌肉滑,氣道通,營衛之行,不失其常,故晝精而夜瞑。老者之氣血衰,其肌肉枯,氣道澀,五藏之氣相搏,其營氣衰少而衛氣內伐,故晝不精,夜不瞑。

黃帝曰:願聞營衛之所行,皆何道從來?岐伯答曰:營出於中焦,衛出於下焦。黃帝曰:願聞三焦之所出。岐伯答曰:上焦出於胃上口,並咽以上,貫膈而布胸中,走腋,循太陰之分而行,還至陽明,上至舌,下足陽明,常與營俱行於陽二十五度,行於陰亦二十五度,一週也,故五十度而復大會於手太陰矣。黃帝曰:人有熱飲食下胃,其氣未定,汗則出,或出於面,或出於背,或出於身半,其不循衛氣之道而出何也?岐伯曰:此外傷於風,內開腠理,毛蒸理洩,衛氣走之,固不得循其道,此氣慓悍滑疾,見開而出,故不得循其道,故命曰漏洩。
黃帝曰;願聞中焦之所出,岐伯答曰:中焦亦並胃中,出上焦之後,此所受氣者,泌糟粕,蒸津液,化其精微,上注於肺脈,乃化而為血,以奉生身,莫貴於此,故獨得行於經隧,命曰營氣。黃帝曰:夫血之與氣,異名同類,何謂也?岐伯答曰:營衛者精氣也,血者神氣也,故血之與氣,異名同類焉。故奪血者無汗,奪汗者無血,故人生有兩死而無兩生。
黃帝曰:願聞下焦之所出。岐伯答曰:下焦者,別迴腸,注於膀胱而滲入焉。故水谷者,常並居於胃中,成糟粕,而俱下於大腸,而成下焦,滲而俱下,濟泌別汁,循下焦而滲入膀胱焉。黃帝曰:人飲酒,酒亦入胃,谷未熟而小便獨先下何也?岐伯答曰:酒者熟谷之液也,其氣悍以清,故後谷而入,先谷而液出焉。黃帝曰:善。余聞上焦如霧,中焦如漚,下焦如瀆,此之謂也。

\section{四時氣第十九}

黃帝問於岐伯曰:夫四時之氣,各不同形,百病之起,皆有所生,灸刺之道,何者為定?岐伯答曰:四時之氣,各有所在,灸刺之道,得氣穴為定。故春取經血脈分肉之間,甚者深刺之,間者淺刺之。夏取盛經孫絡,取分間絕皮膚。秋取經腧,邪在府,取之合。冬取井滎,必深以留之。

溫瘧汗不出,為五十九痏。風(疒水)膚脹,為五十七痏,取皮膚之血者,盡取之。飧洩,補三陰之上,補陰陵泉,皆久留之,熱行乃止。轉筋於陽治其陽,轉筋於陰治其陰,皆卒刺之。徒(疒水),先取環谷下三寸,以鈹針針之,已刺而筩之,而內之,入而復之,以盡其(疒水),必堅,來緩則煩悗,來急則安靜,間日一刺之,(疒水)盡乃止。飲閉藥,方刺之時徒飲之,方飲無食,方食無飲,無食他食,百三十五日。著痹不去,久寒不已,卒取其三里骨為干。腸中不便,取三里,盛瀉之,虛補之。癘風者,素刺其腫上,已刺,以銳針針其處,按出其惡氣,腫盡乃止,常食方食,無食他食。
腹中常鳴,氣上衝胸,喘不能久立,邪在大腸,刺肓之原、巨虛上廉、三里。小腹控睾、引腰脊,上衝心,邪在小腸者,連睾系,屬於脊,貫肝肺,絡心繫。氣盛則厥逆,上衝腸胃,熏肝,散於肓,結於臍。故取之肓原以散之,刺太陰以予之,取厥陰以下之,取巨虛下廉以去之,按其所過之經以調之。善嘔,嘔有苦,長太息,心中憺憺,恐人將捕之,邪在膽,逆在胃,膽液洩則口苦,胃氣逆則嘔苦,故曰嘔膽。取三里以下胃氣逆,則刺少陽血絡以閉膽逆,卻調其虛實以去其邪。飲食不下,膈塞不通,邪在胃脘,在上脘則刺抑而下之,在下脘則散而去之。小腹痛腫,不得小便,邪在三焦約,取之太陽大絡,視其絡脈與厥陰小絡結而血者,腫上及胃脘,取三里。

覩其色,察其以,知其散復者,視其目色,以知病之存亡也。一其形,聽其動靜者,持氣口人迎以視其脈,堅且盛且滑者病日進,脈軟者病將下。諸經實者病三日已。氣口候陰,人迎候陽也。



\section{五邪第二十}

邪在肺,則病皮膚痛,寒熱,上氣喘,汗出,咳動肩背。取之膺中外腧,背三節五藏之傍,以手疾按之,快然,乃刺之,取之缺盆中以越之。

邪在肝,則兩脅中痛,寒中,惡血在內,行善掣,節時腳腫,取之行間以引脅下,補三里以溫胃中,取血脈以散惡血,取耳間青脈,以去其掣。

邪在脾胃,則病肌肉痛。陽氣有餘,陰氣不足,則熱中善飢;陽氣不足,陰氣有餘,則寒中腸鳴腹痛。陰陽俱有餘,若俱不足,則有寒有熱。皆調於三里。

邪在腎,則病骨痛陰痹。陰痹者,按之而不得,腹脹腰痛,大便難,肩背頸項痛,時眩。取之湧泉、崑崙,視有血者盡取之。

邪在心,則病心痛喜悲,時眩僕,視有餘不足而調之其輸也。


\section{寒熱病第二十一}

皮寒熱者,不可附席,毛髮焦,鼻槁臘,不得汗。取三陽之絡,以補手太陰。肌寒熱者,肌痛,毛髮焦而唇槁臘,不得汗。取三陽於下以去其血者,補足太陰以出其汗。骨寒熱者,病無所安,汗注不休。齒未槁,取其少陰於陰股之絡;齒已槁,死不治。骨厥亦然。骨痹,舉節不用而痛,汗注煩心,取三陰之經,補之。身有所傷血出多,及中風寒,若有所墮墜,四支懈惰不收,名曰體惰。取其小腹臍下三結交。三結交者,陽明、太陰也,臍下三寸關元也。厥痹者,厥氣上及腹。取陰陽之絡,視主病也,瀉陽補陰經也。

頸側之動脈人迎。人迎,足陽明也,在嬰筋之前。嬰筋之後,手陽明也,名曰扶突。次脈,足少陽脈也,名曰天牖。次脈,足太陽也,名曰天柱。腋下動脈,臂太陰也,名曰天府。陽迎頭痛,胸滿不得息,取之人迎。暴瘖氣鞭,取扶突與舌本出血。暴聾氣蒙,耳目不明,取天牖。暴攣癇眩,足不任身,取天柱。暴癉內逆,肝肺相搏,血溢鼻口,取天府。此為天牖五部。

臂陽明有入頄遍齒者,名曰大迎,下齒齲取之,臂惡寒補之,不惡寒瀉之。足太陽有入頄遍齒者,名曰角孫,上齒齲取之,在鼻與頄前。方病之時其脈盛,盛則瀉之,虛則補之。一曰取之出鼻外。足陽明有挾鼻入於面者,名曰懸顱,屬口,對入系目本,視有過者取之,損有餘,益不足,反者益。其足太陽有通項入於腦者,正屬目本,名曰眼系,頭目苦痛,取之在項中兩筋間,入腦乃別。陰蹻、陽蹻,陰陽相交,陽入陰,陰出陽,交於目銳眥,陽氣盛則瞋目,陰氣盛則瞑目。

熱厥取足太陰、少陽,皆留之;寒厥取足陽明、少陰於足,皆留之。舌縱涎下,煩悗,取足少陰。振寒灑灑,鼓頷,不得汗出,腹脹煩悗,取手太陰。刺虛者,刺其去也;刺實者,刺其來也。春取絡脈,夏取分腠,秋取氣口,冬取經輸,凡此四時,各以時為齊。絡脈治皮膚,分腠治肌肉,氣口治筋脈,經輸治骨髓、五藏。身有五部:伏免一;腓二,腓者腨也;背三;五藏之腧四;項五。此五部有癰疽者死。病始手臂者,先取手陽明、太陰而汗出;病始頭首者,先取項太陽而汗出;病始足脛者,先取足陽明而汗出。臂太陰可汗出,足陽明可汗出。故取陰而汗出甚者,止之於陽;取陽而汗出甚者,止之於陰。凡刺之害,中而不去則精洩,不中而去則致氣;精洩則病甚而恇,致氣則生為癰疽也。

\section{癲狂第二十二}

目眥外決於面者,為銳眥;在內近鼻者為內眥;上為外眥,下為內眥。癲疾始生,先不樂,頭重痛,視舉目赤,甚作極已,而煩心,候之於顏,取手太陽、陽明、太陰,血變而止。癲疾始作而引口啼呼喘悸者,候之手陽明、太陽,左強者攻其右,右強者攻其左,血變而止。癲疾始作先反僵,因而脊痛,候之足太陽、陽明、太陰、手太陽,血變而止。
治癲疾者,常與之居,察其所當取之處。病至,視之有過者瀉之,置其血於瓠壺之中,至其發時,血獨動矣,不動,灸窮骨二十壯。窮骨者,骶骨也。
骨癲疾者,顑齒諸腧分肉皆滿,而骨居,汗出煩悗。嘔多沃沫,氣下洩,不治。筋癲疾者,身倦攣急大,刺項大經之大杼脈。嘔多沃沫,氣下洩,不治。脈癲疾者,暴僕,四肢之脈皆脹而縱。脈滿,盡刺之出血;不滿,灸之挾項太陽,灸帶脈於腰相去三寸,諸分肉本輸。嘔多沃沫,氣下洩,不治。癲疾者,疾發如狂者,死不治。
狂始生,先自悲也,喜忘苦怒善恐者,得之憂飢,治之取手太陰、陽明,血變而止,及取足太陰、陽明。狂始發,少臥不飢,自高賢也,自辨智也,自尊貴也,善罵詈,日夜不休,治之取手陽明、太陽、太陰、舌下少陰,視之盛者,皆取之,不盛,釋之也。狂言、驚、善笑、好歌樂、妄行不休者,得之大恐,治之取手陽明、太陽、太陰。狂,目妄見、耳妄聞、善呼者,少氣之所生也,治之取手太陽、太陰、陽明、足太陰、頭、兩顑。狂者多食,善見鬼神,善笑而不發於外者,得之有所大喜,治之取足太陰、太陽、陽明,後取手太陰、太陽、陽明。狂而新發,未應如此者,先取曲泉左右動脈,及盛者見血,有頃已,不已,以法取之,灸骨骶二十壯。

風逆暴四肢腫,身漯漯,唏然時寒,飢則煩,飽則善變,取手太陰表裡,足少陰、陽明之經,肉清取滎,骨清取井、經也。厥逆為病也,足暴清,胸若將裂,腸若將以刀切之,煩而不能食,脈大小皆澀,暖取足少陰,清取足陽明,清則補之,溫則瀉之。厥逆腹脹滿,腸鳴,胸滿不得息,取之下胸二脅咳而動手者,與背腧以手按之立快者是也。內閉不得溲,刺足少陰、太陽與骶上以長針,氣逆則取其太陰、陽明、厥陰,甚取少陰、陽明動者之經也。少氣,身漯漯也,言吸吸也,骨痠體重,懈惰不能動,補足少陰。短氣,息短不屬,動作氣索,補足少陰,去血絡也。




\section{熱病第二十三}

偏枯,身偏不用而痛,言不變,志不亂,病在分腠之間,巨針取之,益其不足,損其有餘,乃可復也。痱之為病也,身無痛者,四肢不收,智亂不甚,其言微知,可治,甚則不能言,不可治也。病先起於陽,後入於陰者,先取其陽,後取其陰,浮而取之。

熱病三日而氣口靜、人迎躁者,取之諸陽,五十九刺,以瀉其熱而出其汗,實其陰以補其不足者。身熱甚,陰陽皆靜者,勿刺也;其可刺者,急取之,不汗出則洩。所謂勿刺者,有死征也。熱病七日八日,脈口動喘而短者,急刺之,汗且自出,淺刺手大指間。熱病七日八日,脈微小,病者溲血,口中干,一日半而死,脈代者,一日死。熱病已得汗出,而脈尚躁,喘且復熱,勿刺膚,喘甚者死。熱病七日八日,脈不躁,躁不散數,後三日中有汗;三日不汗,四日死。未曾汗者,勿腠刺之。

熱病先膚痛,窒鼻充面,取之皮,以第一針,五十九,苛軫鼻,索皮於肺,不得索之火,火者心也。熱病先身澀,倚而熱,煩悗,干唇口嗌,取之皮,以第一針,五十九,膚脹口乾,寒汗出,索脈於心,不得索之水,水者腎也。熱病嗌干多飲,善驚,臥不能起,取之膚肉,以第六針,五十九,目青,索肉於脾,不得索之木,木者,肝也。熱病面青腦痛,手足躁,取之筋間,以第四針於四逆,筋目浸,索筋於肝,不得索之金,金者,肺也,熱病數驚,而狂,取之脈,以第四針,急瀉有餘者,癲疾毛髮去,索血於心,不得索之水,水者,腎也。熱病身重骨痛,耳聾而好瞑,取之骨,以第四針,五十九,刺骨病不食,齒耳青,索骨於腎,不得索之土,土者,脾也。熱病不知所痛,耳聾不能自收,口乾,陽熱甚,陰頗有寒者,熱在髓,死,不可治。熱病頭痛,顳目,脈痛善,厥熱病也,取之以第三針,視有餘不足,寒熱痔熱病,體重,腸中熱,取之以第四針,於其俞及下諸指間,索氣於胃絡,得氣也,熱病挾齊急痛,胸脅滿,取之湧泉與陰陵泉,取以第四針,針嗌裡。熱病而汗且出,及脈順可汗者,取之魚際大淵大都大白,瀉之則熱去,補之則汗出,汗出太甚,取內踝上橫脈以止之。熱病已得汗而脈尚躁盛,此陰脈之極也,死。其得汗而脈靜者,生。熱病者脈尚盛躁而不得汗者,此陽脈之極也,死。脈盛躁得汗靜者,生。熱病不可刺者有九,一曰,汗不出,大顴發赤噦者,死,二曰,洩而腹滿甚者,死。三曰,目不明,熱不已者,死。四曰,老人嬰兒,熱而腹滿者,死。五曰,汗不出,嘔下血者,死。六曰,舌本爛,熱不已者,死。七曰,而,汗不出,出不至足者,死。八曰,髓熱者,死。九曰,熱而痙者,死,腰折,齒噤也。凡此尢九者,不可刺也。所謂五十九刺者,兩手外內側各三,凡十二,五指間各一,凡八,足亦如是。頭入發一寸傍三分各三,凡六。更入發三寸邊五,凡十。耳前後口下者各一,項中一,凡六。巔上一,囟會一,髮際一。廉泉一,風池二,天柱二。第三節氣滿胸中喘息,取足太陰大指之端,去爪甲如韭葉,寒則留之,熱則疾之,氣下乃止。心疝暴痛,取足太陰厥陰,盡刺去其血絡。喉痹舌卷,口中干,煩心,心痛,臂內廉痛,不可及頭,取手小指次指爪甲下,去端如韭葉。目中赤痛,從內始,取之陰。風痙身反折,先取足太陽及中及血絡出血,中有寒,取三里。癃,取之陰及三毛上及血絡出血,男子如蠱,女子如,身體腰脊如解,不欲飲食,先取湧泉見血,視跗上盛者,盡見血也。



\section{厥病第二十四}

厥頭痛,面若腫起而煩心,取之足陽明太陰。厥頭痛,頭脈痛,心悲善泣,視頭動脈反盛者,刺盡去血,後調足厥陰。厥頭痛,貞貞頭痛而重,瀉頭上五行行五,先取手少陰,後取足少陰。厥頭痛,意善忘,按之不得,取頭面左右動脈,後取足太陰。厥頭痛,項先痛,腰脊為應,先取天柱,後取足太陽。厥頭痛,頭痛甚,耳前後脈湧有熱,瀉出其血,後取足少陽。真頭痛,頭痛甚,腦盡痛,手足寒至節,死不治。頭痛不可取於俞者,有所擊墮,惡血在於內,若肉傷,痛未已,可則刺,不可遠取也。頭痛不可刺者,大痹為惡,日作者,可令少愈,不可已,頭半寒痛,先取手少陽陽明,後取足少陽陽明。

厥心痛,與背相控善,如從後觸其心,傴僂者,腎心痛也,先取京骨崑崙。發狂不已,取然谷。厥心痛,腹脹胸滿,心尤痛甚,胃心痛也,取之大都太白。厥心痛,痛如以錐針刺其心,心痛甚者,脾心痛也,取之然谷太溪。厥心痛,色蒼蒼如死狀,終日不得太息,肝心痛也,取之行間太沖。厥心痛,臥若徒居,心痛,間動作,痛益甚,色不變,肺心痛也,取之魚際太淵。真心痛,手足清至節,心痛甚,旦發夕死,夕發旦死。心痛不可刺者,中有盛聚,不可取於俞。腸中有蟲瘕及蛟有,皆不可取以小針。心腸痛,作痛,膿聚,往來上下行,痛有休止,腹熱喜渴涎出者,是蛟有也。以手聚按而堅持之,無令得移,以大針刺之,久持之,蟲不動,乃出針也。腹痛。形中上者。

耳聾無聞,取中耳,耳鳴,取耳前動脈。耳痛不可刺者,耳中有膿,若有干盯,耳無聞也。耳聾取手小指次指爪甲上與肉交者,先取手,後取足。耳鳴取手中指爪甲上,左取右,右取左,先取手,後取足。足髀不可舉,側而取之,在樞谷中,以員利針,大針不可刺。病注下血,取曲泉。風痹淫濼,病不可已者,足如履冰,時如入湯中,股脛淫樂,煩心頭痛,時嘔時,眩已汗出,久則目眩,悲以喜恐,短氣,不樂,不出三年,死矣。



\section{病本第二十五}

先病而後逆者,治其本,先逆而後病者,治其本,先寒而後生病者,治其本。先病而後生寒者,治其本。先熱而後生病者,治其本。先洩而後生他病者,治其本。必且調之,乃治其他病。先病而後中滿者,治其標。先病後洩者,治其本。先中滿而後煩心者,治其本。有客氣有同氣,大小便不利,治其標。大小便利,治其本。病發而有餘,本而標之,先治其本,後治其標。病發而不足,標而本之,先治其標,後治其本。謹詳察間甚,以意調之,間者並行,甚為獨行。先小大便不利而後生他病者,治其本也。



\section{雜病第二十六}

厥挾脊而痛者,至頂,頭沉沉然,目然,腰脊強,取足太陽中血絡。厥胸滿面腫,漯漯,然暴言難,甚則不能言,取足陽明。厥氣走喉而不能言,手足清,大便不利,取足少陰。厥而腹向向然,多寒氣,腹中,便溲難,取足太陰。
嗌干,口中熱如膠,取足少陰,膝中痛,取犢鼻,以員利針,發而間之,針大如,刺膝無疑。喉痹不能言,取足陽明。能言,取手陽明。瘧不渴,間日而作,取足陽明。渴而日作,取手陽明。齒痛不惡清飲,取足陽明。惡清飲,取手陽明。聾而不痛者,取足少陽。聾而痛者。取手陽明。而不止,杯血流,取足太陽。杯血,取手太陽,不已,刺宛骨下,不已,刺中出血。腰痛,痛上寒,取足太陽陽明。痛上熱,取足厥陰。不可以仰,取足少陽。中熱而喘,取足少陰中血絡。喜怒而不欲食,言益小,刺足太陰。怒而多言,刺足少陽。痛,刺手陽明與之盛脈出血。項痛不可仰,刺足太陽。不可以顧,刺手太陽也。
小腹滿大,上走胃,至心,淅淅身時寒熱,小便不利,取足厥陰。腹滿,大便不利,腹大,亦上走胸嗌,喘息喝喝然,取足少陰。腹滿食不化,腹向向然,不能大便,取足太陰。心痛引腰脊,欲嘔,取足少陰。心痛腹脹,嗇嗇然,大便不利,取足太陰。
心痛引背,不得息,刺足少陰,不已,取手少陽。心痛引小腹滿,上下無常處,便溲難,刺足厥陰。心痛,但短氣不足以息,刺手太陰。心痛,當九節刺之,按,已刺按之,立已。不已,上下求之,得之立已。
痛,刺足陽明曲周動脈見血,立已。不已,按人迎於經,立已。氣逆上,刺膺中陷者與下胸動脈,腹痛,刺齊左右動脈,已刺按之,立已。不已,刺氣街,已按刺之,立已。痿厥為四末,乃疾解之,日二,不仁者,十日而知,無休,病已止。噦以草刺鼻,嚏,嚏而已。無息而疾迎引之,立已。大驚之,亦可已。



\section{周痹第二十七}

黃帝問於岐伯曰:周痹之在身也,上下移徒隨脈,其上下左右相應,間不容空,願聞此痛,在血脈之中邪,將在分肉之間乎,何以致是。其痛之移也,間不及下針,其痛之時,不及定治,而痛已止矣,何道使然,願聞其故。

岐伯答曰:此眾痹也。非周痹也。黃帝曰:願聞眾痹。岐伯對曰:此各在其處,更發更止,更居更起,以右應左,以左應右。非能周也,更發更休也。黃帝曰:善。刺之奈何?岐伯對曰:刺此者,痛雖已止,必刺其處,勿令復起。

帝曰:善。願聞周痹何如?岐伯對曰:周痹者,在於血脈之中,隨脈以上,隨脈以下,不能左右,各當其所。黃帝曰:刺之奈何?岐伯對曰:痛從上下者,先刺其下以過之,後刺其上以脫之,痛從下上者,先刺其上以過之,後刺其下以脫之。

黃帝曰:善。此痛安生。何因而有名。岐伯對曰:風寒濕氣,客於外分肉之間,迫切而為沫,沫得寒則聚,聚則排分肉而分裂也,分裂則痛,痛則神歸之,神歸之則熱,熱則痛解,痛解則厥,厥則他痹發,發則如是。

帝曰:善。余已得其意矣。此內不在藏,而外未發於皮,獨居分肉之間,真氣不能周,故命曰周痹。故刺痹者,必先切循其下之六經,視其虛實,及大絡之血結而不通,及虛而脈陷空者而調之,熨而通之其手堅,轉引而行之。黃帝曰:善。余已得其意矣。亦得其事也。九者,經巽之理,十二經脈陰陽之病也。




\section{口問第二十八}

黃帝閒居,辟左右而問於岐伯曰:余已聞九針之經,論陰陽逆順,六經已畢,願得口問。岐伯避席再拜曰:善乎哉問也,此先師之所口傳也。黃帝曰:願聞口傳。岐伯答曰:夫百病之始生也,皆生於風雨寒暑,陰陽喜怒,飲食居處,大驚卒恐,則血氣分離,陰陽破散,經絡厥絕,脈道不通,陰陽相逆,衛氣稽留,經脈虛空,血氣不次,乃失其常,論不在經者,請道其方。

黃帝曰:人之欠者,何氣使然。岐伯答曰:衛氣晝日行於陽,夜半則行於陰,陰者主夜,夜者臥,陽者主上,陰者主下,故陰氣積於下,陽氣未盡,陽引而上,陰引而下,陰陽相引,故數欠,陽氣盡,陰氣盛則目瞑,陰氣盡而陽氣盛,盛則寤矣。瀉足少陰,補足太陽。
黃帝曰:入之噦者,何氣使然。岐伯曰:谷入於胃,胃氣上注於肺,今有故寒氣與新谷氣,俱還入於胃,新故相亂,真邪相攻,氣並相逆,復出於胃,故為噦,補手太陰,瀉足少陰。
黃帝曰:人之唏者,何氣使然。岐伯曰:此陰氣盛而陽氣虛,陰氣疾而陽氣徐,陰氣盛而陽氣絕,故為唏,補足太陽,瀉足少陰。
黃帝曰:人之振寒者,何氣使然。岐伯曰:寒氣客於皮膚,陰氣盛,陽氣虛,故為振寒寒慄,補諸陽。
黃帝曰:人之噫者,何氣使然。岐伯曰:寒氣客於胃,厥逆從下上散,復出於胃,故為噫,補足太陰陽明,一曰補眉本也。
黃帝曰:人之嚏者,何氣使然。岐伯曰:陽氣和利,滿於心,出於鼻,故為嚏,補足太陽滎眉本,一曰眉上也。
黃帝曰:人之者,何氣使然。岐伯曰:胃不實則諸脈虛,諸脈虛則筋脈懈惰,筋脈懈惰則行陰用力,氣不能復,故為,因其所在,補分肉間。
黃帝曰:人之哀而泣涕出者,何氣使然。岐伯曰:心者,五藏六府之主也。目者,宗脈之所聚也,上液之道也。口鼻者,氣之門戶也。故悲哀愁憂則心動,心動則五藏六府皆搖,搖則宗脈感,宗脈感則液道開,液道開,故泣涕出焉液者,所以灌精濡空竅者也。故上液之道開,則泣,泣不止則液竭,液竭則精不灌,精不灌則目無所見矣,故命曰奪精,補天柱經俠頸。
黃帝曰:人之太息者,何氣使然。岐伯曰:憂思則心繫急,心繫急則氣道約,約則不利,故太息以伸出之,補手少陰心主,足少陽留之也。
黃帝曰:人之涎下者,何氣使然。岐伯曰:飲食者,皆入於胃,胃中有熱則蟲動,蟲動則胃緩,胃緩則廉泉開,故涎下,補足少陰。
黃帝曰:人之耳中鳴者,何氣使然。岐伯曰:耳者,宗脈之所聚也,故胃中空則宗脈虛,虛則下溜,脈有所竭者,故耳鳴,補客主人,手大指爪甲上與肉交者也。
黃帝曰:人之自舌者,何氣使然。此厥逆走上,脈氣輩至也,少陰氣至則舌,少陽氣至則頰,陽明氣至則矣,視主病者,則補之。

凡此十二邪者,皆奇邪之走空竅者也,故邪之所在,皆為不足,故上氣不足,腦為之不滿,耳為之苦鳴,頭為之苦頃,目為之眩,中氣不足,溲便為之變,腸為之苦鳴。下氣不足,則乃為痿厥心,補足外踝下留之。黃帝曰:治之奈何?岐伯曰:腎主為欠,取足少陰。肺主為噦,取手太陰。足少陰唏者,陰與陽絕,故補足太陽,瀉足少陰。振寒者,補諸陽。噫者,補足太陰陽明。嚏者,補足太陽眉本。因其所在,補分肉間。泣出補天柱經俠頸,俠頸者,頭中分也。太息補手少陰心主,足少陽留之。涎下補足少陰。耳鳴補客主人,手大指爪甲上與肉交者。自舌,視主病者,則補之。目眩頭頃,補足外踝下留之。痿厥心,刺足大指間上二寸留之。一曰足外踝下留之。

\section{師傳第二十九}

黃帝曰:余聞先師,有所心藏,弗著於方,余願聞而藏之,則而行之,上以治民,下以治身,使百姓無病,上下和親,德澤下流,子孫無憂,傳於後世,無所終時,可得聞乎?岐伯曰:遠乎哉問也。夫治與民,自治,治彼與治此,治小與治大,治國與治家,未有逆而能治之也,夫惟順而已矣。順者,非獨陰陽脈,論氣之逆順也,百姓人民,皆欲順其志也。
黃帝曰:順之奈何?岐伯曰:入國問俗,入家問諱,上堂問禮,臨病人問所便。黃帝曰:便病人奈何?岐伯曰:夫中熱消癉則便寒,寒中之屬則便熱,胃中熱則消谷,令人懸心善,齊以上皮熱。腸中熱,則出黃如糜,齊以下皮寒。胃中寒,則腹脹,腸中寒,則腸鳴飧洩。胃中寒,腸中熱,則脹而且洩。胃中熱,腸中寒,則疾小腹痛脹。黃帝曰:胃欲寒飲,腸欲熱飲,兩者相逆,便之奈何?且夫王公大人,血食之君,驕恣從欲輕人,而無能禁之,禁之則逆其志,順之則加其病,便之奈何,治之何先。
岐伯曰:人之情,莫不惡死而樂生。告之以其敗,語之以其善,導之以其所便,開之以其所苦,雖有無道之人,惡有不聽者乎?黃帝曰:治之奈何?岐伯曰:春夏先治其標,後治其本,秋冬先治其本,後治其標。黃帝曰:便其相逆者奈何?岐伯曰:便此者,飲食衣服,亦欲適寒溫,寒無淒愴,暑無出汗。食飲者,熱無灼灼,寒無滄滄,寒溫中適,故氣將持,乃不致邪僻也。

黃帝曰:本藏以身形支節肉,候五藏六府之大小焉。今夫王公大人,臨朝即位之君,而問焉,誰可捫循之。而後答乎?岐伯曰:身形支節者,藏府之蓋也,非面部之閱也。黃帝曰:五藏之氣,閱於面者,余已知之矣,以支節知而閱之,奈何?岐伯曰:五藏六府者,肺為之蓋,巨肩陷,咽喉見其外。黃帝曰:善。岐伯曰:五藏六府,心為之主,缺盆為之道,骷骨有餘,以候。黃帝曰:善。岐伯曰:肝者,主為將,使知候外,欲知堅固,視目小大。黃帝曰:善。岐伯曰:脾者,主為衛,使之迎糧,視舌好惡,以知吉凶。黃帝曰:善。岐伯曰:腎者:主為外,使之遠聽,視耳好惡,以知其性。黃帝曰:善。
願聞六府之候。岐伯曰:六府者,胃為之海,廣骸大頸張胸,五穀乃容,鼻隧以長,以候大腸,厚,人中長,以候小腸,目下果大,其膽乃橫,鼻孔在外,膀胱漏洩,鼻柱中央起。三焦乃約,此所以候六府者也。上下三等,藏安且良矣。



\section{決氣第三十}

黃帝曰:余聞人有精氣津液血脈,余意以為一氣耳,今乃辨為六名,余不知其所以然。岐伯曰:兩神相搏,合而成形,常先身生,是謂精。何謂氣。岐伯曰:上焦開發,宣五穀味,熏膚,充身,澤毛,若霧露之溉,是謂氣。何謂津。岐伯曰:腠理髮洩,汗出溱溱,是謂津。何謂液。岐伯曰:谷入氣滿,淖澤注於骨,骨屬屈伸,澤補益腦髓,皮膚潤澤,是謂液。何謂血。岐伯曰:中焦受氣,取汁變化而赤,是謂血。何謂脈。岐伯曰:壅遏滎氣,令無所避,是謂脈。

黃帝曰:六氣者,有餘不足,氣之多少,腦髓之虛實,血脈之清濁,何以知之。岐伯曰:精脫者,耳聾。氣脫者,目不明。津脫者,腠理開,汗大洩,。液脫者,骨屬屈伸不利,色天,腦髓消,脛,耳數鳴。血脫者,色白,夭然不澤,其脈空虛,此其候也。

黃帝曰:六氣者,貴賤何如?岐伯曰:六氣者,各有部主也,其貴賤善惡,可為常主,然五穀與胃為大海也。


\section{腸胃第三十一}

黃帝問於伯高曰:余願聞六府傳谷者,腸胃之小大長短,受谷之多少奈何?伯高曰:請盡言之,谷所從出入淺深遠近長短之度,唇至齒,長九分,口廣二寸半,齒以後至厭,深三寸半,大容五合,舌重十兩,長七寸,廣二寸半。咽門重十兩,廣二寸半,至胃長一尺六寸。
紆曲屈伸之,長二尺六寸,大一尺五寸徑五寸,大容三斗五升。第三節小腸後附脊左環,回周疊積,其注於迴腸者,外附於齊,上回運環十六曲,大二寸半,徑八分分之少半,長三丈三尺。迴腸當齊左環,回周葉積而下,回運環反十六曲,大四寸,徑一寸寸之少半,長二丈一尺。廣腸傳脊,以受迴腸,左環葉脊上下辟,大八寸,徑二寸寸之大半,長二尺八寸。第四節腸胃所入至所出,長六丈四寸四分,回曲環反,三十二曲也。



\section{平人絕谷第三十二}

黃帝曰:願聞人之不食,七日而死,何也?伯高曰:臣請言其故。胃大一尺五寸,徑五寸,長二尺六寸,橫屈受水谷三斗五升,其中之谷,常留二斗,水一斗五升而滿,上焦洩氣,出其精微,悍滑疾,下焦下溉諸腸。

小腸大二寸半,徑八分分之少半,長三丈二尺,受谷二斗四升,水六升三合合之大半。迴腸大四寸,徑一寸寸之少半,長二丈一尺,受谷一斗,水七升半。廣腸大八寸,徑二寸寸之大半,長二尺八寸,受谷九升三合八分合之一。

腸胃之長,凡五丈八尺四寸,受水谷九斗二升一合合之大半,此腸胃所受水谷之數也。平人則不然,胃滿則腸虛,腸滿則胃虛,更虛更滿,故氣得上下,五藏安定,血脈和則精神乃居,故神者水谷之精氣也。故腸胃之中,當留谷二斗,水一斗五升,故平人日再後,後二升半,一日中五升,七日五七三斗五升,而留水谷盡矣。故平人不食飲七日而死者,水谷精氣津液皆盡故也。


\section{海論第三十三}

黃帝問於岐伯曰:余聞刺法於夫子,夫子之所言,不離於滎衛血氣。夫十二經脈者,內屬於府藏,外絡於支節,夫子乃合之於四海乎?岐伯答曰:人亦有四海,十二經水,經水者,皆注於海,海有東南西北,命曰四海。黃帝曰:以人應之奈何?岐伯曰:人有髓海,有血海,有氣海,有水穀之海,凡此四者,以應四海也。黃帝曰:遠乎哉。夫子之合人天地四海也,願聞應之奈何?岐伯答曰:日必先明知陰陽表裡滎輸所在,四海定矣。

黃帝曰:定之奈何?岐伯曰:胃者水穀之海,其輸上在氣街,下至三里。衝脈者,為十二經之海,其輸上在於大杼,下出於巨虛之上下廉。羶中者,為氣之海,其輸上在柱骨之上下,前在於人迎。腦為髓之海,其輸上在於其蓋,下在風府。黃帝曰:凡此四海者,何利何害,何生何敗。岐伯曰:得順者生,得逆者敗,知調者利,不知調者害。

黃帝曰:四海之逆順奈何?岐伯曰:氣海有餘者,氣滿胸中,息面赤。氣海不足,則氣少不足以言。血海有餘,則常想其身大,怫然不知其所病。血海不足,亦常想身小,狹然不知其所病。水穀之海有餘,則腹滿。水穀之海不足,則飢不受谷食。髓海有餘,則輕勁多力,自過其度。髓海不足,則腦轉耳鳴,脛眩冒,目無所見,懈怠安臥。黃帝曰:余已聞逆順,調之奈何?岐伯曰:審守其輸,而調其虛實,無犯其害,順者得復。逆者必敗。黃帝曰:善。



\section{五亂第三十四}

黃帝曰:經脈十二者,別為五行,分為四時,何失而亂,何得而治。岐伯曰:五行有序,四時有分,相順則治,相逆則亂。黃帝曰:何謂相順。岐伯曰:經脈十二者,以應十二月,十二月者,分為四時,四時者,春夏冬秋,其氣各異,滎衛相隨,陰陽已和,清濁不相干,如是則順之而治。
黃帝曰:何謂逆而亂。岐伯曰:清氣在陰,濁氣在陽,滎氣順脈,衛氣逆行,清濁相干,亂於胸中,是謂大。故氣亂於心,則煩心密嘿,首靜伏。亂於肺,則仰喘喝,接手以呼。亂於腸胃,則為霍亂。亂於臂脛,則為四厥。亂於頭,則為厥逆,頭重眩僕。

黃帝曰:五亂者,刺之有道乎?岐伯曰:有道以來,有道以去,審知其道,是謂身寶。黃帝曰:善。願聞其道。岐伯曰:氣在於心者,取之手少陰心主之輸。氣在於肺者,取之手太陰滎足少陰輸。氣在於腸胃者,取之足太陰陽明,不下者,取之三里。氣在於頭者,取之天柱大杼,不知,取足太陽滎輸。氣在於臂足,取之先去血脈,後取其陽明少陽之滎輸。黃帝曰:補瀉奈何?岐伯曰:徐入徐出,謂之導氣,補瀉無形,謂之同精,是非有餘不足也,亂氣之相交也。黃帝曰:允乎哉道,明乎哉論,請著之玉版,命曰治亂也。

\section{脹論第三十五}

黃帝曰:脈之應於寸口,如何而脹。岐伯曰:其脈大堅以澀者,脹也。黃帝曰:何以知藏府之脹也。岐伯曰:陰為藏,陽為府。黃帝曰:夫氣之令人脹也,在於血脈之中邪藏府之內乎?岐伯曰:三者皆存焉。然非脹之舍也。黃帝曰:願聞脹之舍。岐伯曰:夫脹者,皆在於藏府之外,排藏府而郭胸脅,脹皮膚,故命曰:脹。黃帝曰:藏府之在胸脅腹裡之內也,若匣匱之藏禁器也,各有次舍,異名而同處,一域之中,其氣各異,願聞其故。黃帝曰:未解其意。再問岐伯曰:夫胸腹,藏府之郭也。羶中者,心主之宮城也。胃者,太倉也。咽喉小腸者,傳送也。胃之五竅者,閭裡門戶也。廉泉玉英者,津液之道也。故五藏六府者,各有畔界,其病各有形狀。滎氣循脈,衛氣逆為脈脹,衛氣並脈循分為膚脹,三里而瀉,近者一下,遠者三下,無問虛實,工在疾瀉。

黃帝曰:願聞脹形。岐伯曰:夫心脹者,煩心短氣。臥不安。肺脹者,虛滿而喘。肝脹者,脅下滿而痛引小腹。脾脹者,善噦,四支煩,體重不能勝衣,臥不安。腎脹者,腹滿引背,央央然腰髀痛。六府脹。胃脹者,腹滿,胃脘痛,鼻聞焦臭,妨於食,大便難。大腸脹者,腸鳴而痛濯濯,冬日重感於寒,則飧洩不化。小腸脹者,少腹脹,引腰而痛。膀胱脹者,少腹滿而氣癃。三焦脹者,氣滿於皮膚中,輕輕然而不堅。膽脹者,脅下痛脹,口中苦,善太息。凡此諸脹者,其道在一,明知逆順,針數不失,瀉虛補實,神去其室,致邪失正,真不可定,之所敗,謂之天命,補虛瀉實,神歸其室。久塞其空,謂之良工。

黃帝曰:脹者焉生,何因而有。岐伯曰:衛氣之在身也,常然並脈循分肉,行有逆順,陰陽相隨,乃得天和,五藏更始,四時有序,五穀乃化,然後厥氣在上,滎衛留止,寒氣逆上,真邪相攻,兩氣相搏,乃合為脹也。黃帝曰:善。何以解惑。岐伯曰:合之於真,三合而得。帝曰: 善。黃帝問於岐伯曰:脹論言無問虛實,工在疾瀉,近者一下,遠者三下,今有其三而不下者,其過焉在。岐伯對曰:此言陷於肉盲,而中氣穴者也。不中氣穴,則氣內閉,針不陷盲,則氣不行,上越中肉,則衛氣相亂,陰陽相逐,其於脹也,當瀉不瀉,氣故不下,三而不下,必更其道,氣下乃止,不下復始,可以萬全,烏有殆者乎,其於脹也,必審其,當瀉則瀉,當補則補,如鼓應桴,惡有不下者乎?



\section{五癃津液別第三十六}

黃帝問於岐伯曰:水谷入於口,輸於腸胃,其液別為五,天寒衣薄,則為溺與氣,天熱衣厚則為汗,悲哀氣並則為泣,中熱胃緩則為唾,邪氣內逆則氣為之閉塞而不行,不行則為水脹,余知其然也,不知其所由生,願聞其道。
岐伯曰:水谷皆入於口,其味有五,各注其海,津液各走其道,故三焦出氣,以溫肌肉,充皮膚,為其津,其流而不行者為液。天暑衣厚則腠理開,故汗出,寒留於分肉之間,聚沫則為痛,天寒則腠理閉,氣濕不行,水下留於膀胱,則為溺與氣。五藏六府,心為之主,耳為之聽,目為之候,肺為之相,肝為之將,脾為之衛,腎為之主外。故五藏六府之津液,盡上滲於目,心悲氣並,則心繫急,心繫急則肺舉,肺舉則液上溢。夫心繫與肺,不能盡舉,乍上乍下,故而泣出矣。中熱則胃中消谷,消谷則蟲上下作,腸胃充郭,故胃緩,胃緩則氣逆,故唾出。
五穀之津液和合而為膏者,內滲入於骨空,補益腦髓,而下流於陰陽。陰陽不和,則使液溢而下流於陰,髓液皆減而下,下過度則虛,虛,故腰背痛而脛。陰陽氣道不通,四海塞閉,三焦不瀉,津液不化,水谷並於腸胃之中,別於迴腸,留於下焦,不得滲膀胱,則下焦脹,水溢則為水脹,此津液五別之逆順也。

\section{五閱五使第三十七}

黃帝問於岐伯曰:余聞刺有五官五閱以觀五氣,五氣者,五藏之使也,五時之副也,願聞其五使當安出。岐伯曰:五官者,五藏之閱也。黃帝曰:願聞其所出,令可為常。岐伯曰:脈出於氣口,色見於明堂,五色更出,以應五時,各如其常,經氣入藏,必當治裡。帝曰:善。五色獨決於明堂乎?岐伯曰:五官已辨,闕庭必張,乃立明堂,明堂廣大,蕃蔽見外,方壁高基,引垂居外,五色乃治,平博廣大,壽中百歲。見此者,刺之必已,如是之人者,血氣有餘,肌肉堅致,故可苦以針。

黃帝曰:願聞五官。岐伯曰:鼻者,肺之官也。目者,肝之官也。口者,脾之官也。舌者,心之官也。耳者,腎之官也。黃帝曰:以官何候。岐伯曰:以候五藏。故肺病者,喘息鼻張。肝病者,青。脾病者,黃。心病者,舌卷短,顴赤。腎病者,顴與顏黑。黃帝曰:五脈安出,五色安見,其常色殆者何如?岐伯曰:五官不辨,闕庭不張,小其明堂,蕃蔽不見,又埤其,下無基,垂角去外,如是者,雖平常殆,況加病哉。黃帝曰:五色之見於明堂,以觀五藏之氣,左右高下,各有形乎?岐伯曰:五藏之在中也,各以次舍左右上下,各如其度也。

\section{逆順肥瘦第三十八}

黃帝問於岐伯曰:余聞針道於夫子,眾多畢悉矣。夫子之道,應若失,而據未有堅然者也。夫子之問學熟乎,將審察於物而生之乎?岐伯對曰:聖人之為道者,上合於天,下合於地,中合於人事,必有明法,以起度數,法式檢押,乃後可傳焉。故匠人不能釋尺寸而意短長,廢繩墨而起平水也,工人不能置規而為員,去矩而為方。知用此者,固自然之物,易用之教,逆順之常也。黃帝曰:願聞自然奈何?岐伯曰:臨深決水,不用工力,而水可竭也,循掘決沖,而經可通也,此言氣之滑澀,血之清濁,行之逆順也。

黃帝曰:願聞人之黑白肥瘦小長,各有數乎?岐伯曰:年質壯大,血氣充盈,膚革堅固,因加以邪,刺此者,深而留之,此肥人也。廣肩,腋項肉薄,皮厚而黑色,臨臨然,其血黑以濁,其氣澀以遲,其為人也,貪而於取與。刺此者,深而留之,多益之數也。黃帝曰:刺瘦人奈何?岐伯曰:瘦人者,皮薄色少,肉廉廉然,薄輕言,其血清氣滑,易脫於氣,易損於血,刺此者,淺而疾之。黃帝曰:刺常人奈何?岐伯曰:視其白黑,各為調之,其端正惇厚者,其血氣和調,刺此者,無失常數也。黃帝曰:刺壯士真骨者,奈何?岐伯曰:刺壯士真骨,堅肉緩節監監然,此人重則氣澀血濁,刺此者,深而留之,多益其數。勁則氣滑血清,刺此者,淺而疾之。黃帝曰:刺嬰兒奈何?岐伯曰:嬰兒者,其肉脆,血少氣弱,刺此者,以毫針,淺刺而疾髮針,日再可也。

黃帝曰:臨深決水奈何?岐伯曰:血清氣濁,疾瀉之,則氣竭焉。黃帝曰:循掘決沖,奈何?岐伯曰:血濁氣澀,疾瀉之,則經可通也。黃帝曰:脈行之逆順,奈何?岐伯曰:手之三陰。從藏走手,手之三陽,從手走頭,足之三陽,從頭走足,足之三陰,從足走腹。黃帝曰:少陰之脈獨下行,何也?岐伯曰:不然,夫衝脈者,五藏六府之海也,五藏六府皆稟焉。其上者,出於頏顙。滲諸陽,灌諸精。其下者,注少陰之大絡,出於氣街,循陰股內廉,入中,伏行骨內,下至內踝之後屬而別。其下者,並於少陰之經,滲三陰,其前者,伏行出跗屬,下循跗,入大指間,滲諸絡而溫肌肉。故別絡結則跗上不動,不動則厥,厥則寒矣。黃帝曰:何以明之。岐伯曰:以言導之,切而驗之,其非必動,然後乃可明逆順之行也。黃帝曰:窘乎哉,聖人之為道也,明於日月,微於毫,其非夫子,孰能道之也。



\section{血絡論第三十九}

黃帝曰:願聞其奇邪而不在經者。岐伯曰:血絡是也。黃帝曰:刺血絡而僕者,何也?血出而射者,何也?血少黑而濁者,何也?血出清而半為汁者,何也?髮針而腫者,何也?血出若多若少而面色蒼蒼者,何也?髮針而面色不變而煩者,何也?多出血而不動搖者,何也?願聞其故。
岐伯曰:脈氣盛而血虛者,刺之則脫氣,脫氣則僕。血氣俱盛而陰氣多者,其血滑,刺之則射。陽氣畜積,久留而不瀉者,血黑以濁,故不能射。新飲而液滲於絡,而未合和於血也,故血出而汁別焉。其不新飲者,身中有水,久則為腫。陰氣積於陽,其氣因於絡,故刺之血未出而氣先行,故腫。陰陽之氣,其新相得而未和合,因而瀉之,則陰陽俱脫,表裡相離,故脫色而蒼蒼然。刺之血出多,色不變而煩者,刺絡而虛經,虛經之屬於陰者。陰脫故煩悶。陰陽相得而合為痹者,此為內溢於經,外注於絡,如是者,陰陽俱有餘,雖多出血而弗能虛也。

黃帝曰:相之奈何?岐伯曰:血脈者,盛堅橫以赤,上下無常處,小者如針,大者如筋,則而瀉之萬全也,故無失數矣,失數而反,各如其度。黃帝曰:針入而肉著,何也?岐伯曰:熱氣因於針,則針熱,熱則肉著於針,故堅焉。

\section{陰陽清濁第四十}

黃帝曰:余聞十二經脈,以應十二經水者,其五色各異,清濁不同,人之血氣若一,應之奈何?岐伯曰:人之血氣,苟能若一,則天下為一矣,惡有亂者乎?黃帝曰:余聞一人,非問天下之眾。岐伯曰:夫一人者,亦有亂氣,天下之眾,亦有亂人,其合為一耳。黃帝曰:願聞人氣之清濁。岐伯曰:受谷者濁,受氣者清。清者注陰,濁者注陽。濁而清者,上出於咽。清而濁者,則下行。清濁相干,命曰亂氣。

黃帝曰:夫陰清而陽濁,濁者有清,清者有濁,清濁別之奈何?岐伯曰:氣之大別,清者上注於肺,濁者下走於胃,胃之清氣,上出於口,肺之濁氣,下注於經,內積於海。黃帝曰:諸陽皆濁,何陽獨甚乎?岐伯曰:手太陽獨受陽之濁,手太陰獨受陰之清,其清者上走空竅,其濁者獨下行諸經,諸陰皆清,足太陰獨受其濁。

黃帝曰:治之奈何?岐伯曰:清者其氣滑,濁者其氣澀,此氣之常也。故刺陰者,深而留之,刺陽者,淺而疾之,清濁相干者,以數調之也。

\section{陰陽系日月第四十一}

黃帝曰:余聞天為陽,地為陰,日為陽,月為陰,其合之於人,奈何?岐伯曰:腰以上為天,腰以下為地,故天為陽,地為陰。故足之十二經脈以應十二月,月生於水,故在下者為陰。手之十指,以應十日,日主火,故在上者為陽。

黃帝曰:合之於脈,奈何?岐伯曰:寅者,正月之生陽也,主左足之少陽。未者,六月,主右足之少陽。卯者,二月,主左足之太陽。午者,五月,主右足之太陽。辰者,三月,主左足之陽明。巳者,四月,主右足之陽明,此兩陽合於前,故曰陽明。申者,七月之生陰也,主右足之少陰。丑者,十二月,主左足之少陰。酉者,八月,主右足之太陰。子者,十一月,主左足之太陰。戍者,九月,主右足之厥陰。亥者,十月,主左足之厥陰,此兩陰交盡,故曰厥陰。

甲主左手之少陽,己主右手之少陽,乙主左手之太陽,戊主右手之太陽,丙主左手之陽明,丁主右手之陽明,此兩火併合,故為陽明。庚主右手之少陰,癸主左手之少陰,辛主右手之太陰,壬主左手之太陰。

故足之陽者,陰中之少陽也。足之陰者,陰中之太陰也。手之陽者,陽中之太陽也。手之陰者,陽中之少陰也。腰以上者為陽,腰以下者為陰。其於五藏也,心為陽中之太陽,肺為陽中之少陰,肝為陰中之少陽,脾為陰中之至陰,腎為陰中之太陰。

黃帝曰:以治奈何?岐伯曰:正月二月三月,人氣在左,無刺左足之陽。四月五月六月,人氣在右,無刺右足之陽。七月八月九月,人氣在右,無刺右足之陰。十月十一月十二月,人氣在左,無刺左足之陰。黃帝曰:五行以東方甲乙木王春,春者,蒼色,主肝,肝者,足厥陰也。今乃以甲為左手之少陽,不合於數,何也?岐伯曰:此天地之陰陽也,非四時五行之以次行也。且夫陰陽者,有名而無形,故數之可十,推之可百,數之可千,推之可萬,此之謂也。



\section{病傳第四十二}

黃帝曰:余受九針於夫子,而私覽於諸方,或有導引行氣喬摩炙熨刺炳飲藥之一者,可獨守耶,將盡行之乎?岐伯曰:諸方者,眾人之方也,非一人之所盡行也。黃帝曰:此乃所謂守一勿失,萬物畢者也。今余已聞陰陽之要,虛實之理,頃移之過,可治之屬,願聞病之變化,淫傳絕敗而不可治者,可得聞乎?岐伯曰:要乎哉問道,昭乎其如日醒,窘乎其如夜瞑,能被而服之,神與俱成,畢將服之,神自得之,生神之理,可著於竹帛,不可傳於子孫。黃帝曰:何謂日醒。岐伯曰:明於陰陽,如惑之解,如醉之醒。黃帝曰:何謂夜瞑。岐伯曰:乎其無聲,漠乎其無形,折毛髮理,正氣橫頃,淫恤衍,血脈傳溜,大氣入藏,腹痛下淫,可以致死,不可以致生。

黃帝曰:大氣入藏,奈何?岐伯曰:病先發於心,一日而之肺,三日而之肝,五日而之脾,三日不已,死,冬夜半,夏日中。病先發於肺,三日而之肝,一日而之脾,五日而之胃,十日不已,死,冬日入,夏日出。病先發於肝,三日而之脾,五日而之胃,三日而之腎,三日不已,死,冬日入,夏蚤食。病先發於脾,一日而之胃,二日而之腎,三日而之膂膀胱,十日不已,死,冬人定,夏晏食。病先發於胃,五日而之腎,三日而之膂膀胱,五日而上之心,二日不已,死,冬夜半,夏日。病先發於腎,三日而之膂膀胱,三日而上之心,三日而之小腸,三日不已,死,冬大晨,夏早晡。病先發於膀胱,五日而之腎,一日而之小腸,一日而之心,二日不已,死,冬雞鳴,夏下晡。諸病以次相傳,如是者,皆有死期,不可刺也,間一藏及二三四藏者,乃可刺也。



\section{淫邪發夢第四十三}

黃帝曰:願聞淫恤衍,奈何?岐伯曰:正邪從外襲內,而未有定舍,反淫於藏,不得定處,與滎衛俱行,而與魂魄飛揚,使人臥不得安而喜夢。氣淫於府,則有餘於外,不足於內,氣淫於藏,則有餘於內,不足於外。

黃帝曰:有餘不足有形乎?岐伯曰:陰氣盛,則夢涉大水而恐懼。陽氣盛,則夢大火而燔。陰陽俱盛,則夢相殺。上盛則夢飛,下盛則夢墮,甚則夢取,甚飽則夢予。肝氣盛,則夢怒。肺氣盛,則夢恐懼,哭泣,飛揚。心氣盛,則夢善笑,恐畏。脾氣盛,則夢歌樂,身體重不舉。腎氣盛,則夢腰脊兩解不屬。凡此十二盛者,至而瀉之,立已。

厥氣客於心,則夢見邱山煙火。客於肺,則夢飛揚。見金鐵之奇物。客於肝,則夢山林樹木。客於脾,則夢見邱陵大澤,壞屋風雨。客於腎,則夢臨淵,沒居水中。客於膀胱,則夢遊行。客於胃,則夢飲食。客於大腸,則夢田野。客於小腸,則夢聚邑沖衢。客於膽,則夢訟自刳。客於陰器,則夢接內。客於項,則夢斬首。客於脛,則夢行走而不能前,及居深地苑中。客於股肱,則夢禮節拜起。客於胞植,則夢便。凡此十五不足者,至而補之立已也。
順氣一日分為四時第四十四

黃帝曰:夫百病之所始生者,必起於燥濕寒暑風雨陰陽喜怒飲食居然,氣合而有形,得藏而有名,余知其然也。夫百病者,多以旦慧晝安,夕加夜甚,何也?岐伯曰:四時之氣使然。黃帝曰:願聞四時之氣。岐伯曰:春生夏長,秋收冬藏,是氣之常也,人亦應之。以一日分為四時,朝則為春,日中為夏,日入為秋,夜半為冬,朝則人氣始生,病氣衰,故旦慧。日中人氣長,長則勝邪,故安。夕則人氣始衰,邪氣始生,故加。夜半人氣入藏,邪氣獨居於身,故甚也。黃帝曰:其時有反者何也?岐伯曰:是不應四時之氣,藏獨主其病者,是必以藏氣之所不勝時者甚,以其所勝時者起也。黃帝曰:治之奈何?岐伯曰:順天之時,而病可與期,順者為工,逆者為。

黃帝曰:善。余聞刺有五變,以主五輸,願聞其數。岐伯曰:人有五藏,五藏有五變,五變有五輸,故五五二十五輸,以應五時。黃帝曰:願聞五變。岐伯曰:肝為牡藏,其色青,其時春,其音角,其味酸,其日甲乙。心為牡藏,其色赤,其時夏,其日丙丁,其音徵,其味苦。脾為牝藏,其色黃,其時長夏,其日戊己,其音宮,其味甘。肺為牝藏,其色白,其音商,其時秋,其日庚辛,其味辛。腎為牝藏,其色黑,其時冬,其日壬癸,其音羽,其味咸,是為五變。

黃帝曰:以主五輸奈何?藏主冬,冬刺井。色主春,春刺滎。時主夏,夏刺輸。音主長夏,長夏刺經。味主秋,秋刺合,是謂五變,以主五輸。黃帝曰:諸原安合,以致六輸。岐伯曰:原獨不應五時,以經合之,以應其數,故六六三十六輸。黃帝曰:何謂藏主冬,時主夏,音主長夏,味主秋,色主春,願聞其故。岐伯曰:病在藏者,取之井。病變於色者,取之滎。病時間時甚者,取之輸。病變於音者,取之經。經滿而血者,病在胃,乃以飲食不節得病者,取之於合,故命曰味主合。是謂五變也。



\section{外揣第四十五}

黃帝曰:余聞九針九篇,余親授其調,頗得其意。夫九針者,始於一而終於九,然未得其要道也。夫九針者,小之則無內,大之則無外,深不可為下,高不可為蓋,恍惚無窮,流溢無極,余知其合於天道人事四時之變也,然余願雜之毫毛,渾為一,可乎?岐伯曰:明乎哉問也,非獨針道焉,夫治國亦然。黃帝曰:余願聞針道,非國事也。岐伯曰:夫治國者,夫惟道焉,非道,何可小大深淺,雜合而為一乎?

黃帝曰:願卒聞之。岐伯曰:日與月焉,水與鏡焉,鼓與響焉。夫日月之明,不失其影,水鏡之察,不失其形,鼓響之應,不後其聲,動搖則應和,盡得其情。黃帝曰:窘乎哉,昭昭之明不可蔽,其不可蔽,不失陰陽也。合而察之,切而驗之,見而得之,若清水明鏡之不失其形也。五音不彰,五色不明,五藏波蕩,若是則內外相襲,若鼓之應桴,響之應聲,影之應形。故遠者司外揣內,近者,司內揣外,是謂陰陽之極,天地之蓋,請藏之靈蘭之室,弗敢使洩也。


\section{五變第四十六}

黃帝問於少俞曰:余聞百疾之始期也,必生於風雨寒暑,循毫毛而入腠理,或復還,或留止,或為風腫汗出,或為消癉,或為寒熱,或為留痹,或為積聚,奇邪淫溢,不可勝數,願聞其故。夫同時得病,或病此,或病彼,意者天之為人生風乎,何其異也。少俞曰:夫天之生風者,非以私百姓也,其行公平正直,犯者得之,避者得無殆,非求人而人自犯之。黃帝曰:一時遇風,同時得病,其病各異,願聞其故。少俞曰:善乎哉問,請論以比匠人,匠人磨斧斤,礪刀削,砍材,木之陰陽,尚有堅脆,堅者不入,脆者皮施,至其交節,而缺斤斧焉。夫一木之中,堅脆不同,堅者則剛,脆者易傷,況其材木之不同,皮之厚薄,汁之多少,而各異耶。夫木之蚤花先生葉者,遇春霜烈風,則花落而葉萎。久曝大旱,則脆木薄皮者,枝條汁少而葉萎。久陰淫雨,則薄皮多汁者,皮潰而漉。卒風暴起,則剛脆之木,枝折杌傷。秋霜疾風,則剛脆之木,根搖而葉落。凡此五者,各有所傷,況於人乎?黃帝曰:以人應木,奈何?少俞答曰:木之所傷也,皆傷其枝,枝之剛脆而堅,未成傷也。人之有常病也,亦因其骨節皮膚腠理之不堅固者,邪之所舍也,故常為病也。

黃帝曰:人之善病風厥漉汗者,何以候之,少俞答曰:肉不堅,腠理疏,則善病風。黃帝曰:何以候肉之不堅也。少俞答曰:肉不堅,而無分理,理者理,理而皮不致者,腠理疏,此言其渾然者。
黃帝曰:人之善病消癉者,何以候之。少俞答曰:五藏皆柔弱者,善病消癉。黃帝曰:何以知五藏之柔弱也。少俞答曰:夫柔弱者,必有剛強,剛強多怒,柔者易傷也。黃帝曰:何以候柔弱之與剛強。少俞答曰:此人皮膚薄而目堅固以深者,長沖直揚,其心剛,剛則多怒,怒則氣上逆,胸中畜積,血氣逆留,皮充肌,血脈不行,轉而為熱,熱則消肌膚,故為消癉,此言其人暴剛而肌肉弱者也。
黃帝曰:人之善病寒熱者,何以候之。少俞答曰:小骨弱肉者,善病寒熱。黃帝曰:何以候骨之小大,肉之堅脆,色之不一也。少俞答曰:顴骨者,骨之本也,顴大則骨大,顴小則骨小,皮膚薄而其肉無,其臂懦懦然,其地色殆然,不與其天同色,汗然獨異,此其候也。然後臂薄者,其髓不滿,故善病寒熱也。
黃帝曰:何以候人之善病痹者,少俞答曰:理而肉不堅者,善病痹。黃帝曰:痹之高下有處乎?少俞答曰:欲知其高下者,各視其部。
黃帝曰:人之善病腸中積聚者,何以候之。少俞答曰:皮膚薄而不澤,肉不堅而淖澤,如此,腸胃惡,惡則邪氣留止,積聚乃傷,脾胃之間,寒溫不次,邪氣稍至,畜積留止,大聚乃起。

黃帝曰:余聞病形,已知之矣,願聞其時。少俞答曰:先立其年,以知其時,時高則起,時下則殆,雖不陷下,當年有沖通,其病必起,是謂因形而生病,五變之紀也。



\section{本藏第四十七}

黃帝問於岐伯曰:人之血氣精神者,所以奉生而周於性命者也。經脈者,所以行血氣而滎陰陽,濡筋骨,利關節者也。衛氣者,所以溫分肉,充皮膚,肥腠理,司關闔者也。志意者,所以御精神,收魂魄,適寒溫,和喜怒者也。是故血和則經脈流行,滎覆陰陽,筋骨勁強,關節清利矣。衛氣和則分肉解利,皮膚調柔,腠理緻密矣。志意和則精神專直,魂魄不散,悔怒不起,五藏不受邪矣。寒溫和則六府化谷,風痹不作,經脈通利,支節得安矣。此人之常平也。五藏者,所以藏精神血氣魂魄者也。六府者,所以化水谷而行津液者也。此人之所以具受於天也,無智愚賢不肖,無以相倚也。然有其獨盡天壽,而無邪僻之病,百年不衰,雖犯風雨卒寒大暑,猶有弗能害也。有其不離屏蔽室內,無怵之恐,然猶不免於病,何也,願聞其故。岐伯曰:窘乎哉問也。五藏者,所以參天地,副陰陽,而運四時,化五節者也。五藏者,固有小大高下堅脆端正偏頃者,六府亦有小大長短厚薄結直緩急,凡此二十五者,各不同,或善或惡,或吉或凶,請言其方。

心小則安,邪弗能傷,易傷以憂。心大則憂不能傷,易傷於邪。心高則滿於肺中,而善忘,難開以言。心下則藏外,易傷於寒,易恐以言。心堅則藏安守固,心脆則善病消癉熱中,心端正則和利難傷,心偏頃則操持不一。無守司也。
肺小則少飲,不病喘喝。肺大則多飲,善病胸痹喉痹逆氣。肺高則上氣喘息。肺下則居賁迫肺,善脅下痛。肺堅則不病上氣。肺脆則苦病消癉易傷。肺端正則和利難傷。肺偏頃則胸偏痛也。
肝小則藏安,無脅下之痛。肝大則逼胃,迫咽則苦膈中,且脅下痛。肝高則上支賁切脅,為息賁。肝下則逼胃,脅下空,脅下空則易受邪。肝堅則藏安難傷。肝脆則善病消癉易傷。肝端正則和利難傷。肝偏頃則脅下痛也。
脾小則藏安,難傷於邪也。脾大則苦湊而痛,不能疾行。脾高則引季脅而痛。脾下則下加於大腸,下加於大腸則藏苦受邪,脾堅則藏安難傷。脾脆則善病消癉易傷。脾端正則和利難傷。脾偏傾則善滿善脹也。
腎小則藏安難傷。腎大則善病腰痛,不可以仰,易傷以邪。腎高則苦背膂痛,不可以仰。腎下則腰尻痛,不可以仰,為狐疝。腎堅則不病腰背痛。腎脆則苦病消癉易傷。腎端正則和利難傷。腎偏頃則苦腰尻痛也。凡此二十五變者,人之所苦常病也。

黃帝曰:何以知其然也。岐伯曰:赤色小理者,心小。理者,心大。無者,心高。小短舉者,心下,長者,心下堅。弱小以薄者,心脆。直下不舉者,心端正。倚一方者,心偏頃也。白色小理者,肺小。理者,肺大。巨肩反膺陷喉者,肺高。合腋張脅者,肺下。好肩背厚者,肺堅。肩背薄者,肺脆。背膺厚者。肺端正。脅偏疏者,肺偏頃也。青色小理者,肝小。理者,肝大。廣胸反者,肝高。合脅兔者,肝下。胸脅好者,肝堅。脅骨弱者,肝脆。膺腹好相得者,肝端正。脅骨偏舉者,肝偏頃也。黃色小理者,脾小。理者,脾大。揭唇者,脾高。唇下縱者,脾下。唇堅者,脾堅。唇大而不堅者,脾脆。唇上下好者,脾端正。唇偏舉者,脾偏頃也。黑色小理者,腎小。理者,腎大。高耳者,腎高。耳後陷者,腎下。耳堅者,腎堅。耳薄不堅者,腎脆。耳好前居牙車者,腎端正。耳偏高者,腎偏頃也。凡此諸變者,持則安,減則病也。
黃帝曰:善。然非余之所問也。願聞人之有不可病者,至盡天壽,雖有深憂大恐,怵惕之志,猶不能減也。甚寒大熱,不能傷也。其有不離屏蔽室內,又無怵惕之恐,然不免於病者,何也,願聞其故。岐伯曰:五藏六府,邪之舍也,請言其故。五藏皆小者,少病,苦焦心,大愁憂。五藏皆大者,緩於事,難使以憂。五藏皆高者,好高舉措。五藏皆下者,好出人下。五藏皆堅者,無病。五藏皆脆者,不離於病。五藏皆端正者,和利得人心。五藏皆偏頃者,邪心而善盜,不可以為人,平反覆言語也。

黃帝曰:願聞六府之應。岐伯答曰:肺合大腸,大腸者,皮其應。心合小腸,小腸者,脈其應。肝合膽,膽者,筋其應。脾合胃,胃者,肉其應。腎合三焦膀胱,三焦膀胱者,腠理毫毛其應。
黃帝曰:應之奈何?岐伯曰:肺應皮,皮厚者,大腸厚,皮薄者,大腸薄,皮緩腹裡大者,大腸大而長,皮急者,大腸急而短,皮滑者,大腸直,皮肉不相離者,大腸結。心應脈,皮厚者,脈厚,脈厚者,小腸厚。皮薄者,脈薄。脈薄者,小腸薄。皮緩者,脈緩,脈緩者,小腸大而長。皮薄而脈衝小者,小腸小而短。諸陽經脈皆多紆屈者,小腸結。脾應肉,肉堅大者,胃厚,肉麼者,胃薄,肉小而麼者胃不堅,肉不稱身者,胃下,胃下者,下脘約不利。肉不堅者,胃緩。肉無小裹累者,胃急。肉多少裹累者,胃結。胃結者,上脘約不利也。肝應爪,爪厚色黃者,膽厚。爪薄色紅者,膽薄。爪堅色青者,膽急。爪濡色赤者,膽緩。爪直色白無約者,膽直。爪惡色黑多絞者,膽結也。腎應骨,密理厚皮者,三焦膀胱厚。理薄皮者,三焦膀胱薄。疏腠理者,三焦膀胱緩。皮急而無毫毛者,三焦膀胱急。毫毛美而者,三焦膀胱直。稀毫毛者,三焦肪胱結也。
黃帝曰:厚薄美惡皆有形,願聞其所病。岐伯答曰:視其外應,以知其內藏,則知所病矣。

\section{禁服第四十八}

雷公問於黃帝曰:細子得受業,通於九針六十篇,旦暮勤服之,近者編絕,久者簡垢,然尚諷誦弗置,未盡解於意矣,外揣言渾為一,未知所謂也。夫大則無外,小則無內,大小無極,高下無度,之奈何?士之才力,或有厚薄,智慮褊淺,不能博大深奧,自強於學若細子,細子恐其散於後世,絕於子孫,敢問約之奈何?黃帝曰:善乎哉問也。此先師之所禁,坐私傳之也,割臂歃血之盟也,子若欲得之,何不齋乎,雷公再拜而起曰:請聞命於是也。乃齋宿三日而請曰:敢問今日正陽,細子願以受盟。黃帝乃與俱入齋室,割臂歃血。黃帝親祝曰:今日正陽,歃血傳方,敢有背此言者,反受其殃。雷公再拜曰:細子受之。黃帝乃左握其手,右受之書,曰:慎之慎之,吾為子言之。

凡刺之理,經脈為始,滎其所行,知其度量,內刺五藏,外刺六府,審察衛氣,為百病母,調其虛實,虛實乃止,瀉其血絡,血盡不殆矣。雷公曰:此皆細子之所以通,未知其所約也。

黃帝曰:夫約方者,猶約囊也,囊滿而弗約,則輸洩,方成弗約,則神與弗俱。雷公曰:願為下材者,弗滿而約之。黃帝曰:未滿而知約之以為工,不可以為天下師。雷公曰:願聞為工。

黃帝曰:寸口主中,人迎主外,兩者相應,俱往俱來,若引繩大小齊等,春夏人迎微大,秋冬寸口微大,如是者名曰平人。
人迎大一倍於寸口,病在足少陽,一倍而躁,在手少陽。人迎二倍,病在足太陽,二倍而躁,病在手太陽。人迎三倍,病在足陽明,三倍而躁,病在手陽明。盛則為熱,虛則為寒,緊則為痛痹,代則乍甚乍問。盛則瀉之,虛則補之,緊痛則取之分肉,代則取血絡,具飲藥,陷下則灸之,不盛不虛,以經取之,名曰經刺。人迎四倍者,且大且數,名曰溢陽,溢陽為外格,死不治。必審按其本末,察其寒熱,以驗其藏府之病。
寸口大於人迎一倍,病在足厥陰,一倍而躁,病在手心主。寸口二倍,病在足少陰,二倍而躁。在手少陰。寸口三倍,病在足太陰,三倍而躁,病在手太陰。盛則脹滿,寒中食不化,虛則熱中,出糜少氣,溺色變,緊則痛痹,代則乍痛乍止。盛則瀉之,虛則補之,緊則先刺而後灸之,代則取血絡,而後調之,陷下則徒灸之,陷下者,脈血絡於中,中有著血,血寒,故宜灸之,不盛不虛,以經取之。名曰經刺,寸口四倍者名曰內關,內關者,且大且數,死不治。必審察其本末之寒溫,以驗其藏府之病。

通其滎輸,乃可傳於大數。大數曰:盛則徒瀉之。虛則徒補之,緊則灸刺,且飲藥,陷下則徒灸之,不盛不虛,以經取之。所謂經治者,飲藥,亦曰灸刺。脈急則引,脈大以弱,則欲安靜,用力無勞也。



\section{五色第四十九}

雷公問於黃帝曰:五色獨決於明堂乎,小子未知其所謂也。黃帝曰:明堂者,鼻也。闕者,眉間也。庭者,顏也。蕃者頰側也。蔽者,耳門也。其間欲方大,去之十步,皆見於外,如是者壽,必中百歲。雷公曰:五官之辯,奈何?黃帝曰:明堂骨高以起平以直,五藏次於中央,六府挾其兩側,首面上於闕庭,王宮在於下極,五藏安於胸中,真色以致,病色不見,明堂潤澤以清,五官惡得無辯乎?雷公曰:其不辯者,可得聞乎?黃帝曰:五色之見也,各出其色部,部骨陷者,必不免於病矣。其色部乘襲者,雖病者甚,不死矣。雷公曰:官五色奈何?黃帝曰:青黑為痛,黃赤為熱,白為寒,是謂五官。
雷公曰:病之益甚與其方衰,如何。黃帝曰:外內皆在焉,切其脈口滑小緊以沉者,病益甚,在中。人迎氣大緊以浮者,其病益甚,在外。其脈口浮滑者,病日進。人迎沉而滑者,病日損。其脈口滑以沉者,病日進,在內。其人迎脈滑盛以浮者,其病日進,在外。脈之浮沉及人迎與寸口氣小大等者,病難已。病之在藏,沉而大者,易已,小為逆。病在府,浮而大者,其病易已。人迎盛堅者,傷於寒,氣口盛堅者,傷於食。
雷公曰:以色言病之間甚,奈何?黃帝曰:其色以明,沉夭者為甚,其色上行者,病益甚,其色下行,如雲徹散者,病方已。五色各有藏部,有外部,有內部也。色從外部走內部者,其病從外走內。其色從內走外者,其病從內走外。病生於內者,先治其陰,後治其陽,反者益甚。其病生於陽者,先治其外,後治其內,反者益甚。其脈滑大以代而長者,病從外來,目有所見,志有所惡,此陽氣之並也,可變而已。第四節雷公曰:小子聞風者,百病之始也,厥逆者,寒濕之起也,別之奈何?黃帝曰:常候闕中,薄澤為風,沖濁為痹,在地為厥,此其常也,各以其色言其病。雷公曰:人不病卒死,何以知之。黃帝曰:大氣入於藏府者,不病而卒死矣。雷公曰:病小愈而卒死者,何以知之。黃帝曰:赤色出兩顴,大如母指者,病雖小愈,必卒死。黑色出於庭,大如母指,必不病而卒死。雷公再拜曰:善哉,其死有期乎?黃帝曰:察色以言其時。

雷公曰:善乎,願卒聞之。黃帝曰:庭者,首面也。闕上者,咽喉也。闕中者,肺也。下極者,心也。直下者,肝也。肝左者,膽也。下者,脾也。方上者,胃也。中央者,大腸也。挾大腸者,腎也。當腎者,齊也。面王以上者,小腸也。面王以下者,膀胱子處也。顴者,肩也。顴後者,腎也。臂下者,手也。目內上者,膺乳也。挾繩而上者,背也。循牙車以下者,股也。中央者,膝也。膝以下者,脛也。當脛以下者,足也。巨分者,股裡也。巨屈者,膝臏也。此五藏六府支節之部也。
各有部分,有部分,用陰和陽,用陽和陰,當明部分,萬舉萬當,能別左右,是謂大道,男女異位,故曰陰陽,審察澤夭,謂之良工。沉濁為內,浮澤為外,黃赤為風,青黑為痛,白為寒,黃而膏潤為膿,赤甚者為血痛,甚為攣,寒甚為皮不仁。五色各見其部,察其浮沉,以知淺深,察其澤夭,以觀成敗,察其散搏,以知遠近,視色上下,以知病處,積神於心,以知往今。故相氣不微,不知是非,屬意勿去,乃知新故。色明不,沉夭為甚,不明不澤,其病不甚。其色散,駒駒然未有聚,其病散而氣痛,聚未成也。腎乘心,心先病,腎為應,色皆如是。男子色在於面王,為小腹痛,下為卵痛,其圜直為莖痛,高為本,下為首,狐疝陰之屬也。女子在於面王,為膀胱子處之病,散為痛,搏為聚,方員左右,各如其色形。其隨而下至胝,為淫,有潤如膏狀,為暴食不。左為左,右為右,其色有邪,聚散而不端,面色所指者也。色者,青黑赤白黃,皆端滿有別鄉。別鄉赤者,其色赤,大如榆莢,在面王為不日。其色上銳,首空上向,下銳下向,在左右如法,以五色命藏,青為肝,赤為心,白為肺,黃為脾,黑為腎。肝合筋,心合脈,肺合皮,脾合肉,腎合骨也。

\section{論勇第五十}

黃帝問於少俞曰:有人於此,並行並立,其年之長少等也,衣之厚薄均也,卒然遇烈風暴雨,或病或不病,或皆病,或皆不病,其故何也?少俞曰:帝問何急?黃帝曰:願盡聞之。少俞曰:春青風,夏陽風,秋涼風,各寒風。凡此四時之風者,其所病各不同形。黃帝曰:四時之風,病人如何?少俞曰:黃色薄皮弱肉者,不勝春之虛風;白色薄皮弱肉者,不勝夏之虛風;青色薄皮弱肉,不勝秋之虛風;赤色薄皮弱肉,不勝冬之虛風也。黃帝曰:黑色不病乎?少俞曰:黑色而皮厚肉堅,固不傷於四時之風。其皮薄而肉不堅,色不一者,長夏至而有虛風者,病矣。其皮厚而肌肉堅者,長夏至而有虛風,不病矣。其皮厚而肌肉堅者,必重感於寒,外內皆然,乃病。黃帝曰:善。

黃帝曰:夫人之忍痛與不忍痛者,非勇怯之分也。夫勇士之不忍痛者,見難則前,見痛則止;夫怯士之忍痛者,聞難則恐,遇痛不動。夫勇士之忍痛者,見難不恐,遇痛不動;夫怯士之不忍痛者,見難與痛,目轉面盼,恐不能言,失氣驚,顏色變化,乍死乍生。余見其然也,不知其何由,願聞其故。少俞曰:夫忍痛與不忍痛者,皮膚之薄厚,肌肉之堅脆緩急之分也,非勇怯之謂也。黃帝曰:願聞勇怯之所由然。少俞曰:勇士考,目深以固,長衡直暢,三焦理橫,其心端直,其肝大以堅,其膽滿以傍,怒則氣盛而胸張,肝舉而膽橫,毗裂而目揚,毛起而面蒼,此勇士之由然者也。黃帝曰:願聞怯士之所由然。少俞曰:怯士者,目大而不減,陰陽相失,其焦理縱,(骨曷)(骨亏)短而小,肝系緩,其膽不滿而縱,腸胃挺,脅下空,雖方大怒,氣不能滿其胸,肝肺雖舉,氣衰復下,故不能久怒,此怯士之所由然者也。黃帝曰;怯士之得酒,怒不避勇士者,何藏使然?少俞曰:酒者,水谷之精,熟谷之液也,其氣慓悍,其入於胃中,則胃脹,氣上逆,滿於胸中,肝浮膽橫。當是之時,固比於勇士,氣衰則悔。與勇士同類,不知避之,名曰酒悖也。



\section{背俞第五十一}

黃帝問於岐伯曰:願聞五藏之俞,出於背者。岐伯曰:胸中大俞,在杼骨之端,肺俞在三焦之間,心俞在五焦之間,膈俞在七焦之間,肝俞在九焦之間,脾俞在十一焦之間,腎俞在十四焦之間,皆挾脊相去三寸所,則欲得而驗之,按其處,應在中而痛解,乃其俞也。灸之則可,刺之則不可。氣盛則瀉之,虛則補之。以火補者,無吹其火,須自滅也。以火瀉者,疾吹其火,傳其艾,須其火滅也。



\section{衛氣第五十二}

黃帝曰:五藏者,所以藏精神魂魄者也。六府者,所以受水谷而化行物者也。其氣內干五藏,而外絡支節。其浮氣之不循經者,為衛氣。其精氣之行於經者,為滎氣。陰陽相隨,外內相貫,如環之無端,亭亭淳淳乎,孰能窮之。然其分別陰陽,皆有標本虛實所離之處。能別陰陽十二經者,知病之所生。候虛實之所在者,能得病之高下。知六府之氣街者,能知解結契紹於門戶。能知虛實堅軟者,知補瀉之所在。能知六經標本者,可以無惑於天下。

岐伯曰:博哉,聖帝之論,臣請盡意悉言之。足太陽之本,在跟以上五寸中,標在兩絡命門,命門者,目也。足少陽之本,在竅陰之間,標在窗籠之前,窗籠者,耳也。足少陰之本,在內踝下上三寸中,標在背俞與舌下兩脈也。足厥陰之本,在行間上五寸所,標在背俞也。足陽明之本,在厲兌,標在人迎頰挾頏顙也。足太陰之本,在中封前上四寸之中,標在背俞與舌本也。手太陽之本,在外踝之後,標在命門之上一寸也。手少陽之本,在小指次指之間上二寸,標在耳後上角下外也。手陽明之本,在肘骨中,上至別陽,標在顏下合鉗上也。手太陰之本,在寸口之中,標在腋內動也。手少陰之本,在銳骨之端,標在背俞也。手心主之本,在掌後兩筋之間二寸中,標在腋下下三寸也。凡候此者,下虛則厥,下盛則熱,上虛則眩,上盛則熱痛。故實者,絕而止之,虛者,引而起之。

請言氣街,胸氣有街,腹氣有街,頭氣有街,脛氣有街,故氣在頭者,止之於腦。氣在胸者,止之膺與背俞。氣在腹者,止之背俞,與衝脈於齊左右之動脈者。氣在脛者,止之於氣街,與承山踝上以下。取此者,用毫針,必先按而在,久應於手,乃剌而予之。所治者,頭痛眩僕,腹痛中滿暴脹,及有新積。痛可移者,易已也,積不痛,難已也。



\section{論痛第五十三}

黃帝問於少俞曰:筋骨之強弱,肌肉之堅脆,皮膚之厚薄,腠理之疏密,各不同,其於針石火之痛如何。腸胃之厚薄堅脆亦不等,其於毒藥何如?願盡聞之。少俞曰:人之骨強筋弱肉緩皮膚厚者,耐痛,其於針石之痛火亦然。黃帝曰:其耐火者,何以知之。少俞答曰:加以黑色而美骨者,耐火。黃帝曰:其不耐針石之痛者,何以知之。少俞曰:堅肉薄皮者,不耐針石之痛,於火亦然。
黃帝曰:人之病,或同時而傷,或易已,或難已,其故何如?少俞曰:同時而傷,其身多熱者,易已。多寒者,難已。
黃帝曰:人之勝毒,何以知之。少俞曰:胃厚色黑大骨及肥者,皆勝毒,故其瘦而薄胃者,皆不勝毒也。



\section{天年第五十四}

黃帝問於岐伯曰:願聞人之始生,何氣為基,何立而為,何失而死,何得而生。岐伯曰:以母為基,以父為,失神者死,得神者生也。黃帝曰:何者為神。岐伯曰:血氣已和,滎衛已通,五藏已成,神氣舍心,魂魄畢具,乃成為人。

黃帝曰:人之壽夭各不同,或夭壽,或卒死,或病久,願聞其道。岐伯曰:五藏堅固,血脈和調,肌肉解利,皮膚緻密,滎衛之行,不失其常,呼吸微徐,氣度以行,六府化谷,津液布揚,各如其常,故能長久。黃帝曰:人之壽百歲而死,何以致之。岐伯曰:使道隧以長,基高以方,通調滎衛,三部三里起,骨高肉滿,百歲乃得終。

黃帝曰:其氣之盛衰,以至其死,可得聞乎?岐伯曰:人生十歲,五藏始定,血氣已通,其氣在下,故好走。二十歲,血氣始盛,肌肉方長,故好趨。三十歲,五藏大定,肌肉堅固,血脈盛滿,故好步。四十歲,五藏六府十二經脈,皆大盛以平定,腠理始疏,滎華頹落,發頗頒白,平盛不搖,故好坐。五十歲,肝氣始衰,肝葉始薄,膽汁始滅,目始不明。六十歲,心氣始衰,善憂悲,血氣懈惰,故好臥。七十歲,脾氣虛,皮膚枯。八十歲,肺氣衰,魄離,故言善誤。九十歲,腎氣焦,四藏經脈空虛。百歲,五藏皆虛,神氣皆去,形骸獨居而終矣。

黃帝曰:其不能終壽而死者,何如?岐伯曰:其五藏皆不堅,使道不長,空外以張,喘息暴疾,又卑基,薄脈少血,其肉不石,數中風寒,血氣虛,脈不通,真邪相攻,亂而相引,故中壽而盡也。



\section{逆順第五十五}

黃帝問於伯高曰:余聞氣有逆順,脈有盛衰,刺有大約,可得於聞乎?伯高曰:氣之逆順者,所以應天地陰陽四時五行也。脈之盛衰者,所以候血氣之虛實有餘不足也。刺之大約者,必明知病之可刺,與其未可刺,與其已不可剌也。

黃帝曰:候之奈何?伯高曰:兵法曰:無迎逢逢之氣,無擊堂堂之陣。刺法曰:無刺之熱,無刺漉漉之汗,無刺渾渾之脈,無刺病與脈相逆者。黃帝曰:候其可刺奈何?伯高曰:上工,刺其未病者也。其次,刺其未盛者也。其次,刺其已衰者也。下工,刺其方襲者也,與其形之盛者也,與其病之與脈相逆者也。故曰,方其盛也,勿敢毀傷,刺其已衰,事必大昌。故曰,上工治未病,不治已病,此之謂也。



\section{五味第五十六}

黃帝曰:願聞谷氣有五味,其入五藏,分別奈何?伯高曰:胃者,五藏六府之海也,水谷皆入於胃,五藏六府,皆稟氣於胃。五味各走其所喜,谷味酸,先走肝,谷味苦,先走心,谷味甘,先走脾,谷味辛,先走肺,谷味咸,先走腎,谷氣津液已行,滎衛大通,乃化糟粕,以次傳下。

黃帝曰:滎衛之行奈何?伯高曰:谷始入於胃,其精微者,先出於胃之兩焦,以溉五藏,別出兩行,滎衛之道。其大氣之搏而不行者,積於胸中,命曰氣海,出於肺,循喉咽,故呼則出,吸則入。天地之精氣,其大數常出三入一,故谷不入半日則氣衰,一日則氣少矣。

黃帝曰:谷之五味,可得聞乎?伯高曰:請盡言之。五穀,米甘,麻酸,大豆咸,麥苦,黃黍辛。五果,棗甘,李酸,栗咸,杏苦桃辛。五畜,牛甘,犬酸,豬咸,羊苦,雞辛。五菜,葵甘,韭酸,藿咸,薤苦,蔥辛。五色,黃色宜甘,青色宜酸,黑色宜咸,赤色宜苦,白色宜辛。凡此五者,各有所宜。

所謂五色者,脾病者,宜食米飯牛肉棗葵。心病者,宜食麥羊肉杏薤。腎病者。宜食大豆黃卷豬肉栗藿。肝病者,宜食麻犬肉李韭。肺病者,宜食黃黍雞肉桃蔥。五禁,肝病禁辛,心病禁咸,脾病禁酸,腎病禁甘,肺病禁苦。肝色青,宜食甘,米飯牛肉棗葵皆甘。心色赤,宜食酸,犬肉麻李韭皆酸。脾色黃,宜食咸,大豆豕肉栗藿皆咸。肺色白,宜食苦,麥羊肉杏薤皆苦。腎色黑,宜食辛,黃黍雞肉桃蔥皆辛。



\section{水脹第五十七}

黃帝問於岐伯曰:水與膚脹,鼓脹,腸覃,石瘕,石水,何以別之。岐伯答曰:水始起也,目窠上微腫,如新臥起之狀,其頸脈動,時,陰股間寒,足脛腫,腹乃大,其水已成矣。以手按其腹,隨手而起,如裹水之狀,此其候也。
黃帝曰:膚脹何以候之。岐伯曰:膚脹者,寒氣客於皮膚之間,鼓鼓然不堅,腹大,身盡腫,皮厚,按其腹。而不起,腹色不變,此其候也。
鼓脹何如?岐伯曰:腹脹身皆大,大與膚脹等也,色蒼黃,腹筋起,此其候也。
腸覃何如?岐伯曰:寒氣客於腸外,與衛氣相搏,氣不得滎,因有所繫,癖而內著,惡氣乃起,肉乃生。其始生也,大如雞卵,稍以益大,至其成如懷子之狀,久者離藏,按之則堅,推之則移,月事以時下,此其候也。
石瘕何如?岐伯曰:石瘕生於胞中,寒氣客於子門,子門閉塞,氣不得通,惡血當瀉不瀉,杯以留止,日以益大,狀如懷子,月事不以時下,皆生於女子,可導而下。
黃帝曰:膚脹鼓脹,可刺耶。岐伯曰:先瀉其脹之血絡,後調其經,刺去其血絡也。

\section{賊風第五十八}

黃帝曰:夫子言賊風邪氣之傷人也,令人病焉,今有其不離屏蔽,不出室穴之中,卒然病者,非不離賊風邪氣,其故何也?岐伯曰:此皆嘗有所傷於濕氣,藏於血脈之中,分肉之間,久留而不去,若有所墮墜,惡血在內而不去。卒然喜怒不節,飲食不適,寒溫不時,腠理閉而不通,其開而遇風寒,則血氣凝結,與故邪相襲,則為寒痹。其有熱則汗出,汗出則受風,雖不遇賊風邪氣,必有因加而發焉。

黃帝曰:夫子之所言者,皆病人之所自知也,其無所遇邪氣,又無怵之所志,卒然而病者,其故何也,唯因鬼神之事乎?岐伯曰:此亦有故,邪留而未發,因而志有所惡,及有所慕,血氣內亂,兩氣相搏,其所從來者微,視之不見,聽而不聞,故似鬼神。黃帝曰:其祝而已者,其故何也?岐伯曰:先巫者,因知百病之勝,先知其病之所從生者,可祝而已也。



\section{衛氣失常第五十九}

黃帝曰:衛氣之留於腹中,積不行,苑蘊不得常所,使人支脅胃中滿,喘呼逆息者,何以去之。伯高曰:其氣積於胸中者,上取之。積於腹中者,下取之。上下皆滿者,傍取之。黃帝曰:取之奈何?伯高對曰:積於上,瀉大迎天突喉中。積於下者,瀉三里與氣街。上下皆滿者,上下取之,與季脅之下一寸。重者,雞足取之,診視其脈大而弦急,及絕不至者,及腹皮急甚者,不可刺也。黃帝曰:善。

黃帝問於伯高曰:何以知皮肉氣血筋骨之病也。伯高曰:色起兩眉薄澤者,病在皮。色青黃赤白黑者,病在肌肉。滎氣濡然者,病在血氣。目色青黃赤白黑者,病在筋。耳焦枯,受塵垢,病在骨。黃帝曰:病形何如,取之奈何?伯高曰:夫百病變化,不可勝數,然皮有部,肉有柱,血氣有輸,骨有屬。黃帝曰:願聞其故。伯高曰:皮之部,輸於四末。肉之柱,在臂脛諸陽分肉之間,與足少陰分間。血氣之輸,輸於諸絡,氣血留居則盛而起。筋部無陰無陽,無左無右,候病所在。骨之屬者,骨空之所以受益而益腦髓者也。黃帝曰:取之奈何?伯高曰:夫病變化,浮沉深淺,不可勝窮,各在其處,病間者淺之,甚者深之,間者少之,甚者眾之,隨變而調氣,故曰上工。

黃帝問於伯高曰:人之肥瘦大小寒溫,有老壯少小,別之奈何?伯高對曰:人年五十已上為老,二十已上為壯,十八已上為少,六歲已上為小。黃帝曰:何以度知其肥瘦。伯高曰:人有肥有膏有肉。黃帝曰:別此奈何?伯高曰:肉堅,皮滿者,肥。肉不堅,皮緩者,膏。皮肉不相離者,肉。黃帝曰:身之寒溫何如?伯高曰:膏者,其肉淖而理者,身寒。細理者,身熱脂者,其肉堅,細理者熱,理者寒。黃帝曰:其肥瘦大小奈何?伯高曰:膏者,多氣而皮縱緩,故能縱腹垂腴。肉者,身體容大。脂者,其身收小。黃帝曰:三者之氣血多少何如?伯高曰:膏者,多氣,多氣者,熱,熱者,耐寒。肉者,多血則充形,充形則平。脂者,其血清,氣滑少,故不能大。此別於眾人者也。黃帝曰:眾人奈何?伯高曰:眾人皮肉脂膏,不能相加也,血與氣,不能相多,故其形不小不大,各自稱其身,命曰眾人。黃帝曰:善。治之奈何?伯高曰:必先別其三形,血之多少,氣之清濁,而後調之。治無失常經。是故膏人縱腹垂腴,肉人者,上下容大,脂人者,雖脂不能大也。



\section{玉版第六十}

黃帝曰:余以小針為細物也,夫子乃言上合之於天,下合之於地,中合之於人,余以為過針之意矣,願聞其故。岐伯曰:何物大於天乎,夫大於針者,惟五兵者焉。五兵者,死之備也。非生之具,且夫人者,天地之鎮也,其不可不參乎?夫治民者,亦唯針焉,夫針之與五兵,其孰小乎?

黃帝曰:病之生時,有喜怒不測,飲食不節,陰氣不足,陽氣有餘,滎氣不行,乃發為癰疽。陰陽不通,兩熱相搏,乃化為膿,小針能取之乎?岐伯曰:聖人不能使化者為之,邪不可留也。故兩軍相當,旗幟相望,白刃陳於中野者,此非一日之謀也。能使其民令行,禁止士卒無白刃之難者,非一日之教也,須臾之得也。夫至使身被癰疽之病,膿血之聚者,不亦離道遠乎?夫癰疽之生,膿血之成也,不從天下,不從地出,積微之所生也。故聖人自治於未有形也,愚者遭其已成也。
黃帝曰:其已形,不予遭,膿已成,不予見,為之奈何?岐伯曰:膿已成,十死一生,故聖人勿使已成,而明為良方,著之竹帛,使能者踵而傳之後世,無有終時者,為其不予遭也。黃帝曰:其已有膿血而後遭乎?不道之以小針治乎?岐伯曰:以小治小者,其功小,以大治大者,多害,故其已成膿血者,其唯砭石鈹鋒之所取也。

黃帝曰:多害者其不可全乎?岐伯曰:其在逆順焉。黃帝曰:願聞逆順。岐伯曰:以為傷者,其白眼青,黑眼小,是一逆也。內藥而嘔者,是二逆也。腹痛喝甚,是三逆也。肩項中不便,是四逆也。音嘶色脫,是五逆也。除此五者,為順矣。黃帝曰:諸病皆有逆順,可得聞乎?岐伯曰:腹脹身熱脈大,是一逆也。腹鳴而滿,四支清洩,其脈大。是二逆也。而不止,脈大。是三逆也。且溲血脫形,其脈小勁,是四逆也。脫形,身熱,脈小以疾,是謂五逆也。如是者,不過十五日而死矣。其腹大脹,四末清。形脫,洩甚,是一逆也。腹脹便血,其脈大,時絕,是二逆也。溲血,形肉脫,脈搏,是三逆也。嘔血,胸滿引背,脈小而疾,是四逆也。嘔,腹脹且飧洩,其脈絕,是五逆也。如是者,不過一時而死矣,工不察此者而刺之,是謂逆治。

黃帝曰:夫子之言針甚駿,以配天地,上數天文,下度地紀,內別五藏,外次六府,經脈二十八會,盡有周紀,能殺生人,不能起死者,子能反之乎?岐伯曰:能殺生人,不能起死者也。黃帝曰:余聞之,則為不仁。然願聞其道,弗行於人。岐伯曰:是明道也。其必然也,其如刀劍之可以殺人。如飲酒使人醉也,雖勿胗,猶可知矣。黃帝曰:願卒聞之。岐伯曰:人之所受氣者,谷也。谷之所注者,胃也。胃者,水谷氣血之海也。海之所行雲氣者,天下也。胃之所出氣血者,經隧也。經隧者,五藏六府之大絡也,迎而奪之而已矣。黃帝曰:上下有數乎?岐伯曰:迎之五里,中道而止,五至而已,五往而藏之氣盡矣。故五五二十五,而竭其輸矣。此所謂奪其天氣者也,非能絕其命而頃其壽者也。黃帝曰:願卒聞之。岐伯曰:門而刺之者,死於家中,入門而刺之者,死於堂上。黃帝曰:善乎方,明哉道,請著之玉版,以為重寶,傳之後世,以為刺禁,令民勿敢犯也。



\section{五禁第六十一}

黃帝問於岐伯曰:余聞刺有五禁,何謂五禁。岐伯曰:禁其不可刺也。黃帝曰:余聞刺有五奪。岐伯曰:無瀉其不可奪者也。黃帝曰:余聞刺有五過。岐伯曰:補瀉無過其度。黃帝曰:余聞刺有五逆。岐伯曰:病與脈相逆,命曰五逆。黃帝曰:余聞刺有九宜。岐伯曰:明知九針之論,是謂九宜。
黃帝曰:何謂五禁,願聞其不可刺之時。岐伯曰:甲乙日自乘,無刺頭,無發蒙於耳內。丙丁日自乘,無振埃於肩喉廉泉。戊己日自乘四季,無刺腹去爪瀉水。庚辛日自乘,無刺關節於股膝,壬癸日自乘,無刺足脛,是謂五禁。
黃帝曰:何謂五奪。岐伯曰:形肉已奪,是一奪也。大奪血之後,是二奪也。大汗出之後,是三奪也。大洩之後,是四奪也。新產及大血,是五奪也,此皆不可瀉。
黃帝曰:何謂五逆。岐伯曰:熱病脈靜,汗已出,脈盛躁,是一逆也。病洩脈洪大,是二逆也。著痹不移,肉破,身熱,脈偏絕,是三逆也。淫而奪形,身熱,色夭然白,及後下血杯,血杯篤重,是謂四逆也。寒熱奪形,脈堅搏,是謂五逆也。

\section{動輸第六十二}

黃帝曰:經脈十二,而手太陰足少陰陽明,獨動不休,何也?岐伯曰:是明胃脈也。

胃為五藏六府之海,其清氣上注於肺,肺氣從太陰而行之,其行也,以息往來,故人一呼,脈再動,一吸,脈亦再動,呼吸不已,故動而不止。黃帝曰:氣之過於寸口也,上十焉息,下八焉伏,何道從還,不知其極。岐伯曰:氣之離藏也,卒然如弓弩之發,如水之下岸,上於魚以反衰,其餘氣衰散以上逆,故其行微。

黃帝曰:足之陽明,何因而動。岐伯曰:胃氣上注於肺,其悍氣上衝頭者,循咽上走空竅,循眼系,入絡腦,出,下客主人循牙車,合陽明,並下人迎,此胃氣別走於陽明者也。故陰陽上下,其動也若一。故陽病而陽脈小者為逆,陰病而陰脈大者,為逆,故陰陽俱靜俱動,若引繩相頃者,病。

黃帝曰:足少陰何因而動。岐伯曰:衝脈者,十二經之海也,與少陰之大絡,起於腎下,出於氣街,循陰股內廉,邪入中,循脛股內廉,並少陰之經,下入內踝之後。入足下。其別者,邪入踝,出屬跗上,入大指之間,注諸絡,以溫足脛,此脈之常動者也。

黃帝曰:滎衛之行也,上下相貫,如環之無端。今有其卒然遇邪氣,及逢大寒,手足懈惰,其脈陰陽之道,相輸之會,行相失也。氣何由還。岐伯曰:夫四末陰陽之會者,此氣之大絡也。四街者,氣之徑路也,故絡絕則徑通,四末解則氣從合,相輸如環。黃帝曰:善。此所謂如環無端,莫知其紀,此之謂也。



\section{五味論第六十三}

黃帝問於少俞曰:五味入於口也,各有所走,各有所病。酸走筋,多食之,令人癃。咸走血,多食之,令人喝。辛走氣,多食之,令人洞心。苦走骨,多食之,令人變嘔。甘走肉,多食之,令人心。余知其然也,不知其何由,願聞其故。

少俞答曰:酸入於胃,其氣澀以收,上之兩焦,弗能出入也,不出即留於胃中,胃中和溫,則下注膀胱,膀胱之脆薄以懦,得酸則縮綣,約而不通,水道不行,故癃。陰者,積筋之所終也,故酸入而走筋矣。
黃帝曰:咸走血,多食之,令人喝,何也?少俞曰:咸入於胃,其氣上走中焦,注於脈,則血氣走之,血與咸相得,則凝,凝則胃中汁注之,注之則胃中竭,竭則咽路焦,故舌本干而善喝。血脈者,中焦之道也,故咸入而走血矣。
黃帝曰:辛走氣,多食之,令人洞心,何也?少俞曰:辛入於胃,其氣走於上焦,上焦者,受氣而滎諸陽者也,姜韭之氣薰之,滎衛之氣,不時受之,久留心下,故洞心。辛與氣俱行,故辛入而與汗俱出。
黃帝曰:苦走骨,多食之,令人變嘔,何也?少俞曰:苦入於胃,五穀之氣,皆不能勝苦,苦入下脘,三焦之道,皆閉而不通,故變嘔。齒者,骨之所終也,故苦入而走骨,故入而復出知其走骨也。
黃帝曰:甘走肉,多食之,令人心,何也?少俞曰:甘入於胃,其氣弱小,不能上至於上焦,而與谷留於胃中者,令人柔潤者也,胃柔則緩,緩則蠱動,蠱動則令人心。其氣外通於肉,故甘走肉。



\section{陰陽二十五人第六十四}

黃帝曰:余聞陰陽之人,何如?伯高曰:天地之間,六合之內,不離乎五,人亦應之。故五五二十五人之政,而陰陽之人不與焉。其滎又不合於眾者也,余已知之矣,願聞二十五人之形,血氣之所生別,而以候從外知內,何如?岐伯曰:悉乎哉問也,此先師之也,雖伯高猶不能明之也。黃帝避席遵循而卻曰:余聞之。得其人弗教,是謂重失,得而之,天將厭之。余願得而明之,金匱藏之,不敢揚之。岐伯曰:先立五形金木水火土,別其五色,異其五形之人,而二十五人具矣。
黃帝曰:願卒聞之。岐伯曰:慎之慎之,臣請言之。木形之人,比於上角,似於蒼帝,其為人,蒼色,小頭,長面,大肩背,直身,小手,足好,有才,勞心,少力,多憂,勞於事。能春夏,不能秋冬,感而病生,足厥陰佗佗然。大角之人,比於左足少陽,少陽之上遺遺然,左角之人,比於右足少陽,少陽之下隨隨然。角之人,比於右足少陽,少陽之上推推然。判角之人,比於左足少陽,少陽之下栝栝然。火形之人,比於上徵,似於赤帝。其為人,赤色,廣,脫面,小頭,好肩背髀腹,小手足,行安地,疾心,行搖,肩背肉滿,有氣,輕財,少信,多慮,見事明,好顏,急心,不壽暴死。能春夏,不能秋冬,秋冬感而病生手少陰,核核然。質徵之人,比於左手太陽,太陽之上肌肌然。少徵之人,比於右手太陽,太陽之下,然。右徵之人,比於右手太陽,太陽之上,鮫鮫然。質判之人,比於左手太陽,太陽之下,支支頤頤然。土形之人,比於上宮,似於上古黃帝。其為人,黃色,員面,大頭,美肩背,大腹,美股脛,小手足,多肉,上下相稱,行安地,舉足浮安,心好利人,不喜權勢,善附人也。能秋冬,不能春夏,春夏感而病生,足太陰惇惇然。大宮之人,比於左足陽明,陽明之上婉婉然。加宮之人,比於左足陽明,陽明之下坎坎然,少宮之人,比於右足陽明,陽明之上樞樞然。左宮之人,比於右足陽明,陽明之下兀兀然。金形之人,比於上商,似於白帝。其為人,方面,白色,小頭,小肩背,小腹,小手足,如骨發踵外,骨輕,身清廉,急心靜悍,善為吏,能秋冬,不能春夏,春夏感而病生,手太陰惇惇然。商之人,比於左手陽明,陽明之上廉廉然。右商之人,比於左手陽明,陽明之下脫脫然。左商之人,比於右手陽明,陽明之上監監然。少商之人,比於右手陽明,陽明之下嚴嚴然。水形之人,比上羽,似於黑帝。其為人,黑色面不平,大頭廉頤,小肩,大腹,動手足,發行搖身,下尻長背,延延然,不敬畏,善欺給人戮死,能秋冬,不能春夏,春夏感而病,生足少陰,汗汗然。大羽之人,比於右足太陽,太陽之上,頰頰然。少羽之人,比於左足太陽,太陽之下,紆紆然。眾之為人,比於右足太陽,太陽之下,然。桎之為人,比於左足太陽,太陽之上,安安然。是故五形之人二十五變者,眾之所以相欺者是也。
黃帝曰:得其形,不得其色,何如?岐伯曰:形勝色,色勝形者,至其勝時年加,感則病行,失則憂矣。形色相得者,富貴大樂。黃帝曰:其形色相勝之時,年加可知乎?岐伯曰:凡年忌下上之人,大忌常加七歲,十六歲,二十五歲,三十四歲,四十三歲,五十二歲,六十一歲,皆人之大忌,不可不自安也。感則病行,失則憂矣,當此之時,無為奸事,是謂年忌。
黃帝曰:夫子之言,脈之上下,血氣之候,以知形氣,奈何?岐伯曰:足陽明之上,血氣盛則髯美長,血少氣多則髯短,故氣少血多則髯少,血氣皆少則無髯,兩吻多畫。足陽明之下,血氣盛則下毛美長至胸。血多氣少則下毛美短至齊,行則善高舉足,足指少肉,足善寒,血少氣多,則肉而善瘃,血氣皆少,則無毛,有則稀枯瘁,善痿厥足痹。足少陽之上,氣血盛則通髯美長,血多氣少則通髯美短,血少氣多則少須,血氣皆少則無須,感於寒濕則善痹,骨痛,爪枯也。足少陽之下,血氣盛則脛毛美長,外踝肥,血多氣少則脛毛美短,外踝皮堅而厚,血少氣多則毛少,外踝皮薄而軟,血氣皆少則無毛,外踝瘦無肉。足太陽之上,血氣盛則美眉,眉有毫毛,血多氣少則惡眉,面多少理,血少氣多則面多肉,血氣和則美色。足太陽之下,血氣盛則跟肉滿,踵堅,氣少血多則踵跟空,血氣皆少則善轉筋踵下痛。第四節手陽明之上,血氣盛則髭美,血少氣多則髭惡。血氣皆少則無髭,手陽明之下,血氣盛則腋下毛美,手魚肉以溫,氣血皆少則手瘦以寒。手少陽之上,血氣盛則眉美以長,耳色美,血氣皆少則耳焦惡色。手少陽之下,血氣盛則手卷多肉以溫,血氣皆少則寒以瘦,氣少血多則瘦以多脈。第六節手太陽之上,血氣盛則口多須,面多肉以平,血氣皆少則面瘦惡色。手太陽之下,血氣盛則掌肉充滿,血氣皆少則掌瘦以寒。

黃帝曰:二十五人者,刺之有約乎?岐伯曰:美眉者,足太陽之脈,氣血多。惡眉者,血氣少。其肥而澤者,血氣有餘。肥而不澤者,氣有餘,血不足。瘦而無澤者,氣血俱不足。審察其形氣有餘不足而調之,可以知逆順矣。
黃帝曰:刺其諸陰陽,奈何?岐伯曰:按其寸口人迎,以調陰陽,切循其經絡之凝澀結而不通者,此於身皆為痛痹,甚則不行,故凝澀。凝澀者,致氣以溫之,血和乃止。其結絡者,脈結血不行,決之乃行,故曰:氣有餘於上者,導而下之,氣不足於上者,推而休之,其稽留不至者,因而迎之,必明於經隧,乃能持之。寒與熱爭者,導而行之,其宛陳血不結者,則而予之。必先明知二十五人,則血氣之所在,左右上下,刺約畢也。

\section{五音五味第六十五}

右徵與少徵,調右手太陽上。左商與左徵,調左手陽明上。少徵與太宮,調左手陽明上。右角與太角,調右足少陽下。太徵與少徵,調左手太陽上。眾羽與少羽,調右足太陽下。少商與右商,調右手太陽下。桎羽與眾羽,調右足太陽下。少宮與太宮,調右足陽明下。判角與少角,調右足少陽下。商與上商,調右足陽明下。商與上角,調左足太陽下。
上徵與右徵同,穀麥,畜羊,果杏。手少陰藏心,色赤,味苦,時夏。上羽與太羽同,谷大豆,畜彘,果栗。足少陰藏腎,色黑,味咸,時冬。上宮與太宮同,谷稷,畜牛,果棗。足太陰藏脾,色黃,味甘,時季夏。上商與右商同,谷黍,畜雞,果桃。手太陰藏肺,色白味辛,時秋。上角與太角同。谷麻,畜犬,果李。足厥陰藏肝,色青,味酸,時春。
太宮與上角同,右足陽明上。左角與太角同,左足陽明上。少羽與太羽同,右足太陽下。左商與右商同,左手陽明下。加宮與太宮同,左足少陽上。質判與太宮同,左手太陽下。判角與太角同,左足少陽下。太羽與太角同,右足太陽上。太角與太宮同,右足少陽上。
右徵,少徵,質徵,上徵,判徵。右角,角,上角,太角,判角。右商,少商,商,上商,左商。少宮,上宮,太宮,加宮,左宮。眾羽,桎羽,上羽,太羽,少羽。

黃帝曰:婦人無須者,無血氣乎?岐伯曰:衝脈任脈,皆起於胞中,上循背裡,為經絡之海。其浮而外者,循腹右,上行會於咽喉,別而絡唇口,血氣盛則充膚熱肉,血獨盛則澹滲皮膚,生毫毛。今婦人之生,有餘於氣,不足於血,以其數脫血也。沖任之脈,不滎口,故須不生焉。黃帝曰:士人有傷於陰,陰氣絕而不起,陰不用,然其須不去,其故何也?宦者獨去,何也,願聞其故。岐伯曰:宦者去其宗筋,傷其衝脈,血瀉不復,皮膚內結,口不滎,故須不生。黃帝曰:其有天宦者,未嘗被傷,不脫於血,然其須不生,其故何也?岐伯曰:此天之所不足也,其任沖不盛,宗筋不成,有氣無血,口不滎,故須不生。黃帝曰:善乎哉,聖人之通萬物也,蓋日月之光影,音聲鼓響,聞其聲而知其形,其非夫子,孰能明萬物之精。
是故聖人視其顏色,黃赤者,多熱氣,青白者,少熱氣,黑色者,多血少氣。美眉者,太陽多血,通髯極須者,少陽多血,美眉者,陽明多血,此其時然也。夫人之常數,太陽常多血少氣,少陽常多氣少血,陽明常多血多氣,厥陰常多氣少血,少陰常多氣少血,太陰常多血少氣,此天之常數也。


\section{百病始生第六十六}

黃帝問於岐伯曰:夫百病之始生也,皆生於風雨寒暑,清濕喜怒。喜怒不節則傷藏,風雨則傷上,清濕則傷下三部之氣,所傷異類,願聞其會。岐伯曰:三部之氣各不同,或起於陰,或起於陽,請言其方。喜怒不節,則傷藏,藏傷則病起於陰也。清濕襲虛,則病起於下,風雨襲虛,則病起於上,是謂三部。至於其淫不可勝數。黃帝曰:余固不能數,故問先師,願卒聞其道。岐伯曰:風雨寒熱,不得虛邪,不能獨傷人,卒然逢疾風暴雨而不病者,蓋無虛,故邪不能獨傷人,此必因虛邪之風,與其身形,兩虛相得,乃客其形。兩實相逢,眾人肉堅。其中於虛邪也,因於天時,與其身形,參以虛實,大病乃成。氣有定舍,因處為名,上下中外,分為三員。

是故虛邪之中人也,始於皮膚,皮膚緩則腠理開,開則邪從毛髮入,入則抵深,深則毛髮立,毛髮立則淅然,故皮膚痛。留而不去,則傳舍於絡脈,在絡之時,痛於肌肉,其痛之時息,大經乃代。留而不去,傳舍於經,在經之時,灑淅喜驚。留而不去,傳舍於輸,在輸之時,六氣不通四支,則支節痛,腰脊乃強。留而不去,傳舍於伏沖之脈,在伏沖之時,體重身痛。留而不去,傳舍於腸胃,在腸胃之時,賁響腹脹,脹多寒則腸鳴飧洩,食不化,多熱則溏出麋。留而不出,傳舍於腸胃之外,募原之間,留著於脈,稽留而不去,息而成積,或著孫脈,或著絡脈,或著經脈,或著輸脈,或著於伏沖之脈,或著於膂筋,或著於腸胃之募原,上連於緩筋,邪氣淫,不可勝論。

黃帝曰:願盡聞其所由然。岐伯曰:其著孫絡之脈而成積者,其積往來上下臂手孫絡之居也,浮而緩,不能句積而止之,故往來移行腸胃之間,水湊滲注灌,濯濯有音,有寒則滿雷引,故時切痛。其著於陽明之經,則挾齊而居,飽食則益大,則益小。其著於緩筋也,似陽明之積,飽食則痛,則安。其著於腸胃之募原也,痛而外連於緩筋,飽食則安,則痛。其著於伏沖之脈者,揣之應手而動,發手則熱氣下於兩股如湯沃之狀。其著於膂筋,在腸後者,則積見,飽則積不見,按之不得。其著於輸之脈者,閉塞不通,津液不下,孔竅干塞,此邪氣之從外入內從上下也。

黃帝曰:積之始生,至其已成,奈何?岐伯曰:積之始生,得寒乃生,厥乃成積也。黃帝曰:其成積奈何?岐伯曰:厥氣生足,生脛寒,脛寒則血脈凝澀,血脈凝澀則寒氣上入於腸胃,入於腸胃則脹,脹則腸外之汁沫迫聚不得散,日以成積。卒然多食飲,則腸滿,起居不節,用力過度,則絡脈傷,陽絡傷則血外溢。血外溢則血,陰絡傷則血內溢,血內溢則後血,腸胃之絡傷,則血溢於腸外,腸外有寒,汁沫與血相搏,則併合凝聚不得散,而積成矣。卒然外中於寒,若內傷於憂怒,則氣上逆,氣上逆則六輸不通,溫氣不行,凝血蘊裹而不散,津液澀滲,著而不去,而積皆成矣。黃帝曰:其生於陰者,奈何?岐伯曰:憂思傷心,重寒傷肺,忿怒傷肝,醉以入房,汗出當風傷脾,用力過度,若入房汗出浴,則傷腎,此內外三部之所生病者也。
黃帝曰:善。治之奈何?岐伯答曰:察其所痛,以知其應,有餘不足,當補則補,當瀉則瀉,無逆天時,是謂至治。



\section{行針第六十七}

黃帝問於岐伯曰:余聞九針於夫子,而行之於百姓,百姓之血氣,各不同形,或神動而氣先針行,或氣與針相逢,或針已出,氣獨行,或數刺乃知,或髮針而氣逆,或數刺病益劇,凡此六者,各不同形,願聞其方,
岐伯曰:重陽之人其神易動,其氣易往也。黃帝曰:何謂重陽之人。岐伯曰:重陽之人,高高,言語善疾,舉足善高,心肺之藏氣有餘,陽氣滑盛而揚,故神動而氣先行。
黃帝曰:重陽之人而神不先行者,何也?岐伯曰:此人頗有陰者也。黃帝曰:何以知其頗有陰也。岐伯曰:多陽者,多喜,多陰者,多怒,數怒者,易解。故曰頗有陰,其陰陽之離合難,故其神不能先行也。
黃帝曰:其氣與針相逢,奈何?岐伯曰:陰陽和調,而血氣淖澤滑利,故針入而氣出疾而相逢也。
黃帝曰:針已出而氣獨行者,何氣使然。岐伯曰:其陰氣多而陽氣少,陰氣沉而陽氣浮者內藏,故針已出,氣乃隨其後,故獨行也。
黃帝曰:數刺乃知。何氣使然。岐伯曰:此人之多陰而少陽,其氣沉而氣往難,故數刺乃知也。
黃帝曰:針入而氣逆者,何氣使然。岐伯曰:其氣逆與其數刺病益甚者,非陰陽之氣,浮沉之勢也,此皆之所敗,工之所失,其形氣無過焉。



\section{上膈第六十八}

黃帝曰:氣為上膈者,食飲入而還出,余已知之矣。蟲為下膈,下膈者,食時乃出,余未得其意,願卒聞之。岐伯曰:喜怒不適,食飲不節,寒溫不時,則寒汁流於腸中,流於腸中則蟲寒,蟲寒則積聚,守於下管,則腸胃充郭,衛氣不滎,邪氣居之,人食則蟲上食,蟲上食則下管虛,下管虛則邪氣勝之,積聚以留,留則癰成,癰成則下管約,其癰在管內者,即而痛深,其癰在外者,則癰外而痛浮,癰上皮熱。

黃帝曰:刺之奈何?岐伯曰:微按其癰,視氣所行,先淺刺其傍,稍內益深,還而刺之,無過三行,察其沉浮,以為深淺,已刺必熨,令熱入中,日使熱內,邪氣益衰,大癰乃潰。伍以參禁,以除其內,恬無為,乃能行氣,後以咸苦,化谷乃下矣。



\section{憂恚無言第六十九}

黃帝問於少師曰:人之卒然憂恚,而言無音者,何道之塞,何氣出行,使音不彰,願聞其方。少師答曰:咽喉者,水谷之道也。喉嚨者,氣之所以上下者也。會厭者,音聲之戶也。口者,音聲之扇也。舌者,音聲之機也。懸雍垂者,音聲之關也。頏顙者,分氣之所洩也。橫骨者,神氣所使主發舌者也。故人之鼻洞涕出不收者,頏顙不開,分氣失也。是故厭小而疾薄,則發氣疾,其開闔利,其出氣易。其厭大而厚,則開闔難,其氣出遲,故重言也。人卒然無音者,寒氣客於厭,則厭不能發,發不能下至,其開闔不致,故無音。

黃帝曰:刺之奈何?岐伯曰:足之少陰,上繫於舌,絡於橫骨,終於會厭,兩瀉其血脈,濁氣乃辟,會厭之脈,上絡任脈,取之天突,其厭乃發也。



\section{寒熱第七十}

黃帝問於岐伯曰:寒熱瘰在於頸腋者,皆何氣使生。岐伯曰:此皆鼠寒熱之毒氣也,留於脈而不去者也。黃帝曰:去之奈何?岐伯曰:鼠之本,皆在於藏,其末上出於頸腋之間,其浮於脈中,而未內著於肌肉,而外為膿血者,易去也。黃帝曰:去之奈何?岐伯曰:請從其本引其末,可使衰去,而絕其寒熱。審按其道以予之,徐往徐來以去之,其小如麥者,一刺知,三刺而已。黃帝曰:決其生死奈何?岐伯曰:反其目視之,其中有赤脈,上下貫瞳子,見一脈,一歲死,見一脈半,一歲半死,見二脈,二歲死,見二脈半,二歲半死,見三脈,三歲而死,見赤脈不下貫瞳子,可治也。



\section{邪客第七十一}

黃帝問於伯高曰:夫邪氣之客人也,或令人目不瞑不臥出者,何氣使然。伯高曰:五穀入於胃也,其糟粕津液宗氣,分為三隧,故宗氣積於胸中,出於喉嚨,以貫心脈,而行呼吸焉。滎氣者,泌其津液,注之於脈,化以為血,以態四末,內注五藏六府,以應刻數焉。衛氣者,出其悍氣之疾,而先行於四末分肉皮膚之間,而不休者也,晝日行於陽,夜行於陰,常從足少陰之分間,行於五藏六府。今厥氣客於五藏六府,則衛氣獨衛其外,行於陽,不得入於陰,行於陽則陽氣盛,陽氣盛則陽陷,不得入於陰,陰虛,故目不瞑。黃帝曰:善。治之奈何?伯高曰:補其不足,瀉其有餘,調其虛實,以通其道,而去其邪,飲以半夏湯一劑,陰陽已通,其臥立至。黃帝曰:善。此所謂決瀆壅塞,經絡大通,陰陽和得者也,願聞其方。伯高曰:其湯方以流水千里以外者八升,揚之萬遍,取其清五升煮之,炊以葦薪火沸置秫米一升,治半夏五合徐炊,令竭為一升半,去其滓,飲汁一小杯,日三稍益,以知為度。故其病新發者,覆杯則臥,汗出則已矣。久者,三飲而已也。

黃帝問於伯高曰:願聞人之支節,以應天地奈何?伯高答曰:天員地方,人頭員足方,以應之。天有日月,人有兩目。地有九州,人有九竅。天有風雨,人有喜怒。天有雷電,人有音聲。天有四時,人有四支。天有五音,人有五藏。天有六律,人有六府。天有冬夏,人有寒熱。天有十日,人有手十指。辰有十二,人有足十指莖垂以應之,女子不足二節,以抱人形。天有陰陽,人有夫妻。歲有三百六十五日,人有三百六十節。地有高山,人有肩膝。地有深谷,人有腋。地有十二經水,人有十二經脈。地有泉脈,人有衛氣。地有草,人有毫毛。天有晝夜。人有臥起。天有列星,人有牙齒。地有小山,人有小節。地有山石,人有高骨。地有林木,人有募筋。地有聚邑,人有肉。歲有十二月,人有十二節。地有四時不生草,人有無子。此人與天地相應者也。

黃帝問於岐伯曰:余願聞持針之數,內針之理,縱舍之意,皮開腠理,奈何?脈之屈折,出入之處,焉至而出,焉至而止,焉至而徐,焉至而疾,焉至而入。六府之輸於身者,余願盡聞,少序別離之處,離而入陰,別而入陽,此何道而從行,願盡聞其方。岐伯曰:帝之所問,針道畢矣。黃帝曰:願卒聞之。岐伯曰:手太陰之脈,出於大指之端,內屈,循白肉際。至本節之後太淵,留以澹,外屈,上於本節下,內屈,與陰諸絡會於魚際,數脈並注,其氣滑利,伏行壅骨之下,外屈,出於寸口而行,上至於肘內廉,入於大筋之下,內屈,上行陰,入腋下,內屈,走肺。此順行逆數之曲折也。心主之脈,出於中指之端,內屈,循中指內廉以上,留於掌中,伏行兩骨之間,外屈,出兩筋之間,骨肉之際,其氣滑利,上二寸,外屈,出行兩筋之間,上至肘內廉,入於小筋之下,留兩骨之會,上于于胸中,內絡於心脈。黃帝曰:手少陰之脈,獨無俞,何也?岐伯曰:少陰,心脈也。心者,五藏六府之大主也,精神之所舍也,其藏堅固,邪弗能容也,容之則心傷,心傷則神去,神去則死矣。故諸邪之在於心者,皆在於心之包絡。包絡者,心主之脈也,故獨無俞焉。黃帝曰:少陰獨無俞者,不病乎?岐伯曰:其外經病而藏不病,故獨取其經於掌後銳骨之端,其餘脈出入屈折,其行之徐疾,皆如手少陰心主之脈行也,故本俞者,皆其因氣之虛實疾徐以取之,是謂因沖而瀉,因衰而補,如是者,邪氣得去,真氣堅固,是謂因天之序。
黃帝曰:持針縱舍奈何?岐伯曰:必先明知十二經脈之本末,皮膚之寒熱,脈之盛衰滑澀,其脈滑而盛者,病日進,虛而細者,久以持,大以澀者,為痛痹,陰陽如一者,病難治,其本末尚熱者,病尚在,其熱已衰者,其病亦去矣。持其尺,察其肉之堅脆,大小滑澀,寒溫燥濕,因視目之五色,以知五藏,而決死生,視其血脈,察其色,以知其寒熱痛。黃帝曰:持針縱舍,余未得其意也。岐伯曰:持針之道,欲端以正,安以靜,先知虛實,而行疾徐,左手執骨,右手循之,無與肉果,瀉欲端以正,補必閉膚,輔針導氣,邪得淫,真氣得居。黃帝曰:皮開腠理奈何?岐伯曰:因其分肉,左別其膚,微內而徐端之,視神不散,邪氣得去。

黃帝問於岐伯曰:人有八虛,各何以候。岐伯答曰:以候五藏。黃帝曰:候之奈何?岐伯曰:肺心有邪,其氣留於兩肘。肝有邪,其氣流於兩腋。脾有邪,其氣留於兩髀。腎有邪,其氣留於兩。凡此八虛者,皆機關之室,真氣之所過,血絡之所游,邪氣惡血,固不得住留,住留則傷筋絡骨節,機關不得屈伸,故病攣也。



\section{通天第七十二}

黃帝問於少師曰:余嘗聞人有陰陽,何謂陰人,何謂陽人。少師曰:天地之間,六合之內,不離於五,人亦應之,非徒一陰一陽而已也,而略言耳,口弗能明也。黃帝曰:願略聞其意,有賢人聖人,心能備而行之乎?少師曰:蓋有太陰之人,少陰之人,太陽之人,少陽之人,陰陽和平之人。凡五人者,其態不同,其筋骨氣血各不等。

黃帝曰:其不等者,可得聞乎?少師曰:太陰之人,貪而不仁,下齊湛湛,好內而惡出,心和而不發,不務於時,動而後之,此太陰之人也。少陰之人,小貪而賊心,見人有亡,常若有得,好傷好害,見人有滎,乃反慍怒,心疾而無恩,此少陰之人也。太陽之人,居處于于,好言大事,無能而虛說,志發於四野。舉措不顧是非,為事如常自用,事雖敗,而常無悔,此太陽之人也。少陽之人,諦好自貴,有小小官,則高自宜,好為外交,而不內附,此少陽之人也。陰陽和平之人,居處安靜,無為懼懼,無為欣欣,婉然從物,或與不爭,與時變化,尊則謙謙,譚而不治,是謂至治。

古之善用針艾者,視人五態,乃治之,盛者瀉之,虛者補之。黃帝曰:治人之五態奈何?少師曰:太陰之人,多陰而無陽,其陰血濁,其衛氣澀,陰陽不和,緩筋而厚皮,不之疾瀉,不能移之。少陰之人,多陰少陽,小胃而大腸,六府不調,其陽明脈小,而太陽脈大,必審調之,其血易脫,其氣易敗也。太陽之人,多陽而少陰,必謹調之,無脫其陰,而瀉其陽,陽重脫者,易狂,陰陽皆脫者,暴死不知人也。少陽之人,多陽少陰,經小而脈大,血在中而氣外,實陰而虛陽,獨瀉其絡脈,則強氣脫而疾,中氣不足,病不起也。陰陽和平之人,其陰陽之氣和,血脈調,謹診其陰陽,視其邪正,安容儀,審有餘不足,盛則瀉之,虛則補之,不盛不虛,以經取之。此所以調陰陽,別五態之人者也。

黃帝曰:夫五態之人者,相與無故,卒然新會,未知其行也,何以別之。少師答曰:眾人之屬,不如五態之人者,故五五二十五人,而五態之人不與焉。五態之人,尤不合於眾者也。黃帝曰:別五態之人奈何?少師曰:太陰之人,其狀然黑色,念然下意,臨臨然長大,然未僂,此太陰之人也。少陰之人,其狀清然竊然,固以陰賊,立而躁,行而似伏,此少陰之人也。太陽之人,其狀軒軒儲儲,反身折,此太陽之人也。少陽之人,其狀立則好仰,行則好搖,其兩臂兩肘,則常出於背,此少陽之人也。陰陽和平之人,其狀委委然,隨隨然,然,愉愉然,然,豆豆然,眾人皆曰君子,此陰陽和平之人也。

\section{官能第七十三}

黃帝問於岐伯曰:余聞九針於夫子,眾多矣,不可勝數。余推而論之,以為一紀,余司誦之,子聽其理,非則語余,請其正道,令可久傳後世無患,得其人乃傳,非其人勿言。岐伯稽首再拜曰:請聽聖王之道。黃帝曰:用針之理,必知形氣之所在,左右上下,陰陽表裡,血氣多少,行之逆順,出入之合。謀伐有過。知解結,知補虛瀉實,上下氣門,明通於四海,審其所在,寒熱淋露,以輸異處,審於調氣,明於經隧,左右支絡,盡知其會。寒與熱爭,能合而調之,虛與實鄰,知決而通之,左右不調,把而行之,明於逆順,乃知可治。陰陽不奇,故知起時,害於本末,察其寒熱,得邪所在,萬刺不殆,知官九針,刺道畢矣。

明於五輸徐疾所在,屈伸出入,皆有條理。言陰與陽,合於五行,五藏六府,亦有所藏,四時八風,盡有陰陽,各得其位,合於明堂,各處色部,五藏六府,察其所痛,左右上下,知其寒溫,何經所在。審皮膚之寒溫滑澀,知其所苦,膈有上下,知其氣所在,先得其道,稀而疏之,稍深以留,故能徐入之。大熱在上,推而下之,從下上者,引而去之。視前痛者,常先取之。大寒在外,留而補之。入於中者,從合瀉之。針所不為,炙之所宜。上氣不足,推而揚之。下氣不足,積而從之。陰陽皆虛,火自當之。厥而寒甚,骨廉陷下,寒過於膝,下陵三里。陰絡所過,得之留止。寒入於中,推而行之。經陷下者,火則當之。結絡堅緊,火所治之。不知所苦,兩之下,男陰女陽,良工所禁,針論畢矣。

用針之法,必有法則,上視天光,下司八正,以辟奇邪,而觀百姓,審於虛實,無犯其邪,是得天之露,遇歲之虛,救而不勝,反受其殃。故曰必知天忌,乃言針意,法於往古,驗於來今,觀於窈冥,通於無窮,之所不見,良工之所貴,莫知其形,若神。邪氣之中人也,灑淅動形,正邪之中人也微,先見於色,不知於其身,若有若亡,若亡若存,有形無形,莫知其情。是故上工之取氣,乃救其萌芽,下工守其已成,因敗其形。是故工之用針也,知氣之所在,而守其明戶,明於調氣,補瀉所在,徐疾之意,所取之處。瀉必用員,切而轉之,其氣乃行,疾而徐出,邪氣乃出,伸而迎之,搖大其穴,氣出乃疾。補必用方,外引其皮,令當其門,左引其樞,右推其膚,微旋而徐推之,必端以正,安以靜,堅心無解,欲微以留,氣下而疾出之,推其皮,蓋其外門,真氣乃存,用針之要,無忘其神。

雷公問於黃帝曰:針論曰,得其人乃傳,非其人勿言,何以知其可傳。黃帝曰:各得其人。任之其能,故能明其事。雷公曰:願聞官能奈何?黃帝曰:明目者,可使視色,聰耳者,可使聽音。捷疾辭語者,可使傳論。語徐而安靜,手巧而心審諦者,可使行針艾,理血氣而調諸逆順,察陰陽而兼諸方。緩節柔筋而心和調者,可使導引行氣。疾毒言語輕人者,可使唾癰病。爪苦手毒,為事善傷者,可使按積抑痹,各得其能,方乃可行,其名乃彰。不得其人,其功不成,其師無名。故曰:得其人乃言,非其人勿傳,此之謂也。手毒者,可使試按龜,置龜於器下,而按其上,五十日而死矣。手甘者,復生如故也。



\section{論疾診尺第七十四}

黃帝問於岐伯曰:余欲無視色持脈,獨調其尺,以言其病,從外知內,為之奈何?岐伯曰:審其尺之緩急小大滑澀,肉之堅脆,而病形定矣。視人之目窠上,微癰如新臥起狀,其頸脈動,時,按其手足上,而不起者,風水膚脹也。尺膚滑,其淖澤者,風也。尺肉弱者,解,安臥脫肉者,寒熱,不治。尺膚滑而澤脂者,風也。尺膚澀者,風痹也。尺膚如枯魚之鱗者,水飲也。尺膚熱甚,脈盛躁者,病溫也。其脈盛而滑者,病且出也。尺膚寒,其脈小者,洩,少氣。尺膚炬然先熱後寒者,寒熱也。尺膚先寒,久大之而熱者,亦寒熱也。肘所獨熱者,腰以上熱。手所獨熱者,腰以下熱。肘前獨熱者,膺前熱。肘後獨熱者,肩背熱。臂中獨熱者,腰腹熱。肘後以下三四寸熱者,腸中有蟲。掌中熱者,腹中熱。掌中寒者,腹中寒。魚上白肉有青血脈者,胃中有寒。尺炬然熱,人迎大者,當奪血。尺堅大,脈小甚,少氣有加,立死。

目赤色者病在心,白在肺,青在肝,黃在脾,黑在腎,黃色不可名者病在胸中。診目痛,赤脈從上下者太陽病,從下上者陽明病,從外走內者少陽病。診寒熱,赤脈上下至瞳子,見一脈,一歲死,見一脈半,一歲半死,見二脈,二歲死,見二脈半,二歲半死,見三脈,三歲死。診齲齒痛,按其陽之來,有過者獨熱,在左左熱,在右右熱,在上上熱,在下下熱。
診血脈者,多赤多熱,多青多痛,多黑為久痹。多赤多黑多青皆見者,寒熱身痛,而色微黃,齒垢黃,爪甲上黃,黃疸也。安臥小便黃赤,脈小而澀者,不嗜食。
人病其寸口之脈與人迎之脈小大等,及其浮沉等者,病難已也。女子手少陰脈動甚者妊子。嬰兒病,其頭毛皆逆上者必死。耳間青脈起者掣痛,大便赤硬飧洩,脈小者手足寒難已。飧洩,脈小,手足溫,洩易已。

四時之變,寒暑之勝,重陰必陽,重陽必陰,故陰主寒,陽主熱。故寒甚則熱,熱甚則寒。故曰:寒生熱,熱生寒,此陰陽之變也。故曰:冬傷於寒,春生癉熱。春傷於風,夏生飧洩腸。夏傷於暑,秋生瘧。秋傷於濕,冬生嗽。是謂四時之序也。



\section{刺節真邪第七十五}

黃帝問於岐伯曰:余聞刺有五節,奈何?岐伯曰:固有五節,一曰振埃,二曰發蒙,三曰去爪,四曰徹衣,五曰解惑。黃帝曰:夫子言五節,余未知其意。岐伯曰:振埃者,刺外,去陽病也。發蒙者,刺府輸,去府病也。去爪者,刺關節支絡也。徹衣者,盡刺諸陽之奇輸也。解惑者,盡知調陰陽,補瀉有餘不足,相傾移也。
黃帝曰:刺節言振埃,夫子乃言刺外經,去陽病,余不知其所謂也,願卒聞之。岐伯曰:振埃者,陽氣大逆,上滿於胸中,憤肩息,大氣逆上,喘喝坐伏,病惡埃煙,不得息,請言振埃,尚疾於振埃。黃帝曰:善。取之何如?岐伯曰:取之天容。黃帝曰:其上氣,窮胸痛者,取之奈何?岐伯曰:取之廉泉。黃帝曰:取之有數乎?岐伯曰:取天容者,無過一里。取廉泉者,血變而止。黃帝曰:善哉。
黃帝曰:刺節言發蒙,余不得其意,夫發蒙者,耳無所聞,目無所見,夫子乃言刺府輸,去府病,何輸使然,願聞其故。岐伯曰:妙乎哉問也,此刺之大約,針之極也,神明之類也,口說書卷,猶不能及也,請言發蒙耳,尚疾於發蒙也。黃帝曰:善。願卒聞之。岐伯曰:刺此者,必於日中,刺其聽宮,中其眸子,聲聞於耳,此其輸也。黃帝曰:善。何謂聲聞於耳。岐伯曰:刺邪以手堅按其兩鼻竅而疾偃,其聲必應於針也。黃帝曰:善。此所謂弗見為之,而無目視,見而取之,神明相得者也。
黃帝曰:刺節言去爪,夫子乃言刺關節支絡,願卒聞之。岐伯曰:腰脊者,身之大關節也。支脛者,人之管以趨翔也。莖垂者,身中之機,陰精之候,津液之道也。故飲食不節,喜怒不時,津液內溢,乃下留於睾,血道不通,日大不休,仰不便,趨翔不能,此病滎然有水,不上不下,鈹石所取,形不可匿,常不得蔽,故命曰去爪。帝曰:善。
黃帝曰:刺節言徹衣,夫子乃言盡刺諸陽之奇輸,未有常處也,願卒聞之。岐伯曰:是陽氣有餘,而陰氣不足,陰氣不足則內熱,陽氣有餘則外熱,內熱相搏,熱於懷炭,外畏綿帛近,不可近身,又不可近席,腠理閉塞,則汗不出,舌焦唇槁,干嗌燥,飲食不讓美惡。黃帝曰:善。取之奈何?岐伯曰:取之於其天府大杼三,又刺中膂,以去其熱,補足手太陰,以去其汗,熱去汗稀,疾於徹衣。黃帝曰:善。
黃帝曰:刺節言解惑,夫子乃言盡知調陰陽,補瀉有餘不足,相傾移也,惑何以解之。岐伯曰:大風在身,血脈偏虛,虛者不足,實者有餘,輕重不得,傾側宛伏,不知東西,不知南北,乍上乍下,乍反乍覆,顛倒無常,甚於迷惑。黃帝曰:善。取之奈何?岐伯曰:瀉其有餘,補其不足,陰陽平復,用針若此,疾於解惑。黃帝曰:善。請藏之靈蘭之室,不敢妄出也。

黃帝曰:余聞刺有五邪,何謂五邪。岐伯曰:病有持癰者,有容大者,有狹小者,有熱者,有寒者,是謂五邪。黃帝曰:刺五邪奈何?岐伯曰:凡刺五邪之方,不過五章,癉熱消滅,腫聚散亡,寒痹益溫。小者益陽,大者必去,請道其方。凡刺癰邪,無迎隴,易俗移性,不得膿,脆道更行,去其鄉,不安處所乃散亡,諸陰陽過癰者,取之其輸。瀉之。凡刺大邪,日以小,洩奪其有餘,乃益虛,剽其通,針其邪,肌肉親,視之無有反其真,刺諸陽分肉間。凡刺小邪,日以大,補其不足,乃無害,視其所在,迎之界,遠近盡至,其不得外,侵而行之,乃自費,刺分肉間。凡刺熱邪,越而蒼,出遊不歸,乃無病,為開通,闢門戶,使邪得出,病乃已。凡刺寒邪,日以溫,徐往徐來,致其神,門戶已閉,氣不分,虛實得調,其氣存也。
黃帝曰:官針奈何?岐伯曰:刺癰者,用鈹針。刺大者,用鋒針。刺小者,用員利針。刺熱者,用針。刺寒者,用毫針也。

請言解論,與天地相應,與四時相副,人參天地,故可為解,下有漸洳,上生葦蒲,此所以知形氣之多少也。陰陽者,寒暑也,熱則滋雨而在上,根少汁,人氣在外,皮膚緩腠理開,血氣減,汗大洩,皮淖澤。寒則地凍水冰,人氣在中,皮膚致,腠理閉,汗不出,血氣強,肉堅澀。當是之時,善行水者,不能往冰。善穿地者,不能鑿凍。善用針者,亦不能取四厥。血脈凝結,堅搏不往來者,亦未可即柔。故行水者,必待天溫,冰釋凍解,而水可行,地可穿也。人脈猶是也。治厥者,必先熨調和其經,掌與腋,肘與腳,項與脊以調之,火氣已通,血脈乃行,然後視其病,脈淖澤者,刺而平之,堅緊者,破而散之,氣下乃止,此所謂以解結者也。用針之類,在於調氣,氣積於胃,以通滎衛,各行其道,宗氣留於海,其下者,注於氣街,其上者,走於息道。故厥在於足,宗氣不下,脈中之血,凝而留止,弗之火調,弗能取之。用針者,必先察其經絡之實虛,切而循之,按而彈之,視其應動者,乃後取之而下之。六經調者,謂之不病,雖病,謂之自已也。一經上實下虛而不通者,此必有橫絡盛加於大經,令之不通,視而瀉之,此所謂解結也。上寒下熱,先刺其項太陽,久留之。已刺則熨項與肩胛,令熱下合乃止,此所謂推而上之者也。上熱下寒,視其虛脈而陷之於經絡者,取之,氣下乃止,此所謂引而下之者也。大熱身,狂而妄見妄聞妄言,視足陽明及大絡取之,虛者補之,血而實者瀉之,因其偃臥,居其頭前,以兩手四指挾按頸動脈,久持之,卷而切,推下至缺盆中,而復止如前,熱去乃止,此所謂推而散之者也。

黃帝曰:有一脈生數十病者,或痛,或癰,或熱,或寒,或癢,或痹,或不仁,變化無窮,其故何也?岐伯曰:此皆邪氣之所生也。黃帝曰:余聞氣者,有真氣,有正氣,有邪氣。何謂真氣。岐伯曰:真氣者,所受於天,與谷氣並而充身也。正氣者,正風也,從一方來,非實風,又非虛風也。邪氣者,虛風之賊傷人也,其中人也深,不能自去。正風者,其中人也淺,合而自去,其氣來柔弱,不能勝真氣,故自去。
虛邪之中人也,灑淅動形,起毫毛而發腠理。其入深,內搏於骨,則為骨痹。搏於筋,則為筋攣。搏於脈中,則為血閉不通,則為癰,搏於肉,與衛氣相搏,陽勝者,則為熱,陰勝者,則為寒,寒則真氣去,去則虛,虛則寒。搏於皮膚之間,其氣外發,腠理開,毫毛搖,氣往來行,則為癢,留而不去,則痹,衛氣不行,則為不仁。虛邪偏客於身半,其入深,內居滎衛,滎衛稍衰,則真氣去,邪氣獨留,發為偏枯。其邪氣淺者,脈偏痛。
虛邪之入於深也深,寒與熱相搏,久留而內著,寒勝其熱,則骨痛肉枯,熱勝其寒。則爛肉腐肌為膿,內傷骨,內傷骨為骨蝕。有所疾前筋,筋屈不得伸,邪氣居其間而不反,發於筋溜。有所結,氣歸之,衛氣留之,不得反,津液久留,合而為腸溜,久者,數歲乃成,以手按之柔。已有所結,氣歸之,津液留之,邪氣中之,凝結日以易甚,連以聚居,為昔瘤,以手按之堅。有所結,深中骨,氣因於骨,骨與氣並,日以益大,則為骨疽。有所結,中於肉,宗氣歸之,邪留而不去,有熱則化而為膿,無熱則為肉疽。凡此數氣者,其發無常處,而有常名也。



\section{衛氣行第七十六}

黃帝問於岐伯曰:願聞衛氣之行,出入之合,何如?岐伯曰:歲有十二月,日有十二辰,子午為經,卯酉為緯,天週二十八宿,而一面七星,四七二十八星,房昴為緯,虛張為經,是故房至畢為陽,昴至心為陰,陽主晝,陰主夜。故衛氣之行,一日一夜五十週於身,晝日行於陽二十五週,夜行於陰二十五週,周於五藏。是故平旦陰盡,陽氣出於目,目張則氣上行於頭,循項下足太陽,循背下至小指之端。其散者,別於目銳,下手太陽,下至手小指之間外側。其散者,別於目銳,下足少陽,注小指次指之間,以上循手少陽之分,側下至小指之間,別者以上至耳前,合於頷脈,注足陽明以下行,至跗上,入五指之間。其散者,從耳下下手陽明,入大指之間,入掌中。其至於足也,入足心,出內踝,下行陰分,復合於目,故為一週。

是故日行一舍,人氣行一週與十分身之八。日行二舍,人氣行三週於身與十分身之六。日行三舍,人氣行於身五週與十分身之四。日行四捨,人氣行於身七周與十分身之二。日行五舍,人氣行於身九周。日行六舍,人氣行於身十週與十分身之八。日行七舍,人氣行於身十二周在身與十分身之六。日行十四捨,人氣二十五週於身有奇分與十分身之二,陽盡於陰,陰受氣矣。

其始入於陰,常從足少陰注於腎,腎注於心,心注於肺,肺注於肝,肝注於脾,脾復注於腎為周。是故夜行一舍,人氣行於陰藏一週與十分藏之八,亦如陽行之二十五週,而復合於目。陰陽一日一夜,合有奇分十分身之四。與十分藏之二。是故人之所以臥起之時有早晏者,奇分不盡故也。
黃帝曰:衛氣之在於身也,上下往來不以期,候氣而刺之,奈何?伯高曰:分有多少,日有長短,春秋冬夏,各有分理,然後常以平旦為紀,以夜盡為始。是故一日一夜,水下百刻,二十五刻者,半日之度也,常如是無已。日入而止,隨日之長短,各以為紀而刺之,謹候其時,病可與期。失時反候者,百病不治。故曰:刺實者,刺其來也。刺虛者,刺其去也。此言氣存亡之時,以候虛實而刺之。是故謹候氣之所在而刺之,是謂逢時。在於三陽,必候其氣在於陽而刺之。病在於三陰,必候其氣在陰分而刺之。水下一刻,人氣在太陽。水下二刻,人氣在少陽。水下三刻,人氣在陽明。水下四刻,人氣在陰分。水下五刻,人氣在太陽。水下六刻,人氣在少陽。水下七刻,人氣在陽明。水下八刻,人氣在陰分。水下九刻,人氣在太陽。水下十刻,人氣在少陽。水下十一刻,人氣在陽明。水下十二刻,人氣在陰分。水下十三刻,人氣在太陽。水下十四刻,人氣在少陽。水下十五刻,人氣在陽明。水下十六刻,人氣在陰分。水下十七刻,人氣在太陽。水下十八刻,人氣在少陽。水下十九刻,人氣在陽明。水下二十刻,人氣在陰分。水下二十一刻,人氣在太陽。水下二十二刻,人氣在少陽。水下二十三刻,人氣在陽明。水下二十四刻,人氣在陰分。水下二十五刻,人氣在太陽。此半日之度也。從房至畢一十四捨水下五十刻,日行半度。回行一舍,水下三刻與七分刻之四。大要曰:常以日之加於宿上也,人氣在太陽,是故日行一舍,人氣行三陽行與陰分,常如是無已,天與地同紀,紛紛,終而復始,一日一夜,水下百刻而盡矣。



\section{九宮八風第七十七}

合八風虛實邪正(下為圖)
立秋(坤)玄委秋分(兌)倉果立冬(干)新洛夏至(離)上天招(中央)搖冬至(坎)葉蟄立夏(巽)陰洛春分(震)倉門立春(艮)天留立秋二(玄委西南方)秋分七(倉果西方)立冬六(新洛西北方)夏至九(上天南方)招搖中央冬至一(葉蟄北方)立夏四(陰洛東南方)春分三(倉門東方)立春八(天留東北方)太一常以冬至之日,居葉蟄之宮四十六日,明日居天留四十六日,明日居倉門四十六日,明日居陰洛四十五日,明日居天宮四十六日,明日居玄委四十六日,明日居倉果四十六日,明日居新洛四十五日,明日復居葉蟄之宮,曰冬至矣。太一日遊,以冬至之日,居葉蟄之宮,數所在日從一處,至九日,復反於一,常如是無已,終而復始。

太一移日天必應之以風雨,以其日風雨則吉,歲美民安少病矣。先之則多雨,後之則多旱。太一在冬至之日有變,佔在君。太一在春分之日有變,佔在相。太一在中宮之日有變,佔在吏。太一在秋分之日有變,佔在將。太一在夏至之日有變,佔在百姓。所謂有變者,太一居五宮之日,病風折樹木,揚沙石,各以其所主,佔貴賤。因視風所從來而佔之,風從其所居之鄉來為實風,主生,長養萬物。從其沖後來為虛風,傷人者也,主殺,主害者,謹候虛風而避之,故聖人日避虛邪之道,如避矢石然,邪弗能害,此之謂也。

是故太一入徙立於中宮,乃朝八風,以佔吉凶也。風從南方來,名曰大弱風,其傷人也,內舍於心,外在於脈,氣主熱。風從西南方來,名曰謀風,其傷人也,內舍於脾,外在於肌,其氣主為弱。風從西方來,名曰剛風,其傷人也,內舍於肺,外在於皮膚,其氣主為燥。風從西北方來,名曰折風,其傷人也,內舍於小腸,外在於手太陽脈,脈絕則溢,脈閉則結不通,善暴死。風從北方來,名曰大剛風,其傷人也,內舍於腎,外在於骨與肩背之膂筋,其氣主為寒也。風從東北方來,名曰凶風,其傷人也,內舍於大腸,外在於兩脅腋骨下及支節。風從東方來,名曰嬰兒風,其傷人也,內舍於肝,外在於筋紐,其氣主為身濕。風從東南方來,名曰弱風,其傷人也,內舍於胃,外在肌肉,其氣主體重。此八風皆從其虛之鄉來,乃能病人,三虛相搏,則為暴病卒死,兩實一虛,病則為淋露寒熱。犯其雨濕之地,則為痿。故聖人避風,如避矢石焉。其有三虛而偏中於邪風,則為擊僕偏枯矣。


\section{九針論第七十八}

黃帝曰:余聞九針於夫子,眾多博大矣,余猶不能寤,敢問九針焉生,何因而有名。岐伯曰:九針者,天地之大數也,始於一而終於九,故曰:一以法天,二以法地,三以法人,四以法時,五以法音,六以法律,七以法星,八以法風,九以法野。黃帝曰:以針應九之數,奈何?岐伯曰:夫聖人之起,天地之數也,一而九之,故以立九野,九而九之,九九八十一,以起黃鐘數焉,以針應數也。

一者,天也,天者,陽也,五藏之應天者肺,肺者,五藏六府之盡也,皮者,肺之合也,人之陽也,故為之治針,必以大其頭而銳其末,令無得深入而陽氣出。二者,地也,人之所以應土者,肉也,故為之治針,必其身而員其末,令無得傷肉分,傷則氣得竭。三者,人也,人之所以成生者,血脈也,故為之治針,必大其身而員其末,令可以按脈勿陷,以致其氣,令邪氣獨出。四者,時也,時者,四時八風之客於經絡之中,為瘤病者也,故為之治針,必其身而鋒其末,令可以瀉熱出血,而痼病竭。五者,音也,音者,冬夏之分,分於子午,陰與陽別,寒與熱爭,兩氣相搏,合為癰膿者也,故為之治針必令其末如劍鋒,可以取大膿。六者,律也,律者,調陰陽四時而合十二經脈,虛邪客於經絡而為暴痹者也,故為之治針,必令尖如,且員且銳,中身微大,以取暴氣。七者,星也,星者,人之七竅,邪之所客於經,而為痛痹,舍於經絡者也,故為之治針,令尖如蚊喙,靜以徐往,微以久留,正氣因之,真邪俱往,出針而養者也。八者,風也,風者,人之股肱八節也,八正之虛風,八風傷人,內舍於骨解腰脊節腠理之間,為深痹也,故為之治針,必長其身,鋒其末,可以取深邪遠痹。九者,野也,野者,人之節解皮膚之間也,淫邪流溢於身,如風水之狀,而溜不能過於機關大節者也,故為之治針,令尖如挺,其鋒微員,以取大氣之不能過於關節者也。

黃帝曰:針之長短有數乎?岐伯曰:一曰針者,取法於巾針,去末寸半,卒銳之,長一寸六分,主熱在頭身也。二曰員針,取法於絮針,其身而卯其鋒,長一寸六分,主治分肉間氣。三曰針,取法於黍粟之銳,長三寸半,主按脈取氣,令邪出。四曰鋒針,取法於絮針,其身鋒其末,長一寸六分,主癰熱出血。五曰鈹針,取法於劍鋒,廣二分半,長四寸,主大癰膿,兩熱爭者也。六曰員利針,取法於針微大其末,反小其身,令可深內也,長一寸六分,主取癰痹者也。七曰毫針,取法於毫毛,長一寸六分,主寒熱痛痹在絡者也。八曰長針,取法於綦針,長七寸,主取深邪遠痹者也。九曰大針,取法於鋒針,其鋒微員,長四寸,主取大氣不出關節者也,針形畢矣。此九針大小長短法也。

黃帝曰:願聞身形,應九野,奈何?岐伯曰:請言身形之應九野也,左足應立春,其日戊寅己丑。左脅應春分,其日乙卯。左手應立夏,其日戊辰己巳。膺喉首頭應夏至,其日丙午。右手應立秋,其日戊申己未。右脅應秋分,其日辛酉。右足應立冬,其日戊戌己亥。腰尻下竅應冬至,其日壬子。六府膈下三藏應中州,其大禁,大禁太一所在之日,及諸戊己。凡此九者,善候八正所在之處,所主左右上下,身體有癰腫者,欲治之,無以其所直之日潰治之,是謂天忌日也。

形樂志苦,病生於脈,治之以炙刺。形苦志樂,病生於筋,治之以熨引。形樂志樂,病生於肉,治之以針石。形苦志苦,病生於咽喝,治之以甘藥。形數驚恐,筋脈不通,病生於不仁,治之以按摩醪藥。是謂形。
五藏氣,心主噫,肺主,肝主語,脾主吞,腎主欠。六府氣,膽為怒,胃為氣逆噦,大腸小腸為洩,膀胱不約為遺溺,下焦溢為水。五味,酸入肝,辛入肺,苦入心,甘入脾,咸入腎,淡入胃,是謂五味。五並,精氣並肝則憂,並心則喜,並肺則憂,並腎則恐,並脾則畏,是謂五精之氣,並於藏也。五惡,肝惡風,心惡熱,肺惡寒,腎惡燥,脾惡濕,此五藏氣所惡也。五液,心主汗,肝主泣,肺主涕,腎主唾,脾主涎,此五液所出也。五勞,久視傷血,久臥傷氣,久坐傷肉,久立傷骨,久行傷筋,此五久勞所病也。五走,酸走筋,辛走氣,苦走血,咸走骨,甘走肉,是謂五走也。五裁,病在筋,無食酸,病在氣,無食辛,病在骨,無食咸,病在血,無食苦,病在肉,無食甘,口嗜而欲食之,不可多也,必自裁也,命曰五裁。五發,陰病發於骨,陽病發於血,陰病發於肉,陽病發於冬,陰病發於夏。五邪,邪入於陽,則為狂,邪入於陰,則為血痹,邪入於陽,轉則為癲疾,邪入於陰,轉則為,陽入之於陰,病靜,陰出之於陽,病喜怒。五藏,心藏神,肺藏魄,肝藏魂。脾藏意,腎藏精志也。五主,心主脈,肺主皮,肝主筋,脾主肌,腎主骨。
陽明多血多氣,太陽多血少氣,少陽多氣少血,太陰多血少氣,厥陰多血少氣,少陰多氣少血,故曰刺陽明出血氣,刺太陽出血惡氣,刺少陽出氣惡血,刺太陰出血惡氣,刺厥陰出血惡氣,刺少陰出氣惡血也。足陽明太陰為表裡,少陽厥陰為表裡,太陽少陰為表裡,是謂足之陰陽也。手陽明太陰為表裡,少陽心主為表裡,太陽少陰為表裡,是謂手之陰陽也。



\section{歲露論第七十九}

黃帝問於岐伯曰:經言夏日傷暑,秋病瘧,瘧之發以時,其故何也?岐伯對曰:邪客於風府,病循膂而下,衛氣一日一夜,常大會於風府,其明日日下一節,故其日作晏。此其先客於脊背也,故每至於風府則腠理開,腠理開則邪氣入,邪氣入則病作,此所以日作尚晏也。衛氣之行風府,日下一節,二十一日,下至尾底,二十二日,入脊內,注於伏沖之脈,其行九日,出於缺盆之中,其氣上行,故其病稍益,至其內搏於五藏,橫連募原,其道遠,其氣深,其行遲,不能日作,故次日乃畜積而作焉。黃帝曰:衛氣每至於風府,腠理乃發,發則邪入焉,其衛氣日下一節,則不當風府,奈何?岐伯曰:風府無常,衛氣之所應,必開其腠理,氣之所舍節,則其府也。黃帝曰:善。夫風之與瘧也,相與同類,而風常在,而瘧特以時休,何也?岐伯曰:風氣留其處,瘧氣隨經絡,沉以內搏,故衛氣應,乃作也。帝曰:善。

黃帝問於少師曰:余聞四時八風之中人也,故有寒暑,寒則皮膚急而腠理閉,暑則皮膚緩而腠理開,賊風邪氣因得以入乎,將必須八正虛邪,乃能傷人乎?少師答曰:不然,賊風邪氣之中人也,不得以時,然必因其開也,其入深,其內極病,其病人也,卒暴,因其閉也,其入淺以留,其病也,徐以遲。黃帝曰:有寒溫和適,腠理不開,然有卒病者,其故何也?少師答曰:帝弗知邪入乎?雖平居,其腠理開閉緩急,其故常有時也。黃帝曰:可得聞乎?少師曰:人與天地相參也,與日月相應也。故月滿則海水西盛,人血氣積,肌肉充,皮膚致,毛髮堅,腠理郗,煙垢著,當是之時,雖遇賊風,其入淺不深。至其月郭空,則海水東盛,人氣血虛,其衛氣去,形獨居,肌肉減,皮膚縱,腠理開,毛髮殘,理薄,煙垢落,當是之時,遇賊風,則其入深,其病人也,卒暴。黃帝曰:其有卒然暴死暴病者,何也?少師答曰:三虛者,其死暴疾也。得三實者邪不能傷人也。黃帝曰:願聞三虛。少師曰:乘年之衰,逢月之空,失時之和,因為賊風所傷,是謂三虛,故論不知三虛,工反為。帝曰:願聞三實。少師曰:逢年之盛,遇月之滿,得時之和,雖有賊風邪氣,不能危之也。黃帝曰:善乎哉論,明乎哉道,請藏之金匱,命曰三實,然此一夫之論也。

黃帝曰:願聞歲之所以皆同病者,何因而然。少師曰:此八正之候也。黃帝曰:候之奈何?少師曰:候此者,常以冬至之日,太一立於葉蟄之宮,其至也,天必應之以風雨者矣。風雨從南方來者,為虛風,賊傷人者也。其以夜半至也,萬民皆臥而弗犯也,故其歲民少病。其以晝至者,萬民懈惰而皆中於虛風,故萬民多病。虛邪入客於骨而不發於外,至其立春,陽氣大發,腠理開,因立春之日,風從西方來,萬民又皆中於虛風,此兩邪相搏,經氣結代者矣。故諸逢其風而遇其雨者,命曰遇歲露焉。因歲之和,而少賊風者,民少病而少死,歲多賊風邪氣,寒溫不和,則民多病而死矣。黃帝曰:虛邪之風,其所傷貴賤何如,候之奈何?少師答曰:正月朔日,太一居天留之宮,其日西北風不雨,人多死矣。正月朔日,平旦北風,春,民多死。正月朔日,平旦北風行,民病多者,十月三也。正月朔日,日中北風,夏,民多死。正月朔日,夕時北風,秋,民多死。終日北風,大病,死者十有六。正月朔日,風從南方來,命曰旱鄉,從西方來,命曰白骨,將國有殃,人多死亡。正月朔日,風從東方來,發屋,揚沙石,國有大災也。正月朔日,風從東南方來,春有死亡。正月朔日,天和溫,不風,糴賤民不病,天寒而風,糴貴民多病。此所謂候歲之風,殘傷人者也。二月丑,不風,民多心腹病。三月戍不溫,民多寒熱。四月已不暑,民多癉病。十月申不寒,民多暴死。諸所謂風者,皆發屋,折樹木,揚沙石,起毫毛,發腠理者也。


\section{大惑論第八十}

黃帝問於岐伯曰:余嘗上於清冷之台,中階而顧,匍匐而前,則惑,余私異之,竊內怪之,獨瞑獨視,安心定氣,久而不解,獨博獨眩,披髮長跪,而視之,後久之不已也。卒然自上,何氣使然。岐伯對曰:五藏六府之精氣,皆上於目,而為之精,精之窠為眼,骨之精瞳子,筋之精為黑眼,血之精為絡,其窠氣之精為白眼,肌肉之精為約,裹擷筋骨血氣之精而與脈並為系,上屬於腦,後出於項中,故邪中於項,因逢其身之虛,其入深,則隨眼系以入於腦,入於腦則腦轉,腦轉則引目系急,目系急則目眩以轉矣。邪其精,其精所中,不相比也,則精散,精散則視岐,視岐見兩物。目者,五藏六府之精也,滎衛魂魄之所常滎也,神氣之所生也,故神勞則魂魄散,志意亂,是故瞳子黑眼法於陰,白眼赤脈法於陽也。故陰陽合傳而精明也。目者,心使也,心者,神之舍也,故神精亂而不轉,卒然見非常處,精神魂魄,散不相得,故曰惑也。黃帝曰:余疑其然。余每之東苑,未嘗不惑,去之則復,余唯獨為東苑勞神乎,何其異也。岐伯曰:不然也。心有所喜,神有所惡,卒然相惑,則精氣亂,視誤,故惑,神移乃復,是故間者為迷,甚者為惑。
黃帝曰:人之善忘者,何氣使然。岐伯曰:上氣不足,下氣有餘,腸胃實而心肺虛,虛則滎衛留於下,久之不以時上,故善忘也。
黃帝曰:人之善飢而不嗜食者,何氣使然。岐伯曰:精氣並於脾,熱氣留於胃,胃熱則消谷,谷消故善,胃氣逆上,則胃脘寒,故不嗜食也。
黃帝曰:病而不得臥者,何氣使然。岐伯曰:衛氣不得入於陰,常留於陽,留於陽則陽氣滿,陽氣滿則陽盛,不得入於陰則陰氣虛,故目不瞑矣。
黃帝曰:病目而不得視者,何氣使然。岐伯曰:衛氣留於陰,不得行於陽,留於陰則陰氣盛,陰氣盛則陰滿,不得入於陽則陽氣虛,故目閉也。
黃帝曰:人之多臥者,何氣使然。岐伯曰:此人腸胃大而皮膚濕而分肉不解焉。腸胃大則衛氣留久,皮膚濕則分肉不解,其行遲。夫衛氣者,晝日常行於陽,夜行於陰,故陽氣盡則臥,陰氣盡則寤。故腸胃大,則衛氣行留久,皮膚涇,分肉不解,則行遲,留於陰也久,其氣不清,則欲瞑,故多臥矣。其腸胃小,皮膚滑以緩,分肉解利,衛氣之留於陽也久,故少瞑焉。黃帝曰:其非常經也,卒然多臥者。何氣使然。岐伯曰:邪氣留於上焦,上焦閉而不通,已食若飲湯,衛氣留久於陰而不行,故卒然多臥焉。
黃帝曰:善。治此諸邪,奈何?岐伯曰:先其藏府,誅其小過,後調其氣,盛者瀉之,虛者補之,必先明知其形志之苦樂,定乃取之。

\section{癰疽第八十一}

黃帝曰:余聞腸胃受谷,上焦出氣,以溫分肉,而養骨節,通腠理。中焦出氣如露,上注溪谷,而滲孫脈,津液和調,變化而赤為血,血和則孫脈先滿,溢乃注於絡脈,皆盈,乃注於經脈。陰陽已張,因息乃行,行有經紀,周有道理,與天合同,不得休止。切而調之,從虛去實,瀉則不足,疾則氣減,留則先後,從虛去虛,補則有餘,血氣已調,形氣乃持。余已知血氣之平與不平,未知癰疽之所從生,成敗之時,死生之期,有遠近何以度之,可得聞乎?岐伯曰:經脈留行不止,與天同度,與地合紀。故天宿失度,日月薄蝕,地經失紀,水道流溢,草不成,五穀不殖,徑路不通,民不往來,巷聚邑居,則別離異處,血氣猶然,請言其故。夫血脈滎衛,周流不休,上應星宿,下應經數,寒邪客經絡之中,則血泣,血泣則不通,不通則衛氣歸之,不得復反,故癰腫寒氣化為熱,熱勝則腐肉,肉腐則為膿,膿不瀉則爛筋,筋爛則傷骨,骨傷則髓消,不當骨空,不得洩瀉,血枯空虛,則筋骨肌肉不相榮,經脈敗漏,薰於五藏,藏傷故死矣。

黃帝曰:願盡聞癰疽之形,與忌日名。岐伯曰:癰發於嗌中,名曰猛疽。猛疽不治,化為膿,膿不瀉,塞咽,半日死。其化為膿者,瀉則合豕膏,冷食,三日而已。發於頸,名曰夭疽,其癰大以赤黑,不急治,則熱氣下入淵腋,前傷任脈,內薰肝肺,薰肝肺,十餘日而死矣。陽留大發,消腦留項,名曰腦爍,其色不樂,項痛而如刺以針,煩心者,死不可治。發於肩及,名曰疵癰,其狀赤黑,急治之,此令人汗出至足,不害五藏,癰發四五日,逞之。發於腋下赤堅者,名曰米疽,治之以砭石,欲細而長,疏砭之,涂已豕膏,六日已,勿裹之。其癰堅而不潰者,為馬刀挾癭,急治之。發於胸,名曰井疽,其狀如大豆,三四日起,不早治,下入腹,不治,七日死矣。發於膺,名曰甘疽,色青,其狀如谷實,常苦寒熱,急治之,去其寒熱,十歲死,死後出膿。發於脅,名曰敗疵,敗疵者,女子之病也,炙之,其病大癰膿,治之,其中乃有生肉,大如赤小豆,銼草根各一升,以水一斗六升煮之,竭為取三升,則強飲厚衣,坐於釜上令汗出至足,已。發於股脛,名曰股脛疽,其狀不甚,變而癰膿搏骨,不急治,三十日死矣。發於尻,名曰銳疽,其狀赤堅大,急治之,不治,三十日死矣。發於股陰,名曰赤施,不急治,六十日死,在兩股之內,不治,十日而當死,發於膝,名曰疵癰,其狀大,癰色不變,寒熱,如堅石,勿石,石之者死。須其柔,乃石之者,生。諸癰疽之發於節而相應者,不可治也,發於陽者,百日死,發於陰者,三十日死。發於脛,名曰兔,其狀赤至骨,急治之,不治害人也。發於內踝,名曰走緩,其狀癰也,色不變,數石其輸,而止其寒熱,不死。發於足上下,名曰四淫,其狀大癰,急治之,百日死。發於足傍,名曰厲癰,其狀不大,初如小指發,急治之,去其黑者,不消輒益,不治,百日死。發於足指,名脫癰,其狀赤黑,死不治,不赤黑,不死,不衰,急斬之,不則死矣。

黃帝曰:夫子言癰疽,何以別之,岐伯曰:滎衛稽留於經脈之中,則血泣而不行,不行則衛氣從之而不通,壅遏而不得行,故熱。大熱不止,熱勝,則肉腐,肉腐則為膿,然不能陷骨髓,不為枯,五藏不為傷,故命曰癰。黃帝曰:何謂疽。岐伯曰:熱氣淳盛,下陷肌膚,筋髓枯,內連五藏,血氣竭,當其癰下,筋骨良肉皆無餘,故命曰疽。疽者,上之皮夭以堅,上如牛領之皮,癰者其皮上薄以澤,此其候也。


\EveryShipout{\special{pdf: put @thispage <</Rotate 90>>}}
\newfontfamily\zhfont[RawFeature={vertical:}]{[AdobeSongStd-Light.otf]}
\newfontfamily\zhpunctfont[RawFeature={vertical:+vert:+vhal}]{[AdobeSongStd-Light.otf]}
\begin{withgezhu}
昔在黃帝生而神靈弱而能言幼而徇齊長而敦敏成而登天乃問於天師曰余聞上古之人春秋皆度百歲而動作不衰今時之人年半百而動作皆衰者時世異耶人將失之耶岐伯對曰上古之人其知道者法於陰陽和於術數食飲有節起居有常不妄作勞故能形與神俱而盡終其天年度百歲乃去今時之人不然也以酒為漿以妄為常醉以入房以欲竭其精以耗散其真不知持滿不時御神務快其心逆於生樂起居無節故半百而衰也

夫上古聖人之教下也.皆謂之虛邪賊風.避之有時.恬淡虛無.真氣從之.精神內守.病安從來.是以志閒而少欲.心安而不懼.形勞而不倦.氣從以順.各從其欲.皆得所願.故美其食.任其服.樂其俗.高下不相慕.其民故曰朴.是以嗜欲不能勞其目.淫邪不能惑其心.愚智賢不肖不懼於物.故合於道.所以能年皆度百歲.而動作不衰者.以其德全不危也.

帝曰。人年老而無子者。材力盡耶。將天數然也。岐伯曰。女子七歲。腎氣盛。齒更髮長。二七而天癸至。任脈通。太衝脈盛。月事以時下。故有子。三七。腎氣平均。故真牙生而長極。四七。筋骨堅。髮長極。身體盛壯。五七。陽明脈衰。面始焦。發始墮。六七。三陽脈衰於上。面皆焦。發始白。七七。任脈虛。太衝脈衰少。天癸竭。地道不通。故形壞而無子也。丈夫八歲。腎氣實。髮長齒更。二八。腎氣盛。天癸至。精氣溢寫。陰陽和。故能有子。三八。腎氣平均。筋骨勁強。故真牙生而長極。四八。筋骨隆盛。肌肉滿壯。五八。腎氣衰。發墮齒槁。六八。陽氣衰竭於上。面焦。髮鬢頒白。七八。肝氣衰。筋不能動。天癸竭。精少。腎藏衰。形體皆極。八八。則齒發去。腎者主水。受五藏六府之精而藏之。故五藏盛。乃能寫。今五藏皆衰。筋骨解墮。天癸盡矣。故髮鬢白。身體重。行步不正。而無子耳。

帝曰。有其年已老而有子者何也。岐伯曰。此其天壽過度。氣脈常通。而腎氣有餘也。此雖有子。男不過盡八八。女不過盡七七。而天地之精氣皆竭矣。

帝曰。夫道者年皆百數。能有子乎。岐伯曰。夫道者能卻老而全形。身年雖壽。能生子也。

黃帝曰。余聞上古有真人者。提挈天地。把握陰陽。呼吸精氣。獨立守神。肌肉若一。故能壽敝天地。無有終時。此其道生。中古之時。有至人者。淳德全道。和於陰陽。調於四時。去世離俗。積精全神。遊行天地之間。視聽八達之外。此蓋益其壽命而強者也。亦歸於真人。其次有聖人者。處天地之和。從八風之理。適嗜欲於世俗之間。無恚嗔之心。行不欲離於世。被服章。舉不欲觀於俗。外不勞形於事。內無思想之患。以恬愉為務。以自得為功。形體不敝。精神不散。亦可以百數。其次有賢人者。法則天地。像似日月。辨列星辰。逆從陰陽。分別四時。將從上古合同於道。亦可使益壽而有極時。

  春三月,此謂發陳,天地俱生,萬物以榮,夜臥早起,廣步於庭,被發緩形,以使志生,生而勿殺,予而勿奪,賞而勿罰,此春氣之應,養生之道也。逆之則傷肝,夏為寒變,奉長者少。
  夏三月,此謂蕃秀,天地氣交,萬物華實,夜臥早起,無厭於日,使志無怒,使華英成秀,使氣得洩,若所愛在外,此夏氣之應,養長之道也。逆之則傷心,秋為痎瘧,奉收者少,冬至重病。
  秋三月,此謂容平,天氣以急,地氣以明,早臥早起,與雞俱興,使志安寧,以緩秋刑,收斂神氣,使秋氣平,無外其志,使肺氣清,此秋氣之應,養收之道也。逆之則傷肺,冬為飧洩,奉藏者少。
  冬三月,此謂閉藏,水冰地坼,無擾乎陽,早臥晚起,必待日光,使志若伏若匿,若有私意,若已有得,去寒就溫,無洩皮膚,使氣亟奪,此冬氣之應,養藏之道也。逆之則傷腎,春為痿厥,奉生者少。
  天氣,清淨光明者也,藏德不止,故不下也。天明則日月不明,邪害空竅,陽氣者閉塞,地氣者冒明,雲霧不精,則上應白露不下。交通不表,萬物命故不施,不施則名木多死。惡氣不發,風雨不節,白露不下,則菀槁不榮。賊風數至,暴雨數起,天地四時不相保,與道相失,則未央絕滅。唯聖人從之,故身無奇病,萬物不失,生氣不竭。逆春氣,則少陽不生,肝氣內變。逆夏氣,則太陽不長,心氣內洞。逆秋氣,則太陰不收,肺氣焦滿。逆冬氣,則少陰不藏,腎氣獨沉。夫四時陰陽者,萬物之根本也。所以聖人春夏養陽,秋冬養陰,以從其根,故與萬物沉浮於生長之門。逆其根,則伐其本,壞其真矣。
  故陰陽四時者,萬物之終始也,死生之本也,逆之則災害生,從之則苛疾不起,是謂得道。道者,聖人行之,愚者佩之。從陰陽則生。逆之則死,從之則治,逆之則亂。反順為逆,是謂內格。
  是故聖人不治已病,治未病,不治已亂,治未亂,此之謂也。夫病已成而後藥之,亂已成而後治之,譬猶渴而穿井,而鑄錐,不亦晚乎。


\section{生氣通天論篇第三}

  黃帝曰:夫自古通天者生之本,本於陰陽。天地之間,六合之內,其氣九州、九竅、五藏、十二節,皆通乎天氣。其生五,其氣三,數犯此者,則邪氣傷人,此壽命之本也。
  蒼天之氣清淨,則志意治,順之則陽氣固,雖有賊邪,弗能害也,此因時之序。故聖人傳精神,服天氣,而通神明。失之則內閉九竅,外壅肌肉,衛氣散解,此謂自傷,氣之削也。
  陽氣者若天與日,失其所,則折壽而不彰,故天運當以日光明。是故陽因而上,衛外者也。因於寒,欲如運樞,起居如驚,神氣乃浮。因於暑,汗煩則喘喝,靜則多言,體若燔炭,汗出而散。因於濕,首如裹,濕熱不攘,大筋短,小筋弛長,短為拘,弛長為痿。因於氣,為腫,四維相代,陽氣乃竭。
  陽氣者,煩勞則張,精絕,辟積於夏,使人煎厥。目盲不可以視,耳閉不可以聽,潰潰乎若壞都,汨汨乎不可止。陽氣者,大怒則形氣絕,而血菀於上,使人薄厥。有傷於筋,縱,其若不容,汗出偏沮,使人偏枯。汗出見濕,乃生痤。高粱之變,足生大丁,受如持虛。勞汗當風,寒薄為,郁乃痤。
  陽氣者,精則養神,柔則養筋。開闔不得,寒氣從之,乃生大僂。陷脈為瘻。留連肉腠,俞氣化薄,傳為善畏,及為驚駭。營氣不從,逆於肉理,乃生癰腫。魄汗未盡,形弱而氣爍,穴俞以閉,發為風瘧。
  故風者,百病之始也,清靜則肉腠閉拒,雖有大風苛毒,弗之能害,此因時之序也。
  故病久則傳化,上下不併,良醫弗為。故陽畜積病死,而陽氣當隔,隔者當寫,不亟正治,粗乃敗之。
  故陽氣者,一日而主外,平旦人氣生,日中而陽氣隆,日西而陽氣已虛,氣門乃閉。是故暮而收拒,無擾筋骨,無見霧露,反此三時,形乃困薄。
  岐伯曰:陰者,藏精而起亟也,陽者,衛外而為固也。陰不勝其陽,則脈流薄疾,並乃狂。陽不勝其陰,則五藏氣爭,九竅不通。是以聖人陳陰陽,筋脈和同,骨髓堅固,氣血皆從。如是則內外調和,邪不能害,耳目聰明,氣立如故。
  風客淫氣,精乃亡,邪傷肝也。因而飽食,筋脈橫解,腸澼為痔。因而大飲,則氣逆。因而強力,腎氣乃傷,高骨乃壞。
  凡陰陽之要,陽密乃固,兩者不和,若春無秋,若冬無夏,因而和之,是謂聖度。故陽強不能密,陰氣乃絕,陰平陽秘,精神乃治,陰陽離決,精氣乃絕。
  因於露風,乃生寒熱。是以春傷於風,邪氣留連,乃為洞洩,夏傷於暑,秋為瘧。秋傷於濕,上逆而咳,發為痿厥。冬傷於寒,春必溫病。四時之氣,更傷五藏。
  陰之所生,本在五味,陰之五宮,傷在五味。是故味過於酸,肝氣以津,脾氣乃絕。味過於咸,大骨氣勞,短肌,心氣抑。味過於甘,心氣喘滿,色黑,腎氣不衡。味過於苦,脾氣不濡,胃氣乃厚。味過於辛,筋脈沮弛,精神乃央。是故謹和五味,骨正筋柔,氣血以流,腠理以密,如是,則骨氣以精,謹道如法,長有天命。


\section{金匱真言論篇第四}

  黃帝問曰:天有八風,經有五風,何謂?岐伯對曰:八風發邪,以為經風,觸五藏,邪氣發病。所謂得四時之勝者,春勝長夏,長夏勝冬,冬勝夏,夏勝秋,秋勝春,所謂四時之勝也。
  東風生於春,病在肝,俞在頸項;南風生於夏,病在心,俞在胸脅;西風生於秋,病在肺,俞在肩背;北風生於冬,病在腎,俞在腰股;中央為土,病在脾,俞在脊。故春氣者病在頭,夏氣者病在藏,秋氣者病在肩背,冬氣者病在四支。
  故春善病鼽衄,仲夏善病胸脅,長夏善病洞洩寒中,秋善病風瘧,冬善病痹厥。故冬不按蹻,春不鼽衄,春不病頸項,仲夏不病胸脅,長夏不病洞洩寒中,秋不病風瘧,冬不病痹厥,飧洩而汗出也。
  夫精者身之本也。故藏於精者春不病溫。夏暑汗不出者,秋成風瘧。此平人脈法也。
  故曰:陰中有陰,陽中有陽。平旦至日中,天之陽,陽中之陽也;日中至黃昏,天之陽,陽中之陰也;合夜至雞鳴,天之陰,陰中之陰也;雞鳴至平旦,天之陰,陰中之陽也。
  故人亦應之。夫言人之陰陽,則外為陽,內為陰。言人身之陰陽,則背為陽,腹為陰。言人身之藏府中陰陽。則藏者為陰,府者為陽。肝心脾肺腎五藏,皆為陰。膽胃大腸小腸膀胱三焦六府,皆為陽。所以欲知陰中之陰陽中之陽者何也,為冬病在陰,夏病在陽,春病在陰,秋病在陽,皆視其所在,為施針石也。故背為陽,陽中之陽,心也;背為陽,陽中之陰,肺也;腹為陰,陰中之陰,腎也;腹為陰,陰中之陽,肝也;腹為陰,陰中之至陰,脾也。此皆陰陽表裡內外雌雄相俞應也,故以應天之陰陽也。
  帝曰:五藏應四時,各有收受乎?岐伯曰:有。東方青色,入通於肝,開竅於目,藏精於肝,其病發驚駭。其味酸,其類草木,其畜雞,其穀麥,其應四時,上為歲星,是以春氣在頭也,其音角,其數八,是以知病之在筋也,其臭臊。
  南方赤色,入通於心,開竅於耳,藏精於心,故病在五藏,其味苦,其類火,其畜羊,其谷黍,其應四時,上為熒惑星,是以知病之在脈也,其音徵,其數七,其臭焦。
  中央黃色,入通於脾,開竅於口,藏精於脾,故病在舌本,其味甘,其類土,其畜牛,其谷稷,其應四時,上為鎮星,是以知病之在肉也,其音宮,其數五,其臭香。
  西方白色,入通於肺,開竅於鼻,藏精於肺,故病在背,其味辛,其類金,其畜馬,其穀稻,其應四時,上為太白星,是以知病之在皮毛也,其音商,其數九,其臭腥。
  北方黑色,入通於腎,開竅於二陰,藏精於腎,故病在谿,其味咸,其類水,其畜彘,其谷豆,其應四時,上為辰星,是以知病之在骨也,其音羽,其數六,其臭腐。故善為脈者,謹察五藏六府,一逆一從,陰陽表裡雌雄之紀,藏之心意,合心於精,非其人勿教,非其真勿授,是謂得道。


\section{陰陽應像大論篇第五}

黃帝曰:陰陽者,天地之道也,萬物之綱紀,變化之父母,生殺之本始,神明之府也。治病必求於本。故積陽為天,積陰為地。陰靜陽躁,陽生陰長,陽殺陰藏。陽化氣,陰成形。寒極生熱,熱極生寒。寒氣生濁,熱氣生清。清氣在下,則生飧洩;濁氣在上,則生(月真)脹。此陰陽反作,病之逆從也。
  故清陽為天,濁陰為地;地氣上為雲,天氣下為雨;雨出地氣,雲出天氣。故清陽出上竅,濁陰出下竅;清陽發腠理,濁陰走五藏;清陽實四支,濁陰歸六府。
  水為陰,火為陽,陽為氣,陰為味。味歸形,形歸氣,氣歸精,精歸化,精食氣,形食味,化生精,氣生形。味傷形,氣傷精,精化為氣,氣傷於味。
  陰味出下竅,陽氣出上竅。味厚者為陰,薄為陰之陽。氣厚者為陽,薄為陽之陰。味厚則洩,薄則通。氣薄則發洩,厚則發熱。壯火之氣衰,少火之氣壯。壯火食氣,氣食少火。壯火散氣,少火生氣。
  氣味辛甘發散為陽,酸苦湧洩為陰。陰勝則陽病,陽勝則陰病。陽勝則熱,陰勝則寒。重寒則熱,重熱則寒。寒傷形,熱傷氣。氣傷痛,形傷腫。故先痛而後腫者,氣傷形也;先腫而後痛者,形傷氣也。
  風勝則動,熱勝則腫,燥勝則干,寒勝則浮,濕勝則濡寫。
  天有四時五行,以生長收藏,以生寒暑燥濕風。人有五藏,化五氣,以生喜怒悲憂恐。故喜怒傷氣,寒暑傷形。暴怒傷陰,暴喜傷陽。厥氣上行,滿脈去形。喜怒不節,寒暑過度,生乃不固。故重陰必陽,重陽必陰。
  故曰:冬傷於寒,春必溫病;春傷於風,夏生飧洩;夏傷於暑,秋必痎瘧;秋傷於濕,冬生咳嗽。
  帝曰:余聞上古聖人,論理人形,列別藏府,端絡經脈,會通六合,各從其經,氣穴所發各有處名,谿谷屬骨皆有所起,分部逆從,各有條理,四時陰陽,盡有經紀,外內之應,皆有表裡,其信然乎?
  岐伯對曰:東方生風,風生木,木生酸,酸生肝,肝生筋,筋生心,肝主目。其在天為玄,在人為道,在地為化。化生五味,道生智,玄生神,神在天為風,在地為木,在體為筋,在藏為肝,在色為蒼,在音為角,在聲為呼,在變動為握,在竅為目,在味為酸,在志為怒。怒傷肝,悲勝怒;風傷筋,燥勝風;酸傷筋,辛勝酸。
  南方生熱,熱生火,火生苦,苦生心,心生血,血生脾,心主舌。其在天為熱,在地為火,在體為脈,在藏為心,在色為赤,在音為徵,在聲為笑,在變動為憂,在竅為舌,在味為苦,在志為喜。喜傷心,恐勝喜;熱傷氣,寒勝熱,苦傷氣,咸勝苦。
  中央生濕,濕生土,土生甘,甘生脾,脾生肉,肉生肺,脾主口。其在天為濕,在地為土,在體為肉,在藏為脾,在色為黃,在音為宮,在聲為歌,在變動為噦,在竅為口,在味為甘,在志為思。思傷脾,怒勝思;濕傷肉,風勝濕;甘傷肉,酸勝甘。
  西方生燥,燥生金,金生辛,辛生肺,肺生皮毛,皮毛生腎,肺主鼻。其在天為燥,在地為金,在體為皮毛,在藏為肺,在色為白,在音為商,在聲為哭,在變動為咳,在竅為鼻,在味為辛,在志為憂。憂傷肺,喜勝憂;熱傷皮毛,寒勝熱;辛傷皮毛,苦勝辛。
  北方生寒,寒生水,水生咸,咸生腎,腎生骨髓,髓生肝,腎主耳。其在天為寒,在地為水,在體為骨,在藏為腎,在色為黑,在音為羽,在聲為呻,在變動為栗,在竅為耳,在味為咸,在志為恐。恐傷腎,思勝恐;寒傷血,燥勝寒;咸傷血,甘勝咸。
  故曰:天地者,萬物之上下也;陰陽者,血氣之男女也;左右者,陰陽之道路也;水火者,陰陽之徵兆也;陰陽者,萬物之能始也。故曰:陰在內,陽之守也;陽在外,陰之使也。
  帝曰:法陰陽奈何?岐伯曰:陽勝則身熱,腠理閉,喘粗為之仰,汗不出而熱,齒干以煩冤,腹滿,死,能冬不能夏。陰勝則身寒,汗出,身常清,數栗而寒,寒則厥,厥則腹滿,死,能夏不能冬。此陰陽更勝之變,病之形能也。
  帝曰:調此二者奈何?岐伯曰:能知七損八益,則二者可調,不知用此,則早衰之節也。年四十,而陰氣自半也,起居衰矣。年五十,體重,耳目不聰明矣。年六十,陰痿,氣大衰,九竅不利,下虛上實,涕泣俱出矣。故曰:知之則強,不知則老,故同出而名異耳。智者察同,愚者察異,愚者不足,智者有餘,有餘則耳目聰明,身體輕強,老者復壯,壯者益治。是以聖人為無為之事,樂恬憺之能,從欲快志於虛無之守,故壽命無窮,與天地終,此聖人之治身也。
  天不足西北,故西北方陰也,而人右耳目不如左明也。地不滿東南,故東南方陽也,而人左手足不如右強也。帝曰:何以然?岐伯曰:東方陽也,陽者其精並於上,並於上,則上明而下虛,故使耳目聰明,而手足不便也。西方陰也,陰者其精並於下,並於下,則下盛而上虛,故其耳目不聰明,而手足便也。故俱感於邪,其在上則右甚,在下則左甚,此天地陰陽所不能全也,故邪居之。
  故天有精,地有形,天有八紀,地有五里,故能為萬物之父母。清陽上天,濁陰歸地,是故天地之動靜,神明為之綱紀,故能以生長收藏,終而復始。惟賢人上配天以養頭,下象地以養足,中傍人事以養五藏。天氣通於肺,地氣通於嗌,風氣通於肝,雷氣通於心,谷氣通於脾,雨氣通於腎。六經為川,腸胃為海,九竅為水注之氣。以天地為之陰陽,陽之汗,以天地之雨名之;陽之氣,以天地之疾風名之。暴氣象雷,逆氣象陽。故治不法天之紀,不用地之理,則災害至矣。
  故邪風之至,疾如風雨,故善治者治皮毛,其次治肌膚,其次治筋脈,其次治六府,其次治五藏。治五藏者,半死半生也。故天之邪氣,感則害人五藏;水谷之寒熱,感則害於六府;地之濕氣,感則害皮肉筋脈。
  故善用針者,從陰引陽,從陽引陰,以右治左,以左治右,以我知彼,以表知裡,以觀過與不及之理,見微得過,用之不殆。善診者,察色按脈,先別陰陽;審清濁,而知部分;視喘息,聽音聲,而知所苦;觀權衡規矩,而知病所主。按尺寸,觀浮沉滑澀,而知病所生;以治無過,以診則不失矣。
  故曰:病之始起也,可刺而已;其盛,可待衰而已。故因其輕而揚之,因其重而減之,因其衰而彰之。形不足者,溫之以氣;精不足者,補之以味。其高者,因而越之;其下者,引而竭之;中滿者,寫之於內;其有邪者,漬形以為汗;其在皮者,汗而發之;其慓悍者,按而收之;其實者,散而寫之。審其陰陽,以別柔剛,陽病治陰,陰病治陽,定其血氣,各守其鄉,血實宜決之,氣虛宜掣引之。

\end{withgezhu}


%% \section{异体字注音}

%% 焫:ruo4
%% 鑱:chan2
%% 熇:he4
%% 虙:fu2
%% 皏:peng3
%% 炲:tai2
%% 脽:shui2
%% 齗:yin2
%% 鬄:ti4
%% 稸:xu4
%% 黅:jin1
%% 憹:nao2
%% 爇:ruo4
%% 吤:jie4
%% 蛕:hui2,蛔
%% 腄:chui2
%% 晬:zui4
%% 肬:you2
%% 髃:yu2
%% 覩:du3,睹
%% 昬:hun1,昏
%% 顀:chui2,椎
%% 頄:qiu2,颧骨

%% (月真):chen1,胸膈或上腹部脹滿不適
%% (雩重):zhong1,氣之往來不息
%% (疒肙):juan1
%% (亻亦):yi4
%% (月囷):jun4,肌肉的突起部位
%% (火矣):ai1
%% (目巟):huang1,目昏暗,視物不清
%% (骨行):heng2,脛腓骨的統稱,小腿部,腳脛部
%% (疒龍):
%% (月少):miao3,季脅下挾脊兩旁空軟處
%% (火矣):ai1,火燒,火熨、灸焫等治法
%% (骨盾):tu2:皮肉肥厚之處
%% (月呂):lv2
%% (蕓去草頭令)
%% (疒帬):wan2,痹,麻木
%% (出頁):zhuo1,眼眶下面的骨
%% (疒峻-山):
%% (疒貴):tui2
%% (亻聶):che4,懾
%% (口父):fu3,用嘴咀嚼
%% (去欠):qu4,呿,张口
%% (骨曷):he2,肩骨
%% (骨亏)
%% (疒水):shui4

\end{document}
