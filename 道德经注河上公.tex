\documentclass[a4paper,zihao=-4,oneside,landscape,UTF8]{ctexart}

%页面旋转
\usepackage{everypage}
\AddEverypageHook{\special{pdf: put @thispage <</Rotate 90>>}}

%% font文字旋转
\defaultCJKfontfeatures{RawFeature={vertical:+vert}}

%% 默认字体为源雲明體
\setCJKmainfont[BoldFont=源雲明體 TTF SemiBold,ItalicFont=思源宋体 Light]{源雲明體 TTF Medium}

% 页码编号为汉字
\renewcommand{\thepage}{\Chinese{page}}

\usepackage{geometry}
\newgeometry{top=50pt, bottom=100pt, left=100pt, right=80pt,headsep=0pt}

\usepackage{titlesec}
\titleformat{\section}[hang]{\large\bfseries}{}{0pt}{}
\titlespacing{\section}{4em}{0pt}{0pt}


\ctexset{
	today = big,
	punct = quanjiao, 
	autoindent = 0pt,	
}

%去掉页眉
\usepackage{fancyhdr}
\renewcommand{\headrulewidth}{0.0pt}%
\fancypagestyle{plain}{% Redefine plain pages tyle  
  % Clear header/footer
  \fancyhf{}
  \renewcommand{\headrulewidth}{0.0pt}%
  \fancyfoot[L]{\thepage}
}
\pagestyle{plain}

%%% 命令设置
\usepackage{xcolor}
\definecolor{gray}{gray}{0.3}
\newcommand{\zhushi}[1]{\scriptsize{\textit{\textcolor{gray}{#1}}}\normalsize}

\definecolor{gray1}{gray}{0.2}
\color{gray1}


\title{\textbf{老子}\textit{ 注 \hspace{5em}河上公}\hfil}
\author{Colin Yang \\Email: yangjianlin@gmail.com}
\date{\normalsize\today 版}

\begin{document}

\ziju{0.1}
\large

\maketitle


\section{體道第一}


道可道,
\zhushi{謂經術政教之道也。}
非常道。
\zhushi{非自然生長之道也。常道當以無為養神,無事安民,含光藏暉,滅跡匿端,不可稱道。}
名可名,
\zhushi{謂富貴尊榮,高世之名也。}
非常名。
\zhushi{非自然常在之名也。常名當如嬰兒之未言,雞子之未分,明珠在蚌中,美玉處石間,內雖昭昭,外如愚頑。}
無名,天地之始。
\zhushi{無名者謂道,道無形,故不可名也。始者道本也,吐氣布化,出於虛無,為天地本始也。}
有名,萬物之母。
\zhushi{有名謂天地。天地有形位、有陰陽、有柔剛,是其有名也。萬物母者,天地含氣生萬物,長大成熟,如母之養子也。}
故常無欲,以觀其妙;
\zhushi{妙,要也。人常能無欲,則可以觀道之要,要謂一也。一出布名道,贊敘明是非。}
常有欲,以觀其僥。
\zhushi{僥,歸也。常有欲之人,可以觀世俗之所歸趣也。}
此兩者,同出而異名,
\zhushi{兩者,謂有欲無欲也。同出者,同出人心也。而異名者,所名各異也。名無欲者長存,名有欲者亡身也。}
同謂之玄,
\zhushi{玄,天也。言有欲之人與無欲之人,同受氣於天。}
玄之又玄,
\zhushi{天中復有天也。稟氣有厚薄,得中和滋液,則生賢聖,得錯亂汙辱,則生貪淫也。}
眾妙之門。
\zhushi{能知天中復有天,稟氣有厚薄,除情去欲守中和,是謂知道要之門戶也。}


\section{養身第二}

天下皆知美之為美,
\zhushi{自揚己美,使彰顯也。}
斯惡已;
\zhushi{有危亡也。}
皆知善之為善,
\zhushi{有功名也。}
斯不善已。
\zhushi{人所爭也。}
故有無相生,
\zhushi{見有而為無也。}
難易相成,
\zhushi{見難而為易也。}
長短相較,
\zhushi{見短而為長也。}
高下相傾,
\zhushi{見高而為下也。}
音聲相和,
\zhushi{上唱下必和也。}
前後相隨。
\zhushi{上行下必隨也。}
是以聖人處無為之事,
\zhushi{以道治也。}
行不言之教,
\zhushi{以身師導之也。}
萬物作焉
\zhushi{各自動也。}
而不辭,
\zhushi{不辭謝而逆止。}
生而不有,
\zhushi{元氣生萬物而不有}
為而不恃,
\zhushi{道所施為,不恃望其報也。}
功成而弗居。
\zhushi{功成事就,退避不居其位。}
夫惟弗居,
\zhushi{夫惟功成不居其位。}
是以不去。
\zhushi{福德常在,不去其身也。此言不行不可隨,不言不可知疾。上六句有高下長短,君開一源,下生百端,百端之變,無不動亂。}


\section{安民第三}

不尚賢,
\zhushi{賢謂世俗之賢,辯口明文,離道行權,去質為文也。不尚者,不貴之以祿,不貴之以官。}
使民不爭。
\zhushi{不爭功名,返自然也。}
不貴難得之貨,
\zhushi{言人君不禦好珍寶,黃金棄於山,珠玉捐於淵。}
使民不為盜。
\zhushi{上化清靜,下無貪人。}
不見可欲,
\zhushi{放鄭聲,遠美人。}
使心不亂。
\zhushi{不邪淫,不惑亂也。}
是以聖人之治,
\zhushi{說聖人治國與治身同也。}
虛其心,
\zhushi{除嗜欲,去亂煩。}
實其腹,
\zhushi{懷道抱一守,五神也。}
弱其誌,
\zhushi{和柔謙讓,不處權也。}
強其骨。
\zhushi{愛精重施,髓滿骨堅。強,其良反}
常使民無知無欲。
\zhushi{返樸守淳。}
使夫智者不敢為也。
\zhushi{思慮深,不輕言。夫音符,知音智}
為無為,
\zhushi{不造作,動因循。}
則無不治。
\zhushi{德化厚,百姓安。}


\section{無源第四}

道沖而用之
\zhushi{沖,中也。道匿名藏譽,其用在中。沖,直隆反}
或不盈,
\zhushi{或,常也。道常謙虛不盈滿。}
淵乎似萬物之宗。
\zhushi{道淵深不可知,似為萬物之宗祖。}
挫其銳,
\zhushi{銳,進也。人欲銳精進取功名,當挫止之,法道不自見也。}
解其紛,
\zhushi{紛,結恨也。當念道無為以解釋。}
和其光,
\zhushi{言雖有獨見之明,當知暗昧,不當以擢亂人也。}
同其塵。
\zhushi{當與眾庶同垢塵,不當自別殊。}
湛兮似若存。
\zhushi{言當湛然安靜,故能長存不亡。}
吾不知誰之子,
\zhushi{老子言:我不知,道所從生。}
象帝之先。
\zhushi{道自在天帝之前,此言道乃先天地之生也。至今在者,以能安靜湛然,不勞煩欲使人修身法道。}


\section{虛用第五}

天地不仁,
\zhushi{天施地化,不以仁恩,任自然也。}
以萬物為芻狗。
\zhushi{天地生萬物,人最為貴,天地視之如芻草狗畜,不貴望其報也。}
聖人不仁,
\zhushi{聖人愛養萬民,不以仁恩,法天地行自然。}
以百姓為芻狗。
\zhushi{聖人視百姓如芻草狗畜,不貴望其禮意。}
天地之間,
\zhushi{天地之間空虛,和氣流行,故萬物自生。人能除情欲,節滋味,清五臟,則神明居之也。}
其猶橐龠乎。
\zhushi{橐龠中空虛,人能有聲氣。}
虛而不屈,動而愈出。
\zhushi{言空虛無有屈竭時,動搖之,益出聲氣也。}
多言數窮,
\zhushi{多事害神,多言害身,口開舌舉,必有禍患。}
不如守中。
\zhushi{不如守德於中,育養精神,愛氣希言。}


\section{成象第六}

谷神不死,
\zhushi{谷,養也。人能養神則不死也。神,謂五臟之神也。肝藏魂,肺藏魄,心藏神,腎藏精,脾藏誌,五藏盡傷,則五神去矣。}
是謂玄牝。
\zhushi{言不死之有,在於玄牝。玄,天也,於人為鼻。牝,地也,於人為口。天食人以五氣,從鼻入藏於心。五氣輕微,為精、神、聰、明、音聲五性。其鬼曰魂,魂者雄也,主出入於人鼻,與天通,故鼻為玄也。地食人以五味,從口入藏於胃。五味濁辱,為形、骸、骨、肉、血、脈六情。其鬼曰魄,魄者雌也,主出入於人口,與地通,故口為牝也。}
玄牝之門,是謂天地根。
\zhushi{根,元也。言鼻口之門,是乃通天地之元氣所從往來也。}
綿綿若存,
\zhushi{鼻口呼噏喘息,當綿綿微妙,若可存,復若無有。}
用之不勤。
\zhushi{用氣當寬舒,不當急疾勤勞也。}


\section{韜光第七}

天長地久,
\zhushi{說天地長生久壽,以喻教人也。 }
天地所以能長且久者,以其不自生,
\zhushi{天地所以獨長且久者,以其安靜,施不求報,不如人居處,汲汲求自饒之利,奪人以自與也。}
故能長生。
\zhushi{以其不求生,故能長生不終也。}
是以聖人後其身,
\zhushi{先人而後己也。}
而身先,
\zhushi{天下敬之,先以為長。}
外其身,
\zhushi{薄己而厚人也。}
而身存。
\zhushi{百姓愛之如父母,神明佑之若赤子,故身常存。}
非以其無私邪。
\zhushi{聖人為人所愛,神明所佑,非以其公正無私所致乎。}
故能成其私。
\zhushi{人以為私者,欲以厚己也。聖人無私而己自厚,故能成其私也。}


\section{易性第八}

上善若水。
\zhushi{上善之人,如水之性。}
水善利萬物而不爭,
\zhushi{水在天為霧露,在地為源泉也。}
處眾人之所惡,
\zhushi{眾人惡卑濕垢濁,水獨靜流居之也。}
故幾於道。
\zhushi{水性幾於道同。}
居善地,
\zhushi{水性善喜於地,草木之上即流而下,有似於牝動而下人也。}
心善淵,
\zhushi{水深空虛,淵深清明。}
與善仁,
\zhushi{萬物得水以生。與,虛不與盈也。}
言善信,
\zhushi{水內影照形,不失其情也。}
正善治,
\zhushi{無有不洗,清且平也。}
事善能,
\zhushi{能方能圓,曲直隨形。}
動善時。
\zhushi{夏散冬凝,應期而動,不失天時。}
夫唯不爭,
\zhushi{壅之則止,決之則流,聽從人也。}
故無尤。
\zhushi{水性如是,故天下無有怨尤水者也。}


\section{運夷第九}

持而盈之,不如其已。
\zhushi{盈,滿也。已,止也。持滿必傾,不如止也。}
揣而梲之,不可長保。
\zhushi{揣,治也。先揣之,後必棄捐。}
金玉滿堂,莫之能守。
\zhushi{嗜欲傷神,財多累身。}
富貴而驕,自遺其咎。
\zhushi{夫富當賑貧,貴當憐賤,而反驕恣,必被禍患也。}
功成、名遂、身退,天之道。
\zhushi{言人所為,功成事立,名跡稱遂,不退身避位,則遇於害,此乃天之常道也。譬如日中則移,月滿則虧,物盛則衰,樂極則哀。}


\section{能為第十}

載營魄,
\zhushi{營魄,魂魄也。人載魂魄之上得以生,當愛養之。喜怒亡魂,卒驚傷魄。魂在肝,魄在肺。美酒甘肴,腐人肝肺。故魂靜誌道不亂,魄安得壽延年也。}
抱一,能無離乎,
\zhushi{言人能抱一,使不離於身,則長存。一者,道始所生,太言道行德,玄冥不可得見,欲使人如道也。}


\section{無用第十一}

三十輻共一轂,
\zhushi{古者車三十輻,法月數也。共一轂者,轂中有孔,故眾輻共湊之。治身者當除情去欲,使五藏空虛,神乃歸之。治國者寡能,摠眾弱共使強也。}
當其無,有車之用。
\zhushi{無,謂空虛。轂中空虛,輪得轉行,輿中空虛,人得載其上也。}
埏埴以為器,
\zhushi{埏,和也。埴,土也。和土以為飲食之器。}
當其無,有器之用。
\zhushi{器中空虛,故得有所盛受。}
鑿戶牖以為室,
\zhushi{謂作屋室。}
當其無有室之用。
\zhushi{言戶牖空虛,人得以出入觀視;室中空虛,人得以居處,是其用。}
故有之以為利,
\zhushi{利,物也,利於形用。器中有物,室中有人,恐其屋破壞,腹中有神,畏其形亡也。}
無之以為用。
\zhushi{言虛空者乃可用盛受萬物,故曰虛無能制有形。道者空也。}


\section{檢欲第十二}

五色令人目盲;
\zhushi{貪淫好色,則傷精失明也。}
五音令人耳聾;
\zhushi{好聽五音,則和氣去心,不能聽無聲之聲。}
五味令人口爽;
\zhushi{爽,亡也。人嗜於五味於口,則口亡,言失於道也。}
馳騁畋獵,令人心發狂,
\zhushi{人精神好安靜,馳騁呼吸,精神散亡,故發狂也。}
難得之貨,令人行妨。
\zhushi{妨,傷也。難得之貨,謂金銀珠玉,心貪意欲,不知饜足,則行傷身辱也。}
是以聖人為腹,
\zhushi{守五性,去六情,節誌氣,養神明。}
不為目,
\zhushi{目不妄視,妄視泄精於外。}
故去彼取此。
\zhushi{去彼目之妄視,取此腹之養性。}


\section{厭恥第十三}

寵辱若驚,
\zhushi{身寵亦驚,身辱亦驚。}
貴大患若身。
\zhushi{貴,畏也。若,至也。謂大患至身,故皆驚。}
何謂寵辱。
\zhushi{問何謂寵,何謂辱。寵者尊榮,辱者恥辱。及身還自問者,以曉人也。}
辱為下,
\zhushi{辱為下賤。}
得之若驚,
\zhushi{得寵榮驚者,處高位如臨深危也。貴不敢驕,富不敢奢。}
失之若驚,
\zhushi{失者,失寵處辱也。驚者,恐禍重來也。}
是謂寵辱若驚。
\zhushi{解上得之若驚,失之若驚。}
何謂貴大患若身。
\zhushi{復還自問:何故畏大患至身。}
吾所以有大患者,為吾有身。
\zhushi{吾所以有大患者,為吾有身。有身憂者,勤勞念其饑寒,觸情從欲,則遇禍患也。}
及吾無身,吾何有患。
\zhushi{使吾無有身體,得道自然,輕舉升雲,出入無間,與道通神,當有何患。}
故貴以身為天下者,則可寄天下,
\zhushi{言人君貴其身而賤人,欲為天下主者,則可寄立,不可以久也。}
愛以身為天下,若可托天下。
\zhushi{言人君能愛其身,非為己也,乃欲為萬民之父母。以此得為天下主者,乃可以托其身於萬民之上,長無咎也。}


\section{贊玄第十四}


視之不見名曰夷,
\zhushi{無色曰夷。言一無采色,不可得視而見之。}
聽之不見名曰希,
\zhushi{無聲曰希。言一無音聲,不可得聽而聞之。}
搏之不得名曰微。
\zhushi{無形曰微。言一無形體,不可摶持而得之。}
此三者不可致詰,
\zhushi{三者,謂夷、希、微也。不可致詰者,夫無色、無聲、無形,口不能言,書不能傳,當受之以靜,求之以神,不可問詰而得之也。}
故混而為一。
\zhushi{混,合也。故合於三名之為一。}
其上不皦,
\zhushi{言一在天上,不皦。皦,光明。}
其下不昧。
\zhushi{言一在天下,不昧。昧,有所暗冥。}
繩繩不可名,
\zhushi{繩繩者,動行無窮級也。不可名者,非一色也,不可以青黃白黑別,非一聲也,不可以宮商角徵羽聽,非一形也,不可以長短大小度之也。}
復歸於無物。
\zhushi{物,質也。復當歸之於無質。}
是謂無狀之狀,
\zhushi{言一無形狀,而能為萬物作形狀也。}
無物之象,
\zhushi{一無物質,而為萬物設形象也。}
是謂惚恍。
\zhushi{一忽忽恍恍者,若存若亡,不可見之也。}
迎之不見其首,
\zhushi{一無端末,不可預待也。除情去欲,一自歸之也。}
隨之不見其後,
\zhushi{言一無影跡,不可得而看。}
執古之道,以禦今之有,
\zhushi{聖人執守古道,生一以禦物,知今當有一也。}
能知古始,是謂道紀。
\zhushi{人能知上古本始有一,是謂知道綱紀也。}


\section{顯德第十五}

古之善為士者,
\zhushi{謂得道之君也。}
微妙玄通,
\zhushi{玄,天也。言其誌節玄妙,精與天通也。}
深不可識。
\zhushi{道德深遠,不可識知,內視若盲,反聽若聾,莫知所長。}
夫唯不可識,故強為之容。
\zhushi{謂下句也。}
與兮若冬涉川;
\zhushi{舉事輒加重慎與。與兮若冬涉川,心難之也。}
猶兮若畏四鄰;
\zhushi{其進退猶猶如拘制,若人犯法,畏四鄰知之也。}
儼兮其若容;
\zhushi{如客畏主人,儼然無所造作也。}
渙兮若冰之將釋,
\zhushi{渙者,解散。釋者,消亡。除情去欲,日以空虛。}
敦兮其若樸,
\zhushi{敦者,質厚。樸者,形未分。內守精神,外無文采也。}
曠兮其若谷;
\zhushi{曠者,寬大。谷者,空虛。不有德功名,無所不包也。}
渾兮其若濁。
\zhushi{渾者,守本真,濁者,不照然。與眾合同,不自專也。}
孰能濁以靜之,徐清。
\zhushi{孰,誰也。誰能知水之濁止而靜之,徐徐自清也。}
孰能安以久動之,徐生。
\zhushi{誰能安靜以久,徐徐以長生也。}
保此道者,不欲盈。
\zhushi{保此徐生之道,不欲奢泰盈溢。}
夫惟不盈,故能蔽不新成。
\zhushi{夫為不盈滿之人,能守蔽不為新成。蔽者,匿光榮也。新成者,貴功名。}


\section{歸根第十六}

致虛極,
\zhushi{得道之人,捐情去欲,五內清靜,至於虛極。}
守靜篤,
\zhushi{守清靜,行篤厚。}
萬物並作,
\zhushi{作,生也。萬物並生也。}
吾以觀復。
\zhushi{言吾以觀見萬物無不皆歸其本也。人當念重其本也。}
夫物蕓蕓,
\zhushi{蕓蕓者,華葉盛也。}
各復歸其根,
\zhushi{言萬物無不枯落,各復反其根而更生也。}
歸根曰靜,
\zhushi{靜謂根也。根安靜柔弱,謙卑處下,故不復死也。}
是謂復命。
\zhushi{言安靜者是為復還性命,使不死也。}
復命曰常。
\zhushi{復命使不死,乃道之所常行也。}
知常曰明;
\zhushi{能知道之所常行,則為明。}
不知常,妄作兇。
\zhushi{不知道之所常行,妄作巧詐,則失神明,故兇也。}
知常容,
\zhushi{能知道之所常行,去情忘欲,無所不包容也。}
容乃公,
\zhushi{無所不包容,則公正無私,眾邪莫當。}
公乃王,
\zhushi{公正無私,可以為天下王。治身正則形一,神明千萬,共湊其躬也。}
王乃天,
\zhushi{能王,德合神明,乃與天通。}
天乃道,
\zhushi{德與天通,則與道合同也。}
道乃久。
\zhushi{與道合同,乃能長久。}
沒身不殆。
\zhushi{能公能王,通天合道,四者純備,道德弘遠,無殃無咎,乃與天地俱沒,不危殆也。}


\section{淳風第十七}

太上,下知有之。
\zhushi{太上,謂太古無名之君。下知有之者,下知上有君,而不臣事,質樸也。}
其次,親之譽之。
\zhushi{其德可見,恩惠可稱,故親愛而譽之。}
其次畏之。
\zhushi{設刑法以治之。}
其次侮之。
\zhushi{禁多令煩,不可歸誠,故欺侮之。}
信不足焉,﹝有不信焉﹞。
\zhushi{君信不足於下,下則應之以不信,而欺其君也。}
猶兮其貴言。
\zhushi{說太上之君,舉事猶,貴重於言,恐離道失自然也。}
功成事遂,
\zhushi{謂天下太平也。}
百姓皆謂我自然。
\zhushi{百姓不知君上之德淳厚,反以為己自當然也。}


\section{俗薄第十八}

大道廢,有仁義。
\zhushi{大道之時,家有孝子,戶有忠信,仁義不見也。大道廢不用,惡逆生,乃有仁義可傳道。}
智慧出,有大偽。
\zhushi{智慧之君賤德而貴言,賤質而貴文,下則應之以為大偽奸詐。}
六親不和,有孝慈。
\zhushi{六紀絕,親戚不合,乃有孝慈相牧養也。}
國家昏亂,有忠臣。
\zhushi{政令不明,上下相怨,邪僻爭權,乃有忠臣匡正其君也。此言天下太平不知仁,人盡無欲不知廉,各自潔己不知貞。大道之世,仁義沒,孝慈滅,猶日中盛明,眾星失光。}


\section{還淳第十九}

絕聖
\zhushi{絕聖制作,反初守元。五帝垂象,倉頡作書,不如三皇結繩無文。}
棄智,
\zhushi{棄智慧,反無為。}
民利百倍。
\zhushi{農事修,公無私。}
絕仁棄義,
\zhushi{絕仁之見恩惠,棄義之尚華言。}
民復孝慈。
\zhushi{德化淳也。}
絕巧棄利,
\zhushi{絕巧者,詐偽亂真也。棄利者,塞貪路閉權門也。}
盜賊無有。
\zhushi{上化公正,下無邪私。}
此三者,
\zhushi{謂上三事所棄絕也。}
以為文不足,
\zhushi{以為文不足者,文不足以教民。}
故令有所屬。
\zhushi{當如下句。}
見素抱樸,
\zhushi{見素者,當抱素守真,不尚文飾也。抱樸者,當抱其質樸,以示下,故可法則。}
少私寡欲。
\zhushi{少私者,正無私也。寡欲者,當知足也。}


\section{異俗第二十}

絕學
\zhushi{絕學不真,不合道文。}
無憂。
\zhushi{除浮華則無憂患也。}
唯之與阿,相去幾何。
\zhushi{同為應對而相去幾何。疾時賤質而貴文。}
善之與惡,相去若何。
\zhushi{善者稱知其所窮極也。}
漂兮若無所止。
\zhushi{我獨漂漂,若飛若揚,無所止也,誌意在神域也。}
眾人皆有以,
\zhushi{以,有為也。}
而我獨頑
\zhushi{我獨無為。}
似鄙。
\zhushi{鄙,似若不逮也。}
我獨異於人
\zhushi{我獨與人異也。}
而貴食母。
\zhushi{食,用也。母,道也。我獨貴用道也。}


\section{虛心第二十一}

孔德之容,
\zhushi{孔,大也。有大德之人,無所不容,能受垢濁,處謙卑也。}
唯道是從。
\zhushi{唯,獨也。大德之人,不隨世俗所行,獨從於道也。}
道之為物,唯恍唯忽。
\zhushi{道之於萬物,獨恍忽往來,於其無所定也。}
忽兮恍兮,其中有象;
\zhushi{道唯忽恍無形,之中獨有萬物法象。}
恍兮忽兮,其中有物。
\zhushi{道唯恍忽,其中有一,經營生化,因氣立質。}
窈兮冥兮,其中有精,
\zhushi{道唯窈冥無形,其中有精實,神明相薄,陰陽交會也。}
其精甚真,
\zhushi{言存精氣,其妙甚真,非有飾也。}
其中有信。
\zhushi{道匿功藏名,其信在中也。}
自古及今,其名不去,
\zhushi{自,從也。自古至今,道常在不去。}
以閱眾甫,
\zhushi{閱,稟也。甫,始也。言道稟與,萬物始生,從道受氣。}
吾何以知眾甫之然哉。
\zhushi{吾何以知萬物從道受氣。}
以此。
\zhushi{此,今也。以今萬物皆得道精氣而生,動作起居,非道不然。}


\section{益謙第二十二}

曲則全,
\zhushi{曲己從眾,不自專,則全其身也。}
枉則直,
\zhushi{枉,屈己而伸人,久久自得直也。}
窪則盈,
\zhushi{地窪下,水流之;人謙下,德歸之。}
弊則新,
\zhushi{自受弊薄,後己先人,天下敬之,久久自新也。}
少則得,
\zhushi{自受取少則得多也,天道佑謙,神明托虛。}
多則惑。
\zhushi{財多者,惑於所守,學多者,惑於所聞。}
是以聖人抱一為天下式。
\zhushi{抱,守也。式,法也。聖人守一,乃知萬事,故能為天下法式也。}
不自見故明,
\zhushi{聖人不以其目視千裏之外也,乃因天下之目以視,故能明達也。}
不自是故彰,
\zhushi{聖人不自以為是而非人,故能彰顯於世。}
不自伐故有功,
\zhushi{伐,取也。聖人德化流行,不自取其美,故有功於天下。}
不自矜故長。
\zhushi{矜,大也。聖人不自貴大,故能久不危。}
夫唯不爭,故天下莫能與之爭。
\zhushi{此言天下賢與不肖,無能與不爭者爭也。}
古之所謂曲則全者,豈虛言哉。
\zhushi{傳古言,曲從則全身,此言非虛妄也。}
誠全而歸之。
\zhushi{誠,實也。能行曲從者,實其肌體,歸之於父母,無有傷害也。}


\section{虛無第二十三}

希言自然。
\zhushi{希言者,謂愛言也。愛言者,自然之道。}
故飄風不終朝,驟雨不終日。
\zhushi{飄風,疾風也。驟雨,暴雨也。言疾不能長,暴不能久也。}
孰為此者?天地。
\zhushi{孰,誰也。誰為此飄風暴雨者乎?天地所為。}
天地尚不能久,
\zhushi{不能終於朝暮也。}
而況於人乎?
\zhushi{天地至神合為飄風暴雨,尚不能使終朝至暮,何況人欲為暴卒乎。}
故從事於道者,
\zhushi{從,為也。人為事當如道安靜,不當如飄風驟雨也。}
道者同於道,
\zhushi{道者,謂好道人也。同於道者,所謂與道同也。}
德者同於德,
\zhushi{德者,謂好德之人也。同於德者,所謂與德同也。}
失者同於失。
\zhushi{失,謂任己而失人也。同於失者,所謂與失同也。}
同於道者,道亦樂得之。
\zhushi{與道同者,道亦樂得之也。}
同於德者,德亦樂得之,
\zhushi{與德同者,德亦樂得之也。}
同於失者,失亦樂失之。
\zhushi{與失同者,失亦樂失之也。}
信不足焉,
\zhushi{君信不足於下,下則應君以不信也。}
有不信焉。
\zhushi{此言物類相歸,同聲相應,同氣相求。雲從龍,風從虎,水流濕,火就燥,自然之類也。}


\section{苦恩第二十四}

跂者不立,
\zhushi{跂,進也。謂貪權慕名,進取功榮,則不可久立身行道也。}
跨者不行,
\zhushi{自以為貴而跨於人,眾共蔽之,使不得行。}
自見者不明,
\zhushi{人自見其形容以為好,自見其所行以為應道,殊不知其形醜,操行之鄙。}
自是者不彰,
\zhushi{自以為是而非人,眾共蔽之,使不得彰明。}
自伐者無功,
\zhushi{所謂輒自伐取其功美,即失有功於人也。}
自矜者不長。
\zhushi{好自矜大者,不可以長久。}
其其於道也,曰:餘食贅行。
\zhushi{贅,貪也。使此自矜伐之人,在治國之道,日賦斂餘祿食以為貪行。}
物或惡之。
\zhushi{此人在位,動欲傷害,故物無有不畏惡之者。}
故有道者不處也。
\zhushi{言有道之人不居其國也。}


\section{象元第二十五}

有物混成,先天地生。
\zhushi{謂道無形,混沌而成萬物,乃在天地之前。}
寂兮寥兮,獨立而不改,
\zhushi{寂者,無音聲。寥者,空無形。獨立者,無匹雙。不改者,化有常。}
周行而不殆,
\zhushi{道通行天地,無所不入,在陽不焦,托蔭不腐,無不貫穿,而不危怠也。}
可以為天下母。
\zhushi{道育養萬物精氣,如母之養子。}
吾不知其名,字之曰道,
\zhushi{我不見道之形容,不知當何以名之,見萬物皆從道所生,故字之曰道。}
強為之名曰大。
\zhushi{不知其名,強曰大者,高而無上,羅而無外,無不包容,故曰大也。}
大曰逝,
\zhushi{其為大,非若天常在上,非若地常在下,乃復逝去,無常處所也。}
逝曰遠,
\zhushi{言遠者,窮乎無窮,布氣天地,無所不通也。}
遠曰反。
\zhushi{言其遠不越絕,乃復反在人身也。}
故道大,天大,地大,王亦大。
\zhushi{道大者,包羅天地,無所不容也。天大者,無所不蓋也。地大者,無所不載也。王大者,無所不制也。}
域中有四大,
\zhushi{四大,道、天、地、王也。凡有稱有名,則非其極也。言道則有所由,有所由然後謂之為道,然則是道稱中之大也,不若無稱之大也,無稱不可而得為名,曰域也。天地王皆在乎無稱之內也,故曰域中有四大者也。}
而王居其一焉。
\zhushi{八極之內有四大,王居其一也。}
人法地,
\zhushi{人當法地安靜和柔也,種之得五谷,掘之得甘泉,勞而不怨也,有功而不制也。}
地法天,
\zhushi{天湛泊不動,施而不求報,生長萬物,無所收取。}
天法道,
\zhushi{道清靜不言,陰行精氣,萬物自成也。}
道法自然。
\zhushi{道性自然,無所法也。}


\section{重德第二十六}

重為輕根,
\zhushi{人君不重則不尊,治身不重則失神,草木之花葉輕,故零落,根重故長存也。}
靜為躁君。
\zhushi{人君不靜則失威,治身不靜則身危,龍靜故能變化,虎躁故夭虧也。躁早報反}
是以聖人終日行,不離輜重。
\zhushi{輜,靜也。聖人終日行道,不離其靜與重也。離音利輜側基反重直用反}
雖有榮觀,燕處超然。
\zhushi{榮觀,謂宮𨵗。燕處,後妃所居也。超然,逺避而不處也。觀古亂反}
奈何萬乘之主
\zhushi{奈何者,疾時主傷痛之辭。萬乗之主謂,王乗繩證反}
而以身輕天下?
\zhushi{王者至尊,而以其身行輕躁乎。疾時王奢恣輕淫也。}
輕則失臣,
\zhushi{王者輕滔則失其臣,治身輕淫則失其精。}
躁則失君。
\zhushi{王者行躁疾則失其君位,治身躁疾則失其精神也。}


\section{巧用第二十七}

善行無轍跡,
\zhushi{善行道者求之於身,不下堂,不出門,故無轍跡。}
善言無瑕謫,
\zhushi{善言謂擇言而出之,則無瑕疵謫過於天下。}
善計不用籌策,
\zhushi{善以道計事者,則守一不移,所計不多,則不用籌策而可知也。}
善閉無關楗而不可開
\zhushi{善以道閉情欲、守精神者,不如門戶有關楗可得開。楗其偃反}
善結無繩約而不可解。
\zhushi{善以道結事者,乃可結其心,不如繩索可得解也。}
是以聖人常善救人,
\zhushi{聖人所以常教人忠孝者,欲以救人性命。}
故無棄人;
\zhushi{使貴賤各得其所也。}
常善救物,
\zhushi{聖人所以常教民順四時者,欲以救萬物之殘傷。}
故無棄物。
\zhushi{聖人不賤名而貴玉視之如一。}
是謂襲明。
\zhushi{聖人善救人物,是謂襲明大道。}
故善人者,不善人之師;
\zhushi{人之行善者,聖人即以為人師。}
不善人者,善人之資。
\zhushi{資,用也。人行不善者,聖人猶教導使為善,得以給用也。}
不貴其師,
\zhushi{獨無輔也。不愛其資無所使也。}
雖智大迷,
\zhushi{雖自以為智。言此人乃大迷惑。}
是謂要妙。
\zhushi{能通此意,是謂知微妙要道也。}


\section{反樸第二十八}

知其雄,守其雌,為天下溪。
\zhushi{雄以喻尊,神也。}


\section{無為第二十九}

將欲取天下
\zhushi{欲為天下主也。}
而為之,
\zhushi{欲以有為治民。}
吾見其不得已。
\zhushi{我見其不得天道人心已明矣,天道惡煩濁,人心惡多欲。}
天下神器,不可為也。
\zhushi{器,物也。人乃天下之神物也,神物好安靜,不可以有為治。}
為者敗之,
\zhushi{以有為治之,則敗其質性。}
執者失之。
\zhushi{強執教之,則失其情實,生於詐偽也。}
故物或行或隨,
\zhushi{上所行,下必隨之也。}
或呴或吹,
\zhushi{歔,溫也。吹,寒也。有所溫必有所寒也。}
或強或羸,
\zhushi{有所強大,必有所羸弱也。}
或載或隳。
\zhushi{載,安也。隳,危也。有所安必有所危,明人君不可以有為治國與治身也。}
是以聖人去甚,去奢,去泰。
\zhushi{甚謂貪淫聲色。奢謂服飾飲食。泰謂宮室臺榭。去此三者,處中和,行無為,則天下自化。}


\section{儉武第三十}

以道佐人主者,
\zhushi{謂人主能以道自輔佐也。}
不以兵強天下。
\zhushi{以道自佐之主,不以兵革,順天任德,敵人自服。}
其事好還。
\zhushi{其舉事好還自責,不怨於人也。}
師之所處,荊棘生焉。
\zhushi{農事廢,田不修。}
大軍之後,必有兇年。
\zhushi{天應之以惡氣,即害五谷,盡傷人也。}
善有果而已,
\zhushi{善用兵者,當果敢而已,不美之。}
不敢以取強。
\zhushi{不以果敢取強大之名也。}
果而勿矜
\zhushi{當果敢謙卑,勿自矜大也。}
果而勿伐,
\zhushi{當果敢推讓,勿自伐取其美也。}
果而勿驕,
\zhushi{驕,欺也。果敢勿以驕欺人。}
果而不得已,
\zhushi{當過果敢至誠,不當逼迫不得已也。}
果而勿強
\zhushi{果敢勿以為強兵、堅甲以欺淩人也。}
物壯則老,
\zhushi{草木壯極則枯落,人壯極則衰老也。言強者不可以久。}
是謂不道。
\zhushi{枯老者,坐不行道也。}
不道早已。
\zhushi{不行道者早死。}


\section{偃武第三十一}

處之。
\zhushi{上將軍居右,喪禮尚右,死人貴陰也。}
殺人之眾,以哀悲泣之;
\zhushi{傷己德薄,不能以道化人,而害無辜之民。}
戰勝,以喪禮處之。
\zhushi{古者戰勝,將軍居喪主禮之位,素服而哭之,明君子貴德而賤兵,不得以而誅不祥,心不樂之,比於喪也,知後世用兵不已故悲痛之。}


\section{聖德第三十二}

道常無名,
\zhushi{道能陰能陽,能弛能張,能存能亡,故無常名也。}
樸雖小,天下莫敢臣。
\zhushi{道樸雖小,微妙無形,天下不敢有臣使道者也。}
侯王若能守之,萬物將自賔。
\zhushi{侯王若能守道無為,萬物將自賓,服從於德也。}
天地相合,以降甘露,
\zhushi{侯王動作能與天相應和,天即降下甘露善瑞也。}
民莫之令而自均。
\zhushi{天降甘露善瑞,則萬物莫有教令之者,皆自均調若一也。}
始制有名,
\zhushi{始,道也。有名,萬物也。道無名能制於有名,無形,能制於有形也。}
名亦既有,
\zhushi{既,盡也。有名之物,盡有情欲,叛道離德,故身毀辱也。}
夫亦將知之。
\zhushi{人能法道行德,天亦將自知之。}
知之,可以不殆。
\zhushi{天知之,則神靈佑助,不復危怠。}
譬道之在天下,猶川谷之與江海。
\zhushi{譬言道之在天下,與人相應和,如川谷與江海相流通也。}


\section{辯德第三十三}

知人者智,
\zhushi{能知人好惡,是為智。}
自知者明。
\zhushi{人能自知賢與不肖,是為反聽無聲,內視無形,故為明也。}
勝人者有力,
\zhushi{能勝人者,不過以威力也。}
自勝者強。
\zhushi{人能自勝己情欲,則天下無有能與己爭者,故為強也。}
知足者富,
\zhushi{人能知足,則長保福祿,故為富也。}
強行者有誌,
\zhushi{人能強力行善,則為有意於道,道亦有意於人。}
不失其所者乆,
\zhushi{人能自節養,不失其所受天之精氣,則可以長久。}
死而不亡者壽。
\zhushi{目不妄視,耳不妄聽,口不妄言,則無怨惡於天下,故長壽。}


\section{任成第三十四}

大道泛兮,
\zhushi{言道泛泛,若浮若沈,若有若無,視之不見,說之難殊。泛音泛}
其可左右。
\zhushi{道可左右,無所不宜。}
萬物恃之而生,
\zhushi{恃,待也。萬物皆恃道而生。}
而不辭,
\zhushi{道不辭謝而逆止也。}
功成不名有,
\zhushi{有道不名其有功也。}
愛養萬物而不為主。
\zhushi{道雖愛養萬物,不如人主有所收取。}
常無欲,可名於小。
\zhushi{道匿德藏名,怕然無為,似若微小也。}
萬物歸焉而不為主,
\zhushi{萬物皆歸道受氣,道非如人主有所禁止也。}
可名為大。
\zhushi{萬物橫來橫去,使名自在,故不若於大也。}
是以聖人終不為大,
\zhushi{聖人法道匿德藏名,不為滿大。}
故能成其大。
\zhushi{聖人以身師導,不言而化,萬事修治,故能成其大。}


\section{仁德第三十五}

執大象,天下往。
\zhushi{執,守也。象,道也。聖人守大道,則天下萬民移心歸往之也。治身則天降神明,往來於己也。}
往而不害,安、平、太。
\zhushi{萬物歸往而不傷害,則國家安寕而致太平矣。治身不害神明,則身安而大壽也。}
樂與餌,過客止,
\zhushi{餌,美也。過客,一也。人能樂美於道,則一留止也。一者,去盈而處虛,忽忽如過客。}
道之出口,淡乎其無味,
\zhushi{道出入於口,淡淡非如五味有酸鹹苦甘辛也。}
視之不足見,
\zhushi{足,德也。道無形,非若五色有青黃赤白黑可得見也。}
聽之不足聞,
\zhushi{道非若五音有宮商角徵羽可得聽聞也。}
用之不足既。
\zhushi{用道治國,則國安民昌。治身則壽命延長,無有既盡時也。}


\section{微明第三十六}

將欲歙之,必固張之。
\zhushi{先開張之者,欲極其奢淫。}
將欲弱之,必固強之。
\zhushi{先強大之者,欲使遇禍患。}
將欲廢之,必固興之。
\zhushi{先興之者,欲使其驕危。}
將欲奪之,必固與之。
\zhushi{先與之者,欲極其貪心。}
是謂微明。
\zhushi{此四事,其道微,其效明也。}
柔弱勝剛強。
\zhushi{柔弱者久長,剛強者先亡也。}
魚不可脫於淵,
\zhushi{魚脫於淵,謂去剛得柔,不可復制焉。}
國之利器,不可以示人。
\zhushi{利器者,謂權道也。治國權者,不可以示執事之臣也。治身道者,不可以示非其人也。}


\section{為政第三十七}

道常無為而無不為。
\zhushi{道以無為為常也。}
侯王若能守之,萬物將自化。
\zhushi{言侯王若能守道,萬物將自化效於己也。}
化而欲作,吾將鎮之以無名之樸。
\zhushi{吾,身也。無明之樸,道德也。萬物已化效於己也。復欲作巧偽者,侯王當身鎮撫以道德也。}
無名之樸,夫亦將無欲。不欲以靜,
\zhushi{言侯王鎮撫以道德,民亦將不欲,故當以清靜導化之也。}
天下將自定。
\zhushi{能如是者,天下將自正定也。}


\section{論德第三十八}

上德不德,
\zhushi{上德,謂太古無名號之君,德大無上,故言上德也。不德者,言其不以德教民,因循自然,養人性命,其德不見,故言不德也。}
是以有德。
\zhushi{言其德合於天地,和氣流行,民德以全也。}
下德不失德,
\zhushi{下德,謂號謚之君,德不及上德,故言下德也。不失德者,其德可見,其功可稱也。}
是以無德。
\zhushi{以有名號及其身故。}
上德無為
\zhushi{謂法道安靜,無所施為也。}
而無以為,
\zhushi{言無以名號為也。}
下德為之
\zhushi{言為教令,施政事也。}
而有以為。
\zhushi{言以為己取名號也}
上仁為之
\zhushi{上仁謂行仁之君,其仁無上,故言上仁。為之者,為人恩也。}
而無以為,
\zhushi{功成事立,無以執為。}
上義為之
\zhushi{為義以斷割也。}
而有以為。
\zhushi{動作以為己,殺人以成威,賊下以自奉也。}
上禮為之
\zhushi{謂上禮之君,其禮無上,故言上禮。為之者,言為禮制度,序威儀也。}
而莫之應,
\zhushi{言禮華盛實衰,飾偽煩多,動則離道,不可應也。}
則攘臂而扔之。
\zhushi{言禮煩多不可應,上下忿爭,故攘臂相仍引。}
故失道而後德,
\zhushi{言道衰而德化生也。}
失德而後仁,
\zhushi{言德衰而仁愛見也。}
失仁而後義,
\zhushi{言仁衰而分義明也。}
失義而後禮。
\zhushi{言義衰則失禮聘,行玉帛也。}
夫禮者,忠信之薄
\zhushi{言禮廢本治末,忠信日以衰薄。}
而亂之首。
\zhushi{禮者賤質而貴文,故正直日以少,邪亂日以生。}
前識者,道之華
\zhushi{不知而言知為前識,此人失道之實,得道之華。}
而愚之始。
\zhushi{言前識之人,愚暗之倡始也。}
是以大丈夫處其厚,
\zhushi{大丈夫謂得道之君也。處其厚者,謂處身於敦樸。}
不居其薄,
\zhushi{不處身違道,為世煩亂也。}
處其實,
\zhushi{處忠信也。}
不居其華。
\zhushi{不尚華言也。}
故去彼取此。
\zhushi{去彼華薄,取此厚實。}


\section{法本第三十九}


昔之得一者:
\zhushi{昔,往也。一,無為,道之子也。}
天得一以清,
\zhushi{言天得一故能垂象清明。}
地得一以寧,
\zhushi{言地得一故能安靜不動搖。}
神得一以靈,
\zhushi{言神得一故能變化無形。}
谷得一以盈,
\zhushi{言谷得一故能盈滿而不絕也}
萬物得一以生,
\zhushi{言萬物皆須道以生成也。}
侯王得一以為天下貞。
\zhushi{言侯王得一故能為天下平正}
其致之。
\zhushi{致,誡也。謂下六事也。}
天無以清將恐裂,
\zhushi{言天當有陰陽弛張,晝夜更用,不可但欲清明無已時,將恐分裂不為天。}
地無以寧將恐發,
\zhushi{言地當有高下剛柔,節氣五行,不可但欲安靜無已時,將恐發泄不為地。}
神無以靈將恐歇,
\zhushi{言神當有王相囚死休廢,不可但欲靈變無已時,將恐虛歇不為神。}
谷無以盈將恐竭,
\zhushi{言谷當有盈縮虛實,不可但欲盈滿無已時,將恐枯竭不為谷。}
萬物無以生將恐滅,
\zhushi{言萬物當隨時生死,不可但欲長生無已時,將恐滅亡不為物。}
侯王無以貴高將恐蹶。
\zhushi{言侯王當屈己以下人,汲汲求賢,不可但欲貴高於人無已時,將恐顛蹶失其位。}
故貴以賤為本,
\zhushi{言必欲尊貴,當以薄賤為本,若禹稷躬稼,舜陶河濱,周公下白屋也。}
高以下為基
\zhushi{言必欲尊貴,當以下為本基,猶築墻造功,因卑成高,下不堅固,後必傾危。}
是以侯王自謂孤、寡、不轂。
\zhushi{孤寡喻孤獨,不轂喻不能如車轂為眾輻所湊。}
此非以賤為本邪?
\zhushi{言侯王至尊貴,能以孤寡自稱,此非以賤為本乎,以曉人?}
非乎!
\zhushi{嗟嘆之辭。}
故致數輿無輿,
\zhushi{致,就也。言人就車數之為輻、為輪、為轂、為衡、為輿,無有名為車者,故成為車,以喻侯王不以尊號自名,故能成其貴。}
不欲琭琭如玉,珞珞如石。
\zhushi{琭琭喻少,落落喻多,玉少故見貴,石多故見賤。言不欲如玉為人所貴,如石為人所賤,當處其中也。}

\section{去用第四十}

反者道之動,
\zhushi{反,本也。本者,道之所以動,動生萬物,背之則亡也。}
弱者道之用。
\zhushi{柔弱者,道之所常用,故能常久。}
天下萬物生於有,
\zhushi{天下萬物皆從天地生,天地有形位,故言生於有也。}
有生於無。
\zhushi{天地神明,蜎飛蠕動,皆從道生。道無形,故言生於無也。此言本勝於華,弱勝於強,謙虛勝盈滿也。}


\section{同異第四十一}

章上士聞道,勤而行之。
\zhushi{上士聞道,自勤苦竭力而行之。}
中士聞道,若存若亡。
\zhushi{中士聞道,治身以長存,治國以太平,欣然而存之,退見財色榮譽,惑於情欲,而復亡之也。}
下士聞道,大笑之。
\zhushi{下士貪狠多欲,見道柔弱,謂之恐懼,見道質樸,謂之鄙陋,故大笑之。}
不笑不足以為道。
\zhushi{不為下士所笑,不足以名為道。}
故建言有之:
\zhushi{建,設也。設言以有道,當如下句。}
明道若昧,
\zhushi{明道之人,若暗昧無所見。}
進道若退,
\zhushi{進取道者,若退不及。}
夷道若纇。
\zhushi{夷,平也。大道之人不自別殊,若多比類也。}
上德若谷,
\zhushi{上德之人若深谷,不恥垢濁也。}
大白若辱,
\zhushi{大潔白之人若汙辱,不自彰顯。}
廣德若不足,
\zhushi{德行廣大之人,若愚頑不足也。}
建德若偷,
\zhushi{建設道德之人,若可偷引使空虛也。}
質真若渝,
\zhushi{質樸之人,若五色有渝淺不明也。}
大方無隅,
\zhushi{大方正之人,無委屈廉隅。}
大器晚成,
\zhushi{大器之人,若九鼎瑚璉,不可卒成也。}
大音希聲,
\zhushi{大音猶雷霆待時而動,喻當愛氣希言也。}
大象無形,
\zhushi{大法象之人,質樸無形容。}
道隱無名。
\zhushi{道潛隱,使人無能指名也。}
夫惟道,善貸且成。
\zhushi{成,就也。言道善稟貸人精氣,且成就之也。}


\section{道化第四十二}

道生一,
\zhushi{道使所生者一也。}
一生二,
\zhushi{一生陰與陽也。}
二生三,
\zhushi{陰陽生和、清、濁三氣,分為天地人也。}
三生萬物。
\zhushi{天地人共生萬物也,天施地化,人長養之也。}
萬物負陰而抱陽,
\zhushi{萬物無不負陰而向陽,回心而就日。}
沖氣以為和。
\zhushi{萬物中皆有元氣,得以和柔,若胸中有藏,骨中有髓,草木中有空虛與氣通,故得久生也。}
人之所惡,惟孤、寡、不谷,而王公以為稱。
\zhushi{孤寡不轂者,不祥之名,而王公以為稱者,處謙卑,法空虛和柔。}
故物或損之而益,
\zhushi{引之不得,推之必還。}
或益之而損。
\zhushi{夫增高者誌崩,貪富者致患。}
人之所教,
\zhushi{謂眾人所教,去弱為強,去柔為剛。}
我亦教之。
\zhushi{言我教眾人,使去強為弱,去剛為柔。}
強梁者不得其死,
\zhushi{強粱者,謂不信玄妙,背叛道德,不從經教,尚勢任力也。不得其死者,為天命所絕,兵刃所伐,王法所殺,不得以壽命死。}
吾將以為教父。
\zhushi{父,始也。老子以強梁之人為教,誡之始也。}


\section{偏用第四十三}

天下之至柔,馳騁天下之至堅。
\zhushi{至柔者,水也。至堅者,金石也。水能貫堅入剛,無所不通。}
無有入無間。
\zhushi{無有謂道也。道無形質,故能出入無間,通神明濟群生也。}
吾是以知無為之有益。
\zhushi{吾見道無為而萬物自化成,是以知無為之有益於人也。}
不言之教,
\zhushi{法道不言,師之以身。}
無為之益,
\zhushi{法道無為,治身則有益於精神,治國則有益於萬民,不勞煩也。}
天下希及之。
\zhushi{天下,人主也。希能有及道無為之治身治國也。}


\section{立戒第四十四}

名與身孰親。
\zhushi{名遂則身退也。}
身與貨孰多。
\zhushi{財多則害身也。}
得與亡孰病。
\zhushi{好得利則病於行也。}
甚愛必大費,
\zhushi{甚愛色,費精神。甚愛財,遇禍患。所愛者少,所亡者多,故言大費。}
多藏必厚亡。
\zhushi{生多藏於府庫,死多藏於丘墓。生有攻劫之憂,死有掘冢探柩之患。}
知足不辱,
\zhushi{知足之人絕利去欲,不辱於身。}
知止不殆,
\zhushi{知可止,則財利不累於身,聲色不亂於耳目,則身不危殆也。}
可以長久。
\zhushi{人能知止足則福祿在己,治身者,神不勞;治國者,民不擾,故可長久。}


\section{洪德第四十五}

大成若缺,
\zhushi{謂道德大成之君也。若缺者,滅名藏譽,如毀缺不備也。}
其用不弊,
\zhushi{其用心如是,則無敝盡時也。}
大盈若沖,
\zhushi{謂道德大盈滿之君也。若沖者,貴不敢驕也,富不敢奢也。}
其用不窮。
\zhushi{其用心如是,則無窮盡時也。}
大直若屈,
\zhushi{大直,謂修道法度正直如一也。若屈者,不與俗人爭,若可屈折。}
大巧若拙,
\zhushi{大巧謂多才術也。若拙者,亦不敢見其能。}
大辯若訥。
\zhushi{大辯者,智無疑。若訥者,口無辭。}
躁勝寒,
\zhushi{勝,極也。春夏陽氣躁疾於上,萬物盛大,極則寒,寒則零落死亡也。言人不當剛躁也。}
靜勝熱,
\zhushi{秋冬萬物靜於黃泉之下,極則熱,熱者生之源。}
清靜能為天下正。
\zhushi{能清靜則為天下之長,持身正則無終已時也。}


\section{儉欲第四十六}

天下有道,
\zhushi{謂人主有道也。}
卻走馬以糞,
\zhushi{糞者,糞田也。兵甲不用,卻走馬治農田,治身者卻陽精以糞其身。}
天下無道,
\zhushi{謂人主無道也。}
戎馬生於郊。
\zhushi{戰伐不止,戎馬生於郊境之上,久不還也。}
罪莫大於可欲。
\zhushi{好淫色也。}
禍莫大於不知足,
\zhushi{富貴不能自禁止也。}
咎莫大於欲得。
\zhushi{欲得人物,利且貪也。}
故知足之足,
\zhushi{守真根也。}
常足。
\zhushi{無欲心也。}


\section{鑒遠第四十七}

不出戶知天下,
\zhushi{聖人不出戶以知天下者,以己身知人身,以己家知人家,所以見天下也。}
不窺牖見天道,
\zhushi{天道與人道同,天人相通,精氣相貫。人君清凈,天氣自正,人君多欲,天氣煩濁。吉兇利害,皆由於己。}
其出彌遠,其知彌少。
\zhushi{謂去其家觀人家,去其身觀人身,所觀益遠,所見益少也。}
是以聖人不行而知,
\zhushi{聖人不上天,不入淵,能知天下者,以心知之也。}
不見而名,
\zhushi{上好道,下好德;上好武,下好力。聖人原小知大,察內知外。}
不為而成。
\zhushi{上無所為,則下無事,家給人足,萬物自化就也。}


\section{忘知第四十八}

為學日益,
\zhushi{學謂政教禮樂之學也。日益者,情欲文飾日以益多。}
為道日損。
\zhushi{道謂之自然之道也。日損者,情欲文飾日以消損。}
損之又損,
\zhushi{損情欲也。又損之,所以漸去。}
以至於無為,
\zhushi{當恬淡如嬰兒,無所造為也。}
無為而無不為。
\zhushi{情欲斷絕,德於道合,則無所不施,無所不為也。}
取天下常以無事,
\zhushi{取,治也。治天下當以無事,不當以勞煩也。}
及其有事,不足以取天下。
\zhushi{及其好有事,則政教煩,民不安,故不足以治天下也。}


\section{任德四十九}

聖人無常心,
\zhushi{聖人重改更,貴因循,若自無心。}
以百姓心為心。
\zhushi{百姓心之所便,聖人因而從之。}
善者吾善之,
\zhushi{百姓為善,聖人因而善之。}
不善者吾亦善之,
\zhushi{百姓雖有不善者,聖人化之使善也。}
德善。
\zhushi{百姓德化,聖人為善}
信者吾信之,
\zhushi{百姓為信,聖人因而信之。}
不信者吾亦信之,
\zhushi{百姓為不信,聖人化之為信者也。}
德信。
\zhushi{百姓德化,聖人以為信。}
聖人在天下怵怵,
\zhushi{聖人在天下怵怵常恐怖,富貴不敢驕奢。}
為天下渾其心。
\zhushi{言聖人為天下百姓混濁其心,若愚暗不通也。}
百姓皆註其耳目,
\zhushi{註,用也。百姓皆用其耳目為聖人視聽也。}
聖人皆孩之。
\zhushi{聖人愛念百姓如嬰孩赤子,長養之而不責望其報。}


\section{貴生第五十}

出生入死。
\zhushi{出生,謂情欲出五內,魂靜魄定,故生。入死,謂情欲入於胸臆,精勞神惑,故死。}
生之徒十有三,死之徒死十有三,
\zhushi{言生死之類各有十三,謂九竅四關也。其生也目不妄視,耳不妄聽,鼻不妄嗅,口不妄言,味,手不妄持,足不妄行,精神不妄施。其死也反是也。}
人之生,動之死地十有三。
\zhushi{人知求生,動作反之十三死也。}
夫何故,
\zhushi{問何故動之死地也。}
以其求生之厚。
\zhushi{所以動之死地者,以其求生活之事太厚,違道忤天,妄行失紀。}
蓋以聞善攝生者,
\zhushi{攝,養也。}
路行不遇兕虎,
\zhushi{自然遠離,害不幹也。}
入軍不披甲兵,
\zhushi{不好戰以殺人。}
兕無投其角,虎無所措爪,兵無所容其刃。
\zhushi{養生之人,兕虎無由傷,兵刃無從加之也。}
夫何故,
\zhushi{問兕虎兵甲何故不加害之。}
以其無死地。
\zhushi{以其不犯十三之死地也。言神明營護之,此物不敢害。}


\section{養德第五十一}

道生之,
\zhushi{道生萬物。}
德畜之,
\zhushi{德,一也。一主布氣而蓄養}
物形之,
\zhushi{一為萬物設形像也。}
勢成之。
\zhushi{一為萬物作寒暑之勢以成之。}
是以萬物莫不尊道而貴德。
\zhushi{道德所為,無不盡驚動,而尊敬之。}
道之尊,德之貴,夫莫之命而常自然。
\zhushi{道一不命召萬物,而常自然應之如影響。}
故道生之,德畜之,長之育之,成之孰之,養之覆之。
\zhushi{道之於萬物,非但生而已,乃復長養、成孰、覆育,全其性命。人君治國治身,亦當如是也。}
生而不有,
\zhushi{道生萬物,不有所取以為利也。}
為而不恃,
\zhushi{道所施為,不恃望其報也。}
長而不宰,
\zhushi{道長養萬物,不宰割以為利也。}
是謂玄德。
\zhushi{道之所行恩德,玄暗不可得見。}


\section{歸元第五十二}

天下有始,
\zhushi{始有道也。}
以為天下母。
\zhushi{道為天下萬物之母}
既知其母,復知其子,
\zhushi{子,一也。既知道己,當復知一也。}
既知其子,復守其母,
\zhushi{己知一,當復守道反無為也。}
沒身不殆。
\zhushi{不危殆也。}
塞其兌,
\zhushi{兌,目也。目不妄視也。}
閉其門,
\zhushi{門,口也。使口不妄言}
終身不勤。
\zhushi{人當塞目不妄視,閉口不妄言,則終生不勤苦。}
開其兌,
\zhushi{開目視情欲也。}
濟其事,
\zhushi{濟,益也。益情欲之事。}
終身不救。
\zhushi{禍亂成也。}
見小曰明,
\zhushi{萌芽未動,禍亂未見為小,昭然獨見為明。}
守柔日強。
\zhushi{守柔弱,日以強大也。}
用其光,
\zhushi{用其目光於外,視時世之利害。}
復歸其明。
\zhushi{復當返其光明於內,無使精神泄也。}
無遺身殃,
\zhushi{內視存神,不為漏失。}
是謂習常。
\zhushi{人能行此,是謂修習常道。}


\section{益證第五十三}

使我介然有知,行於大道。
\zhushi{介,大也。老子疾時王不行大道,故設此言。使我介然有知於政事,我則行於大道,躬行無為之化。}
唯施是畏。
\zhushi{唯,獨也。獨畏有所施為,恐失道意。欲賞善,恐偽善生;欲信忠恐詐忠起。}
大道甚夷,而民好徑。
\zhushi{夷,平易也。徑,邪、不平正也。大道甚平易,而民好從邪徑也。}
朝甚除,
\zhushi{高臺榭,宮室修。}
田甚蕪,
\zhushi{農事廢,不耕治。}
倉甚虛,
\zhushi{五谷傷害,國無儲也。}
服文彩,
\zhushi{好飾偽,貴外華。}
帶利劍,
\zhushi{尚剛強,武且奢。}
厭飲食,財貨有餘,
\zhushi{多嗜欲,無足時。}
是謂盜誇。
\zhushi{百姓而君有餘者,是由劫盜以為服飾,持行誇人,不知身死家破,親戚並隨也。}
非道哉。
\zhushi{人君所行如是,此非道也。復言也哉者,痛傷之辭。}


\section{修觀第五十四}

天下。
\zhushi{以修道之主,觀不修道之主也。}
吾何以知天下之然哉,以此。
\zhushi{老子言,吾何知天下修道者昌,背道者亡。以此五事觀而知之也。}


\section{玄符第五十五}

含德之厚,
\zhushi{謂含懷道德之厚也。}
比於赤子。
\zhushi{神明保佑含德之人,若父母之於赤子也。}
毒蟲不螫,
\zhushi{蜂蠇蛇虺不螫。}
猛獸不據,玃鳥不搏。
\zhushi{赤子不害於物,物亦不害之。故太平之世,人無貴賤,仁心,有刺之物,還返其本,有毒之蟲,不傷於人。}
骨弱筋柔而握固。
\zhushi{赤子筋骨柔弱而持物堅固,以其意心不移也。}
未知牝牡之合而峻作精之至也。
\zhushi{赤子未知男女會合而陰陽作怒者,由精氣多之所致也。}
終日號而不啞,和之至也。
\zhushi{赤子從朝至暮啼號聲不變易者,和氣多之所至也。}
知和日常,
\zhushi{人能和氣柔弱有益於人者,則為知道之常也。}
知常日明,
\zhushi{人能知道之常行,則日以明達於玄妙也。}
益生日祥,
\zhushi{祥,長也。言益生欲自生,日以長大。}
心使氣日強。
\zhushi{心當專一和柔而神氣實內,故形柔。而反使妄有所為,和氣去於中,故形體日以剛強也。}
物壯則老,
\zhushi{萬物壯極則枯老也。}
謂之不道,
\zhushi{枯老則不得道矣。}
不道早已。
\zhushi{不得道者早死。}


\section{玄德第五十六}

知者不言,
\zhushi{知者貴行不貴言也。}
言者不知。
\zhushi{駟不及舌,多言多患。}
塞其兌,閉其門,
\zhushi{塞閉之者,欲絕其源。}
挫其銳,
\zhushi{情欲有所銳為,當念道無為以挫止之。}
解其紛,
\zhushi{紛,結恨不休也。當念道恬怕以解釋之。}
和其光,
\zhushi{雖有獨見之明,當和之使暗昧,不使曜亂。}
同其塵,
\zhushi{不當自別殊也。}
是謂玄同。
\zhushi{玄,天也。人能行此上事,是謂與天同道也。}
故不可得而親,
\zhushi{不以榮譽為樂,獨立為哀。}
亦不可得而踈;
\zhushi{誌靜無欲,故與人無怨。}
不可得而利,
\zhushi{身不欲富貴,口不欲五味。}
亦不可得而害,
\zhushi{不與貪爭利,不與勇爭氣。}
不可得而貴,
\zhushi{不為亂世主,不處暗君位。}
亦不可得而賤,
\zhushi{不以乘權故驕,不以失誌故屈。}
故為天下貴。
\zhushi{其德如此,天子不得臣,諸侯不得屈,與世沈浮容身避害,故天下貴也。}


\section{淳風第五十七}

以正治國,
\zhushi{以,至也。天使正身之人,使有國也。}
以奇用兵,
\zhushi{奇,詐也。天使詐偽之人,使用兵也。}
以無事取天下。
\zhushi{以無事無為之人,使取天下為之主。}
吾何以知其然哉,以此。
\zhushi{此,今也。老子言,我何以知天意然哉,以今日所見知。}
天下多忌諱而民彌貧。
\zhushi{天下謂人主也。忌諱者防禁也。今煩則奸生,禁多則下詐,相殆故貧。}
民多利器,國家滋昏。
\zhushi{利器者,權也。民多權則視者眩於目,聽者惑於耳,上下不親,故國家昏亂。}
人多伎巧,奇物滋起。
\zhushi{人謂人君、百裏諸侯也。多技巧,謂刻畫宮觀,雕琢章服,奇物滋起,下則化上,飾金鏤玉,文繡彩色日以滋甚。}
法物滋彰,盜賊多有。
\zhushi{法物,好物也。珍好之物滋生彰著,則農事廢,饑寒並至,而盜賊多有也。}
故聖人雲:
\zhushi{謂下事也。}
我無為而民自化,
\zhushi{聖人言:我修道承天,無所改作,而民自化成也。}
我好靜而民自正,
\zhushi{聖人言:我好靜,不言不教,而民自忠正也。}
我無事而民自富,
\zhushi{我無徭役徵召之事,民安其業故皆自富也。}
我無欲而民自樸。
\zhushi{我常無欲,去華文,微服飾,民則隨我為質樸也。聖人言:我修道守真,絕去六情,民自隨我而清也。}


\section{順化第五十八}

其政悶悶,
\zhushi{其政教寬大,悶悶昧昧,似若不明也。}
其民醇醇,
\zhushi{政教寬大,故民醇醇富厚,相親睦也。}
其政察察,
\zhushi{其政教急疾,言決於口,聽決於耳也。}
其民缺缺。
\zhushi{政教急疾。民不聊生。故缺缺日以踈薄。}
禍兮福所倚,
\zhushi{倚,因也。夫福因禍而生,人遭禍而能悔過責己,修道行善,則禍去福來。}
福兮禍所伏。
\zhushi{禍伏匿於福中,人得福而為驕恣,則福去禍來。}
孰知其極,
\zhushi{禍福更相生,誰能知其窮極時。}
其無正,
\zhushi{無,不也。謂人君不正其身,其無國也。}
正復為奇,
\zhushi{奇,詐也。人君不正,下雖正,復化上為詐也。}
善復為訞。
\zhushi{善人皆復化上為訞祥也。}
人之迷,其日固久。
\zhushi{言人君迷惑失正以來,其日已固久。}
是以聖人方而不割,
\zhushi{聖人行方正者,欲以率下,不以割截人也。}
廉而不害,
\zhushi{聖人廉清,欲以化民,不以傷害人也。今則不然,正己以害人也。}
直而不肆,
\zhushi{肆,申也。聖人雖直,曲己從人,不自申也。}
光而不曜。
\zhushi{聖人雖有獨見之明,當如暗昧,不以曜亂人也。}


\section{守道第五十九}

治人,
\zhushi{謂人君治理人民。}
事天,
\zhushi{事,用也。當用天道,順四時。}
莫若嗇。
\zhushi{嗇,愛惜也。治國者當愛民財,不為奢泰。治身者當愛精氣,不為放逸。}
夫為嗇,是謂早服。
\zhushi{早,先也。服,得也。夫獨愛民財,愛精氣,則能先得天道也。}
早服謂之重積德。
\zhushi{先得天道,是謂重積得於己也。}
重積德則無不克,
\zhushi{克,勝也。重積德於己,則無不勝。}
無不克則莫知其極,
\zhushi{無不克勝,則莫知有知己德之窮極也。}
莫知其極可以有國。
\zhushi{莫知己德者有極,則可以有社稷,為民致福。}
有國之母,可以長久。
\zhushi{國身同也。母,道也。人能保身中之道,使精氣不勞,五神不苦,則可以長久。}
是謂深根固蒂,
\zhushi{人能以氣為根,以精為蒂,如樹根不深則拔,蒂不堅則落。言當深藏其氣,固守其精,使無漏泄。}
長生久視之道。
\zhushi{深根固蒂者,乃長生久視之道。}


\section{居位第六十}

治大國者若烹小鮮。
\zhushi{鮮,魚。烹小魚不去腸、不去鱗、不敢撓,恐其糜也。治國煩則下亂,治身煩則精散。}
以道蒞天下,其鬼不神。
\zhushi{以道德居位治天下,則鬼不敢以其精神犯人也。}
非其鬼不神,其神不傷人。
\zhushi{其鬼非無精神也,非不入正,不能傷自然之人。}
非其神不傷人,聖人亦不傷。
\zhushi{非鬼神不能傷害人。以聖人在位不傷害人,故鬼不敢幹之也。}
夫兩不相傷,
\zhushi{鬼與聖人俱兩不相傷也。}
故德交歸焉。
\zhushi{夫兩不相傷,則人得治於陽,鬼神得治於陰,人得保全其性命,鬼得保其精神,故德交歸焉。}


\section{謙德第六十一}

大國者下流,
\zhushi{治大國,當如居下流,不逆細微。}
天下之交,
\zhushi{大國,天下士民之所交會。}
天下之牝。
\zhushi{牝者,陰類也。柔謙和而不昌也。}
牝常以靜勝牡,
\zhushi{女所以能屈男,陰勝陽,以,安靜不先求之也。}
以靜為下。
\zhushi{陰道以安靜為謙下。}
故大國以下小國,則取小國,
\zhushi{能謙下之,則常有之。}
小國以下大國,則取大國。
\zhushi{此言國無大小,能持謙畜人,則無過失也。}
故或下以取,或下而取。
\zhushi{下者謂大國以下小國,小國以下大國,更以義相取。}
大國不過欲兼畜人,
\zhushi{大國不失下,則兼並小國而牧畜之。}
小國不過欲入事人。
\zhushi{使為臣仆。}
夫兩者各得其所欲,大者宜為下。
\zhushi{大國小國各欲得其所,大國又宜為謙下}


\section{為道第六十二}

道者萬物之奧,
\zhushi{奧,藏也。道為萬物之藏,無所不容也。}
善人之寶,
\zhushi{善人以道為身寶,不敢違也。}
不善人之所保。
\zhushi{道者,不善人之保倚也。遭患逢急,猶知自悔卑下。}
美言可以市,
\zhushi{美言者獨可於市耳。夫市交易而退,不相宜善言美語,求者欲疾得,賣者欲疾售也。}
尊行可以加入。
\zhushi{加,別也。人有尊貴之行,可以別異於凡人,未足以尊道。}
人之不善,何棄之有。
\zhushi{人雖不善,當以道化之。蓋三皇之前,無有棄民,德化淳也。}
故立天子,置三公,
\zhushi{欲使教化不善之人。}
雖有拱璧以先駟馬,不如坐進此道。
\zhushi{雖有美璧先駟馬而至,故不如坐進此道。}
古之所以貴此道者,何不日以求得?
\zhushi{古之所以貴此道者,不日日遠行求索,近得之於身。}
有罪以免耶,
\zhushi{有罪謂遭亂世,暗君妄行形誅,修道則可以解死,免於眾也。}
故為天下貴。
\zhushi{道德洞遠,無不覆濟,全身治國,恬然無為,故可為天下貴也。}


\section{恩始第六十三}

為無為,
\zhushi{因成循故,無所造作。}
事無事,
\zhushi{預有備,除煩省事也。}
味無味。
\zhushi{深思遠慮,味道意也。}
大小多少,
\zhushi{陳其戒令也。欲大反小,欲多反少,自然之道也。}
報怨以德。
\zhushi{修道行善,絕禍於未生也。}
圖難於其易,
\zhushi{欲圖難事,當於易時,未及成也。}
為大於其細。
\zhushi{欲為大事,必作於小,禍亂從小來也。}
天下難事必作於易,天下大事必作於細。
\zhushi{從易生難,從細生著。}
是以聖人終不為大,故能成其大。
\zhushi{處謙虛,天下共歸之也。}
夫輕諾必寡信,
\zhushi{不重言也。}
多易必多難。
\zhushi{不慎患也。}
是以聖人猶難之,
\zhushi{聖人動作舉事,猶進退,重難之,欲塞其源也。}
故終無難矣。
\zhushi{聖人終生無患難之事,猶避害深也}


\section{守微第六十四}

其安易持,
\zhushi{治身治國安靜者,易守持也。}
其未兆易謀,
\zhushi{情欲禍患未有形兆時,易謀止也。}
其脆易破,
\zhushi{禍亂未動於朝,情欲未見於色,如脆弱易破除。}
其微易散。
\zhushi{其未彰著,微小易散去也。}
為之於未有,
\zhushi{欲有所為,當於未有萌芽之時塞其端也。}
治之於未亂。
\zhushi{治身治國於未亂之時,當豫閉其門也。}
合抱之木生於毫末;
\zhushi{從小成大。}
九層之臺起於累土;
\zhushi{從卑立高。}
千裏之行始於足下。
\zhushi{從近至遠。}
為者敗之,
\zhushi{有為於事,廢於自然;有為於義,廢於仁;有為於色,廢於精神也。}
執者失之。
\zhushi{執利遇患,執道全身,堅持不得,推讓反還。}
是以聖人無為故無敗,
\zhushi{聖人不為華文,不為色利,不為殘賊,故無敗壞。}
無執故無失。
\zhushi{聖人有德以教愚,有財以與貧,無所執藏,故無所失於人也。}
民之從事,常於幾成而敗之。
\zhushi{從,為也。民之為事,常於功德幾成,而貪位好名,奢泰盈滿而自敗之也。}
慎終如始,則無敗事。
\zhushi{終當如始,不當懈怠。}
是以聖人欲不欲,
\zhushi{聖人欲人所不欲。人欲彰顯,聖人欲伏光;人欲文飾,聖人欲質樸;人欲色,聖人欲於德。}
不貴難得之貨;
\zhushi{聖人不眩為服,不賤石而貴玉。}
學不學,
\zhushi{聖人學人所不能學。人學智詐,聖人學自然;人學治世,聖人學治身;守道真也。}
復眾人之所過;
\zhushi{眾人學問反,過本為末,過實為華。復之者,使反本也。}
以輔萬物之自然。
\zhushi{教人反本實者,欲以輔助萬物自然之性也。}
而不敢為。
\zhushi{聖人動作因循,不敢有所造為,恐遠本也。}


\section{淳德第六十五}

古之善為道者,非以明民,將以愚之。
\zhushi{說古之善以道治身及治國者,不以道教民明智巧詐也,將以道德教民,使質樸不詐偽。}
民之難治,以其智多。
\zhushi{民之所以難治者,以其智多而為巧偽。}
故以智治國,國之賊;
\zhushi{使智慧之人治國之政事,必遠道德,妄作威福,為國之賊也。}
不以智治國,國之福。
\zhushi{不使智慧之人治國之政事,則民守正直,不為邪飾,上下相親,君臣同力,故為國之福也。}
知此兩者亦稽式。
\zhushi{兩者謂智與不智也。常能智者為賊,不智者為福,是治身治國之法式也。}
常知稽式,是謂玄德。
\zhushi{玄,天也。能知治身及治國之法式,是謂與天同德也。}
玄德深矣,遠矣,
\zhushi{玄德之人深不可測,遠不可及也。}
與物反矣!
\zhushi{玄德之人與萬物反異,萬物欲益己,玄德施與人也。}
然後乃至於大順。
\zhushi{玄德與萬物反異,故能至大順。順天理也。}


\section{後己第六十六}

江海所以能為百谷王者,以其善下之,故能為百谷王。
\zhushi{江海以卑,故眾流歸之,若民歸就王。以卑下,故能為百谷王也。}
是以欲上民,
\zhushi{欲在民之上也。}
必以言下之;
\zhushi{法江海處謙虛。}
欲先民,
\zhushi{欲在民之前也。}
必以身後之。
\zhushi{先人而後己也。}
是以聖人處上而民不重,
\zhushi{聖人在民上為主,不以尊貴虐下,故民戴而不為重。}
處前而民不害。
\zhushi{聖人在民前,不以光明蔽後,民親之若父母,無有欲害之心也。}
是以天下樂推而不厭。
\zhushi{聖人恩深愛厚,視民如赤子,故天下樂推進以為主,無有厭也。}
以其不爭,
\zhushi{天下無厭聖人時,是由聖人不與人爭先後也。}
故天下莫能與之爭。
\zhushi{言人皆有為,無有與吾爭無為。}


\section{三寶第六十七}

天下皆謂我大,似不肖。
\zhushi{老子言:天下謂我德大,我則佯愚似不肖。}
夫唯大,故似不肖,
\zhushi{唯獨名德大者為身害,故佯愚似若不肖。無所分別,無所割截,不賤人而自責。}
若肖久矣。
\zhushi{肖,善也。謂辨惠也。若大辨惠之人,身高自貴行察察之政所從來久矣。}
其細也夫。
\zhushi{言辨惠者唯如小人,非長者。}
我有三寶,持而保之。
\zhushi{老子言:我有三寶,抱持而保倚。}
一曰慈,
\zhushi{愛百姓若赤子。}
二曰儉,
\zhushi{賦斂若取之於己也。}
三曰不敢為天下先。
\zhushi{執謙退,不為倡始也。}
慈故能勇,
\zhushi{以慈仁,故能勇於忠孝也。}
儉故能廣,
\zhushi{天子身能節儉,故民日用廣矣。}
不敢為天下先,
\zhushi{不為天下首先。}
故能成器長。
\zhushi{成器長,謂得道人也。我能為得道人之長也。}
今舍慈且勇,
\zhushi{今世人舍慈仁,但為勇武。}
舍儉且廣,
\zhushi{舍其儉約,但為奢泰。}
舍後且先,
\zhushi{舍其後己,但為人先。}
死矣!
\zhushi{所行如此,動入死地。}
夫慈以戰則勝,以守則固。
\zhushi{夫慈仁者,百姓親附,並心一意,故以戰則勝敵,以守衛則堅固。}
天將救之,以慈衛之。
\zhushi{天將救助善人,必與慈仁之性,使能自營助也。}


\section{配天第六十八}

善為士者不武,
\zhushi{言貴道德,不好武力也。}
善戰者不怒,
\zhushi{善以道戰者,禁邪於胸心,絕禍於未萌,無所誅怒也。}
善勝敵者不與,
\zhushi{善以道勝敵者,附近以仁,來遠以德,不與敵爭,而敵自服也。}
善用人者為之下。
\zhushi{善用人自輔佐者,常為人執謙下也。}
是謂不爭之德,
\zhushi{謂上為之下也。是乃不與人爭之道德也。}
是謂用人之力,
\zhushi{能身為人下,是謂用人臣之力也。}
是謂配天古之極。
\zhushi{能行此者,德配天也。是乃古之極要道也。}


\section{玄用第六十九}

用兵有言:
\zhushi{陳用兵之道。老子疾時用兵,故托己設其義也。}
吾不敢為主而為客,
\zhushi{主,先也。不敢先舉兵。客者,和而不倡。用兵當承天而後動。}
不敢進寸而退尺。
\zhushi{侵人境界,利人財寶,為進;閉門守城,為退。}
是謂行無行,
\zhushi{彼遂不止,為天下賊,雖行誅之,不成行列也。}
攘無臂,
\zhushi{雖欲大怒,若無臂可攘也。}
扔無敵,
\zhushi{雖欲仍引之,若無敵可仍也。}
執無兵。
\zhushi{雖欲執持之,若無兵刃可持用也。何者?傷彼之民罹罪於天,遭不道之君,湣忍喪之痛也。}
禍莫大於輕敵。
\zhushi{夫禍亂之害,莫大於欺輕敵家,侵取不休,輕戰貪財也。}
輕敵,幾喪吾寶。
\zhushi{幾,近也。寶,身也。欺輕敵者,近喪身也。}
故抗兵相加,
\zhushi{兩敵戰也。}
哀者勝矣。
\zhushi{哀者慈仁,士卒不遠於死。}


\section{知難第七十}

章吾言甚易知,甚易行。
\zhushi{老子言:吾所言省而易知,約而易行。}
天下莫能知,莫能行。
\zhushi{人惡柔弱,好剛強也。}
言有宗,事有君。
\zhushi{我所言有宗祖根本,事有君臣上下,世人不知者,非我之無德,心與我之反也。}
夫唯無知,是以不我知。
\zhushi{夫唯世人之無知者,是我德之暗,不見於外,窮微極妙,故無知也。}
知我者希,則我者貴。
\zhushi{希,少也。唯達道者乃能知我,故為貴也。}
是以聖人被褐懷玉。
\zhushi{被褐者薄外,懷玉者厚內,匿寶藏德,不以示人也。}


\section{知病第七十一}

知不知上,
\zhushi{知道言不知,是乃德之上。,}
不知知病。
\zhushi{不知道言知,是乃德之病。}
夫唯病病,是以不病。
\zhushi{夫唯能病苦眾人有強知之病,是以不自病也。}
聖人不病,以其病病,是以不病。
\zhushi{聖人無此強知之病者,以其常苦眾人有此病,以此非人,故不自病。夫聖人懷通達之知,托於不知者,欲使天下質樸忠正,各守純性。小人不知道意,而妄行強知之事以自顯著,內傷精神,減壽消年也。}


\section{愛己第七十二}

民不畏威,則大威至。
\zhushi{威,害也。人不畏小害則大害至。大害者,謂死亡也。畏之者當愛精神,承天順地也。}
無狹其所居,
\zhushi{謂心居神,當寬柔,不當急狹也。}
無厭其所生,
\zhushi{人所以生者,以有精神。托空虛,喜清靜,飲食不節,忽道念色,邪僻滿腹,為伐本厭神也。}
夫唯不厭,是以不厭。
\zhushi{夫唯獨不厭精神之人,洗心濯垢,恬泊無欲,則精神居之不厭也。}
是以聖人自知,不自見,
\zhushi{自知己之得失,不自顯見德美於外,藏之於內。}
自愛不自貴。
\zhushi{自愛其身以保精氣,不自貴高榮名於世。}
故去彼取此。
\zhushi{去彼自見、自貴,取此自知、自愛。}


\section{任為第七十三}

勇於敢則殺,
\zhushi{勇敢有為,則殺其身。}
勇於不敢則活。
\zhushi{勇於不敢有為,則活其身。}
此兩者,
\zhushi{謂敢與不敢也。}
或利或害,
\zhushi{活身為利,殺身為害。}
天之所惡。
\zhushi{惡有為也。}
孰知其故?
\zhushi{誰能知天意之故而不犯?}
是以聖人猶難之。
\zhushi{言聖人之明德猶難於勇敢,況無聖人之德而欲行之乎?}
天之道,不爭而善勝,
\zhushi{天不與人爭貴賤,而人自畏之。}
不言而善應,
\zhushi{天不言,萬物自動以應時。}
不召而自來,
\zhushi{天不呼召,萬物皆負陰而向陽。}
繟然而善謀。
\zhushi{繟,寬也。天道雖寬博,善謀慮人事,修善行惡,各蒙其報也。}
天網恢恢,疏而不失。
\zhushi{天所網羅恢恢甚大,雖疏遠,司察人善惡,無有所失。}


\section{制惑第七十四}

民不畏死,
\zhushi{治國者刑罰酷深,民不聊生,故不畏死也。治身者嗜欲傷神,貪財殺身,民不知畏之也。}
奈何以死懼之?
\zhushi{人君不寬刑罰,教民去情欲,奈何設刑法以死懼之?}
若使民常畏死,
\zhushi{當除己之所殘克,教民去利欲也。}
而為奇者,吾得執而殺之。孰敢?
\zhushi{以道教化而民不從,反為奇巧,乃應王法執而殺之,誰敢有犯者?老子疾時王不先道德化之,而先刑罰也。}
常有司殺者。
\zhushi{司殺者,謂天居高臨下,司察人過。天網恢恢,疏而不失也。}
夫代司殺者,是謂代大匠斫。
\zhushi{天道至明,司殺有常,猶春生夏長,秋收冬藏,鬥杓運移,以節度行之。人君欲代殺之,是猶拙夫代大匠斫木,勞而無功也。}
夫代大匠斫者,希有不傷手矣。
\zhushi{人君行刑罰,猶拙夫代大匠斫,則方圓不得其理,還自傷。代天殺者,失紀綱,不得其紀綱還受其殃也。}


\section{貪損第七十五}

民之饑,以其上食稅之多,是以饑。
\zhushi{人民所以饑寒者,以其君上稅食下太多,民皆化上為貪,叛道違德,故饑。}
民之難治,以其上之有為,是以難治。
\zhushi{民之不可治者,以其君上多欲,好有為也。是以其民化上有為,情偽難治。}
民之輕死,以其上求生之厚,
\zhushi{人民所以侵犯死者,以其求生活之道太厚,貪利以自危。}
是以輕死。
\zhushi{以求生太厚之故,輕入死地也。}
夫唯無以生為者,是賢於貴生。
\zhushi{夫唯獨無以生為務者,爵祿不幹於意,財利不入於身,天子不得臣,諸侯不得使,則賢於貴生也。}


\section{戒強第七十六}

人之生也柔弱,
\zhushi{人生含和氣,抱精神。故柔弱也。}
其死也堅強。
\zhushi{人死和氣竭,精神亡,故堅強也。}
萬物草木之生也柔脆,
\zhushi{和氣存也。}
其死也枯槁。
\zhushi{和氣去也。}
故堅強者死之徒,柔弱者生之徒。
\zhushi{以上二事觀之,知堅強者死,柔弱者生也。}
是以兵強則不勝,
\zhushi{強大之兵輕戰樂殺,毒流怨結,眾弱為一強,故不勝。}
木強則共。
\zhushi{本強大則枝葉共生其上。}
強大處下,柔弱處上。
\zhushi{興物造功,大木處下,小物處上。天道抑強扶弱,自然之效。}


\section{天道第七十七}

天之道,其猶張弓乎!
\zhushi{天道暗昧,舉物類以為喻也。}
髙者抑之,下者舉之,有餘者損之,不足者與之。
\zhushi{言張弓和調之,如是乃可用。夫抑髙舉下,損強益弜,天之道也。}
天之道,損有餘而補不足。
\zhushi{天道損有餘而益謙,常以中和為上。}
人之道則不然,損不足以奉有餘。
\zhushi{人道則與天道反,世俗之人損貧以奉富,奪弱以益強也。}
孰能有餘以奉天下?唯有道者。
\zhushi{言誰能居有餘之位,自省爵祿以奉天下不足者乎?唯有道之君能行也。}
是以聖人為而不恃,
\zhushi{聖人為德施,不恃其報也。}
功成而不處,
\zhushi{功成事就,不處其位。}
其不欲見賢。
\zhushi{不欲使人知己之賢,匿功不居榮,畏天損有餘也。}


\section{任信第七十八}

天下柔弱莫過於水,
\zhushi{圓中則圓,方中則方,壅之則止,決之則行。}
而攻堅強者莫知能勝,
\zhushi{水能懷山襄陵,磨鐵消銅,莫能勝水而成功也。}
其無以易之。
\zhushi{夫攻堅強者,無以易於水。}
弱之勝強,
\zhushi{水能滅火,陰能消陽。}
柔之勝剛,
\zhushi{舌柔齒剛,齒先舌亡。}
天下莫不知,
\zhushi{知柔弱者久長,剛強者折傷。}
莫能行。
\zhushi{恥謙卑,好強梁。}
故聖人雲:
\zhushi{謂下事也。}
受國之垢,是謂社稷主;
\zhushi{人君能受國之垢濁者,若江海不逆小流,則能長保其社稷,為一國之君主也。}
受國不祥,是謂天下王。
\zhushi{人君能引過自與,代民受不祥之殃,則可以王天下。}
正言若反。
\zhushi{此乃正直之言,世人不知,以為反言。}


\section{任契第七十九}

和大怨,
\zhushi{殺人者死,傷人者刑,以相和報。}
必有餘怨,
\zhushi{任刑者失人情,必有餘怨及於良人也。}
安可以為善?
\zhushi{言一人,則先天心,安可以和怨為善?}
是以聖人執左契而不責於人。
\zhushi{古者聖人執左契,合符信也。無文書法律,刻契合符以為信也。但刻契為信,不責人以他事也。}
有德司契,
\zhushi{有德之君,司察契信而已。}
無德司徹。
\zhushi{無德之君,背其契信,司人所失。}
天道無親,常與善人。
\zhushi{天道無有親疏,唯與善人,則與司契同也。}


\section{獨立第八十}

小國寡民,
\zhushi{聖人雖治大國,猶以為小,儉約不奢泰。民雖眾,猶若寡少,不敢勞之也。}
使有什伯人之器而不用,
\zhushi{使民各有部曲什伯,貴賤不相犯也。器謂農人之器。而不用,不徵召奪民良時也。}
使民重死而不遠徙。
\zhushi{君能為民興利除害,各得其所,則民重死而貪生也。政令不煩則民安其業,故不遠遷徙離其常處也。}
雖有舟輿,無所乘之;
\zhushi{清靜無為,不作煩華,不好出入遊娛也。}
雖有甲兵,無所陳之。
\zhushi{無怨惡於天下。}
使民復結繩而用之,
\zhushi{去文反質,信無欺也。}
甘其食,
\zhushi{甘其蔬食,不漁食百姓也。}
美其服,
\zhushi{美其惡衣,不貴五色。}
安其居,
\zhushi{安其茅茨,不好文飾之屋。}
樂其俗。
\zhushi{樂其質樸之俗,不轉移也。}
鄰國相望,雞犬之聲相聞,
\zhushi{相去近也。}
民至老不相往來。
\zhushi{其無情欲。}


\section{顯質第八十一}

信言不美,
\zhushi{信者,如其實也。不美者,樸且質也。}
美言不信。
\zhushi{美言者,滋美之華辭。不信者,飾偽多空虛也。}
善者不辯,
\zhushi{善者,以道修身也。不彩文也。}
辯者不善。
\zhushi{辯者,謂巧言也。不善者,舌致患也。山有玉,掘其山;水有珠,濁其淵;辯口多言,亡其身。}
知者不博,
\zhushi{知者,謂知道之士。不博者,守一元也。}
博者不知。
\zhushi{博者,多見聞也。不知者,失要真也。}
聖人不積,
\zhushi{聖人積德不積財,有德以教愚,有財以與貧也。}
既以為人己愈有,
\zhushi{既以為人施設德化,己愈有德。}
既以與人己愈多。
\zhushi{既以財賄布施與人,而財益多,如日月之光,無有盡時。}
天之道,利而不害;
\zhushi{天生萬物,愛育之,令長大,無所傷害也。}
聖人之道,為而不爭。
\zhushi{聖人法天所施為,化成事就,不與下爭功名,故能全其聖功也。}


\end{document}
