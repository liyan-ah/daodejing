\section{經脈第十}
雷公問於黃帝曰:禁脈之言,凡刺之理,經脈為始,營其所行,制其度量,內次五藏,外別六府,願盡聞其道。黃帝曰:人始生,先成精,精成而腦髓生,骨為干,脈為營,筋為剛,肉為牆,皮膚堅而毛髮長,谷入於胃,脈道以通,血氣乃行。雷公曰:願卒聞經脈之始生。黃帝曰:經脈者,所以能決死生,處百病,調虛實,不可不通。

肺手太陰之脈,起於中焦,下絡大腸,還循胃口,上膈屬肺,從肺系橫出腋下,下循臑內,行少陰心主之前,下肘中,循臂內上骨下廉,入寸口,上魚,循魚際,出大指之端;其支者,從腕後直出次指內廉,出其端。是動則病肺脹滿膨膨而喘咳,缺盆中痛,甚則交兩手而瞀,此為臂厥。是主肺所生病者,咳,上氣喘渴,煩心胸滿,臑臂內前廉痛厥,掌中熱。氣盛有餘,則肩背痛風寒,汗出中風,小便數而欠。氣虛則肩背痛寒,少氣不足以息,溺色變。為此諸病,盛則瀉之,虛則補之,熱則疾之,寒則留之,陷下則灸之,不盛不虛,以經取之。盛者寸口大三倍於人迎,虛者則寸口反小於人迎也。

大腸手陽明之脈,起於大指次指之端,循指上廉,出合谷兩骨之間,上入兩筋之中,循臂上廉,入肘外廉,上臑外前廉,上肩,出\ytz{髃}骨之前廉,上出於柱骨之會上,下入缺盆絡肺,下膈屬大腸;其支者,從缺盆上頸貫頰,入下齒中,還出挾口,交人中,左之右,右之左,上挾鼻孔。是動則病齒痛頸腫。是主津液所生病者,目黃口乾,鼽衄,喉痹,肩前臑痛,大指次指痛不用。氣有餘則當脈所過者熱腫,虛則寒慄不復。為此諸病,盛則瀉之,虛則補之,熱則疾之,寒則留之,陷下則灸之,不盛不虛,以經取之。盛者人迎大三倍於寸口,虛者人迎反小於寸口也。

胃足陽明之脈,起於鼻之交頞中,旁納太陽之脈,下循鼻外,入上齒中,還出挾口環唇,下交承漿,卻循頤後下廉,出大迎,循頰車,上耳前,過客主人,循髮際,至額顱;其支者,從大迎前下人迎,循喉嚨,入缺盆,下膈屬胃絡脾;其直者,從缺盆下乳內廉,下挾臍,入氣街中;其支者,起於胃口,下循腹裡,下至氣街中而合,以下髀關,抵伏兔,下膝臏中,下循脛外廉,下足跗,入中指內間;其支者,下廉三寸而別,下入中指外間;其支者,別跗上,入大指間,出其端。是動則病灑灑振寒,善呻數欠顏黑,病至則惡人與火,聞木聲則惕然而驚,心欲動,獨閉戶塞牖而處,甚則欲上高而歌,棄衣而走,賁響腹脹,是為骭厥。是主血所生病者,狂瘧溫淫汗出,鼽衄,口喎唇胗,頸腫喉痹,大腹水腫,膝臏腫痛,循膺、乳、氣街、股、伏兔、骭外廉、足跗上皆痛,中指不用。氣盛則身以前皆熱,其有餘於胃,則消谷善飢,溺色黃。氣不足則身以前皆寒慄,胃中寒則脹滿。為此諸病,盛則瀉之,虛則補之,熱則疾之,寒則留之,陷下則灸之,不盛不虛,以經取之。盛者人迎大三倍於寸口,虛者人迎反小於寸口也。
脾足太陰之脈,起於大指之端,循指內側白肉際,過核骨後,上內踝前廉,上踹內,循脛骨後,交出厥陰之前,上膝股內前廉,入腹屬脾絡胃,上膈,挾咽,連舌本,散舌下;其支者,復從胃,別上膈,注心中。是動則病舌本強,食則嘔,胃脘痛,腹脹善噫,得後與氣則快然如衰,身體皆重。是主脾所生病者,舌本痛,體不能動搖,食不下,煩心,心下急痛,溏、瘕、洩,水閉、黃疸,不能臥,強立股膝內腫厥,足大指不用。為此諸病,盛則瀉之,虛則補之,熱則疾之,寒則留之,陷下則灸之,不盛不虛,以經取之。盛者,寸口大三倍於人迎,虛者,寸口反小於人迎也。

心手少陰之脈,起於心中,出屬心繫,下膈絡小腸;其支者,從心繫上挾咽,系目系;其直者,復從心繫卻上肺,下出腋下,下循臑內後廉,行太陰心主之後,下肘內,循臂內後廉,抵掌後銳骨之端,入掌內後廉,循小指之內出其端。是動則病嗌干心痛,渴而欲飲,是為臂厥。是主心所生病者,目黃脅痛,臑臂內後廉痛厥,掌中熱痛。為此諸病,盛則瀉之,虛則補之,熱則疾之,寒則留之,陷下則灸之,不盛不虛,以經取之。盛者寸口大再倍於人迎,虛者寸口反小於人迎也。

小腸手太陽之脈,起於小指之端,循手外側上腕,出踝中,直上循臂骨下廉,出肘內側兩筋之間,上循臑外後廉,出肩解,繞肩胛,交肩上,入缺盆絡心,循嚥下膈,抵胃屬小腸;其支者,從缺盆循頸上頰,至目銳眥,卻入耳中;其支者,別頰上\ytzn{䪼}抵鼻,至目內眥,斜絡於顴。是動則病嗌痛頷腫,不可以顧,肩似拔,臑似折。是主液所生病者,耳聾目黃頰腫,頸頷肩臑肘臂外後廉痛。為此諸病,盛則瀉之,虛則補之,熱則疾之,寒則留之,陷下則灸之,不盛不虛,以經取之。盛者人迎大再倍於寸口,虛者人迎反小於寸口也。

膀胱足太陽之脈,起於目內眥,上額交巔;其支者,從巔至耳上角;其直者,從巔入絡腦,還出別下項,循肩髆內,挾脊抵腰中,入循膂,絡腎屬膀胱;其支者,從腰中下挾脊貫臀,入膕中;其支者,從髆內左右,別下貫胛,挾脊內,過髀樞,循髀外從後廉下合膕中,以下貫踹內,出外踝之後,循京骨,至小指外側。是動則病沖頭痛,目似脫,項如拔,脊痛,腰似折,髀不可以曲,膕如結,踹如裂,是為踝厥。是主筋所生病者,痔瘧狂顛疾,頭囟項痛,目黃淚出鼽衄,項背腰尻膕踹腳皆痛,小指不用。為此諸病,盛則瀉之,虛則補之,熱則疾之,寒則留之,陷下則灸之,不盛不虛,以經取之。盛者人迎大再倍於寸口,虛者人迎反小於寸口也。

腎足少陰之脈,起於小指之下,邪走足心,出於然谷之下,循內踝之後,別入跟中,以上踹內,出膕內廉,上股內後廉,貫脊屬腎絡膀胱;其直者,從腎上貫肝膈,入肺中,循喉嚨,挾舌本;其支者,從肺出絡心,注胸中。是動則病飢不欲食,面如漆柴,咳唾則有血,喝喝而喘,坐而欲起,目\ytz{䀮}\ytz{䀮}如無所見,心如懸若飢狀,氣不足則善恐,心惕惕如人將捕之,是為骨厥。是主腎所生病者,口熱舌干,咽腫上氣,嗌干及痛,煩心心痛,黃疸腸澼,脊股內後廉痛,痿厥嗜臥,足下熱而痛。為此諸病,盛則瀉之,虛則補之,熱則疾之,寒則留之,陷下則灸之,不盛不虛,以經取之。灸則強食生肉,緩帶披髮,大杖重履而步。盛者寸口大再倍於人迎,虛者寸口反小於人迎也。

心主手厥陰心包絡之脈,起於胸中,出屬心包絡,下膈,歷絡三膲;其支者,循胸出脅,下腋三寸,上抵腋,下循臑內,行太陰少陰之間,入肘中,下臂行兩筋之間,入掌中,循中指出其端;其支者,別掌中,循小指次指出其端。是動則病手心熱,臂肘攣急,腋腫,甚則胸脅支滿,心中憺憺大動,面赤目黃,喜笑不休。是主脈所生病者,煩心心痛,掌中熱。為此諸病,盛則瀉之,虛則補之,熱則疾之,寒則留之,陷下則灸之,不盛不虛,以經取之。盛者寸口大一倍於人迎,虛者寸口反小於人迎也。

三焦手少陽之脈,起於小指次指之端,上出兩指之間,循手錶腕,出臂外兩骨之間,上貫肘,循臑外上肩,而交出足少陽之後,入缺盆,布羶中,散落心包,下膈,循屬三焦;其支者,從羶中上出缺盆,上項,系耳後,直上出耳上角,以屈下頰至\ytzn{䪼};其支者,從耳後入耳中,出走耳前,過客主人前,交頰,至目銳眥。是動則病耳聾渾渾焞焞,嗌腫喉痹。是主氣所生病者,汗出,目銳眥痛,頰痛,耳後肩臑肘臂外皆痛,小指次指不用。為此諸病,盛則瀉之,虛則補之,熱則疾之,寒則留之,陷下則灸之,不盛不虛,以經取之。盛者人迎大一倍於寸口,虛者人迎反小於寸口也。

膽足少陽之脈,起於目銳眥,上抵頭角,下耳後,循頸行手少陽之前,至肩上,卻交出手少陽之後,入缺盆;其支者,從耳後入耳中,出走耳前,至目銳眥後;其支者,別銳眥,下大迎,合於手少陽,抵於\ytzn{䪼},下加頰車,下頸合缺盆以下胸中,貫膈絡肝屬膽,循脅裡,出氣街,繞毛際,橫入髀厭中;其直者,從缺盆下腋,循胸過季脅,下合髀厭中,以下循髀陽,出膝外廉,下外輔骨之前,直下抵絕骨之端,下出外踝之前,循足跗上,入小指次指之間;其支者,別跗上,入大指之間,循大指岐骨內出其端,還貫爪甲,出三毛。是動則病口苦,善太息,心脅痛不能轉側,甚則面微有塵,體無膏澤,足外反熱,是為陽厥。是主骨所生病者,頭痛頷痛,目銳眥痛,缺盆中腫痛,腋下腫,馬刀俠癭,汗出振寒,瘧,胸脅肋髀膝外至脛絕骨外髁前及諸節皆痛,小指次指不用。為此諸病,盛則瀉之,虛則補之,熱則疾之,寒則留之,陷下則灸之,不盛不虛,以經取之。盛者人迎大一倍於寸口,虛者人迎反小於寸口也。

肝足厥陰之脈,起於大指叢毛之際,上循足跗上廉,去內踝一寸,上踝八寸,交出太陰之後,上膕內廉,循股陰入毛中,過陰器,抵小腹,挾胃屬肝絡膽,上貫膈,布脅肋,循喉嚨之後,上入頏顙,連目系,上出額,與督脈會於巔;其支者,從目系下頰裡,環唇內;其支者,復從肝別貫膈,上注肺。是動則病腰痛不可以俯仰,丈夫\ytz{㿉}疝,婦人少腹腫,甚則嗌干,面塵脫色。是主肝所生病者,胸滿嘔逆飧洩,狐疝遺溺閉癃。為此諸病,盛則瀉之,虛則補之,熱則疾之,寒則留之,陷下則灸之,不盛不虛,以經取之。盛者寸口大一倍於人迎,虛者寸口反小於人迎也。

手太陰氣絕則皮毛焦,太陰者行氣溫於皮毛者也,故氣不榮則皮毛焦,皮毛焦則津液去皮節,津液去皮節者則爪枯毛折,毛折者則毛先死,丙篤丁死,火勝金也。手少陰氣絕則脈不通,脈不通則血不流,血不流則髦色不澤,故其面黑如漆柴者,血先死,壬篤癸死,水勝火也。足太陰氣絕者則脈不榮肌肉,唇舌者肌肉之本也,脈不榮則肌肉軟,肌肉軟則舌萎人中滿,人中滿則唇反,唇反者肉先死,甲篤乙死,木勝土也。足少陰氣絕則骨枯,少陰者冬脈也,伏行而濡骨髓者也,故骨不濡則肉不能著也,骨肉不相親則肉軟卻,肉軟卻故齒長而垢發無澤,發無澤者骨先死,戊篤己死,土勝水也。足厥陰氣絕則筋絕,厥陰者肝脈也,肝者筋之合也,筋者聚於陰氣,而脈絡於舌本也,故脈弗榮則筋急,筋急則引舌與卵,故唇青舌卷卵縮則筋先死,庚篤辛死,金勝木也。五陰氣俱絕則目系轉,轉則目運,目運者為志先死,志先死則遠一日半死矣。六陽氣絕,則陰與陽相離,離則腠理髮洩,絕汗乃出,故旦佔夕死,夕佔旦死。

經脈十二者,伏行分肉之間,深而不見;其常見者,足太陰過於外踝之上,無所隱故也。諸脈之浮而常見者,皆絡脈也。六經絡手陽明少陽之大絡,起於五指間,上合肘中。飲酒者,衛氣先行皮膚,先充絡脈,絡脈先盛,故衛氣已平,營氣乃滿,而經脈大盛。脈之卒然動者,皆邪氣居之,留於本末;不動則熱,不堅則陷且空,不與眾同,是以知其何脈之動也。雷公曰:何以知經脈之與絡脈異也?黃帝曰:經脈者常不可見也,其虛實也以氣口知之,脈之見者皆絡脈也。雷公曰:細子無以明其然也。黃帝曰:諸絡脈皆不能經大節之間,必行絕道而出,入復合於皮中,其會皆見於外。故諸刺絡脈者,必刺其結上,甚血者雖無結,急取之以瀉其邪而出其血,留之發為痹也。凡診絡脈,脈色青則寒且痛,赤則有熱。胃中寒,手魚之絡多青矣;胃中有熱,魚際絡赤;其暴黑者,留久痹也;其有赤有黑有青者,寒熱氣也;其青短者,少氣也。凡刺寒熱者皆多血絡,必間日而一取之,血盡乃止,乃調其虛實;其小而短者少氣,甚者瀉之則悶,悶甚則僕不得言,悶則急坐之也。

手太陰之別,名曰列缺,起於腕上分間,並太陰之經直入掌中,散入於魚際。其病實則手銳掌熱,虛則欠\ytz{㰦},小便遺數,取之去腕半寸,別走陽明也。手少陰之別,名曰通裡,去腕一寸半,別而上行,循經入於心中,系舌本,屬目系。其實則支膈,虛則不能言,取之掌後一寸,別走太陽也。手心主之別,名曰內關,去腕二寸,出於兩筋之間,循經以上,繫於心包絡。心繫實則心痛,虛則為頭強,取之兩筋間也。手太陽之別,名曰支正,上腕五寸,內注少陰;其別者,上走肘,絡肩\ytz{髃}。實則節弛肘廢,虛則生\ytz{肬},小者如指痂疥,取之所別也。手陽明之別,名曰偏歷,去腕三寸,別入太陰;其別者,上循臂,乘肩\ytz{髃},上曲頰偏齒;其別者,入耳合於宗脈。實則齲聾,虛則齒寒痹隔,取之所別也。手少陽之別,名曰外關,去腕二寸,外遶臂,注胸中,合心主。病實則肘攣,虛則不收,取之所別也。足太陽之別,名曰飛揚,去踝七寸,別走少陰。實則鼽窒頭背痛,虛則鼽衄,取之所別也。足少陽之別,名曰光明,去踝五寸,別走厥陰,下絡足跗。實則厥,虛則痿躄,坐不能起,取之所別也。足陽明之別,名曰豐隆,去踝八寸,別走太陰;其別者,循脛骨外廉,上絡頭項,合諸經之氣,下絡喉嗌。其病氣逆則喉痹瘁瘖,實則狂巔,虛則足不收脛枯,取之所別也。足太陰之別,名曰公孫,去本節之後一寸,別走陽明;其別者,入絡腸胃。厥氣上逆則霍亂,實則腸中切痛,虛則鼓脹,取之所別也。足少陰之別,名曰大鍾,當踝後繞跟,別走太陽;其別者,並經上走於心包,下外貫腰脊。其病氣逆則煩悶,實則閉癃,虛則腰痛,取之所別者也。足厥陰之別,名曰蠡溝,去內踝五寸,別走少陽;其別者,徑脛上睾,結於莖。其病氣逆則睾腫卒疝,實則挺長,虛則暴癢,取之所別也。任脈之別,名曰尾翳,下鳩尾,散於腹。實則腹皮痛,虛則癢搔,取之所別也。督脈之別,名曰長強,挾膂上項,散頭上,下當肩胛左右,別走太陽,入貫膂。實則脊強,虛則頭重,高搖之,挾脊之有過者,取之所別也。脾之大絡,名曰大包,出淵腋下三寸,布胸脅。實則身盡痛,虛則百節盡皆縱,此脈若羅絡之血者,皆取之脾之大絡脈也。凡此十五絡者,實則必見,虛則必下,視之不見,求之上下,人經不同,絡脈異所別也。
